%Document-Author: Bonato Paolo + Biggeri Mattia
%Document-Date: 2016/03/24
%Document-Description: Documento di Specifica tecnica del gruppo SWEeneyThreads 

\documentclass[a4paper]{article}
\usepackage[english, italian]{babel}
\usepackage[T1]{fontenc}
\usepackage[utf8]{inputenc}
\usepackage{url}
\usepackage{graphicx}
\usepackage[hidelinks]{hyperref}
\usepackage{booktabs}
\usepackage{eurosym}
\usepackage{tabularx}
\usepackage{pifont}
\usepackage[table]{xcolor}
\usepackage{float}
\usepackage[]{appendix}
\usepackage{ltxtable} 
\usepackage{geometry}
\geometry{margin=1in}
\usepackage{longtable}
\usepackage{multirow}

\graphicspath{{Immagini/}}

\newcolumntype{Y}{>{\centering\arraybackslash}X}
\newcolumntype{s}{>{\hsize=.21\hsize}X}
\newcolumntype{f}{>{\hsize=.37\hsize}X}
\newcolumntype{m}{>{\hsize=.42\hsize}X}
\newcolumntype{t}{>{\hsize=.1\hsize}X}
\newcolumntype{r}{>{\hsize=.3\hsize}X}
\newcolumntype{k}{>{\hsize=.4\hsize}X}

\renewcommand{\abstractname}{Tabella contenuti}

\begin{document}
	
	\begin{titlepage}
		% Defines a new command for the horizontal lines, change thickness here
		\newcommand{\HRule}{\rule{\linewidth}{0.5mm}} 
		\center  
		
		% HEADING SECTION
		\textsc{\LARGE SWEeneyThreads}\\[1.5cm] 
		\textsc{\Large Actorbase}\\[0.5cm] 
		\textsc{\large a NoSQL DB based on the Actor model}\\[0.5cm]
		
		
		% TITLE SECTION
		\HRule \\[0.4cm]
		{ \huge \bfseries Specifica Tecnica}\\[0.4cm] 
		\HRule \\[1.5cm]
		
		% AUTHOR SECTION
		\begin{minipage}{0.4\textwidth}
			\begin{flushleft} \large
				\emph{Redattori:}\\
				Bonato Paolo \\
				Biggeri Mattia
			\end{flushleft}
		\end{minipage}
		~
		\begin{minipage}{0.4\textwidth}
			\begin{flushright} \large
				\emph{Approvazione:} \\
				\emph{Verifica:} \\
				 
			\end{flushright}
		\end{minipage}
		
		%immagine
		\begin{figure}[H]
			\centering
			\includegraphics[scale=0.8]{sweeney.png}
		\end{figure}
		\begin{center}
			Versione 1.0.0
		\end{center}
		% Date, change the \today to a set date if you want to be precise
		{\large \today}\\[3cm] 
		% Fill the rest of the page with whitespace
		\vfill  
	\end{titlepage}
	
	
	\tableofcontents
	
	\newpage 
	\section*{Diario delle modifiche}
		\begin{table}[H]
			\begin{tabularx}{\textwidth}{s f m X}
				\noalign{\hrule height 1.5pt}
				\rowcolor{orange!85} Versione & Data & Autore & Descrizione \\
				\noalign{\hrule height 0.5pt}
				1.0.0 & 2016-03-24 & \emph{Analisti} \newline Bonato Paolo \newline Biggeri Mattia & Creazione scheletro documento, stesura introduzione, definizione di metodo e formalismo di specifica. \\
				\noalign{\hrule height 1.5pt}
			\end{tabularx}
			\caption{Diario delle modifiche \label{tab:table_label}}
		\end{table}
	



	\newpage \section{Introduzione}
	\subsection{Scopo del documento}
		Il documento definisce la progettazione ad alto livello del progetto Actorbase.
		Verrà presentata l'architettura generale, le componenti, le classi e i design pattern utilizzati per realizzare il prodotto.
	\subsection{Scopo del prodotto}
		Il progetto consiste nella realizzazione di un DataBase NoSQL key-value basato sul modello ad 
		Attori con l'obiettivo di fornire una tecnologia adatta allo sviluppo di moderne 
		applicazioni che richiedono brevissimi tempi di risposta e che elaborano enormi quantità 
		di dati. Lo sviluppo porterà al rilascio del software sotto licenza MIT.
	\subsection{Glossario}
		Al fine di evitare ambiguità di linguaggio e di massimizzare la comprensione dei documenti, il 
      gruppo ha steso un documento interno che è il \emph{Glossario v1.0.3}. In esso saranno definiti, in modo
      chiaro e conciso i termini che possono causare ambiguità o incomprensione del testo.
	\subsection{Riferimenti}
		\begin{itemize}
			\item \textbf{Slide dell'insegnamento Ingegneria del software mod.A:} \\
			\url{http://www.math.unipd.it/~tullio/IS-1/2015/Dispense/E02.pdf}
			\item \textbf{Scala:} \\
			\url{http://www.scala-lang.org/}
			\item \textbf{Java:} \\
			\url{http://www.java.com/}
			\item \textbf{Akka:} \\
			\url{http://akka.io/}
		\end{itemize}
	\subsubsection{Normativi}
		\begin{itemize}
			\item \textbf{Norme di progetto:} \emph{Norme di progetto v1.1.2}
			\item \textbf{Capitolato d'appalto Actorbase (C1):} \\ 
			\url{http://www.math.unipd.it/~tullio/IS-1/2015/Progetto/C1p.pdf}
		\end{itemize}
		
		
	\newpage 
	\section{Tecnologie utilizzate}
	\subsection{Scala}
		Le possibili scelte dettate dal capitolato sono Java e Scala. Si è scelto di utilizzare Scala perché offre i seguenti vantaggi:
		\begin{itemize}
			\item \textbf{Concorrenza e distribuzione:} Ottimo supporto alla programmazione multi-threaded e distribuita, essenziale per la realizzazione di un prodotto responsive e scalabile.
			\item \textbf{Supporto di Akka:} Il linguaggio supporta la libreria Akka che è richiesta dal capitolato.
			\item \textbf{Preferenza del Committente:} Il Committente ha espresso la sua preferenza sull'utilizzo di Scala.
		\end{itemize}
		
	\subsection{Akka}
		L'utilizzo della libreria Akka è reso obbligatorio dal capitolato ed è la base del modello ad attori che costituisce il progetto.
	
	\newpage 
	\section{Descrizione dell'architettura}
		\subsection{Metodo e formalismo di specifica}
			Nell'esposizione dell'architettura del prodotto si procederà con un approccio di tipo top-down, ovvero dal generale al particolare.
			Inizialmente si descriverà la distinzione tra le componenti Client e Server con le rispettive dipendenze. Questa suddivisione verrà mantenuta nelle sezioni a seguire.
			Per ogni package saranno descritti brevemente il tipo, l'obiettivo e la funzione e saranno specificati eventuali figli, classi ed interazioni con altri package.
			Ogni classe sarà dotata di una breve descrizione e ne saranno specificate le responsabilità, le classi ereditate, le sottoclassi e le relazioni con altre classi.
			Successivamente saranno mostrati e descritti i diagrammi delle attività che coinvolgono l'utente.
			Infine si illustreranno degli esempi di utilizzo dei design pattern nell'architettura del sistema.
		\subsection{Architettura generale} 
 
	 
	\newpage 
	\section{Componenti}
	
	\newpage 
	\section{Package}
		\subsection{Lato server}
			\subsubsection{package server 1}
				% Immagine
				% Descrizione
				% Figli
				% Classi
				% Interazioni con altri package
				
		\subsection{Lato client}
			\subsubsection{package client 1}
				% Immagine
				% Descrizione
				% Figli
				% Classi
				% Interazioni con altri package
	\newpage 
	\section{Classi}
		\subsection{Lato server}
			\subsection{classe server 1}
				% Descrizione
				% Responsabilità
				% Classi ereditate
				% Sottoclassi
				% Relazioni con altre classi
		\subsection{Lato client}
			\subsection{classe client 1}
				% Descrizione
				% Responsabilità
				% Classi ereditate
				% Sottoclassi
				% Relazioni con altre classi
	\newpage 
	\section{Diagrammi delle attività}
	
	\newpage 
	\section{Design pattern}
	
	\newpage 
	\section{Stime di fattibilità e di bisogno di risorse}
	
	\newpage 
	\section{Tracciamento}
		\subsection{Tracciamento componenti-requisiti}
		\subsection{Tracciamento requisiti-componenti}
		
	\newpage 
	\section{Appendice}
	
	\cleardoublepage
	\addcontentsline{toc}{section}{\listfigurename}
	\listoffigures
	
	\cleardoublepage
	\addcontentsline{toc}{section}{\listtablename}
	\listoftables
		
\end{document}
	
