%Document-Author: Nicoletti Luca
%Document-Date: 2016/01/12
%Document-Description: Template tabelle

\documentclass[a4paper]{report}
\usepackage[english, italian]{babel}
\usepackage[T1]{fontenc}
\usepackage[utf8]{inputenc}
\usepackage{url}
\usepackage{graphicx}
\usepackage[hidelinks]{hyperref}
\usepackage{booktabs}
\usepackage{tabularx}
\usepackage{pifont}
\usepackage[table]{xcolor}
\usepackage{float}

%Tutte le colonne sotto sono ristrette rispetto a quelle normali, se si vuole una colonna
% più grande usare X
%Colonna piccola, per campi piccoli
\newcolumntype{s}{>{\hsize=.21\hsize}X} 
%Colonna media, per campi medi
\newcolumntype{f}{>{\hsize=.37\hsize}X}
%Colonna grande
\newcolumntype{m}{>{\hsize=.42\hsize}X}

\begin{document}
	\begin{table}[H]
		\begin{tabularx}{\textwidth}{s f m X}
			%Linea grossa per inizio tabella
			\noalign{\hrule height 1.5pt}
			%Riga colorata di titolo
			\rowcolor{orange!85} Campo1 & Campo2 & Campo3 & Campo4 \\
			%Linea grossa per separazione head da corpo
			\noalign{\hrule height 1.5pt}
			Dato1 & Dato2 & Dato 3 & Dato4 \\
			%Linea sottile per separare dati, non da usare sempre
			\noalign{\hrule height 0.5pt}
			Dato1 & Dato2 & Dato 3 & Dato4 \\
			\noalign{\hrule height 0.5pt}
			Dato1 & Dato2 & Dato 3 & Dato4 \\
			\noalign{\hrule height 1.5pt}
		\end{tabularx}
	\end{table}
\end{document}
