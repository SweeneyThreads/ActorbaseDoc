%Document-Author: Bonato Paolo + Bortolazzo Matteo + Maino Elia
%Document-Date: 2016-01-16
%Document-Description: Glossario


\documentclass[a4paper]{report}
\usepackage[english, italian]{babel}
\usepackage[T1]{fontenc}
\usepackage[utf8]{inputenc}
\usepackage{url}
\usepackage{graphicx}
\graphicspath{{../Immagini/}}
\usepackage[hidelinks]{hyperref}
\usepackage{booktabs}
\usepackage{tabularx}
\usepackage{pifont}
\usepackage[table]{xcolor}
\usepackage{float}
\usepackage{longtable}
\usepackage{booktabs}
\usepackage{ltxtable} 
\usepackage{geometry}
\geometry{margin=1in}
\newcolumntype{s}{>{\hsize=.21\hsize}X}
\newcolumntype{f}{>{\hsize=.37\hsize}X}
\newcolumntype{m}{>{\hsize=.42\hsize}X}

\newcommand{\mychapter}[2]{
	\setcounter{chapter}{#1}
	\setcounter{section}{0}
	\setcounter{subsection}{1}
	\chapter*{#2}
	\addcontentsline{toc}{chapter}{#2}
}

\renewcommand{\abstractname}{Tabella contenuti}

\begin{document}
	
	\begin{titlepage}
		% Defines a new command for the horizontal lines, change thickness here
		\newcommand{\HRule}{\rule{\linewidth}{0.5mm}} 
		\center  
		
		% HEADING SECTION
		\textsc{\LARGE SweeneyThreads}\\[1.5cm] 
		\textsc{\Large Actorbase}\\[0.5cm] 
		\textsc{\large a NoSQL DB based on the Actor model}\\[0.5cm]
		
		
		% TITLE SECTION
		\HRule \\[0.4cm]
		{ \huge \bfseries Glossario}\\[0.4cm] 
		\HRule \\[1.5cm]
		
		% AUTHOR SECTION
		\begin{minipage}{0.4\textwidth}
			\begin{flushleft} \large
				\emph{Redattori:}\\
				Bonato Paolo \\
				Bortolazzo Matteo \\
				Maino Elia
			\end{flushleft}
		\end{minipage}
		~
		\begin{minipage}{0.4\textwidth}
			\begin{flushright} \large
				\emph{Approvazione:} \\
				Padovan Tommaso \\
				\emph{Verifica:} \\
				Biggeri Mattia \\
				Tommasin Davide 
			\end{flushright}
		\end{minipage}
		
		%immagine
		\begin{figure}[H]
			\centering
			\includegraphics[scale=0.8]{sweeney.png}
		\end{figure}
		\begin{center}
			Versione 1.0.0
		\end{center}
		% Date, change the \today to a set date if you want to be precise
		{\large \today}\\[3cm] 
		% Fill the rest of the page with whitespace
		\vfill  
	\end{titlepage}
	
	
	\tableofcontents
	
	\mychapter{0}{Diario delle modifiche}
		\begin{table}[H]
			\begin{tabularx}{\textwidth}{s f m X}
				\noalign{\hrule height 1.5pt}
				\rowcolor{orange!85} Versione & Data & Autore & Descrizione \\
				\noalign{\hrule height 0.5pt}
				1.0.0 & 2016-01-21 & \emph{Analisti} \newline Bonato Paolo \newline Bortolazzo Matteo\newline Maino Elia  & Stesura scheletro documento e Introduzione
				 \\
				\noalign{\hrule height 1.5pt}
			\end{tabularx}
			\caption{Diario delle modifiche \label{tab:table_label}}
		\end{table}

	\mychapter{1}{Introduzione}
	\section{Scopo del documento}
		In questo documento sono raccolti tutti i termini che possono risultare sconosciuti ad un lettore
		esterno o che possono generare ambiguità. Per ognuno di essi si riporta una breve definizione
		che aiuti a chiarirne il significato.
	\section{Scopo del prodotto}
		Lo scopo del progetto è la realizzazione di un DataBase NoSQL key-value basato sul modello ad 
		Attori\ped{\textit{G}} con l'obiettivo di fornire una tecnologia adatta allo sviluppo di moderne 
		applicazioni che richiedono brevissimi tempi di risposta e che elaborano enormi quantità 
		di dati. Lo sviluppo porterà al rilascio del software sotto licenza MIT.
	\section{Riferimenti}
	\subsection{Informativi}	
		\begin{itemize}
			\item \textbf{Slide dell'insegnamento Ingegneria del software mod.A:} \\
			\url{http://www.math.unipd.it/~tullio/IS-1/2015/Dispense/E02.pdf}
		\end{itemize}
	\subsection{Normativi}
		\begin{itemize}
			\item \textbf{Norme di progetto:} \emph{Norme di progetto v1.1.1}
		\end{itemize}
		


\mychapter{1}{A}
\section{Actor} Actor si riferisce agli attori all'interno del progetto Actorbase, non al ruolo dell'utente nell'interazione con il sistema. Un Actor può essere StoreKeeper, Ninja, StoreFinder, Manager.
\section{Ambiente di lavoro} Quanto serve ai processi di produzione, è fatto persone, ruoli e procedure, infrastrutture.
\section{Approccio sistematico} Sapere perché si fa una cosa, approcciare un problema conoscendo i passi da svolgere. Essere sistematici porta sia all’efficacia che all’efficienza.
\section{Approccio disciplinato} Approccio basato su delle regole fissate in partenza da rispettare in ogni situazione al fine di rendere possibile l’organizzazione di un lavoro di gruppo e la pianificazione di tempi e costi del lavoro.
\section{Approccio quantificabile} Ottenere un sistema in cui si riesca a dare un costo affidabile alle operazioni da svolgere. La disciplina e la sistematicità rendono possibile la quantificazione del lavoro.
\section{Architettura SW} [...] che deve avere le seguenti qualità:
  * Sufficienza
  * Comprensibilità
  * Modularità
  * Robustezza 
  * Flessibilità
  * Riusabilità
  * Efficienza
  * Affidabiltà
  * Disponibilità
  * Security
  * Safety
  * Semplicità
  * Incapsulazione
  * Coesione
  * Basso accoppiamento 
\section{Attore} ruolo dell'utente nell'interazione con il sitema, svolgono il caso d'uso per raggiungere l'obiettivo, sono un buon mezzo di individuazione dei casi d'uso, non includono dettagli implementativi, sui modi di interazione. Vanno identificati secondo un processo.

\mychapter{2}{B}

\section{Baseline} E' l'insieme di Configuration Item ad una specifica Milestone. Rappresenta cioè un punto dello sviluppo verificato, approvato e garantito. Una volta fissata una baseline il progetto può solo avanzare, è una configurazione a cui si può tornare senza perdite in caso di fallimento.
\section{Best practice} Prassi (modo di fare) che per esperienza e per studio abbia mostrato di garantire i migliori risultati in circostanze note e specifiche. 
\section{Brainstorming} Tecnica di analisi dei requisiti che prevede discussioni creative e paritetiche con o senza il committente. Solitamente necessita di un facilitatore, ovvero un partecipante neutrale che fa si che si rispettino le regole, ed un rapporteur, ovvero colui che tiene nota della discussione riportando solo le parti importanti producendo così le “miniature”.

\mychapter{3}{C}

\section{Casi d'uso} Un caso d'uso é uno specifico modo di utilizzare il sistema da parte di un attore per eseguire una certa funzionalitá del sistema stesso; Una sequenza di transizioni tramite la quale si ottiene un risultato di valore misurabile.
\section{Configuration} E' un sistema in un suo istante; cambia a seconda di alcune svolte importanti (milestone).
\section{Configurazione} Vedi Configuration.
\section{Ciclo di vita del software} Gli stati che il prodotto assume dal concepimento al ritiro.
\section{Coesione} Fa si che le attività di processo siano ben definite e correlate tra loro (e così anche i compiti al loro interno).

\mychapter{4}{D}

\section{Design pattern} Soluzione progettuale ad un problema ricorrente. Definisce una funzionalità, ma lasciando dei gradi di libertà d'uso.

\mychapter{5}{E}

\section{Efficacia} È determinata dal grado di conformità del prodotto rispetto alle norme vigenti e agli obiettivi prefissati. Un prodotto software finale è da considerarsi tanto più efficace quanto più si avvicina agli obiettivi fissati inizialmente. Perseguire l’efficacia equivale a garantire la qualità del prodotto.
\section{Efficienza} È inversamente proporzionale alla quantità di risorse impiegate nell’esecuzione delle attività richieste. Perseguire l’efficienza equivale a contenere i costi e i tempi di produzione.
\section{Elicitare} riferito a concetti o informazioni, ottenerli mediante domande o altri stimoli.

\mychapter{6}{F}

\section{Fase} Durata temporale entro uno stato di ciclo di vita o una transazione tra essi.
\section{Flipped-classroom} una flipped-classroom inverte i metodi tradizionali di insegnamento. Gli alunni guardano i concetti a casa, per conto loro, comunicando con altri alunni e con il docente. Durante la lezione disegnata sono loro a condurre l'insegnamento, esponendo concetti e informazioni elicitate a casa.

\mychapter{7}{G}

\section{Gantt (diagramma di)} Diagramma per la dislocazione temporale delle attività. Per ogni attività indica durata temporale, sequenzialità/parallelismo con le altre. Permette di confrontare le stime con progressi reali.

\mychapter{8}{I}

\section{Incremento} Procedere per incrementi significa aggiungere (o togliere) a un impianto base, la caratteristica fondamentale dell’incremento è l’utilizzare un solo tipo di operazione, non si torna sui propri passi come per l’iterazione. Ha caratteristiche preferibili perché permette di stabilire una data di termine.
\section{Ingegneria (Engineering)} L’applicazione di principi scientifici e matematici a fini pratici (civili, sociali, prodotti di consumo).
\section{Ingegneria del software} Lo studio e l’applicazione dell’ingegneria al design, allo sviluppo e al mantenimento del software. L’approccio utilizzato deve essere sistematico, disciplinato e quantificabile. Non è una branchia dell’informatica ma un interfacciarsi di diverse discipline dell’informatica, matematica, economia, ingegneria, psicologia e sociologia.
\section{Ingegneria dei requisiti} È parte integrante dell’ingegneria di sistema e corrisponde all’insieme di attività necessarie per il trattamento sistematico dei requisiti. Tali attività si suddividono in due insiemi principali:
  1 Analisi
  2 Specifica di verifica e validazione
\section{ISO 8601:2004} (Data elements and interchange formats - Information interchange - Representation of dates and times) E' lo standard di riferimento per la rappresentazione di date ed orari.
\section{ISO/IEC 12207}  ISO 12207 è uno standard dell'ISO per la gestione del Ciclo di vita del software. Si propone di diventare lo standard di riferimento definendo tutte le attività svolte nel processo di sviluppo e mantenimento del software. Lo standard ISO 12207 stabilisce un processo di ciclo di vita del software, compreso processi ed attività relative alle specifiche ed alla configurazione di un sistema. Ad ogni processo corrisponde un insieme di risultati (outcome).
\section{Iterazione} Fare un passo indietro per fare un passo avanti, operare raffinamenti o rivisitazioni in maniera ciclica finché certe condizioni non vengono soddisfatte. L’iterazione ha caratteristiche molto pericolose perché è una operazione distruttiva (elimina quanto fatto e sovrascrive) e potenzialmente può durare all'infinito: non sa garantire un termine al processo di sviluppo, quindi è non quantificabile.

\mychapter{9}{M}

\section{Manutenibilità/Manutenzione} Qualità fondamentale del sw che si presenta in diverse forme:
  * Correttiva: per correggere difetti eventualmente rilevati
  * Adattativa: per adattare il sistema alla variazione dei requisiti
  * Evolutiva: per aggiungere funzionalità al sistema
Più risulta semplice eseguire queste operazioni più si considera il prodotto software mantenibile.
\section{Milestone} Letteralmente pietre miliari, vengono tipicamente utilizzate nella pianificazione e gestione di progetti complessi al fine di importanti traguardi intermedi nello svolgimento del progetto stesso. Molto spesso sono rappresentate da eventi, cioè da attività con durata zero o di un giorno, e vengono evidenziate in maniera diversa dalle altre attività nell’ambito dei documenti di progetto.
\section{Modello di ciclo di vita} Descrive come i processi si relazionano tra loro nel tempo rispetto agli stati di ciclo di vita, è la base concettuale intorno alla quale pianificare, organizzare, eseguire e controllare lo svolgimento delle attività necessarie. Esistono diversi possibili cicli di vita, non diversi per numero e significato degli stati ma diversi per le transizioni tra essi e le relative regole di attivazione.
\section{Modularità} Fa si che i processi siano tra loro relazionati in modo chiaro e distinto.

\mychapter{10}{O}

\section{Opportunità} Un’offerta relativa a un prodotto software non implica necessariamente un’opportunità (sw già esistente, progetto irrealizzabile, ...) dunque è compito del software engineer valutare la presenza di opportunità al momento di accettare o meno la commissione.
\section{Organizzazione di processo} 
  \section{Pianificare (plan)} Definire attività, scadenze, responsabilità, risorse utili a raggiungere specifici obiettivi di miglioramento.
  \section{Eseguire (do)} Eseguire le attività secondo i piani.
  \section{Valutare (check)} Verificare l’esito delle azioni di miglioramento rispetto alle attese.
  \section{Agire (act)} Applicare soluzioni correttive alle carenze rilevate.

\mychapter{11}{P}

\section{PERT} Acronimo di Program Evaluation and Review Technique. È una tecnica di project management (un grafico) nato per ridurre i tempi ed i costi per la progettazione. Con questa tecnica si tengono sotto controllo le attività di un progetto utilizzando una rappresentazione reticolare che aiuta ad individuare il cammino critico e a ragionare sulle scadenze di un progetto.
\section{Problematiche essenziali} Problematiche del software relative al suo sviluppo concettuale e dunque non eliminabili dal progresso tecnologico.
\section{Problematiche accidentali} Problematiche dello sviluppo software dovute all’inadeguatezza temporanea delle tecnologie di supporto, più legate agli strumenti che ai concetti, tali problemi possono essere risolti/eliminati grazie alle evoluzioni tecnologiche.
\section{Prodotto software} The set of computer programs, procedures, and possibly associated documentation and data (ISO/IEC 12207) Il software vero e proprio associato eventualmente alla relativa documentazione.
\section{Processo di ciclo di vita} Un processo di ciclo di vita specifica le attività che vanno svolte per causare transizioni nel ciclo di vita di un prodotto sw.
\section{Processo software (Way of working)} Il quadro metodologico, normativo e strategico delle attività di progetto per organizzare al meglio le attività necessarie all’interno di vincoli dati di tempo, di risorse e di obiettivi. Un processo è un insieme di attività correlate e coese che trasformano ingressi in uscite secondo regole fissate, consumando risorse nel farlo (Glossario ISO 9000).
\section{Progetto software} L’insieme delle attività di produzione di un software:
  * Pianificazione
  * Analisi dei requisiti
  * Progettazione
  * Realizzazione
  * Verifica e validazione
  * Manutenzione
\section{Progetto didattico} Sequenza di revisioni di avanzamento come principale modalità di interazione basata sulla logica di relazione cliente-fornitore. Ciascuna revisione di avanzamento richiede diversi adempimenti formali e valuta il raggiungimento di uno specifico stato di avanzamento.
\section{Programmatore} Figura professionale che svolge un’attività creativa fortemente personalizzata, scrive programmi per se stesso, da solo, sotto la propria esclusiva responsabilità tecnica.
\section{Prototipo} Serve per provare e scegliere soluzioni a lavoro in corso, possono essere di due tipi: “usa e getta” (iterazione) o “baseline” (fornire stati di incremento). L’utilizzo di prototipi comporta in ogni caso un costo che se supera il beneficio ottenibile ne rende improduttivo l’uso.

\mychapter{12}{R}

\section{Requisito}  Una delle primissime attività da svolgere per quanto riguarda lo sviluppo è la definizione dei requisiti del prodotto, un sistema software sarà considerato efficace se e solo se si dimostrerà in grado di soddisfare i requisiti. Inoltre la definizione dei requisiti fornisce subito un’idea sulla realizzabilità del prodotto (l’opportunità è sostenibile?) e sui tempi di sviluppo.
  1 Condizione (capability) necessaria a un utente per risolvere un problema o raggiungere un obiettivo [lato bisogno]
  2 Condizione (capability) che deve essere soddisfatta o posseduta da un sistema per adempiere a un obbligo (contratto, standard, specifica, documento formale) [lato soluzione]
Spesso un requisito lato utente si trasforma in molti requisiti lato soluzione.
\section{Riuso} Molte volte nel swe risulta decisamente produttiva la tecnica del riuso a patto che lo si faccia in maniera sistematica e consapevole. Adattare componenti preesistenti alle proprie esigenze se fatto con criterio può far risparmiare al team molte risorse, aumentando l’efficienza.
\section{Roulo} Funzione aziendale assegnata a progetto.

\mychapter{13}{S}

\section{Servizio} Mezzo per fornire valore all'utente, agevolando il raggiungimento dei suoi obiettivi, sollevandolo da costi e rischi.
\section{Software engineer} Realizza parte di un sistema complesso con la consapevolezza che potrà essere usato, completato e modificato da altri. Deve guardare e comprendere il quadro generale nel quale il sistema cui contribuisce si colloca, deve operare compromessi intelligenti e lungimiranti tra visioni e spinte contrapposte.
\section{Standard di processo} Sono fondamentalmente di due tipi: Standard come modello di azione e Standard come modello di valutazione. I primi definiscono e impongono/propongono delle procedure da seguire, i secondi cercano di identificare una “best practice” da usare come linea di valutazione dell’operato.
\section{Standard ISO/IEC 12207} Modello più noto e riferito, è ad alto livello e divide i processi in tre macro-aree:
  * Processi primari
  * Processi di supporto
  * Processi organizzativi
\section{Stakeholder (People)} L’insieme di persone a vario titolo coinvolte nel ciclo di vita del SW con influenza sul prodotto.
  * Business management
  * Project management
  * Team di sviluppo
  * Clienti
  * Utenti Finali

\mychapter{14}{T}

\section{Task} Compito organizzativo ottenuto dalla frantumazione di un processo. Generalmente è delle dimensioni adeguate per essere svolto da una solo persona ed è parallelizzabile.

\mychapter{15}{U}

\section{UTF-8} (Unicode Transformation Format, 8 bit) è una codifica dei caratteri Unicode in sequenze di lunghezza variabile di byte, creata da Rob Pike e Ken Thompson. UTF-8 usa gruppi di byte per rappresentare i caratteri Unicode.

\mychapter{16}{V}

\section{Validazione} Serve ad accertare che il prodotto realizzato corrisponda alle attese (Did I built the right system?) ed è rivolta ai prodotti finali.
\section{Verifica} Serve ad accertare che l’esecuzione delle attività di processo non abbia introdotto errori. Si confronta ciò che si sta facendo con le regole che si devono rispettare. (Am I building the system right?)

\mychapter{17}{W}

\section{WBS} Vedi Work Breakdown Structrure.
\section{Work Breakdown Structure} Identifica l'elenco di tutte le attività di un progetto. Le WBS sono usate dal Project manager come supporto per le attività di cui è responsabile.
\end{document}