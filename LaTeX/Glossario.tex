%Document-Author: Maino Elia + Nicoletti Luca
%Document-Date: 2016-01-16
%Document-Description: Glossario


\documentclass[a4paper]{article}
\usepackage[english, italian]{babel}
\usepackage[T1]{fontenc}
\usepackage[utf8]{inputenc}
\usepackage{url}
\usepackage{graphicx}
\graphicspath{{Immagini/}}
\usepackage[hidelinks]{hyperref}
\usepackage{booktabs}
\usepackage{tabularx}
\usepackage{pifont}
\usepackage[table]{xcolor}
\usepackage{float}
\usepackage{booktabs}
\usepackage{geometry}
\geometry{margin=1in}
\newcolumntype{s}{>{\hsize=.21\hsize}X}
\newcolumntype{f}{>{\hsize=.37\hsize}X}
\newcolumntype{m}{>{\hsize=.42\hsize}X}


\begin{document}
	
	\begin{titlepage}
		% Defines a new command for the horizontal lines, change thickness here
		\newcommand{\HRule}{\rule{\linewidth}{0.5mm}} 
		\center  
		
		% HEADING SECTION
		\textsc{\LARGE SweeneyThreads}\\[1.5cm] 
		\textsc{\Large Actorbase}\\[0.5cm] 
		\textsc{\large a NoSQL DB based on the Actor model}\\[0.5cm]
		
		
		% TITLE SECTION
		\HRule \\[0.4cm]
		{ \huge \bfseries Glossario}\\[0.4cm] 
		\HRule \\[1.5cm]
		
		% AUTHOR SECTION
		\begin{minipage}{0.4\textwidth}
			\begin{flushleft} \large
				\emph{Redattori:}\\
				Maino Elia\\
				Nicoletti Luca
			\end{flushleft}
		\end{minipage}
		~
		\begin{minipage}{0.4\textwidth}
			\begin{flushright} \large
				\emph{Approvazione:} \\
				Padovan Tommaso \\
				\emph{Verifica:} \\
				Biggeri Mattia \\
				Tommasin Davide 
			\end{flushright}
		\end{minipage}
		
		%immagine
		\begin{figure}[H]
		\centering
			\includegraphics[scale=0.8]{sweeney.png}
		\end{figure}
		\begin{center}
			Versione 1.0.2
		\end{center}
		% Date, change the \today to a set date if you want to be precise
		{\large \today}\\[3cm] 
		% Fill the rest of the page with whitespace
		\vfill  
	\end{titlepage}
	
	
	
	\section*{Diario delle modifiche}
		\begin{table}[H]
			\begin{tabularx}{\textwidth}{s f m X}
				\noalign{\hrule height 1.5pt}
				\rowcolor{orange!85} Versione & Data & Autore & Descrizione \\
				\noalign{\hrule height 1.5pt}
				1.0.2 & 2016-02-23 & \emph{Responsabile} \newline Nicoletti Luca & Rivista l'intera tabella del diario delle modifiche 
				secondo le indicazioni  \\
				\noalign{\hrule height 0.5pt}
				1.0.1 & 2016-02-19 & \emph{Responsabile} \newline Nicoletti Luca & Prime modifiche post consegna RR: tolto il comando
				per la creazione di paragrafi da noi introdotto, sostituiti tutti i "paragrafi" con sezioni; cambiata classe del 
				documento da "report" ad "article" come consigliato nella correzione. \\
				\noalign{\hrule height 0.5pt}
				1.0.0 & 2016-01-21 & \emph{Verificatore} \newline Tommasin Davide & Verificato il documento \\
				\noalign{\hrule height 0.5pt}
				0.0.4 & 2016-01-21 & \emph{Analista} \newline Maino Elia & Inserimento voci \\
				\noalign{\hrule height 0.5pt}
				0.0.3 & 2016-01-21 & \emph{Amministratore} \newline Nicoletti Luca & Riordinato il documento, tolto indice \\
				\noalign{\hrule height 0.5pt}
				0.0.2 & 2016-01-21 & \emph{Amministratore} \newline Nicoletti Luca & Sistemati elenchi puntati interni \\
				\noalign{\hrule height 0.5pt}
				0.0.1 & 2016-01-21 & \emph{Analista} \newline Maino Elia & Stesura scheletro documento e Introduzione \\
				\noalign{\hrule height 1.5pt}
			\end{tabularx}
			\caption{Diario delle modifiche \label{tab:table_label}}
		\end{table}

%	\subsection{Informativi}	
%		\begin{itemize}
%			\item \textbf{Slide dell'insegnamento Ingegneria del software mod.A:} \\
%			\url{http://www.math.unipd.it/~tullio/IS-1/2015/Dispense/E02.pdf}
%			\item \textbf{Wikipedia:} \\
%			\url{https://it.wikipedia.org/wiki/Pagina_principale}
%		\end{itemize}
	
	\newpage

	\section*{A}
		\begin{itemize}
			\item \textbf{Actor:} Actor si riferisce agli attori all'interno del progetto Actorbase, 
			non al ruolo dell'utente nell'interazione con il sistema. Un Actor può essere StoreKeeper, 
			Ninja, StoreFinder, Manager.
			\item \textbf{Ambiente di lavoro:} Quanto serve ai processi di produzione, è fatto di persone, 
			ruoli e procedure, infrastrutture.
			\item \textbf{Approccio sistematico:} Sapere perché si fa una cosa, approcciare un problema 
			conoscendo i passi da svolgere. Essere sistematici porta sia all’efficacia che all’efficienza.
			\item \textbf{Approccio disciplinato:} Approccio basato su delle regole fissate in partenza da 
			rispettare in ogni situazione al fine di rendere possibile l’organizzazione di un lavoro di gruppo 
			e la pianificazione di tempi e costi del lavoro.
			\item \textbf{Approccio quantificabile:} Ottenere un sistema in cui si riesca a dare un costo 
			affidabile alle operazioni da svolgere. La disciplina e la sistematicità rendono possibile la 
			quantificazione del lavoro.
			\item \textbf{Architettura SW:} Insieme di regole che danno una struttura al software. Deve avere le seguenti qualità:
			\begin{itemize}
				\item  Sufficienza
				\item  Comprensibilità
				\item  Modularità
				\item  Robustezza 
				\item  Flessibilità
				\item  Riusabilità
				\item  Efficienza
				\item  Affidabiltà
				\item  Disponibilità
				\item  Security
				\item  Safety
				\item  Semplicità
				\item  Incapsulazione
				\item  Coesione
				\item  Basso accoppiamento
			\end{itemize}
			\item \textbf{Attore:} Elemento esterno di un caso d'uso che interagisce con il sistema. 
		\end{itemize}
		
	\section*{B}
		\begin{itemize}
			\item \textbf{Baseline:} E' l'insieme di Configuration Item consolidato ad una specifica Milestone. Rappresenta 
			cioè un punto dello sviluppo verificato, approvato e garantito.\'E una configurazione a cui si può tornare senza perdite
			 in caso di fallimento.
			\item \textbf{Best practice:} Prassi (modo di fare) che per esperienza e per studio abbia mostrato di 
			garantire i migliori risultati in circostanze note e specifiche. 
			\item \textbf{Brainstorming:} Tecnica di analisi dei requisiti che prevede discussioni creative e paritetiche 
			con o senza il committente. Solitamente necessita di un facilitatore, ovvero un partecipante neutrale che fa 
			si che si rispettino le regole, ed un rapporteur, ovvero colui che tiene nota della discussione riportando solo 
			le parti importanti producendo così le “miniature”.
		\end{itemize}
		
	\section*{C}
		\begin{itemize}
			\item \textbf{Casi d'uso:} Un caso d'uso é la descrizione di un'interazione tra uno o più attori esterni ed il sistema, al
			fine di effettuare una determinata operazione. I casi d'uso sono utili per identificare requisiti in quanto descrivono
			i modi in cui il prodotto verrà utilizzato.
			\item \textbf{Commit:} Un commit si effettua quando si copiano le modifiche eseguite sui file locali nella directory del repository  (il software di controllo versione controlla quali file sono stati modificati dall’ultima sincronizzazione).
			\item \textbf{Configuration:} E' un sistema in un suo istante; cambia a seconda di alcune svolte importanti (milestone).
			\item \textbf{Configurazione:} Vedi Configuration.
			\item \textbf{Ciclo di vita del software:} Gli stati che il prodotto assume dal concepimento al ritiro.
			\item \textbf{Coesione:} Fa si che le attività di processo siano ben definite e correlate tra loro (e così anche i 
			compiti al loro interno).
			\item \textbf{Ciclo PDCA:} Organizzazione dei processi basata sul principio del miglioramento continuo
			\begin{itemize}
				\item \textbf{Pianificare (plan):} Definire attività, scadenze, responsabilità, risorse utili a raggiungere 
				specifici obiettivi di miglioramento.
				\item \textbf{Eseguire (do):} Eseguire le attività secondo i piani.
				\item \textbf{Valutare (check):} Verificare l’esito delle azioni di miglioramento rispetto alle attese.
				\item \textbf{Agire (act):} Applicare soluzioni correttive alle carenze rilevate.
			\end{itemize}
		\end{itemize}
		
	\section*{D}
		\begin{itemize}
			\item \textbf{Design pattern:} Soluzione progettuale ad un problema ricorrente. 
		\end{itemize}
	\section*{E}
		\begin{itemize}
			\item \textbf{Efficacia:} È determinata dal grado di conformità del prodotto rispetto alle norme vigenti e 
			agli obiettivi prefissati. Un prodotto software finale è da considerarsi tanto più efficace quanto più si 
			avvicina agli obiettivi fissati inizialmente. Perseguire l'efficacia equivale a garantire la qualità del prodotto.
			\item \textbf{Efficienza:} È inversamente proporzionale alla quantità di risorse impiegate nell'esecuzione 
			delle attività richieste. Perseguire l'efficienza equivale a contenere i costi e i tempi di produzione.
			\item \textbf{Elicitare:} Riferito a concetti o informazioni, ottenerli mediante domande o altri stimoli.
		\end{itemize}
		
	\section*{F}
		\begin{itemize}
			\item \textbf{Fase:} Durata temporale entro uno stato di ciclo di vita o una transazione tra essi.
		\end{itemize}
		
	\section*{G}
		\begin{itemize}
			\item \textbf{Gantt (diagramma di):} Diagramma per la dislocazione temporale delle attività. Per ogni 
			attività indica durata temporale, sequenzialità/parallelismo con le altre. Permette di confrontare le 
			stime con progressi reali.
		\end{itemize}
		
	\section*{I}
		\begin{itemize}
			\item \textbf{Incremento:} Procedere per incrementi significa aggiungere (o togliere) a un impianto base, 
			la caratteristica fondamentale dell’incremento è l’utilizzare un solo tipo di operazione, non si torna sui 
			propri passi come per l’iterazione. Ha caratteristiche preferibili perché permette di stabilire una data di 
			termine.
			\item \textbf{Ingegneria (Engineering):} L’applicazione di principi scientifici e matematici a fini pratici 
			(civili, sociali, prodotti di consumo).
			\item \textbf{Ingegneria del software:} Lo studio e l’applicazione dell’ingegneria al design, allo sviluppo 
			e al mantenimento del software. L’approccio utilizzato deve essere sistematico, disciplinato e quantificabile. 
			Non è una branchia dell’informatica ma un interfacciarsi di diverse discipline dell’informatica, matematica, 
			economia, ingegneria, psicologia e sociologia.
			\item \textbf{Ingegneria dei requisiti:} È parte integrante dell’ingegneria di sistema e corrisponde all’insieme 
			di attività necessarie per il trattamento sistematico dei requisiti. Tali attività si suddividono in due insiemi 
			principali:
			\begin{itemize}
				\item Analisi
			  	\item Specifica di verifica e validazione
			\end{itemize}
			\item \textbf{ISO 8601:2004:} (Data elements and interchange formats - Information interchange - Representation 
			of dates and times) E' lo standard di riferimento per la rappresentazione di date ed orari.
			\item \textbf{ISO/IEC 12207:}  ISO 12207 è uno standard dell'ISO per la gestione del Ciclo di vita del software. 
			Si propone di diventare lo standard di riferimento definendo tutte le attività svolte nel processo di sviluppo e 
			mantenimento del software. Lo standard ISO 12207 stabilisce un processo di ciclo di vita del software, compreso 
			processi ed attività relative alle specifiche ed alla configurazione di un sistema. Ad ogni processo corrisponde 
			un insieme di risultati (outcome).
			\item \textbf{Iterazione:} Fare un passo indietro per fare un passo avanti, operare raffinamenti o rivisitazioni 
			in maniera ciclica finché certe condizioni non vengono soddisfatte. L’iterazione ha caratteristiche molto pericolose 
			perché è una operazione distruttiva (elimina quanto fatto e sovrascrive) e potenzialmente può durare all'infinito: 
			non sa garantire un termine al processo di sviluppo, quindi è non quantificabile.
		\end{itemize}
	
	\section*{J}	
	\begin{itemize}
		\item \textbf{JVM:} La macchina virtuale Java, detta anche Java Virtual Machine o JVM, è il componente della piattaforma
		 Java che esegue i programmi tradotti in bytecode dopo una prima compilazione.
	\end{itemize}
	
	\section*{L}
	\begin{itemize}
		\item \textbf{LaTeX:} LaTeX  è un linguaggio di markup usato per la preparazione di testi basato sul programma di
		 composizione tipografica TEX.
	\end{itemize}
		
	\section*{M}
		\begin{itemize}
			\item \textbf{Manutenibilità/Manutenzione:} Qualità fondamentale del sw che si presenta in diverse forme:
			\begin{itemize}
				\item  \textbf{Correttiva:} per correggere difetti eventualmente rilevati
			  	\item  \textbf{Adattativa:} per adattare il sistema alla variazione dei requisiti
			  	\item  \textbf{Evolutiva:} per aggiungere funzionalità al sistema
			\end{itemize}
			Più risulta semplice eseguire queste operazioni più si considera il prodotto software mantenibile.
			\item \textbf{Milestone:} Letteralmente pietre miliari, vengono tipicamente utilizzate nella pianificazione 
			e gestione di progetti complessi al fine di importanti traguardi intermedi nello svolgimento del progetto stesso. 
			Molto spesso sono rappresentate da eventi, cioè da attività con durata zero o di un giorno, e vengono evidenziate 
			in maniera diversa dalle altre attività nell’ambito dei documenti di progetto.
			\item \textbf{Modello di ciclo di vita:} Descrive come i processi si relazionano tra loro nel tempo rispetto agli 
			stati di ciclo di vita, è la base concettuale intorno alla quale pianificare, organizzare, eseguire e controllare 
			lo svolgimento delle attività necessarie. Esistono diversi possibili cicli di vita, non diversi per numero e significato 
			degli stati ma diversi per le transizioni tra essi e le relative regole di attivazione.
			\item \textbf{Modularità:} Fa si che i processi siano tra loro relazionati in modo chiaro e distinto.		
		\end{itemize}
		
	\section*{N}
	\begin{itemize}
		\item \textbf{NoSQL:} NoSQL è un movimento che promuove sistemi software dove la persistenza dei dati è caratterizzata dal fatto di non utilizzare il modello relazionale, di solito usato dai database tradizionali (RDBMS).
	\end{itemize}
		
	\section*{O}
		\begin{itemize}	
			\item \textbf{Opportunità:} Un’offerta relativa a un prodotto software non implica necessariamente un’opportunità 
			(sw già esistente, progetto irrealizzabile, ...) dunque è compito del software engineer valutare la presenza di 
			opportunità al momento di accettare o meno la commissione.	
		\end{itemize}
		
	\section*{P}
		\begin{itemize}
			\item \textbf{PERT (diagramma di):} Acronimo di Program Evaluation and Review Technique. È una tecnica di project management 
			(un diagramma) nato per ridurre i tempi ed i costi per la progettazione. Con questa tecnica si tengono sotto 
			controllo le attività di un progetto utilizzando una rappresentazione reticolare che aiuta ad individuare il 
			cammino critico e a ragionare sulle scadenze di un progetto.
			\item \textbf{Problematiche essenziali:} Problematiche del software relative al suo sviluppo concettuale e dunque 
			non eliminabili dal progresso tecnologico.
			\item \textbf{Problematiche accidentali:} Problematiche dello sviluppo software dovute all’inadeguatezza temporanea 
			delle tecnologie di supporto, più legate agli strumenti che ai concetti, tali problemi possono essere risolti/eliminati 
			grazie alle evoluzioni tecnologiche.
			\item \textbf{Prodotto software:} Il software vero e proprio associato eventualmente alla relativa documentazione.
			\item \textbf{Processo di ciclo di vita:} Un processo di ciclo di vita specifica le attività che vanno svolte per 
			causare transizioni nel ciclo di vita di un prodotto sw.
			\item \textbf{Processo software (Way of working):} Il quadro metodologico, normativo e strategico delle attività di 
			progetto per organizzare al meglio le attività necessarie all’interno di vincoli dati di tempo, di risorse e di obiettivi. 
			Un processo è un insieme di attività correlate e coese che trasformano ingressi in uscite secondo regole fissate, consumando 
			risorse nel farlo (Glossario ISO 9000).
			\item \textbf{Progetto software:} L’insieme delle attività di produzione di un software:
			\begin{itemize}
				\item  Pianificazione
			  	\item  Analisi dei requisiti
			  	\item  Progettazione
			  	\item  Realizzazione
			  	\item  Verifica e validazione
			  	\item  Manutenzione
			\end{itemize}
			\item \textbf{Progetto didattico:} Sequenza di revisioni di avanzamento come principale modalità di interazione basata 
			sulla logica di relazione cliente-fornitore. Ciascuna revisione di avanzamento richiede diversi adempimenti formali e 
			valuta il raggiungimento di uno specifico stato di avanzamento.
			\item \textbf{Prototipo:} Serve per provare e scegliere soluzioni a lavoro in corso, possono essere di due tipi: “usa e getta” 
			(iterazione) o “baseline” (fornire stati di incremento). L’utilizzo di prototipi comporta in ogni caso un costo che se supera 
			il beneficio ottenibile ne rende improduttivo l’uso.
		\end{itemize}
		
	\section*{R}
		\begin{itemize}
			\item \textbf{Requisito:}  Una delle primissime attività da svolgere per quanto riguarda lo sviluppo è la 
			definizione dei requisiti del prodotto, un sistema software sarà considerato efficace se e solo se si dimostrerà 
			in grado di soddisfare i requisiti. Inoltre la definizione dei requisiti fornisce subito un’idea sulla realizzabilità 
			del prodotto (l'opportunità è sostenibile?) e sui tempi di sviluppo.
			\begin{enumerate}
				\item \textbf{Condizione (capability):} necessaria a un utente per risolvere un problema o raggiungere un obiettivo [lato bisogno]
				\item \textbf{Condizione (capability):} che deve essere soddisfatta o posseduta da un sistema per adempiere a un obbligo (contratto, 
			  standard, specifica, documento formale) [lato soluzione]
			\end{enumerate}
			Spesso un requisito lato utente si trasforma in molti requisiti lato soluzione.
			\item \textbf{Riuso:} Molte volte nell'ingegneria del software risulta decisamente produttiva la tecnica del riuso a patto che lo si faccia 
			in maniera sistematica e consapevole. Adattare componenti preesistenti alle proprie esigenze se fatto con criterio può far 
			risparmiare al team molte risorse, aumentando l'efficienza.
			\item \textbf{Ruolo:} Funzione aziendale assegnata a progetto.
		\end{itemize}
		
	\section*{S}
		\begin{itemize}
			\item \textbf{Servizio:} Mezzo per fornire valore all'utente, agevolando il raggiungimento dei suoi obiettivi, 
			sollevandolo da costi e rischi.
			\item \textbf{Scala:} Scala (da Scalable Language) è un linguaggio di programmazione di tipo general-purpose
			 multi-paradigma studiato per integrare le caratteristiche e funzionalità dei linguaggi orientati agli oggetti e dei linguaggi
			  funzionali. La compilazione di codice sorgente Scala produce Java bytecode per l'esecuzione su una JVM.
			\item \textbf{Software engineer:} Realizza parte di un sistema complesso con la consapevolezza che potrà essere 
			usato, completato e modificato da altri. Deve guardare e comprendere il quadro generale nel quale il sistema cui 
			contribuisce si colloca, deve operare compromessi intelligenti e lungimiranti tra visioni e spinte contrapposte.
			\item \textbf{Standard di processo:} Sono fondamentalmente di due tipi: Standard come modello di azione e Standard 
			come modello di valutazione. I primi definiscono e impongono/propongono delle procedure da seguire, i secondi 
			cercano di identificare una “best practice” da usare come linea di valutazione dell’operato.
			\item \textbf{Stakeholder (People):} L’insieme di persone a vario titolo coinvolte nel ciclo di vita del SW con 
			influenza sul prodotto.
			\begin{itemize}
				\item  Business management
			  	\item  Project management
			  	\item  Team di sviluppo
			  	\item  Clienti
			  	\item  Utenti Finali
			\end{itemize}
		\end{itemize}
		
	\section*{T}
		\begin{itemize}	
			\item \textbf{Task:} Compito organizzativo ottenuto dalla frantumazione di un processo. Generalmente è delle 
			dimensioni adeguate per essere svolto da una solo persona ed è parallelizzabile.
		\end{itemize}
		
	\section*{U}
		\begin{itemize}	
			\item \textbf{UML:} Famiglia di notazioni grafiche che si basano su un singolo meta-modello e servono a supportare la descrizione e il progetto dei sistemi software	
			\item \textbf{UTF-8:} (Unicode Transformation Format, 8 bit) è una codifica dei caratteri Unicode in sequenze 
			di lunghezza variabile di byte, creata da Rob Pike e Ken Thompson. UTF-8 usa gruppi di byte per rappresentare i caratteri Unicode.		
		\end{itemize}
		
	\section*{V}
		\begin{itemize}
			\item \textbf{Validazione:} Serve ad accertare che il prodotto realizzato corrisponda alle attese (Did I 
			built the right system?) ed è rivolta ai prodotti finali.
			\item \textbf{Verifica:} Serve ad accertare che l’esecuzione delle attività di processo non abbia introdotto 
			errori. Si confronta ciò che si sta facendo con le regole che si devono rispettare. (Am I building the system right?)	
		\end{itemize}
		
	\section*{W}
		\begin{itemize}
			\item \textbf{WBS:} Vedi Work Breakdown Structrure.
			\item \textbf{Work Breakdown Structure:} Identifica l'elenco di tutte le attività di un progetto. Le WBS sono 
			usate dal Project manager come supporto per le attività di cui è responsabile.
		\end{itemize}

\end{document}