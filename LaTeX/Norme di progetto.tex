%Document-Author: Nicoletti Luca + Maino Elia
%Document-Date: 2016/01/08
%Document-Description: Documento di norme di progetto, include tutte le norme del gruppo SWEeneyThreads

\documentclass[a4paper]{article}
\usepackage[english, italian]{babel}
\usepackage[T1]{fontenc}
\usepackage[utf8]{inputenc}
\usepackage{url}
\usepackage{graphicx}
\usepackage[hidelinks]{hyperref}
\usepackage{booktabs}
\usepackage{tabularx}
\usepackage{pifont}
\usepackage[table]{xcolor}
\usepackage{float}
\usepackage{longtable}
\usepackage{multirow}
\usepackage{ltxtable} 
\usepackage{geometry}
\geometry{margin=1in}

\usepackage{multirow}


\graphicspath{{Immagini/}}
 
\newcolumntype{s}{>{\hsize=.21\hsize}X}
\newcolumntype{f}{>{\hsize=.37\hsize}X}
\newcolumntype{m}{>{\hsize=.42\hsize}X}

\renewcommand{\abstractname}{Tabella contenuti}

\begin{document}

	\begin{titlepage}
		% Defines a new command for the horizontal lines, change thickness here
		\newcommand{\HRule}{\rule{\linewidth}{0.5mm}}
		\center

		% HEADING SECTION
		\textsc{\LARGE SWEeneyThreads}\\[1.5cm]
		\textsc{\Large Actorbase}\\[0.5cm]
		\textsc{\large a NoSQL DB based on the Actor model}\\[0.5cm]


		% TITLE SECTION
		\HRule \\[0.4cm]
		{ \huge \bfseries Norme di progetto}\\[0.4cm]
		\HRule \\[1.5cm]

		% AUTHOR SECTION
		\begin{minipage}{0.4\textwidth}
			\begin{flushleft} \large
				\emph{Redattori:}\\
				Nicoletti Luca \\
				Tommasin Davide\\
			\end{flushleft}
		\end{minipage}
		~
		\begin{minipage}{0.4\textwidth}
			\begin{flushright} \large
				\emph{Approvazione:} \\
                Maino Elia \\
				\emph{Verifica:} \\
                Bonato Paolo \\
			\end{flushright}
		\end{minipage}

		%immagine
		\begin{figure}[H]
			\centering
			\includegraphics[scale=0.8]{sweeney.png}
		\end{figure}
		\begin{center}
			Versione 2.0.3
		\end{center}
		% Date, change the \today to a set date if you want to be precise
		{\large \today}\\[3cm]
		% Fill the rest of the page with whitespace
		\vfill
	\end{titlepage}


	\tableofcontents

	\newpage
	\section*{Diario delle modifiche}
		\LTXtable{\textwidth}{Tabelle/tabelle_diario_modifiche/tabella_norme.tex}	

	\newpage
	\section{Introduzione}
		\subsection{Scopo del documento}
			Nel seguente documento sono definite le norme che l'intero gruppo SWEeneyThreads si
			impegna a rispettare durante lo svolgimento
			del progetto Actorbase.

			Ogni membro è tenuto a leggere il documento e a rispettare le norme al fine di dare maggiore uniformità
			allo svolgimento dei processi,
			migliorandone l'efficacia, riducendo il numero di errori e i tempi di sviluppo.

			Poiché il gruppo ha deciso di basarsi sulla struttura a processi \emph{ISO/IEC 12207}
			la struttura di questo documento ne rispecchia
			l'organizzazione. In particolare la suddivisione in processi primari, di supporto e organizzativi.
			%immagine
			\begin{figure}[H]
				\centering
				\includegraphics[scale=0.4]{Immagini/"processi 12207".png}
				\caption{Processi ISO/IEC 12207}
			\end{figure}
		\subsection{Scopo del prodotto}
			Lo scopo del progetto è la realizzazione di un DataBase NoSQL key-value basato sul modello ad
			Attori con l'obiettivo di fornire una tecnologia adatta allo sviluppo di moderne
			applicazioni che richiedono brevissimi tempi di risposta e che elaborano enormi quantità
			di dati. Lo sviluppo porterà al rilascio del software sotto licenza MIT.
		\subsection{Glossario}
			Al fine di evitare ambiguità di linguaggio e di massimizzare la comprensione dei documenti, il
	      	gruppo ha steso un documento interno che è il \emph{Glossario v2.0.0}. In esso saranno definiti, in modo
	      	chiaro e conciso i termini che possono causare ambiguità o incomprensione del testo. In ogni documento,
	      	ogni occorrenza di un termine presente nel glossario, verrà contrassegnata con una \emph{G} a pedice come la seguente:
	      	\newline
	      	\begin{center}
	      		\ped{\textit{G.}}
	      	\end{center}
	      	L'inserimento della \emph{G} a pedice è automatizzato grazie ad una funzionalità della piattaforma \emph{SPitFire}, insieme di tool al supporto della documentazione e del tracciamento sviluppato dal gruppo. Essa permette di effettuare l'upload di un file .tex a cui si vogliono aggiungere le \emph{G} e di ottenere un file con le aggiunte corrette in corrispondenza dei termini del glossario. \emph{SPitFire} permette di salvare la lista dei termini di glossario semplicemente effettuando l'upload del file .tex del Glossario, ricava in automatico la lista delle parole e le utilizza per aggiungere le \emph{G} correttamente.
		\subsection{Riferimenti}
			\subsubsection{Informativi}
				\begin{itemize}
					\item Specifiche UTF-8: \\ \url{http://unicode.org/faq/utf_bom.html}
					\item Licenza MIT: \\ \url{https://opensource.org/licenses/MIT}
					\item Scala Programming Language: \\ \url{http://www.scala-lang.org/}
					\item ISO/IEC 12207: \\ \url{http://www.iso.org/iso/catalogue_detail?csnumber=43447}
					\item ISO 8601:2004: \\ \url{http://www.iso.org/iso/home/standards/iso8601.htm}
					\item \LaTeX: \\ \url{https://www.latex-project.org}
					\item UML: \\ \url{http://www.uml.org}
					\item Astah Professional: \\ \url{http://astah.net/}
					\item Telegram: \\ \url{https://telegram.org}
					\item Google Drive: \\ \url{https://www.google.com/intl/it_it/drive/}
					\item Google Hangouts: \\ \url{https://hangouts.google.com}
					\item TeamWork: \\ \url{https://www.teamwork.com/}
					\item IntelliJ: \\ \url{https://www.jetbrains.com/idea/}
					\item ProjectLibre: \\ \url{http://www.projectlibre.org/}
					\item GitHub: \\ \url{https://github.com/}
					\item Texmaker: \\ \url{http://www.xm1math.net/texmaker/}
					\item altervista: \\ \url{http://it.altervista.org/}
					\item SPitFire: \\ \url{http://sweeneytreadaas.altervista.org/}
					\item SweeneyThreads: \\ \url{http://sweeneythreads.github.io/}
					\item Piano di progetto: \\ \emph{Piano di Progetto v2.0.0}
					\item Piano di qualifica: \\ \emph{Piano di Qualifica v2.0.0}
				\end{itemize}
			\subsubsection{Normativi}
				\begin{itemize}
					\item Capitolato d'appalto Actorbase (C1): \\
					\url{http://www.math.unipd.it/~tullio/IS-1/2015/Progetto/C1p.pdf}
				\end{itemize}

	\newpage
	\section{Processi primari}
		Sono state definite delle norme relative ai processi primari che maggiormente riguardano le attività
		svolte dal gruppo: fornitura e sviluppo.
		\subsection{Fornitura}
		\subsubsection{Studio di fattibilità}
		Lo \emph{Studio di fattibilità} del progetto deve essere steso dai membri che ricoprono il ruolo di
		\emph{Analisti} sulla base delle prime riunioni effettuate, decise dal \emph{Responsabile di progetto},
		e delle preferenze espresse da ogni singolo membro del gruppo.
		In seguito il documento verrà analizzato e valutato da altri membri del gruppo.
		Lo \emph{Studio di fattibilità} deve contenere:
		\begin{itemize}
			\item \textbf{Dominio:} conoscenza delle tecnologie richieste e del dominio applicativo;
			\item \textbf{Rapporto costo/benefici:} eventuali prodotti simili già presenti sul mercato,
			competitori, costo della
			realizzazione del prodotto e quantità di requisiti obbligatori;
			\item \textbf{Individuazione dei rischi:} evidenziare lacune tecniche e di conoscenza del dominio
			dei membri del gruppo, comprensione
			dei punti critici, difficoltà nel determinare i requisiti obbligatori e opzionali e nella loro
			classificazione.
		\end{itemize}
		Tale valutazione deve essere svolta per tutti i capitolati presenti, eventualmente con un maggior livello di dettaglio per quanto riguarda il \emph{capitolato C1}.

		\subsection{Sviluppo}
		
		\subsubsection{Casi d'uso}
			Come detto in precedenza alcuni requisiti possono essere ricavati dai Casi d'uso\ped{\emph{G}}, ad essi si
			può fare riferimento anche con la dicitura \emph{use case} o con l'acronimo \emph{UC} nel caso
			fosse  necessario utilizzarli in tabelle o diagrammi.

			 I Casi d'uso\ped{\emph{G}} vanno identificati dagli \emph{Analisti}, attraverso una procedura che va dal
			 generale al particolare.
			  \\ \\
			 Un \emph{caso d'uso} richiede la definizione dei seguenti campi:
			 \begin{itemize}
			 	\item Codice gerarchico
			 	\item Nome sintetico
			 	\item Attori
			 	\begin{itemize}
			 		\item Principali
			 		\item Secondari
			 	\end{itemize}
			 	\item Pre-condizione
			 	\item Post-condizione
			 	\item Flusso degli eventi relativi allo scenario principale
			 	\item Eventuali scenari alternativi
			 	\item Lista di requisiti dedotti
			 \end{itemize}
			 A tale insieme di informazioni va associato un diagramma di caso d'uso in UML. Per la
			 scrittura dei Casi d'uso\ped{\emph{G}} in UML il gruppo ha deciso di utilizzare il software
			  \emph{Astah Professional} alla versione 7.0. La scelta è  caduta su tale programma poiché
			  è disponibile per tutti i principali sistemi operativi desktop e permette di
			   ottenere una licenza gratuita per studenti.
			  \\ \\
			  È degli \emph{Analisti} il compito di inserire tutte le informazioni relative ai casi
			  d'uso e i loro diagrammi nel documento di \emph{Analisi dei requisiti}.
			  \\ \\
			 Il codice gerarchico del \emph{caso d'uso} ha la forma seguente:
			 \begin{center}
			 	UC[codice univoco del padre].[codice progressivo]
			 \end{center}
			 Dove il codice progressivo può definire diversi livelli di gerarchia separati da un punto.

			 I Casi d'uso\ped{\emph{G}} saranno espressi in forma tabellare nel documento di \emph{Analisi dei requisiti}
			 nel seguente modo, in più, ogni caso d'uso sarà accompagnato da un diagramma UML che ne
			 descrive il flusso di eventi.
			 \\ \\
			 Di seguito si riporta un esempio di caso d'uso comprensivo di diagramma UML e di descrizione tabellare.
			 \begin{figure}[H]
				\centering
				\includegraphics[scale=0.3]{UC/"UC 1 Connessione".png}
				\caption{Diagramma di UC1: Connessione al server}
			\end{figure}
	\textbf{Descrizione} 
	\\ \\
	L'utente ha appena avviato l'interfaccia CLI dell'applicativo e intende connettersi ad un server contenente i database utilizzando il comando di connessione (UC 1.1). L'utente deve inserire l'indirizzo del server (UC 1.2) e successivamente l'username e la password necessari per l'accesso (UC 1.3), nel caso in cui il tentativo di connessione dovesse fallire, l'utente riceve un messaggio di errore informativo.
	\begin{table}[H]
			\begin{tabularx}{\textwidth}{r X}
				\textbf{Codice gerarchico} & UC1 \\
				\noalign{\hrule height 0.5pt}
				\textbf{Nome sintetico} & Connessione al server \\
				\noalign{\hrule height 0.5pt}
				\textbf{Attore principale} & Utente non autenticato\\
				\noalign{\hrule height 0.5pt}
				\textbf{Attori secondari} & Nessuno \\
				\noalign{\hrule height 0.5pt}
				\textbf{Pre-condizione} & L'utente ha avviato l'interfaccia CLI, non è autenticato e intende connettersi ad un server\\
				\noalign{\hrule height 0.5pt}
				\textbf{Post-condizione} & L'utente risulta connesso al server \\
				\noalign{\hrule height 0.5pt}
				\textbf{Flusso eventi} & \begin{enumerate}
				\item L'utente inserisce il comando di connessione (UC 1.1)
				\item L'utente inserisce l'indirizzo del server (UC 1.2)
				\item L'utente inserisce username e password (UC 1.3) e preme invio
				\end{enumerate} \\
				\noalign{\hrule height 0.5pt}
				\textbf{Scenari alternativi} & Nessuno \\
				\noalign{\hrule height 0.5pt}
				\textbf{Lista requisiti\newline dedotti} & \dots \\ 
			\end{tabularx}
			\caption{Caso d'uso UC 1 - Connessione al server}
		 \end{table} 
		 
		 
		\subsubsection{Tecniche di analisi e classificazione requisiti}
			Sempre compito degli \emph{Analisti}, sarà quello di stilare l'\emph{Analisi dei requisiti}. Essi potranno ricavarli
			da eventuali Casi d'uso\ped{\emph{G}} emersi da Brainstorming o riunioni con il committente.
	
			I requisiti saranno elencati secondo un ordine. Ogni requisito seguirà la seguente codifica: \\
			\begin{center}
				R[Codice][Importanza][Tipo]
			\end{center}
			\textbf{Codice} \\ \\ Un codice univoco ed espresso in modo gerarchico;\\ \\
			\textbf{Importanza} \\ \\Può assumere i seguenti valori:
			\begin{itemize}
				\item \textbf{N:} Necessario;
				\item \textbf{D:} Desiderabile;
				\item \textbf{O:} Opzionale.
			\end{itemize}
			\textbf{Tipo} \\ \\Può assumere i seguenti valori:
			\begin{itemize}
				\item \textbf{F:} funzionale;
				\item \textbf{Q:} di qualità;
				\item \textbf{P:} prestazionale;
				\item \textbf{V:} vincolo.
			\end{itemize}
			I requisiti saranno inseriti in una tabella, che includerà anche un nome per ogni attributo, una fonte
			 e una breve descrizione. La forma tabellare di un requisito risulta quindi essere:
			\begin{table}[H]
				\begin{tabularx}{\textwidth}{X X X X}
					\noalign{\hrule height 1.5pt}
					\rowcolor{orange!85}Codice & Nome & Fonte & Descrizione \\
					\noalign{\hrule height 1.5pt}
					R[1.4.5][N][P] & Requisito di qualità & Capitolato & Garantire \dots \\
					\noalign{\hrule height 1.5pt}
				\end{tabularx}
			\end{table}
			

		\subsubsection{Gestione dei requisiti}
        Per automatizzare il più possibile la gestione dei requisiti ed il relativo tracciamento il gruppo ha sviluppato una piattaforma accessibile via web denominata \emph{SPitFire}.
        \begin{figure}[H]
				\centering
				\includegraphics[width=\textwidth]{spitfire.png}
				\caption{Pagina principale di SPitFire}
			\end{figure}
        \emph{SPitFire} si basa sul salvataggio dei dati in un database mySQL e permette di svolgere le seguenti operazioni sui requisiti:
        \begin{itemize}
        	\item Inserire un requisito tramite interfaccia web
        	\item Visualizzare i requisiti presenti, navigando dai requisiti di livello superiore a quelli di livello inferiore
        	\item Segnare il completamento di un requisito
        	\item Importare i requisiti da una tabella .tex
        	\item Esportare i requisiti in una tabella .tex
        \end{itemize}
        \begin{figure}[H]
				\centering
				\includegraphics[width=\textwidth]{spitfire-opzioni-requisiti.png}
				\caption{Opzioni di gestione dei requisiti offerte da \emph{SPitFire}}
			\end{figure}
			In particolare l'esportazione da database a file .tex e l'importazione da file .tex a database permettono di ridurre notevolmente il carico di lavoro degli analisti per quanto riguarda la semplice stesura della documentazione. Le tabelle dei requisiti presenti nel documento di \emph{Analisi dei requisiti} sono infatti generate in automatico.
        
        
        \subsubsection{Tracciamento dei requisiti}
            Il tracciamento dei requisiti è anch'esso automatizzato grazie alla piattaforma \emph{SPitFire}. 
            \\ \\
            Il tool individua automaticamente le dipendenze di tipo padre-figlio e genera in tal modo un albero dei requisiti navigabile semplicemente. Sono stati definiti dei trigger SQL per cui non risulta possibile segnare come completato un requisito che presenta requisiti figli incompleti. Inoltre il completamento di tutti i requisiti figli rende automatico il completamento del padre.
            \\ \\
            Il tracciamento tra fonti e requisiti è automatico: \emph{SPitFire} permette di generare le tabelle di tracciamento in formato .tex basandosi sui dati presenti nel database.
            \begin{figure}[H]
				\centering
				\includegraphics[width=\textwidth]{spitfire-tracciamento.png}
				\caption{Tabelle di tracciamento generate da  \emph{SPitFire}}
			\end{figure}
		\subsubsection{Gestione cambiamento requisiti}
		Per quanto riguarda il cambiamento dei requisiti, nel Database verrà tenuta una tabella di backup dei "vecchi" requisiti,
		nella loro forma di dichiarazione, con un puntatore al "nuovo" requisito, che invece avrà le specifiche aggiornate. È compito del
		\emph{Responsabile di progetto} mantenere aggiornata la tabella dei requisiti, copiando prima il requisito che necessita di cambiamento
		nella tabella dei vecchi requisiti, e poi aggiornando il requisito stesso.
		\\ \\
		Inoltre \emph{SPitFire } individua automaticamente eventuali incongruenze tra i file \LaTeX presenti e le informazioni contenute nel database, in tal modo è semplice tenere traccia dei requisiti non aggiornati.

		
		\subsubsection{Progettazione architetturale}
		La progettazione architetturale prevede la definizione delle componenti architetturali. La descrizione di ogni  componente package deve avvenire sia in forma testuale, che in forma di diagrammi UML. La descrizione di classi e interfacce può invece avvenire semplicemente a livello testuale.  \\
		Nello specifico i diagrammi UML che il gruppo dovrà utilizzare in fase di progettazione sono i seguenti:
		\begin{itemize}
			\item Diagrammi dei package
			\item Diagrammi delle classi
			\item Diagrammi di attività
			\item Diagrammi di sequenza
		\end{itemize}
		
		\subsubsection{Definizione testuale delle componenti architetturali}
		Per garantire uniformità nel documento di \emph{Specifica Tecnica}, i membri del gruppo dovranno rispettare una struttura definita per quanto riguarda la descrizione testuale di componenti architetturali, siano esse Package,  classi o interfacce. 
		\paragraph{Componente package} 
		La descrizione deve essere composta da al più i seguenti elementi:
		\begin{itemize}
			\item Descrizione
			\item Package figli
			\item Classi
			\item Interfacce
			\item Relazioni con altre componenti
		\end{itemize}
		
		\paragraph{Componente classe/interfaccia} 
		La descrizione deve essere composta da al più i seguenti elementi:
		\begin{itemize}
			\item Descrizione
			\item Utilizzo
			\item Classi estese
			\item Interfacce implementate
			\item Classi figlie
			\item Relazioni con altre classi
		\end{itemize}
		
		\subsubsection{Diagrammi delle componenti architetturali}
		In linea con quanto definito per la stesura dei diagrammi dei casi d'uso, i diagrammi delle componenti dovranno essere realizzati utilizzando il software \emph{Astah Professional} alla versione 7.0 .
		\paragraph{Diagrammi dei package} 
		La distinzione tra package si basa principalmente sul colore, per tale motivo ogni qual volta si utilizzi una combinazione di colori differente, essa deve essere accompagnata da una legenda  descrittiva.
		\begin{figure}[H]
				\centering
				\includegraphics[scale=0.5]{ST/Legenda.png}
				\caption{Esempio di legenda per un diagramma dei package}
			\end{figure}
			In particolare è consigliato utilizzare colori diversi per distinguere:
			\begin{itemize}
				\item Componenti di livello diverso
				\item Componenti interni da librerie esterne e framework
			\end{itemize}
			
		\paragraph{Diagrammi delle classi}
		Poiché i diagrammi delle classi raggiungono facilmente un elevato livello di complessità, i membri del gruppo devono assicurarsi di inserire nei documenti diagrammi che siano leggibili. Nel caso in cui un diagramma risultasse troppo dettagliato la sua immagine andrà inserita in repository ma non nel documento, in esso sarà necessario dividere il diagramma in più parti, ed inserire immagini distinte e leggibili per ognuna di esse.
		
		\paragraph{Diagrammi di attività e di sequenza}
		Valgono le stesse norme di leggibilità definite per i diagrammi delle classi, inoltre ogni diagramma di attività o sequenza dovrà essere accompagnato da una descrizione testuale.
		
		\subsubsection{Design Pattern}
		I membri del gruppo sono invitati ad utilizzare design pattern noti dove possibile in fase di progettazione. L'utilizzo di essi permetterà di avere un'architettura più comprensibile ed estendibile sia dal punto di vista del codice che dal punto di vista della documentazione che lo accompagna. \\ \\
		Ogni volta che una parte dell'architettura del sistema si basa su un design pattern sarà necessario fornire una descrizione dello stesso in un'apposita sezione del documento di \emph{Specifica Tecnica}.		
		
				
		
		\subsubsection{Codifica in \LaTeX}
		Regole riguardanti la stesura di codice \LaTeX :
		\begin{itemize}
			\item Ogni file deve iniziare con 3 righe di commento come quelle riportate in seguito:
			%immagine
			\begin{figure}[H]
				\centering
				\includegraphics[scale=1]{Immagini/"3commentsline".png}
				\caption{\LaTeX \space - Commenti ad inizio file}
			\end{figure}
			\item Ogni file deve contenere nella prima parte tutti gli \verb|\usepackage{}| necessari
			\item I commenti andranno inseriti in una riga vuota, eventualmente prima della riga di
			codice a cui fanno riferimento
			\item I commenti su più righe useranno il comando \\ \verb|\begin{comment} - \end{comment}|
			\item Tra ogni \verb|\begin{PART}| e \verb|\end{PART}| tutto il testo e il codice andrà indentato:
			%immagine
			\begin{figure}[H]
				\centering
				\includegraphics[scale=0.8]{Immagini/"indent".png}
				\caption{\LaTeX \space - Indentazione 1}
			\end{figure}
			\item Per quanto riguarda il comando personalizzato \verb|\newpage \section{}| o altre sezione \verb|\subsection{}|, \verb|\subsubsection{}|
			verranno comunque indentate le parti innestate al loro interno come segue:
			%immagine
			\begin{figure}[H]
				\centering
				\includegraphics[scale=0.8]{Immagini/"indent2".png}
				\caption{\LaTeX \space - Indentazione 2}
			\end{figure}
			\item Verrà utilizzato T1 come encoding del font: \\ \verb|\usepackage[T1]{fontenc}|
			\item Verrà utilizzato utf8 come encoding dell'input: \\ \verb|\usepackage[utf8]{inputenc}|
			\item Verrà utilizzato english, italian come parametro per babel: \\
			\verb|\usepackage[english, italian]{babel}| in modo da usare inglese e italiano nello
			stesso documento tenendo italiano come lingua principale
			\item Prima di ogni immagine, verrà inserito un commento su una riga, come definito
			sopra, per facilitarne l'individuazione:
			%immagine
			\begin{figure}[H]
				\centering
				\includegraphics[scale=0.8]{Immagini/"immagini".png}
				\caption{\LaTeX \space - Commento prima di ogni immagine}
			\end{figure}
			\item A fine documento, come commento su più righe, andrà inserita la documentazione e la descrizione (anche breve) del file
		\end{itemize}
		
		
		\subsubsection{Codifica in \emph{Scala}}
		Tutte le regole di indentazione, assegnazione dei nomi, scrittura delle parentesi, nomina file, e documentazione sono quelle
		definite dalla documentazione ufficiale di \emph{Scala}: \url{http://docs.scala-lang.org/style/}
		\begin{figure}[H]
				\centering
				\includegraphics[scale=0.5]{scala-style.png}
				\caption{Scala Style Guide}
			\end{figure}
			
		\subsubsection{Integrazione del software}
		Per l'integrazione continua delle componenti del sistema ed il relativo testing il gruppo ha deciso di utilizzare \emph{Travis CI}. Sarà compito dei \emph{progettisti} definire il sistema di build.
		\begin{figure}[H]
				\centering
				\includegraphics[scale=0.5]{travis.png}
				\caption{Logo di Travis CI}
			\end{figure}

	\newpage
	\section{Processi di supporto}
	
		\subsection{Documentazione}
		In questo capitolo si descrivono le convenzioni definite e adottate dal gruppo riguardanti le
		modalità di redazione, verifica e approvazione dei documenti.

		Tutti i documenti formali prodotti da SWEeneyThreads sono scritti utilizzando il linguaggio \LaTeX,
		compilati e forniti in formato PDF (per quanto riguarda le versioni digitali). Per la stesura dei
		documenti il gruppo utilizzerà il software \emph{Texmaker}.
		\subsubsection{Template}
		Al fine di rendere più rapida e meno incline a differenziazioni la stesura dei diversi documenti è stato prodotto un
		template \LaTeX, reperibile nel repository in \verb|Actorbase/LaTeX/Templates|.
		\subsubsection{Struttura documenti}
		La struttura dei documenti presenta una suddivisione in sezioni, sottosezioni e ulteriori sotto-sottosezioni.
		Tutti le sezioni, sottosezioni e sotto-sottosezioni sono state create usando i comandi standard \LaTeX \space
		\verb|\section{}|, \verb|\subsection{}| e \verb|\subsubsection{}|.

		La numerazione delle sezioni è utilizzata fino al terzo livello di profondità (x.y.z), dal quarto livello in poi le sottosezioni
		 non presentano numerazione. Tale scelta è stata presa al fine di rendere più leggibile l'indice.
		\\ \\
		Di seguito viene fornita una descrizione più dettagliata di alcuni elementi di un documento:
		\subsubsection{Prima pagina}
		La prima pagina di un documento presenta gli elementi seguenti:
		\begin{itemize}
			\item Nome del gruppo
			\item Nome del progetto
			\item Sottotitolo del progetto
			\item Titolo del documento
			\item Cognome e nome dei redattori del documento
			\item Cognome e nome di chi approva il progetto in qualità di responsabile
			\item Cognome e nome dei verificatori del documento
			\item Logo del gruppo
			\item Numero di versione del documento
			\item Data di rilascio del documento
		\end{itemize}
		La prima pagina è parte del template disponibile nel repository.
		
		I cognomi e i nomi dei redattori, del responsabile e dei verificatori, verranno cambiati di versione in versione. Questo è possibile
		grazie al versionamento dei documenti, infatti, i nomi di chi lavora al documento, fanno riferimento solo alla versione in cui il documento
		è in un determinato momento. Questo comporta un cambiamento dei nomi dei responsabili di un documento di versione in versione. Il numero di nomi
		è quindi variabile, in una fase ci possono essere 2 persone per la scrittura del documento, in quella successiva 3, o solamente una.
		
		\subsubsection{Indice}
			In ogni documento sono presenti in ordine
			\begin{itemize}
				\item Un indice delle sezioni;
				\item Un indice delle tabelle;
				\item Un indice delle figure.
			\end{itemize}
			Tali indici sono generati automaticamente tramite appositi comandi \LaTeX, l'assenza di figure
			e/o tabelle nel documento comporta l'omissione del corrispondente indice.

		Data la natura secondaria degli indici relativi alle tabelle e alle figure, si è deciso di posizionarli alla fine del documento.
		L'indice dei contenuti si trova invece subito dopo la pagina iniziale.
		\subsubsection{Diario delle modifiche}
			Ogni documento deve contenere una sezione denominata "Diario delle modifiche" in cui annotare tutte le
			attività svolte sul documento.
			\\ \\
			Lo schema della tabella è il seguente:
			\begin{itemize}
				\item \textbf{Versione:} numero di versione del documento dopo le modifiche;
				\item \textbf{Data:} data in cui sono state apportate le modifiche;
				\item \textbf{Autore:} ruolo, cognome e nome dell'autore che apportato le modifiche (gli autori possono essere
				più di uno);
				\item \textbf{Descrizione:} descrizione delle modifiche apportate al documento.
			\end{itemize}
			La compilazione della tabella è un attività obbligatoria nel caso di modifiche rilevanti al documento. Un documento
			non può cambiare di versione senza che tale cambiamento venga annotato nella tabella, nel caso di modifiche
			minori che non cambiano di molto il contenuto (es. correggere un accento) la tabella può rimanere invariata.
			\begin{table}[H]
				\begin{tabularx}{\textwidth}{s f m X}
					\noalign{\hrule height 1.5pt}
					\rowcolor{orange!85} Versione & Data & Autore & Descrizione \\
					\noalign{\hrule height 1.5pt}
				\end{tabularx}
				\caption{Schema del diario delle modifiche \label{tab:table_label}}
			\end{table}
			Il Diario delle modifiche non è incluso nella numerazione delle sezioni, si trova dopo l'indice e prima di qualsiasi
			 capitolo numerato
		\subsubsection{Formattazione generale delle pagine}
			La formattazione generale di una pagina prevede la diminuzione dei margini destri e sinistri, ma non prevede
			altre modifiche importanti e si basa, per tutte le altre regole, sulla formattazione standard di \LaTeX \space
			usata per i documenti di classe "Report".\\
			Per effettuare la modifica viene usato i seguenti comandi: \\ \\
			\verb|\usepackage{geometry}| \\ \verb|\geometry{margin=1in}|. \\
		\subsubsection{Norme tipografiche}
			Questa sezione contiene norme tipografiche e ortografiche adottate dal gruppo al fine di garantire uno stile
			uniforme e una semantica coerente per tutti i documenti.
		\subsubsection{Stile del testo}
			Il font utilizzato in tutti i documenti formali scritti dal gruppo sarà il \emph{Computer Modern}, ovvero
			quello standard utilizzato da \LaTeX.
		\begin{itemize}
			\item \textbf{Corsivo:} il corsivo va utilizzato nei casi seguenti:
			\begin{itemize}
				\item Citazioni;
				\item Nomi particolari;
				\item Documenti;
				\item Riferimenti;
			\end{itemize}
			A seconda della semantica del testo si utilizzano i comandi \LaTeX \space \verb|\emph{}| e \verb|\textit{}|.
			\item \textbf{Grassetto:} il grassetto va utilizzato nei casi seguenti:
			\begin{itemize}
				\item Elenchi puntati: evidenzia il concetto sviluppato nella continuazione del punto.
			\end{itemize}
			\item \textbf{Maiuscolo:} una parola completamente in maiuscolo deve indicare un acronimo o una sigla.
			\item \textbf{\LaTeX:} ogni riferimento al linguaggio \LaTeX \space va scritto utilizzando il comando
			\verb|\LaTeX|.
		\end{itemize}
		\subsubsection{Formati}
		\begin{itemize}
			\item \textbf{Percorsi:}
			\begin{itemize}
				\item Indirizzi email : comando \LaTeX \space \verb|\href{mailto:nome@dominio}{nome@dominio}|;
				\item Indirizzi web completi: comando \LaTeX \space \verb|\url|;
				\item Indirizzi relativi: comando \LaTeX  \space \verb|\verb|.
			\end{itemize}
			\item \textbf{Date:} le date presenti nei documenti seguono lo standard ISO 8601:2004:
			\begin{center}
				AAAA - MM - GG
			\end{center}
			Dove:
			\begin{itemize}
				\item AAAA rappresenta l'anno;
				\item MM rappresenta il mese;
				\item GG rappresenta il giorno.
			\end{itemize}
			\item \textbf{Ruoli di progetto:} quando si fa riferimento ad un ruolo di progetto questo va scritto in corsivo
			e con la prima lettera maiuscola (es. \textit{Responsabile}).
			\item \textbf{Documenti:} i riferimenti vanno scritti in corsivo (es. \textit{Analisi dei requisiti}).
			\item \textbf{Nomi dei file:} i nomi dei file vanno scritti utilizzando il comando \LaTeX \space
			\verb|\verb| (es. \verb|immagine.png|).
			\item \textbf{Nomi propri:} I nomi propri seguono la forma "Cognome Nome".
			\item \textbf{Nome del gruppo:} il nome del gruppo è SWEeneyThreads, la distinzione tra lettere maiuscole e
			minuscole va rispettata ogni volta che vi si fa riferimento.
		\end{itemize}
		\subsubsection{Sigle}
		L'utilizzo di sigle e abbreviazioni per riferirsi a documenti va limitato il più possibile, tuttavia nel caso il loro uso
		fosse funzionale alla lettura (come nel caso di tabelle o diagrammi) il loro uso è consentito:
		\begin{itemize}
			\item \textbf{SdF:} Studio di Fattibilità;
			\item \textbf{AdR:} Analisi dei Requisiti;
			\item \textbf{GL:} Glossario;
			\item \textbf{NdP:} Norme di Progetto;
			\item \textbf{PdQ:} Piano di Qualifica;
			\item \textbf{PdP:} Piano di Progetto;
			\item \textbf{ST:} Specifica Tecnica;
			\item \textbf{RR:} Revisione dei Requisiti;
			\item \textbf{RP:} Revisione di Progettazione;
			\item \textbf{RQ:} Revisione di Qualifica;
			\item \textbf{RA:} Revisione di Accettazione.
		\end{itemize}
		\subsubsection{Componenti grafiche}
		Le componenti grafiche previste all'interno dei documenti sono immagini e tabelle. Ogni occorrenza di un
		elemento grafico è accompagnata da una didascalia indicizzata, in modo da poterla associare alla sezione
		relativa del documento.
		\textbf{Tabelle}
		Le tabelle sono definite utilizzando un template in \LaTeX \space realizzato dal gruppo e disponibile nel
		repository all'indirizzo \verb|Actorbase/LaTeX/Templates|
		\textbf{Immagini}
		Il formato scelto per le immagini è Portable Network Graphics (PNG). \\
		Le immagini vanno sempre inserite utilizzando la seguente sequenza di comandi \LaTeX:
		\begin{verbatim}
			\begin{figure}[H]
				\centering
				\includegraphics[scale=0-1]{Immagini/nome.png}
				\caption{Titolo - didascalia}
			\end{figure}
		\end{verbatim}
		\subsubsection{Classificazione documenti}
		I documenti prodotti dal gruppo si dividono in formali e informali.
		\textbf{Documenti formali}
		Quando un documento riceve l'approvazione del \emph{Responsabile} viene definito formale e risulta idoneo
		al rilascio all'esterno del gruppo. \\
		Per risultare approvato un documento deve aver completato con successo il percorso di verifica e validazione
		descritto nel \emph{Piano di Qualifica}.
		\textbf{Documenti informali}
		Un documento rimane informale finché non viene approvato dal \emph{Responsabile}, durante tale fase
		il suo uso è da considerarsi esclusivamente interno al gruppo. \\
		Alcuni documenti prodotti dal gruppo possono rimanere informali per l'intera durata del loro ciclo di vita.
		\subsubsection{Versionamento documenti}
		I documenti prodotti dal gruppo devono essere sempre identificati da un numero di versione del tipo:
		\begin{center}
			X.Y.Z
		\end{center}
		Dove:
		\begin{itemize}
			\item X: è il numero principale di versione, viene incrementato ad ogni uscita formale del documento;
			\item Y: viene incrementato quando il documento entra in una fase successiva del suo ciclo di vita;
			\item Z: viene incrementato quando si apportano modifiche minori al documento.
		\end{itemize}
		All'interno di un documento quando si intende fare riferimento ad una specifica versione di un altro documento la
		notazione da utilizzare è:
		\begin{center}
			\emph{Nome Documento vX.Y.Z}.
		\end{center}
		Mentre per fare riferimento ad un file vero e proprio:
		\begin{center}
			\verb|NomeDocumento_vX.Y.Z.estensione|
		\end{center}
		\subsubsection{Ciclo di vita dei documenti}
		Ogni documento prodotto dal gruppo rispetta il seguente ciclo di vita:
		\begin{itemize}
			\item \textbf{Lavorazione/Modifica:} il documento entra in questa fase al momento della sua creazione e vi
			rimane per tutto il tempo in cui il suo contenuto viene modificato. Prima di terminare la sua fase di modifica ed essere
			messo a disposizione dei verificatori, su ogni documento deve essere effettuato il controllo ortografico messo a disposizione
			dal software \emph{TexMaKer}:
			%Image
			\begin{figure}[H]
				\centering
				\includegraphics[scale=0.8]{controlloOrtografico}
				\caption{Controllo ortografico - strumento di TexMaKer}
				\label{controlloOrtografico}
			\end{figure}
			\item \textbf{Verifica:} quando termina la fase di modifica, il documento passa nelle mani dei
			\emph{Verificatori}	che lo analizzano al fine di individuare eventuali errori o incongruenze sintattiche e semantiche;
			\item \textbf{Approvazione:} dopo essere stato verificato il documento deve essere approvato dal
			\emph{Responsabile}. Se il documento ottiene l'approvazione diventa ufficiale e raggiunge lo stato finale del
			suo ciclo di vita per quanto riguarda la corrente versione.
		\end{itemize}
		Ogni documento prodotto può attraversare più volte ogni fase del suo ciclo di vita, allo stesso modo può non
		attraversarle tutte. Quando si inizia una revisione formale su un documento già approvato questo ricomincia il
		ciclo da capo con un numero di versione incrementato.
		
				\subsection{Accertamento della qualità}
			
		\subsubsection{Analisi dei processi}
			I processi compiuti durante una fase verranno analizzati secondo il seguente protocollo:
			\begin{itemize}
				\item \textbf{Controllo delle metriche:}
					Alla conclusione di ogni fase del progetto si calcolano gli indici definiti nella sezione 2.2.1 confrontando 
						le macro-attività preventivate nel \emph{Piano di progetto} con i dati effettivi riscontrati dal sistema di ticketing.
				\item \textbf{Analisi PDCA:}
					Secondo il ciclo PDCA una fase di progetto ha inizio con la pianificazione di come possono essere migliorati i processi. 
					Durante la fase vengono attuati i cambiamenti prefissati e alla fine viene effettuato un controllo. Se un processo 
					risulta essere migliore viene adottato in modo definitivo, se invece risulta inalterato o addirittura peggiore i 
						cambiamenti vengono scartati.
			\end{itemize}
		\subsubsection{Analisi dei documenti}
			Ogni documento redatto è verificato mediante il seguente protocollo:
			\begin{itemize}
				\item \textbf{Controllo sintattico:} Il testo deve venire sottoposto a controllo dell'ortografia con il tool
				fornito dall'ambiente di sviluppo \LaTeX utilizzato. Alcuni errori non possono essere comunque rilevati da
				meccanismi automatici, quindi al fine di ottenere correttezza sintattica e semantica i Verificatori
				effettueranno un walkthrough al fine di ricercare errori sfuggiti al correttore ortografico.
				\item \textbf{Controllo semantico:}
		I Verificatori sono tenuti a leggere il documento, controllando che le frasi lette abbiano
		senso compiuto e che siano pertinenti all'argomento trattato nella sezione. Errori di questo
		tipo possono essere dovuti a sviste o errori di copia nella stesura.
				\item \textbf{Rispetto delle Norme di progetto:} I Verificatori sono tenuti a verificare che il documento
				rispetti tutte le norme tipografiche e di struttura del documento riportate nelle \emph{Norme di progetto}. Porzioni
				di questa verifica sono automatizzabili, i Verificatori dovranno quindi usare tool ove possibile. 
				\item \textbf{Inspection secondo checklist:} I Verificatori dovranno scorrere la lista di controllo e
				verificare che non sia presenti gli errori comuni lì riportati.
				\item \textbf{Verifica Glossario:} I Verificatori dovranno controllare che tutti i termini contenuti nel 
				Glossario siano indicati all'interno dei documenti con la G a pedice. 
				\item \textbf{Calcolo delle metriche:} I Verificatori dovranno calcolare le seguenti metriche per ogni documento:
				\begin{itemize}
					\item \textbf{Gulpease:}I verificatori calcoleranno l'indice Gulpease del documento tramite strumenti automatici forniti su \url{http://sweeneytreadaas.altervista.org/menuPrincipale/documentation_tools/}, in caso i risultati non siano soddisfacenti presenterà al redattore le sezioni peggiori;
					\item \textbf{Numero figure e tabelle su numero pagine:}Tramite gli elenchi delle figure e delle tabelle sarà calcolato il rapporto tra la somma delle stesse e il numero di pagine, e se risulterà scadente il verificatore lo segnalerà al redattore del documento.
					\item \textbf{Media parole per section:}La metrica sarà calcolata tramite strumenti automatici e in caso di valori non accettabili le sezioni peggiori saranno comunicate al redattore.
				\end{itemize}
				\item \textbf{Miglioramento:} Se nello svolgimento di uno qualunque dei punti precedenti un Verificatore
					notasse nuove possibilità di automatizzazione dovrà segnalarle e si dovrà cercare o costruire uno strumento per
					rendere effettiva tale automatizzazione. Inoltre durante le esecuzioni dei walkthrough, i verificatori sono
					tenuti ad annotare gli errori più frequenti o potenzialmente dannosi rilevati. Tali errori andranno poi ad
					aggiornare la lista di controllo che verrà utilizzata per le successive inspection.
			\end{itemize}
		
		\subsection{Gestione delle versioni}
			Il processo di verifica inizia fin dal primo rilascio ufficiale nella repository di un documento vX.0.0, il risultato 
			del processo di verifica cambia la versione del documento, incrementandone la versione al secondo livello di profondità, 
			il documento quindi passa a vX.1.0. Ogni modifica apportata al documento ne aumenta la versione all'ultimo livello di 
			profondità. Una volta completate le modifiche e le verifiche continue sui documenti, esso passa nella fase di validazione,
			a carico del responsabile, che ne fa cambiare la versione al secondo livello di priorità. 
			
			Grazie al diario delle modifiche è più facile individuare dove concentrare l'attenzione nelle verifiche successive 
			alla prima, in quanto sono segnate le sezioni che sono state aggiunte o che hanno subito dei cambiamenti dall'ultima
			 verifica, evitando così di controllare sempre tutto il documento ad ogni rilascio di versione, per cui ottimizzando 
			 il tempo necessario al controllo.

			Per ognuna delle 5 fasi del progetto descritte nel \emph{Piano di Progetto v2.0.0}, sono necessarie diverse attività 
			di verifica, descritte nella sezione Analisi del \emph{Piano di Qualifica v3.0.0}, a causa dei diversi output ottenuti:

			\begin{itemize}
				\item \textbf{Scelta ed approccio al capitolato:} Si devono eseguire le attività di verifica sui processi e sui documenti.
				
				\item \textbf{Analisi di dettaglio:} Si devono eseguire le attività di verifica sui processi e sui documenti.
				
				\item \textbf{Progettazione e sviluppo:} Si devono eseguire le attività di verifica sui processi, sui documenti e sul codice prodotto.
				
				\item \textbf{Sviluppo ulteriore ed incremento:} Si devono eseguire le attività di verifica sui processi, sui documenti e sul codice prodotto.
				
				\item \textbf{Progettazione e sviluppo:} Si devono eseguire le attività di verifica sui processi, sui manuali e sul prodotto finito.
			\end{itemize}
			
			
			Come detto in precedenza conclusa la verifica, ha inizio il processo di approvazione. Durante questo processo è compito 
			del Responsabile di Progetto accertarsi che i prodotti ottenuti siano conformi a quanto pianificato e progettato.
			
				\subsection{Qualifica}
		\subsubsection{Procedure di controllo qualità per i processi}
			La qualità del processo viene garantita da:
			\begin{itemize}
				\item \textbf{Pianificazione:} i processi devono essere pianificati nel dettaglio, in maniera da determinare 
				i punti e le tempistiche in cui effettuare controlli;
				\item \textbf{Controllo:} i controlli pianificati devono essere eseguiti in maniera oggettiva e neutrale, 
				quindi con strumenti automatici ovunque possibile;
				\item \textbf{Miglioramento continuo:} l'adozione del principio di \emph{PDCA} aiuterà a migliorare i processi durante l'intera durata del progetto..
			\end{itemize}
		\subsubsection{Procedure di controllo qualità per il prodotto}
			La qualità del prodotto viene garantita da:
			\begin{itemize}
				\item \textbf{Comprensione ed analisi del dominio};
				\item \textbf{Verifica:} determina che l'output di una fase sia consistente, completo e corretto. 
				Deve essere eseguita costantemente per tutta la durata del progetto, ma cercando di essere minimamente invasiva;
				\item \textbf{Validazione:} conferma oggettivamente che il prodotto sia conforme alle aspettative;
				\item \textbf{Quality Assurance:} garantisce il raggiungimento degli obiettivi di qualità, in maniera preventiva. 
				In questo modo si riduce drasticamente il ricorso a tecniche retrospettive, e con esse si riducono le iterazioni.
			\end{itemize}
				
						
		
		%DA INSERIRE APPENA AVVIATO IL PDQ!
		% IMPORTANTE RICORDARSI
		%\subsection{Accertamento qualità}
		%\subsection{Qualifica}
		%\subsection{Risoluzione di problemi}
		%Per quanto riguarda l'individuazione, il tracciamento e la risoluzione di bug e problemi il gruppo ha deciso di
		%affidarsi ad il servizio di issue tracking \emph{YouTrack} offerto da \emph{JetBrains}. \\
		%Per i progetti fino a dieci utenti, il servizio offre 10GB di spazio sul cloud dell'azienda in maniera gratuita.
		
				\subsection{Strumenti e metodi per l'applicazione delle metriche}
			Le risorse software che si utilizzeranno durante il processo di verifica sono:
			\begin{itemize}
				\item \textbf{\emph{TexMaker}:} un ambiente grafico Open-Source per \LaTeX cross-platform, permette 
				la compilazione rapida e la visualizzazione del PDF generato, sarà anche usato per il controllo ortografico;
				\item \textbf{\emph{Scalastyle}:} analizzatore statico che rileva potenziali problemi nel codice, da utilizzare sia durante la produzione del codice, sia durante la sua verifica
				 (\url{https://github.com/scalastyle/scalastyle});
				\item \textbf{\emph{Scapegoat}:} un altro analizzatore statico che si concentra maggiormente sugli 
				standard di stile e di coding (\url{https://github.com/sksamuel/scapegoat});
				\item \textbf{\emph{CLOC (Count Lines Of Code)}:} misura alcune metriche riguardanti il codice 
				sorgente in vari linguaggi, tra cui \emph{Scala}, servirà a calcolare le linee di codice per linee di commento (\url{cloc.sourceforge.net});
				\item \textbf{\emph{ScalaTest}:} framework per i test su Scala (\url{http://www.scalatest.org/}).
				
				\item \textbf{Repo's Outpost:} tool su piattaforma web sviluppato dai componenti del gruppo,
				disponibile (previo login) all'indirizzo: \\
				\url{http://sweeneytreadaas.altervista.org/menuPrincipale/documentation_tools/documenti_latex.php} \\
				Fornisce una \emph{dashboard} aggiornata in tempo reale che riporta le principali
				metriche dei documenti presenti nel repository.
				I valori sono evidenziati con diversi colori che ne caratterizzano ottimalità,
				accettabilità o non accettabilità.
				%immagine
				\begin{figure}[H]
					\centering
					\includegraphics[scale=0.35]{immagini/Pdq/repoOutpost.png}
					\caption{Screenshot del tool Repo's Outpost}
				\end{figure}
				
				\item \textbf{Camel Calculator:} tool su piattaforma web sviluppato dai componenti del gruppo,
				disponibile (previo login) all'indirizzo: \\
				\url{http://sweeneytreadaas.altervista.org/menuPrincipale/documentation_tools/metric_calc.php} \\
				Permette di calcolare nel dettaglio le metriche di uno specifico documento. È possibile caricare
				il file mediante diverse modalità, compresi dei link diretti per i documenti nel repository del gruppo.
				
				\item \textbf{Gloss Buddy:} tool su piattaforma web sviluppato dai componenti del gruppo,
				disponibile (previo login) all'indirizzo: \\
				\url{http://sweeneytreadaas.altervista.org/menuPrincipale/documentation_tools/glossario.php} \\
				Marca i termini del glossario con la simbologia corretta.
				
			\end{itemize}
			
			\subsection{Comunicazione e risoluzione di anomalie}
			Una anomalia è una violazione da parte di un documento, o unità di codice, di una o più delle seguenti condizioni:
			\begin{itemize}
				\item Conformità alla norme tipografiche o di codifica;
				\item Appartenenza al range di accettabilità per tutte le metriche descritte nella sezione Visione generale della strategia di verifica del\emph{Piano di Qualifica v3.0.0};
				\item Congruenza del prodotto con funzionalità indicate nell'\emph{analisi dei requisiti};
				\item Congruenza del codice con il design del prodotto.
			\end{itemize}
			Se un \emph{Verificatore} dovesse trovare una anomalia egli è tenuto ad aprire un sotto-ticket all'interno 
			della task-list a lui assegnata. Nel caso la risoluzione del ticket avesse la necessità di essere strutturato 
			in sotto-attività sarà compito del \emph{Responsabile} aprire una nuova task-list ed assegnarla alle figure coinvolte.
			

	\newpage
	\section{Processi organizzativi}
	\subsection{Processi di gestione dell'infrastruttura}
	\subsubsection{Documentazione di pianificazione}
		Per quanto riguarda la documentazione della pianificazione si è scelto di adoperare
		\emph{ProjectLibre}, un software Open-source per il project management, che permette
		di automatizzare molte mansioni che altrimenti il \emph{Progettista} dovrebbe svolgere
		a mano. \\ \\
		\emph{ProjectLibre} è stato scelto per le sue ottime caratteristiche:
		\begin{itemize}
			\item Portabilità, in quanto basato su \emph{Java};
			\item Open-source;
			\item Genera automaticamente diagrammi di Gantt, WBS e PERT;
			\item Calcola automaticamente i costi, sia totali che per singola attività/risorsa,
			aiutando a tenere sotto controllo il budget.
		\end{itemize}
	\subsubsection{Ticketing}
	\textbf{Scelta della piattaforma di ticketing}
	Per quanto riguarda l'emissione e la gestione dei ticket si è scelto di affidarsi alla piattaforma \emph{Teamwork}
	in quanto:
	\begin{itemize}
		\item Ha ottenuto buoni punteggi da reviews di utenti e di critica;
		\item Fornisce 100Mb di storage e la possibilità di avere due progetti attivi, contemporaneamente;
		\item Fornisce un analizzatore di rischi e benefici;
		\item Genera automaticamente diagrammi di Gantt interattivi;
		\item Include un ottimo Task management (priorità, task history, possibilità di aggiungere in automatico task ricorrenti);
		\item Notifiche sms e \emph{Notification group}.
	\end{itemize}


	La principale alternativa presa in considerazione è stata \emph{Zoho}, ma non è stata ritenuta all'altezza in quanto offre
	meno features. Segue una breve lista per mettere a confronto le principali funzionalità messe a disposizione dalle due piattaforme: \\

	\begin{table}[H]
		\begin{tabularx}{\textwidth}{*2{>{\centering\arraybackslash}X}}
			\noalign{\hrule height 1.5pt}
			\rowcolor{orange!85} ZOHO & TEAMWORK \\
			\noalign{\hrule height 0.5pt}
			Calendar & Calendar \\
			Gant & Gantt interattivi \\
			Task management & To-do list\\
			Time tracking & Track Project Hours\\
			Bug tracking & Analizzatore rischi/benefici  \\
			Document management & Template di progetto \\
			& Priorities \\
			& Track Burn Rate \\
			& Track Staff Hours \\
			& SMS di notifica \\
			\noalign{\hrule height 1.5pt}
		\end{tabularx}
		\caption{Zoho / Teamwork - Lista features \label{tab:table_label}}
	\end{table}

	%immagine
	\begin{figure}[H]
		\centering
		\includegraphics[scale=0.4]{Immagini/"zohovstw".png}
		\caption{Zoho / Teamwork - Rating delle features a confronto}
	\end{figure}

	Secondo \emph{SoftwareInsider} (\url{softwareinsider.com}) sono molto simili nelle funzionalità principali;
	ma \emph{Teamwork} offre alcuni strumenti in più per la gestione di processi software tradizionali.

	I principali:\\
	\begin{table}[H]
		\begin{tabularx}{\textwidth}{*3{>{\centering\arraybackslash}X}}
			\noalign{\hrule height 1.5pt}
			\rowcolor{orange!85} & ZOHO & TEAMWORK \\
			\noalign{\hrule height 0.5pt}
			Calendar & \ding{51} & \ding{51} \\
			Gantt interattivi & \ding{51} & \ding{51} \\
			Template di progetto & \ding{51} & \ding{51} \\
			Risk/benefits analyzer & \ding{53} & \ding{51} \\
			Scheduling & \ding{53} & \ding{51} \\
			\noalign{\hrule height 1.5pt}
		\end{tabularx}
		\caption{Zoho / Teamwork - Differenza strumenti \label{tab:table_label}}
	\end{table}

	Come task management \emph{Zoho} offre solamente delle To-do List, mentre \emph{Teamwork} ha anche le seguenti
	feature:
	\begin{itemize}
		\item Add Recurring Tasks;
		\item Group Tasks by Projects;
		\item Set Priorities;
		\item Task History.
	\end{itemize}
	\emph{Zoho} offre alcune funzionalità in più in quanto a comunicazione real-time tra membri del gruppo,
	ma questo risulta irrilevante per il nostro gruppo, in quanto per la comunicazione real-time viene
	adottato un sistema diverso.
	\textbf{Politiche di ticketing}
		I Ticket devono essere assegnati dal \emph{Responsabile}. Poiché un ticket assegna una o più attività ad una o più
		persone, chi riceve un ticket può generare altri ticket relativi a sotto attività da lui individuate per tenere traccia
		dello sviluppo in maniera più chiara. In ogni caso a tutti gli altri membri del gruppo non è assolutamente permessa la
		 generazione di ticket esterni ad attività già assegnate.
	\subsubsection{Versioning}
	Per gestire il versionamento il gruppo utilizza \emph{GitHub}. Tale scelta è dovuta sia ad un apprezzamento
	comune da parte dei membri del gruppo per la piattaforma, che ad una richiesta esplicita di pubblicazione del
	progetto sulla stessa da parte del committente. \\
	\'E stato creato un account ufficiale del gruppo, raggiungibile all'indirizzo
	\url{https://github.com/SweeneyThreads}
	%immagine
	\begin{figure}[H]
		\centering
		\includegraphics[scale=0.25]{Immagini/"sweeneygithub".png}
		\caption{Account GitHub SWEeneyThreads}
	\end{figure}
		\subsubsection{Repository}
	Sono state previste diverse \emph{Repository} necessarie allo sviluppo del progetto: Actorbase, ActorbaseDoc, RR, RP, RQ, RA e ConsegnaActorbase. 	
    Actorbase conterrà tutti i file del prodotto da sviluppare, mentre le \emph{Repository} RR, RP, RQ e RA si riferiscono alle 4 consegne del 
    progetto previste: revisione dei requisiti, revisione di progettazione, revisione di qualifica, revisione di accettazione.
    ActorbaseDoc conterrà tutta la documentazione necessaria per il progetto.  
    Effettuata la consegna del materiale, la \emph{Repository} \verb|Actorbase/| verrà copiata in quella corrispondente, che servirà quindi come 
    backup della \emph{Baseline} a cui fa riferimento.
    La \emph{Repository} ConsegnaActorbase servirà per effettuare le consegne, il link alla repository verrà consegnato al committente nel giorno delle 
    consegne.
	%immagine
  	\begin{figure}[H]
		\centering
		\includegraphics[scale=0.5]{"repositories".png}
		\caption{Struttura delle \emph{Repository GitHub}}
	\end{figure}
	In \verb|ActorbaseDoc/| saranno presenti le seguenti sottocartelle:
	\begin{itemize}
		\item Documenti;
		\item LaTeX;
	\end{itemize}
	\textbf{Documenti} \\ \\
	Nella cartella \verb|ActorbaseDoc/Documenti/| verranno inseriti tutti i PDF generati dal comando \\ \verb|pdflatex nome-documento.tex|. Non
	saranno presenti altri file in questa cartella. I documenti saranno divisi in \emph{Interni} ed \emph{Esterni} per questo saranno create 
    delle sottocartelle: \verb|ActorbaseDoc/Documenti/Interni| e \verb|ActorbaseDoc/Documenti/Esterni|.
	\\ \\
    \textbf{LaTeX} \\ \\
	Nella cartella \verb|ActorbaseDoc/LaTeX/| saranno presenti tutti i file \verb|*.tex| pronti per la compilazione. In questa cartella verrà inserita
	anche una cartella\\ \verb|ActorbaseDoc/LaTeX/Immagini/| contenente tutte le immagini necessarie alla compilazione dei file.
	Inoltre verrà aggiunta una cartella \verb|Actorbase/LaTeX/Templates/| contenente i template per la stesura di documenti e per il
	disegno appropriato di tabelle. 
    Per le tabelle su più pagine, si terranno dei file separati dal file principale in cui sono incluse. Per questo è stata creata la tabella 
    \verb|ActorbaseDoc/LaTeX/Tabelle|.
    \\ \\
	\textbf{Progetto} \\ \\
    Per il progetto in sé, è stata creata una repository a parte \verb|Actorbase| in modo da semplificare l'integrazione continua con \emph{Travis}.
	\\ \\
    \textbf{Normative per i commit} \\ \\
		Si è deciso di dare a tutti i membri del gruppo la possibilità di effettuare commit sul master-branch del repository
		senza dover attendere l'approvazione di un account centrale. Tale scelta impone però la definizione di alcune norme
		di commit da rispettare:
		\begin{itemize}
			\item I commit devono seguire l'emissione di un ticket. Tale norma serve ad evitare un eccessivo numero di
			 commit sul repository contenenti poche modifiche;
			\item Nel caso il commit interessasse un documento o un file, il numero di versione dello stesso deve risultare
			 aggiornato;
			\item Nella descrizione del commit è obbligatorio inserire una descrizione delle modifiche effettuate. \'E preferito
			l'uso di elenchi puntati, la forma discorsiva va utilizzata solo se non è possibile esprimere il contenuto come
			elenco.
		\end{itemize}
	\subsection{Processi di management}
	\subsubsection{Ruoli}
		Per quanto riguarda i ruoli, il gruppo utilizzerà quelli definiti nelle slide 7-11 disponibili all'indirizzo:
		\url{http://www.math.unipd.it/~tullio/IS-1/2015/Dispense/L04.pdf}. È stato deciso, unanimemente, che le rotazioni dei ruoli principali
		come \emph{Amministratore} e \emph{Responsabile di progetto} avverranno ogni 2 settimane. Come stabilito, una persona può ricoprire
		contemporaneamente più ruoli, la rotazione di altri ruoli come \emph{Analista}, \emph{Progettista} e \emph{Programmatore} potrà avvenire
		meno frequentemente, in quanto potrebbe risultare dannoso dover abbandonare un'attività di analisi o di programmazione prima della sua
		conclusione.
\\ \\
		Il gruppo stabilisce la rotazione dei ruoli in base alle attività, ai task, e alla disponibilità fornita da ogni membro.
\\ \\
		È possibile ottenere in qualsiasi momento una panoramica dei ruoli assegnati tramite l'apposita sezione del
		progetto creato su \emph{Teamwork}, all'indirizzo\\ \url{https://actorbase.teamwork.com/projects/188894/projectroles}. 
		%immagine
	\begin{figure}[H]
		\centering
		\includegraphics[scale=0.5]{ruoli.png}
		\caption{Accesso alla sezione ruoli da Teamwork}
	\end{figure}
		Grazie a \emph{SPitFire} è inoltre possibile importare i ruoli definiti in \emph{Teamwork} in un progetto \emph{ProjectLibre} grazie ad un parser che automatizza tale operazione.
	\subsubsection{Comunicazioni}
	Le comunicazioni possono avvenire tra membri del gruppo (interne), o tra il gruppo e terzi (esterne). Gli
	strumenti utilizzati differiscono a seconda della tipologia della comunicazione.
	\\ \\
	\textbf{Interne}
	\begin{itemize}
		\item \textbf{Chat:} Per le comunicazioni interne il gruppo ha deciso di adottare una chat di messaggistica
		istantanea: \emph{Telegram}. All'interno di questo mezzo di comunicazione verranno concordate date e orari
		delle riunioni; comunicati eventuali ritardi ai meeting; proposte idee informali, che verranno poi riproposte
		in modo ufficiale alle riunioni (questo per evitare di dimenticarsene o per lasciare tempo agli altri membri
		del gruppo di ragionare più tempo su una proposta); inoltre \emph{Telegram} verrà utilizzato per l'invio di files
		temporanei, di documentazione o informativi. Sarà compito del \emph{Responsabile di progetto} prelevare file di documentazione
		e riportarli nella repository adatta, e nel Drive del gruppo.

		La scelta di \emph{Telegram} è dovuta alla possibilità di utilizzare il servizio sia da desktop che da mobile, e alla
		possibilità di inviare qualsiasi tipo di file;
		\item \textbf{Videoconferenze:} Per le videoconferenze di gruppo si utilizzerà \emph{Google Hangouts}.

		È utilizzabile da tutti i dispositivi e richiede semplicemente un account \emph{Google} di cui disponevano
		già tutti i membri del gruppo.
	\end{itemize}
	\textbf{Esterne}
	\\ \\
	Per tutte le comunicazioni esterne va utilizzata la mail ufficiale del gruppo: \href{mailto:sweeneythreads@gmail.com}{sweeneythreads@gmail.com}.
	 \\ La gestione di tale indirizzo email spetta al \emph{Responsabile} che dunque risulta essere l'unico componente del gruppo a poter comunicare
	con il committente i maniera ufficiale. Il \emph{Responsabile} ha il compito di informare gli altri membri del gruppo sulle
	discussioni avute con il committente, tale aggiornamento può avvenire a voce durante le riunioni e gli incontri oppure tramite
	l'inoltro delle email ricevute agli indirizzi personali dei componenti interessati.
	\\ \\
	Le email ufficiali devono rispettare le seguenti linee guida:
	\begin{itemize}
		\item \textbf{Destinatario:} poiché questo indirizzo email va usato esclusivamente per comunicazioni ufficiali il destinatario
		del messaggio va salvato tra i contatti (funzione di Gmail), nel caso non dovesse già farne parte;
		\item \textbf{Oggetto:} l'oggetto deve esprimere in maniera chiara ed esaustiva il contenuto dell'email, deve essere breve
		 e non deve rendere l'email confondibile con le altre preesistenti.
		 Nel caso il messaggio fosse una risposta l'oggetto deve essere preceduto dalla particella "Re:", nel caso di un inoltro dalla
		 particella "I:";
		\item \textbf{Corpo:} nel caso il messaggio fosse una risposta o un inoltro, il contenuto aggiunto va sempre scritto in testa al
		fine di non costringere i lettori a scorrere tutta l'email. La cancellazione della restante parte del messaggio è sconsigliata, per
		facilitare una visione completa della conversazione;
		\item \textbf{Allegati:} L'aggiunta di allegati al messaggio è consentita con l'unico vincolo di inviare file che possiedono un nome
		esplicativo o di specificare il contenuto dell'allegato nel corpo se il nome del file potrebbe essere poco comprensibile.
	\end{itemize}
	Inoltre il gruppo ha creato una pagina web di presentazione all'indirizzo \url{sweeneythreads.github.io}. Tale pagina, oltre a permettere un rapido accesso alla piattaforma di amministrazione \emph{SPitFire}, contiene diverse informazioni relative al gruppo e allo svolgimento del progetto: consegne effettuate, news, \dots
	\\ \\
	L'aggiornamento della pagina è compito del \emph{Responsabile}.
	\subsubsection{Riunioni}
		\textbf{Ufficiali} \\ \\
		Le riunioni possono essere due tipi: con la presenza del committente o senza la presenza del committente. Il gruppo si impegna a tenere almeno una riunione
		ufficiale senza presenza del committente ogni due settimane. Le riunioni hanno una durata minima di due ore, che potrà
		essere prolungata a piacere, in questo caso, nel verbale di riunione dovrà comparire di quanto si è superato il tempo
		previsto durante la riunione, e il motivo del prolungamento. Queste modifiche sono a carico del \emph{Responsabile di progetto}.
		I verbali prodotto andranno inseriti nella \emph{Repository} \verb|Actorbase/Documenti/Verbali|, che verrà suddivisa in due
		sottocartelle per i verbali interni e quelli esterni.
\\ \\
		Le riunioni con presenza del committente, andranno concordate secondo le norme di comunicazioni esterne con quest'ultimo
		e comunicate tramite i mezzi di comunicazione interni a tutti i membri del gruppo, ognuno dei quali è fortemente tenuto ad
		essere presente. Potranno verificarsi casi in cui non tutti i membri del gruppo potranno presentarsi alle riunioni con
		presenza del committente, ma non potrà verificarsi l'assenza del \emph{Responsabile di progetto} e dell'\emph{Amministratore}.
		I quali sono tenuti a riferire quanto emerso dalle riunioni a tutti i restanti membri assenti. Nelle riunioni con il
		committente può verificarsi un cambiamento riguardante ad un requisito, in questo caso il cambiamento va inserito nel
		verbale, che deve essere messo a disposizione del committente nella cartella \verb|Actorbase/Documenti/Verbali/Esterni|.
		\\ \\
		\textbf{Non ufficiali} \\ \\
		Le riunioni non ufficiali sono da considerarsi riunioni tra pochi membri del gruppo, ad esempio tra i due realizzatori di questo
		stesso documento, o incontri occasionali avvenuti senza comunicazioni nei canali ufficiali. Queste riunioni non necessitano di
		una stesura di un verbale; se da queste riunioni emergesse un grave errore, o una comunicazione importante, i membri presenti
		sono tenuti a richiedere una riunioni ufficiale straordinaria, che dovrà essere approvata dal'\emph{Amministratore}. In caso
		contrario, tutte le scelte non rilevanti non necessitano di approvazione.
		\textbf{Brainstorming} \\
		I \emph{Brainstorming} vengono tenuti sotto richiesta di qualsiasi membro del gruppo, e approvati, se per motivazioni
		valide, dal \emph{Responsabile di progetto}. Un \emph{Brainstorming}ha durata minima di un'ora e massima di due; durante il quale
		ogni membro ricopre un ruolo di egual importanza rispetto agli altri, le decisioni vengono prese all'unisono o con la maggioranza
		dei membri a favore, non è compito del \emph{Responsabile di progetto} approvare le soluzioni emerse da un \emph{Brainstorming}.

		Durante un \emph{Brainstorming} ci sarà un membro con il compito di scrivere le \emph{Minute}, ovvero un \emph{Notaio}.
		Ad ogni \emph{Brainstorming} sarà anche scelto un \emph{Moderatore} che ricoprirà un ruolo di servizio. Ovvero dovrà far
		rispettare le regole di base. Una volta finito il \emph{Brainstorming}, il \emph{Notaio} dovrà riorganizzare gli appunti
		presi in un verbale ordinato.
		
		\subsection{Meccanismi di controllo e rendicontazione}
		
			\subsubsection{Meccanismi di controllo}
				All'interno dell'ambiente di lavoro sono stati predisposti meccanismi per:
				\begin{itemize}
					\item Controllare l'andamento delle attività ed eventuali ritardi;
					\item Permettere un aggiornamento facilitato della pianificazione;
					\item Rendicontare le ore di lavoro spese nelle varie attività.
				\end{itemize}
				
			\subsubsection{Andamento delle attività}
				Per monitorare i ritardi sulle attività e acquisire maggiore esperienza per stime future si adotta la 
				funzione timer di Teamwork. Ogni componente del gruppo è invitato a tenere attivo il timer durante 
				tutto lo svolgimento delle attività a lui assegnate. In questo modo si può avere una misurazione del 
				tempo effettivo impiegato da ogni membro per svolgere le attività, che può poi essere confrontata con 
				la stima fatta a priori.
				%image
				\begin{figure}[H]
					\centering
					\includegraphics[scale=0.4]{PdP/teamworkTimer}
					\caption{Timer da attivare durante il lavoro svolto.}
				\end{figure}
				Inoltre per ogni attività è predisposta anche una \emph{due to date}, ovvero la data entro la quale la task deve 
				essere soddisfatta. Teamwork segnala ogni attività nel riepilogo non completata entro la data di fine con 
				una scritta rossa che riporta il ritardo. È facile per il responsabile individuare a colpo d'occhio le 
				task in ritardo e provvedere a comunicare con il/i componenti del gruppo a cui essa è assegnata per capire 
				le motivazioni del ritardo ed eventualmente rivedere le stime future.
				%image
				\begin{figure}[H]
					\centering
					\includegraphics[scale=0.4]{PdP/teamworkTaskinRitardo}
					\caption{Visualizzazione data di scadenza di un task.}
				\end{figure}
				Se necessario è possibile impostare notifiche automatiche in prossimità o al superare di una scadenza.
				Il sistema di ticketing adottato fornisce un calendario in cui vengono indicate le date stimate di inizio e fine
				di ogni attività.\\
				Ogni membro del gruppo può consultarlo liberamente per pianificare il proprio lavoro in base agli altri
				impegni privati.
				
			\subsubsection{Controllo metriche di progetto}
				L'introduzione delle metriche nel progetto fornisce una maniera il più possibile oggettiva e sistematica per
				misurare le performance del gruppo. Dal punto di vista del controllo del progetto le metriche impiegate sono:
				\begin{itemize}
					\item Budget Variance
					\item Schedule Variance
				\end{itemize}
				Questi indicatori permetto al team di:
				\begin{itemize}
					\item Identificare i problemi di costo/schedulazione prima che diventino criticità;
					\item Aiutare il team a focalizzarsi sul completamento delle proprie attività.
				\end{itemize}

	\cleardoublepage
	\addcontentsline{toc}{section}{\listfigurename}
	\listoffigures

	\cleardoublepage
	\addcontentsline{toc}{section}{\listtablename}
	\listoftables

\end{document}
