%Document-Author: Nicoletti Luca + Padovan Tommaso + Bortolazzo Matteo + Maino Elia + Bonato Paolo
%Document-Date: 2016/01/13
%Document-Description: Documento di Piano di qualifica

\documentclass[a4paper]{report}
\usepackage[english, italian]{babel}
\usepackage[T1]{fontenc}
\usepackage[utf8]{inputenc}
\usepackage{url}
\usepackage{graphicx}
\usepackage[hidelinks]{hyperref}
\usepackage{booktabs}
\usepackage{tabularx}
\usepackage{pifont}
\usepackage[table]{xcolor}
\usepackage{float}
\usepackage[]{appendix}

\graphicspath{{Immagini/}}

\newcolumntype{s}{>{\hsize=.21\hsize}X}
\newcolumntype{f}{>{\hsize=.37\hsize}X}
\newcolumntype{m}{>{\hsize=.42\hsize}X}

\newcommand{\mychapter}[2]{
	\setcounter{chapter}{#1}
	\setcounter{section}{0}
	\setcounter{subsection}{1}
	\chapter*{#2}
	\addcontentsline{toc}{chapter}{#2}
}

\renewcommand{\abstractname}{Tabella contenuti}

\begin{document}
	\begin{titlepage}
		% Defines a new command for the horizontal lines, change thickness here
		\newcommand{\HRule}{\rule{\linewidth}{0.5mm}} 
		\center  
		
		% HEADING SECTION
		\textsc{\LARGE SweeneyThreads}\\[1.5cm] 
		\textsc{\Large Actorbase}\\[0.5cm] 
		\textsc{\large a NoSQL DB based on the Actor model}\\[0.5cm]
		
		
		% TITLE SECTION
		\HRule \\[0.4cm]
		{ \huge \bfseries Piano di qualifica}\\[0.4cm] 
		\HRule \\[1.5cm]
		
		% AUTHOR SECTION
		\begin{minipage}{0.4\textwidth}
			\begin{flushleft} \large
				\emph{Redattori:}\\
				Bonato \textsc{Paolo} \\
				Bortolazzo \textsc{Matteo} \\
				Maino \textsc{Elia} \\
				Nicoletti \textsc{Luca} \\
				Padovan \textsc{Tommaso} \\
			\end{flushleft}
		\end{minipage}
		~
		\begin{minipage}{0.4\textwidth}
			\begin{flushright} \large
				\emph{Approvazione:} \\
				Padovan \textsc{Tommaso} \\
				\emph{Verifica:} \\
				Biggeri \textsc{Mattia} \\
				Tommasin \textsc{Davide} \\
			\end{flushright}
		\end{minipage}
		
		%immagine
		\begin{figure}[H]
			\centering
			\includegraphics[scale=0.8]{sweeney.png}
		\end{figure}
		\begin{center}
			Versione 1.0.0
		\end{center}
		% Date, change the \today to a set date if you want to be precise
		{\large \today}\\[3cm] 
		% Fill the rest of the page with whitespace
		\vfill  
	\end{titlepage}
	
	\tableofcontents
	
	\mychapter{0}{Diario delle modifiche}
		\begin{table}[H]
			\begin{tabularx}{\textwidth}{sfmX}
				\noalign{\hrule height 1.5pt}
				\rowcolor{orange!85} Versione & Data & Autore & Descrizione \\
				\noalign{\hrule height 1.5pt}
				1.0.0 & 2016-01-17 & \emph{Amministratore} Nicoletti Luca & Scrittura scheletro logico del documento \\
				\noalign{\hrule height 1.5pt}
			\end{tabularx}
			\caption{Diario delle modifiche \label{tab:table_label}}
		\end{table}
		
	\mychapter{1}{Introduzione}
		\section{Scopo del documento}
			Lo scopo di questo documento è di descrivere le scelte effettuate in merito alle strategie 
			che il gruppo ha deciso di adottare per raggiungere obiettivi qualitativi e misurabili da 
			applicare al proprio prodotto. Per soddisfare questi obiettivi sarà necessario attuare un 
			processo di verifica continuo sulle attività svolte in modo da poter rilevare ed eventualmente 
			correggere anomalie e incongruenze in modo tempestivo per evitare danni e sprechi di risorse.
		\section{Scopo del prodotto}
			Lo scopo del progetto è la realizzazione di un DataBase NoSQL key-value basato sul modello ad 
			Attori\ped{\textit{G}} con l'obiettivo di fornire una tecnologia adatta allo sviluppo di moderne 
			applicazioni che richiedono brevissimi tempi di risposta e che elaborano enormi quantità 
			di dati. Lo sviluppo porterà al rilascio del software sotto licenza MIT.
		\section{Glossario}
			Con lo scopo di evitare ambiguità di linguaggio e di massimizzare la comprensione dei documenti, il 
			gruppo ha steso un documento interno che è il \emph{Glossario v1.0.0}. La prima occorrenza
			di ogni termine termine contenuto nel \emph{Glossario} e presente in questo documento verrà 
			marcato con una "\textit{G}" maiuscola in pedice.
		\section{Riferimenti}
			\subsection{Normativi}
				\begin{itemize}
					\item \textbf{Norme di progetto:} \\ \emph{Norme di progetto v1.1.1};
					\item \textbf{Capitolato d'appalto:} \\ \url{http://www.math.unipd.it/~tullio/IS-1/2015/Progetto/C1p.pdf}.
				\end{itemize}
			\subsection{Formativi}
				\begin{itemize}
					\item \textbf{Piano di progetto:} \\ \emph{Piano di progetto v1.0.4};
					\item \textbf{Slide del corso:} \\ \url{http://www.math.unipd.it/~tullio/IS-1/2015/};
					\item \textbf{SWEBOK - Version 3:} \\ \url{http://www.computer.org/web/swebok/v3}
					\item \textbf{\dots}
				\end{itemize}
	\mychapter{2}{Visione generale della strategia di verifica}
		\section{Definizione obiettivi}
			In questa sezione verranno descritti gli obiettivi di qualità relativi al prodotto che il 
			gruppo ha deciso di raggiungere e gli obiettivi relativi ai processi che saranno svolti per 
			il completamento del progetto.
			\subsection{Qualità di processo}
			\subsection{Qualità di prodotto}
		\section{Organizzazione}
		\section{Pianificazione strategica e temporale}
		\section{Responsabilità}
		\section{Strumenti}
		\section{Analisi}
			\subsection{Tecniche per l'analisi statica}
			\subsection{Tecniche per l'analisi dinamica}
		\section{Misure e metriche}
			\subsection{Metriche per i processi}
			\subsection{Metriche per i documenti}
			\subsection{Metriche per il software}
		\section{Metodi}
			\subsection{Analisi dei processi}
			\subsection{Analisi dei documenti}
	\mychapter{3}{Gestione amministrativa della revisione}
		\section{Comunicazione e risoluzione di anomalie}
			%descrizione anomalia
		\section{Procedure di controllo qualità per i processi}
		\section{Procedure di controllo qualità per il prodotto}
	\mychapter{4}{Resoconto delle attività di verifica}
		\section{Riassunto delle attività di verifica}
		%Revisione dei requisiti
		\section{Tracciamento componenti - requisiti}
		\section{Dettaglio delle verifiche tramite analisi}
			\subsection{Processi}
			\subsection{Documenti}
		\section{Dettaglio dell'esito delle revisioni}
	
\end{document}
