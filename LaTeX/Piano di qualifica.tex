%Document-Author: Nicoletti Luca + Padovan Tommaso + Bortolazzo Matteo + Maino Elia + Bonato Paolo
%Document-Date: 2016/01/13
%Document-Description: Documento di Piano di qualifica

\documentclass[a4paper]{report}
\usepackage[english, italian]{babel}
\usepackage[T1]{fontenc}
\usepackage[utf8]{inputenc}
\usepackage{url}
\usepackage{graphicx}
\usepackage[hidelinks]{hyperref}
\usepackage{booktabs}
\usepackage{tabularx}
\usepackage{pifont}
\usepackage[table]{xcolor}
\usepackage{float}
\usepackage[]{appendix}

\graphicspath{{Immagini/}}

\newcolumntype{s}{>{\hsize=.21\hsize}X}
\newcolumntype{f}{>{\hsize=.37\hsize}X}
\newcolumntype{m}{>{\hsize=.42\hsize}X}

\newcommand{\mychapter}[2]{
	\setcounter{chapter}{#1}
	\setcounter{section}{0}
	\setcounter{subsection}{1}
	\chapter*{#2}
	\addcontentsline{toc}{chapter}{#2}
}

\renewcommand{\abstractname}{Tabella contenuti}

\begin{document}
	\begin{titlepage}
		% Defines a new command for the horizontal lines, change thickness here
		\newcommand{\HRule}{\rule{\linewidth}{0.5mm}} 
		\center  
		
		% HEADING SECTION
		\textsc{\LARGE SweeneyThreads}\\[1.5cm] 
		\textsc{\Large Actorbase}\\[0.5cm] 
		\textsc{\large a NoSQL DB based on the Actor model}\\[0.5cm]
		
		
		% TITLE SECTION
		\HRule \\[0.4cm]
		{ \huge \bfseries Piano di qualifica}\\[0.4cm] 
		\HRule \\[1.5cm]
		
		% AUTHOR SECTION
		\begin{minipage}{0.4\textwidth}
			\begin{flushleft} \large
				\emph{Redattori:}\\
				Bonato \textsc{Paolo} \\
				Bortolazzo \textsc{Matteo} \\
				Maino \textsc{Elia} \\
				Nicoletti \textsc{Luca} \\
				Padovan \textsc{Tommaso} \\
			\end{flushleft}
		\end{minipage}
		~
		\begin{minipage}{0.4\textwidth}
			\begin{flushright} \large
				\emph{Approvazione:} \\
				Padovan \textsc{Tommaso} \\
				\emph{Verifica:} \\
				Biggeri \textsc{Mattia} \\
				Tommasin \textsc{Davide} \\
			\end{flushright}
		\end{minipage}
		
		%immagine
		\begin{figure}[H]
			\centering
			\includegraphics[scale=0.8]{sweeney.png}
		\end{figure}
		\begin{center}
			Versione 1.0.1
		\end{center}
		% Date, change the \today to a set date if you want to be precise
		{\large \today}\\[3cm] 
		% Fill the rest of the page with whitespace
		\vfill  
	\end{titlepage}
	
	\tableofcontents
	
	\mychapter{0}{Diario delle modifiche}
		\begin{table}[H]
			\begin{tabularx}{\textwidth}{sfmX}
				\noalign{\hrule height 1.5pt}
				\rowcolor{orange!85} Versione & Data & Autore & Descrizione \\
				\noalign{\hrule height 1.5pt}
				1.0.1 & 2016-01-17 & \emph{Amministratori} Nicoletti Luca + Tommaso Padovan & Visione generale della 
				strategia di verifica e Gestione amministrativa della revisione \\
				\noalign{\hrule height 0.5pt}
				1.0.0 & 2016-01-17 & \emph{Amministratore} Nicoletti Luca & Scrittura scheletro logico del documento \\
				\noalign{\hrule height 1.5pt}
			\end{tabularx}
			\caption{Diario delle modifiche \label{tab:table_label}}
		\end{table}
		
	\mychapter{1}{Introduzione}
		\section{Scopo del documento}
			Lo scopo di questo documento è di descrivere le scelte effettuate in merito alle strategie 
			che il gruppo ha deciso di adottare per raggiungere obiettivi qualitativi e misurabili da 
			applicare al proprio prodotto. Per soddisfare questi obiettivi sarà necessario attuare un 
			processo di verifica continuo sulle attività svolte in modo da poter rilevare ed eventualmente 
			correggere anomalie e incongruenze in modo tempestivo per evitare danni e sprechi di risorse.
		\section{Scopo del prodotto}
			Lo scopo del progetto è la realizzazione di un DataBase NoSQL key-value basato sul modello ad 
			Attori\ped{\textit{G}} con l'obiettivo di fornire una tecnologia adatta allo sviluppo di moderne 
			applicazioni che richiedono brevissimi tempi di risposta e che elaborano enormi quantità 
			di dati. Lo sviluppo porterà al rilascio del software sotto licenza MIT.
		\section{Glossario}
			Con lo scopo di evitare ambiguità di linguaggio e di massimizzare la comprensione dei documenti, il 
			gruppo ha steso un documento interno che è il \emph{Glossario v1.0.0}. La prima occorrenza
			di ogni termine termine contenuto nel \emph{Glossario} e presente in questo documento verrà 
			marcato con una "\textit{G}" maiuscola in pedice.
		\section{Riferimenti}
			\subsection{Normativi}
				\begin{itemize}
					\item \textbf{Norme di progetto:} \\ \emph{Norme di progetto v1.1.1};
					\item \textbf{Capitolato d'appalto:} \\ \url{http://www.math.unipd.it/~tullio/IS-1/2015/Progetto/C1p.pdf}.
				\end{itemize}
			\subsection{Formativi}
				\begin{itemize}
					\item \textbf{Piano di progetto:} \\ \emph{Piano di progetto v1.0.4};
					\item \textbf{Slide del corso:} \\ \url{http://www.math.unipd.it/~tullio/IS-1/2015/};
					\item \textbf{SWEBOK - Version 3:} \\ \url{http://www.computer.org/web/swebok/v3}
					\item \textbf{\dots}
				\end{itemize}
	\mychapter{2}{Visione generale della strategia di verifica}
		\section{Definizione obiettivi}
			In questa sezione verranno descritti gli obiettivi di qualità relativi al prodotto che il 
			gruppo ha deciso di raggiungere e gli obiettivi relativi ai processi che saranno svolti per 
			il completamento del progetto.
			\subsection{Qualità di processo}
				Per garantire la qualità del prodotto è necessario garantire anche quella dei processi necessari 
				al suo completamento. A questo scopo si è deciso di adottare lo standard 
				\emph{ISO/IEC 15504} denominato \emph{SPICE}.
				
				Questo modello descrive come ogni processo debba essere controllato costantemente in maniera da 
				rilevare possibili errori o debolezze e correggerli prima che essi si diffondano, facendo 
				aumentare esponenzialmente il carico di lavoro. Affinché le singole valutazioni contribuiscano 
				all'effettivo miglioramento dei processi devono essere sempre ripetibili, oggettivi e comparabili.
				\emph{SPICE} definisce livelli di maturità del processo:
				\begin{itemize}
					\item[0] - Incompete
					\item[1] - Performed
					\item[2] - Managed
					\item[3] - Established
					\item[4] - Predictable
					\item[5] - Optimizing					
				\end{itemize}
				Al fine di applicare correttamente questo modello è evidentemente indispensabile adottare il 
				principio \emph{PDCA} il quale definisce una metodologia di controllo dei processi durante il loro 
				ciclo di vita che consente di migliorarne in modo continuativo la qualità. \\ 
				Esso di compone di 4 fasi:
				\begin{itemize}
					\item \textbf{Plan:} definire dettagliatamente cosa deve essere realizzato rispetto agli 
					obiettivi di miglioramento, e come questi controlli saranno effettuati;
					\item \textbf{Do:} fase di esecuzione delle attività pianificate;
					\item \textbf{Check:} vengono confrontati i dati in uscita dalla fase \emph{Do} con quelli 
					pianificati nella fase \emph{Plan}, per intervenire in tempo e migliorare i risultati;
					\item \textbf{Act:} fase in cui si mette in pratica il miglioramento continuo dei processi
					 utilizzando i risultati della verifica per modificare gli aspetti critici dei processi in esame.
				\end{itemize}
				\begin{figure}[H]
					\centering
					\includegraphics[scale=0.4]{PDCA.jpg}
					\caption{Fasi del principio PDCA.}
				\end{figure}
			\subsection{Qualità di prodotto}
		\section{Organizzazione}
		\section{Pianificazione strategica e temporale}
			Avendo lo scopo di rispettare le scadenze riportate nel Piano di progetto, è necessario che l'attività 
			di verifica, sia del codice che della documentazione, sia sistematica e ben organizzata; in questo modo 
			l'individuazione, e quindi la correzione degli errori avverrà il prima possibile, limitando la diffusione 
			degli stessi.\\
			Per cercare di ridurre il numero degli errori, e quindi semplificare l'attività di verifica, ogni fase 
			di codifica o documentazione sarà preceduta da una fase di studio preliminare. Evitando le imprecisioni 
			di natura concettuale si ridurranno le correzioni necessarie.
			
			Di seguito vengono riportate le scadenze previste:
			\begin{itemize}
				\item \textbf{Revisione dei requisiti:} 2016-01-22;
				\item \textbf{Revisione di progettazione:} 2016-04-18;
				\item \textbf{Revisione di qualifica:} 2016-05-23;
				\item \textbf{Revisione di accettazione:} 2016-06-17.
			\end{itemize}
		\section{Responsabilità}
			Il \emph{Responsabile di progetto} ha il compito di:
			\begin{itemize}
				\item Accertarsi che le attività di verifica vengano svolte sistematicamente secondo quanto 
				riportato nelle \emph{Norme di progetto};
				\item Accertarsi che vengano rispettati ruoli e competenze assegnate nel \emph{Piano di progetto};
				\item Verificare che non ci siano conflitti di interesse tra redattori e \emph{Verificatori};
				\item Aprire ed assegnare i ticket principali e le task-list;
				\item Approvare un documento e sancirne la distribuzione.
			\end{itemize}
			I \emph{Verificatori} hanno il compito di:
			\begin{itemize}
				\item Effettuare la verifica dei documenti con strumenti e metodi proposti nel \emph{Piano di Qualifica};
				\item Attenersi rigidamente a quanto sancito nelle \emph{Norme di progetto};
				\item Segnalare tempestivamente un errore, qualora riscontrato;
				\item Sottoporre i documenti all'approvazione del \emph{Responsabile}, una volta giunti ad uno stadio finale.
			\end{itemize}
		\section{Risorse}
			Per la realizzazione del progetto sono necessarie risorse sia umane che tecnologiche.
			\subsection{Risorse umane}
				Vengono descritte nel dettaglio nel \emph{Piano di progetto} e sono:
				\begin{itemize}
					\item \emph{Responsabile di progetto};
					\item \emph{Amministratore};
					\item \emph{Analista};
					\item \emph{Progettista};
					\item \emph{Programmatore};
					\item \emph{Verificatore}.
				\end{itemize}
			\subsection{Risorse software}
				Sono necessari tutti i software utili
				\begin{itemize}
					\item alla gestione di documentazione in \LaTeX;
					\item alla creazione di diagrammi UML;
					\item allo sviluppo di codice \emph{Scala};
					\item a semplificare ed automatizzare la verifica;
					\item a semplificare ed automatizzare la pianificazione e la documentazione della stessa;
					\item a semplificare ed automatizzare la comunicazione interna tra i membri del gruppo;
					\item a gestire test ed analisi sul codice.
				\end{itemize}
				\subsection{Risorse hardware}
					\begin{itemize}
						\item computer dotati di tutti i software descritti nel \emph{Piano di qualifica} e nelle
						\emph{Norme di progetto};
						\item luoghi dove effettuare le riunioni del gruppo.
					\end{itemize}
		\section{Strumenti}
			Le risorse software che si utilizzeranno durante il processo di verifica sono:
			\begin{itemize}
				\item \textbf{\emph{TexMaker}:} un ambiente grafico Open-Source per \LaTeX cross-platform, permette 
				la compilazione rapida e la visualizzazione del PDF generato;
				\item \textbf{\emph{Scalastyle}:} analizzatore statico che rileva potenziali problemi nel codice
				 (\url{https://github.com/scalastyle/scalastyle});
				\item \textbf{\emph{Scapegoat}:} un altro analizzatore statico che si concentra maggiormente sugli 
				standard di stile e di coding (\url{https://github.com/sksamuel/scapegoat});
				\item \textbf{\emph{CLOC (Count Lines Of Code)}:} misura alcune metriche riguardanti il codice 
				sorgente in vari linguaggi, tra cui \emph{Scala} (\url{cloc.sourceforge.net});
				\item \textbf{\emph{ScalaTest}:} framework per i test su Scala (\url{http://www.scalatest.org/}).
			\end{itemize}
		\section{Analisi}
			\subsection{Tecniche per l'analisi statica}
				L'analisi statica non richiede l'esecuzione del codice in oggetto, ed è quindi applicabile sia 
				alla documentazione che al codice. Permette di individuare errori ed anomalie al più presto possibile, 
				scongiurandone la diffusione. \\ Essa può essere svolta in due modi distinti.
				\subsubsection{Walkthrough}
					Si svolge effettuando una lettura critica a pettine. Questa tecnica viene utilizzata prevalentemente 
					nelle prime fasi del progetto, in cui non si ha ne una adeguata esperienza, ne uno storico degli errori 
					più comuni che permetta una indagine più mirata. I Verificatori, tramite questa tecnica, saranno in 
					grado di stilare una lista di errori più frequenti, potendo così applicare successivamente la tecnica 
					\emph{Inspection}. Il \emph{Walkthrough} è una tecnica onerosa e richiede l'intervento di più persone. 
					Dopo una fase iniziale in cui i Verificatori leggono il documento ed individuano potenziali errori essi 
					devono essere discussi in una riunione con altri componenti del gruppo per accertare che non siano dei 
					falsi positivi.
				\subsubsection{Inspection}
					È una tecnica molto meno onerosa. Consiste nel controllare alcune parti dei documenti che si sono rivelate 
					maggiormente prone ad errori. Per ottenere questo risultato è necessario avere una lista di controllo che 
					indichi quali sono le parti da controllare in maniera mirata. Essa viene stilata durante le fasi di 
					\emph{Walkthrough}. Un altro motivo per cui la \emph{Inspection} è preferibile è il fatto che essa richiede 
					l'intervento dei soli verificatori, che poi possono procedere alla correzione della maggior parte degli errori, 
					oppure ad aprire un ticket riguardante quelli che non sono di immediata risoluzione.\\
					
					
				Durante l'applicazione del \emph{Walkthrough} ai documenti sono state riportate le tipologie di errori più frequenti, 
				esse costituiscono quindi la lista di controllo per le verifiche ad \emph{Inspection}:
				\begin{itemize}
					\item \textbf{Norme stilistiche:}
					\begin{itemize}
						\item Nome del documento: non viene utilizzata la macro predisposta;
						\item Versione del documento in prima pagina errata;
						\item Immagini mancanti;
						\item Spazi lasciati vuoti per aggiunte successive e non rimossi;
						\item Mancanza di uniformità delle espressioni all'interno dello stesso documento;
						\item Mancanze nella sezione dei riferimenti.
					\end{itemize}
					\item \textbf{Italiano:}
					\begin{itemize}
						\item Doppie;
						\item Accenti.
					\end{itemize}
					\item \textbf{\LaTeX :}
					\begin{itemize}
						\item mancanza dell'indice delle immagini e delle tabelle.
					\end{itemize}
					\item \textbf{UML:}
					\begin{itemize}
						\item incongruenze tra l'immagine contenente i diagrammi e la descrizione testuale della stessa;
						\item errori nel testo delle immagini dovute a copia-incolla.
					\end{itemize}
				\end{itemize}
			\subsection{Tecniche per l'analisi dinamica}
				Questo tipo di analisi richiede una esecuzione di parte del programma, quindi ovviamente non applica 
				ai documenti ma solo al codice. Il suo obiettivo è rilevare errori o difetti di implementazione 
				mediante l'uso di test che devono essere necessariamente ripetibili: solo un test che produca lo 
				stesso output partendo dallo stesso ambiente e lo stesso input può è capace di riscontrare problemi. 
				L'attore che esegue un test deve definire a priori ed avere il pieno controllo su:
				\begin{itemize}
					\item \textbf{Ambiente:} insieme di hardware a software come sistema operativo e altri programmi o 
					processi in esecuzione;
					\item \textbf{Specifiche:} definizione degli input e dei relativi output attesi, che sono ripetibili 
					in quanto si postula di essere in un ambiente deterministico;
					\item \textbf{Procedure:} descrizione delle azioni compiute dall'attore (umano o computer che sia) 
					per arrivare allo stato iniziale, far partire l'esecuzione, inserire gli input specificati e verificare 
					che l'output sia uguale a quello atteso.
				\end{itemize}
		\section{Misure e metriche}
			\subsection{Metriche per i processi}
			\subsection{Metriche per i documenti}
			\subsection{Metriche per il software}
		\section{Metodi}
			\subsection{Analisi dei processi}
			\subsection{Analisi dei documenti}
	\mychapter{3}{Gestione amministrativa della revisione}
		\section{Comunicazione e risoluzione di anomalie}
			Una anomalia è una violazione da parte di un documento, o unità di codice, di una o più delle seguenti condizioni:
			\begin{itemize}
				\item Conformità alla norme tipografiche o di codifica;
				\item Appartenenza al range di accettabilità per tutte le metriche descritte nella \emph{Sezione 2};
				\item Congruenza del prodotto con funzionalità indicate nell'\emph{analisi dei requisiti};
				\item Congruenza del codice con il design del prodotto.
			\end{itemize}
			Se un \emph{Verificatore} dovesse trovare una anomalia egli è tenuto ad aprire un sotto-ticket all'interno 
			della task-list a lui assegnata. Nel caso la risoluzione del ticket avesse la necessità di essere strutturato 
			in sotto-attività sarà compito del \emph{Responsabile} aprire una nuova task-list ed assegnarla alle figure coinvolte.

		\section{Procedure di controllo qualità per i processi}
			La qualità del processo viene garantita da:
			\begin{itemize}
				\item \textbf{Pianificazione:} i processi devono essere pianificati nel dettaglio, in maniera da determinare 
				i punti e le tempistiche in cui effettuare controlli;
				\item \textbf{Controllo:} i controlli pianificati devono essere eseguiti in maniera oggettiva e neutrale, 
				quindi con strumenti automatici ovunque possibile;
				\item \textbf{Miglioramento continuo:} è garantita dall'applicazione del principio \emph{PDCA}.
			\end{itemize}
		\section{Procedure di controllo qualità per il prodotto}
			La qualità del prodotto viene garantita da:
			\begin{itemize}
				\item \textbf{Comprensione ed analisi del dominio};
				\item \textbf{Verifica:} determina che l'output di una fase sia consistente, completo e corretto. 
				Deve essere eseguita costantemente per tutta la durata del progetto, ma cercando di essere minimamente invasiva;
				\item \textbf{Validazione:} conferma oggettivamente che il prodotto sia conforme alle aspettative;
				\item \textbf{Quality Assurance:} garantisce il raggiungimento degli obiettivi di qualità, in maniera preventiva. 
				In questo modo si riduce drasticamente il ricorso a tecniche retrospettive, e con esse si riducono le iterazioni.
			\end{itemize}
	\mychapter{4}{Resoconto delle attività di verifica}
		\section{Riassunto delle attività di verifica}
		%Revisione dei requisiti
		\section{Tracciamento componenti - requisiti}
		\section{Dettaglio delle verifiche tramite analisi}
			\subsection{Processi}
			\subsection{Documenti}
		\section{Dettaglio dell'esito delle revisioni}
	
\end{document}
