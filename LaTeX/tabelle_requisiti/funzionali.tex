
\newcolumntype{s}{>{\hsize=.37\hsize}X}
\newcolumntype{f}{>{\hsize=.42\hsize}X}
\newcolumntype{m}{>{\hsize=.21\hsize}X}

\begin{longtable}{s f m X}  
			\rowcolor{orange!85}Codice & Nome & Fonte & Descrizione \\
\endhead
R[1][N][F] & STOREKEEPER & Capitolato
	& Creazione dell'actor STOREKEEPER per mantenere i dati su memoria interna \\
	\hline
		R[1.1][N][F] & Elaborazione messaggi ricevuti STOREKEEPER & Capitolato
		& Gestire i messaggi ricevuti da STOREKEEPER \\
		\hline
			R[1.1.1][N][F] & Elaborazione messaggi aggiunta riga & Capitolato
			& Aggiunge una riga con la chiave ed il valore presenti nel messaggio \\
			\hline
			R[1.1.2][N][F] & Elaborazione messaggi rimozione riga & Capitolato
			& Rimuove la riga con chiave uguale a quella presente nel messaggio\\
			\hline
			R[1.1.3][N][F] & Elaborazione messaggi aggiornamento riga & Capitolato
			& Aggiorna il valore della riga con chiave uguale a quella presente nel messaggio con il valore presente nel messaggio\\
			\hline
			R[1.1.4][N][F] & Elaborazione messaggi lettura riga & Capitolato
			& Se presente restituisce il valore della riga con chiave uguale a quella presente nel messaggio \\
			\hline
			R[1.1.5][O][F] & Elaborazione messaggi di risposta da attori MANAGER & Capitolato
			& Gestione dei messaggi inviati da attori MANAGER in risposta alla richieste dell'actor STOREKEEPER \\
			\hline
		R[1.2][N][F] & Invio messaggi STOREKEEPER & Capitolato
		& Creazione messaggi da inviare da STOREKEEPER \\
		\hline
			R[1.2.1][N][F] & Invio messaggi a WAREHOUSEMAN & Capitolato
			& Creazione messaggi da inviare a WAREHOUSEMAN \\
			\hline
				R[1.2.1.1][N][F] & Invio messaggi aggiunta a WAREHOUSEMAN & Capitolato
				& Creazione messaggi di aggiunta riga da inviare a WAREHOUSEMAN \\
				\hline
				R[1.2.1.2][N][F] & Invio messaggi rimozione a WAREHOUSEMAN & Capitolato
				& Creazione messaggi di rimozione riga da inviare a WAREHOUSEMAN \\
				\hline
				R[1.2.1.3][N][F] & Invio messaggi aggiornamento a WAREHOUSEMAN & Capitolato
				& Creazione messaggi di aggiornamento riga da inviare a WAREHOUSEMAN \\
				\hline
			R[1.2.2][N][F] & Invio messaggi risposta a STOREFINDER & Capitolato
			& Creazione messaggi di risposta a comando da inviare a STOREFINDER \\
			\hline
			R[1.2.3][O][F] & Invio messaggi richiesta proprietà a MANAGER & Capitolato
			& Creazione messaggi di richiesta proprietà interne a MANAGER \\
			\hline
			R[1.2.4][O][F] & Invio messaggi a NINJA & Capitolato
			& Creazione messaggi da inviare a NINJA \\
			\hline
				R[1.2.4.1][O][F] & Invio messaggi aggiunta a NINJA & Capitolato
				& Creazione messaggi di aggiunta riga da inviare a NINJA \\
				\hline
				R[1.2.4.2][O][F] & Invio messaggi rimozione a NINJA & Capitolato
				& Creazione messaggi di rimozione riga da inviare a NINJA \\
				\hline
				R[1.2.4.3][O][F] & Invio messaggi aggiornamento a NINJA & Capitolato
				& Creazione messaggi di aggiornamento riga da inviare a NINJA \\
				\hline
		R[1.3][O][F] & Gestione memoria interna & Decisione interna
		& Implementazione gestione controllata dell'uso della memoria interna \\
		\hline
			R[1.3.1][O][F] & Impostazione memoria massima utilizzabile dal sistema & Decisione interna
			& Implementare la possibilità di impostare un valore massimo di memoria occupabile dal sistema \\
			\hline
			R[1.3.2][O][F] & Rilevazione memoria totale ed in uso & Decisione interna
			& Implementare la rilevazione della memoria totale del sistema e di quella utilizzata \\
			\hline
			R[1.3.3][O][F] & Valutazione svuotamento memoria & Decisione interna
			& Implementare un sistema che valuti quando è necessario svuotare la memoria interna \\
			\hline
			R[1.3.4][O][F] & Selezionare elementi da rimuovere dalla memoria & Decisione interna
			& Implementare la selezione degli elementi che possono essere rimossi dalla memoria \\
			\hline
			R[1.3.5][O][F] & Eliminare elementi selezionati	 & Decisione interna
			& Implementare l'eliminazione degli elementi selezionati dalla memoria interna \\
			\hline
			R[1.3.6][O][F] & Recuperare elementi da disco se necessario & Decisione interna
			& Implementare il recupero da disco degli elementi rimossi dalla memoria interna \\
			\hline
		R[1.4][O][F] & Creazione attori NINJA & Capitolato 
		& Implementare la creazione di nuovi attori NINJA da parte dell'actor STOREKEEPER \\
		\hline
	R[2][N][F] & STOREFINDER & Capitolato
	& Creazione dell'actor STOREFINDER per gestire le richieste \\
	\hline
		R[2.1][N][F] & MAIN & Capitolato
		& Creazione dell'actor MAIN, sottotipo di STOREFINDER\\
		\hline		
			R[2.1.1][N][F] & Gestione connessioni & Capitolato
			& Implementare la connessione via rete al server \\
			\hline
			R[2.1.2][N][F] & Gestione accesso & Capitolato
			& Implementare l'accesso al server tramite apposite credenziali\\
			\hline
				R[2.1.2.1][N][F] & Aggiunta credenziali & Capitolato
				& Implementare l'aggiunta di nuove credenziali \\
				\hline
				R[2.1.2.2][N][F] & Rimozione credenziali & Capitolato
				& Implementare la rimozione di credenziali \\
				\hline
				R[2.1.2.3][N][F] & Memorizzazione tabella credenziali su disco & Capitolato
				& Implementare il salvataggio su disco delle credenziali \\
				\hline
				R[2.1.2.4][N][F] & Lettura tabella credenziali da disco & Capitolato
				& Implementare la lettura da disco delle credenziali \\
				\hline
			R[2.1.3][D][F] & Gestione database (o schemi) & Decisione interna
			& Implementare la gestione di più database all'interno dello stesso server \\
			\hline
				R[2.1.3.1][D][F] & Aggiunta database & Decisione interna
				& Implementare l'aggiunta di un nuovo database \\
				\hline
				R[2.1.3.2][D][F] & Rimozione database & Decisione interna
				& Implementare la rimozione di un database \\
				\hline
				R[2.1.3.3][D][F] & Rinomina database & Decisione interna
				& Implementare la rinomina di un database\\
				\hline
				R[2.1.3.4][D][F] & Associazione database-utente & Decisione interna
				& Implementare l'associazione utente-database per la visibilità controllata \\
				\hline	
					R[2.1.3.4.1][D][F] & Aggiunta privilegi database a utente & Decisione interna
					& Aggiungere ai privilegi di un utente la modifica di un database \\
					\hline	
					R[2.1.3.4.2][D][F] & Rimozione privilegi database a utente & Decisione interna
					& Rimuovere dai privilegi di un utente la modifica di un database \\
					\hline				
			R[2.1.4][N][F] & Creazione attori & Capitolato
			& Gestire la creazione di nuovi attori da parte degli attori MAIN \\
			\hline
				R[2.1.4.1][N][F] & Creazione attori MAIN ad ogni nuova connesione & Capitolato
				& Permettere all'actor MAIN principale di creare nuovi MAIN per gestire le richieste esterne \\
				\hline
				R[2.1.4.2][N][F] & Creazione attori WAREHOUSEMAN per lettura disco & Capitolato
				& Permettere all'actor MAIN di creare attori WAREHOUSEMAN per leggere i dati presenti su disco all'avvio \\
				\hline
				R[2.1.4.3][N][F] & Creazione attori STOREKEEPER & Capitolato
				& Permettere all'actor MAIN di creare attori STOREKEEPER quando una nuova mappa viene creata dall'utente o quando viene letta da disco \\
				\hline
				R[2.1.4.4][N][F] & Creaizone attori STOREFINDER & Capitolato
				& Permettere all'actor MAIN di creare attori STOREFINDER per gestire le richieste interne \\
				\hline
			R[2.1.5][O][F] & Gestione richieste esterne & Capitolato
			& Implementare il riconoscimento di un DSL \\
			\hline
				R[2.1.5.1][N][F] & Parsing stringa comando & Capitolato
				& Implementare la gestione dei comandi inseriti dall'utente \\
				\hline	
					R[2.1.5.1.1][N][F] & Invio messaggi in base al comando & Capitolato
					& Inviare messaggi agli attori adatti dopo un comando utente \\
					\hline			
			R[2.1.6][O][F] & Gestione STOREFINDER & Capitolato
			& Gestire le comunicazioni tra attori MAIN e STOREFINDER \\
			\hline
				R[2.1.6.1][N][F] & Invio messaggi a STOREFINDER & Capitolato
				& Creazione messaggi da inviare a STOREFINDER \\
				\hline
				R[2.1.6.2][N][F] & Elaborazione messaggi da STOREFINDER & Capitolato
				& Gestire i messaggi di risposta ricevuti da STOREFINDER \\
				\hline
				R[2.1.6.3][N][F] & Gestione numero massimo di STOREFINDER & Capitolato
				& Gestire le impostazioni per il numero massimo di STOREFINDER possibili \\
				\hline
			R[2.1.7][O][F] & Importazione & Decisione interna, UC3.3
			& Implementare l'importazione di database \\
			\hline
			R[2.1.8][O][F] & Esportazione & Decisione interna, UC3.4
			& Implementare l'esportazione di database \\
			\hline
		R[2.2][N][F] & Elaborazione messaggi ricevuti STOREFINDER & Capitolato
		& Gestione messaggi ricevuti da STOREFINDER \\
		\hline		
			R[2.2.1][N][F] & Elaborazione messaggi di aggiunta da MAIN & Capitolato
			& Gestire i messaggi ricevuti di aggiunta riga da MAIN \\
			\hline
			R[2.2.2][N][F] & Elaborazione messaggi di rimozione riga da MAIN & Capitolato
			& Gestire i messaggi ricevuti di rimozione riga da MAIN \\
			\hline
			R[2.2.3][N][F] & Elaborazione messaggi di aggiornamento riga da MAIN & Capitolato
			& Gestire i messaggi ricevuti di aggiornamento riga da MAIN \\
			\hline
			R[2.2.4][N][F] & Elaborazione messaggi da STOREKEEPER & Capitolato
			& Gestire i messaggi di risposta ricevuti da MAIN \\
			\hline
		R[2.3][N][F] & Invio messaggi STOREFINDER & Capitolato
		& Creazione messaggi da inviare da STOREFINDER \\
		\hline		
			R[2.3.1][N][F] & Invio messaggi lettura riga a STOREKEEPER & Capitolato
			& Creazione messaggi di lettura riga da inviare a STOREKEEPER\\
			\hline
			R[2.3.2][N][F] & Invio messaggi aggiunta riga a STOREKEEPER & Capitolato
			& Creazione messaggi di aggiunta riga da inviare a STOREKEEPER\\
			\hline
			R[2.3.3][N][F] & Invio messaggi rimozione riga a STOREKEEPER & Capitolato
			& Creazione messaggi di rimozione riga da inviare a STOREKEEPER\\
			\hline
			R[2.3.4][N][F] & Invio messaggi aggiornamento riga a STOREKEEPER & Capitolato
			& Creazione messaggi di aggiornamento riga da inviare a STOREKEEPER\\
			\hline
	R[3][N][F] & WAREHOUSEMAN & Capitolato
		& Creazione dell'actor WAREHOUSEMAN per l'interazione con il filesystem \\
		\hline
		R[3.1][N][F] & Gestione file & Capitolato
		& Implementare l'integrazione tra WAREHOUSEMAN e filesystem \\
		\hline		
			R[3.1.1][N][F] & Creazione file & Capitolato
			& Implementare la creazione di file su disco \\
			\hline
			R[3.1.2][N][F] & Rimozione file & Capitolato
			& Implementare la rimozione di file su disco \\
			\hline
			R[3.1.3][N][F] & Modifica file & Capitolato
			& Implementare la modifica di file su disco \\
			\hline
			R[3.1.4][N][F] & Rinomina file & Capitolato
			& Implementare la rinomina dei file su disco \\
			\hline		
		R[3.2][N][F] & Elaborazione messaggi ricevuti WAREHOUSEMAN & Capitolato
		& Gestire i messaggi ricevuti da WAREHOUSEMAN \\
		\hline		
			R[3.2.1][N][F] & Elaborazione messaggi aggiunta riga su file & Capitolato
			& Gestire i messaggi di aggiunta riga su file ricevuti da STOREKEEPER  \\
			\hline
			R[3.2.2][N][F] & Elaborazione messaggi rimozione riga da file & Capitolato
			& Gestire i messaggi di rimozione riga su file ricevuti da STOREKEEPER  \\
			\hline
			R[3.2.3][N][F] & Elaborazione messaggi aggiornamento riga su file & Capitolato
			& Gestire i messaggi di aggiornamento riga su file ricevuti da STOREKEEPER  \\
			\hline
			R[3.2.4][N][F] & Elaborazione messaggi rimozione mappa & Capitolato
			& Gestire i messaggi di rimozione mappa ricevuti da STOREKEEPER  \\
			\hline
			R[3.2.4][N][F] & Elaborazione messaggi lettura mappe da disco & Capitolato
			& Gestire i messaggi di richiesta di lettura filesystem ricevuti da MAIN  \\
			\hline
		R[3.3][N][F] & Invio messaggi WAREHOUSEMAN & Capitolato
		& Creazione messaggi da inviare da WAREHOUSEMAN \\
		\hline		
			R[3.3.1][N][F] & Invio messaggi conferma a STOREKEEPER & Capitolato
			& Creazione messaggi di risposta da inviare da STOREKEEPER \\
			\hline
			R[3.3.2][N][F] & Invio messaggi risposta a MAIN & Capitolato
			& Creazione messaggi di risposta da inviare da MAIN \\
			\hline
	R[4][O][F] & NINJA & Capitolato
		& Creazione dell'actor NINJA per il recupero dei dati \\
		\hline
		R[4.1][O][F] & Elaborazione messaggi ricevuti NINJA & Capitolato
		& Gestire i messaggi ricevuti da NINJA  \\
		\hline		
			R[4.1.1][O][F] & Elaborazione messaggi aggiunta riga a mappa NINJA & Capitolato
			& Gestire i messaggi di aggiunta riga ricevuti da STOREKEEPER  \\
			\hline
			R[4.1.2][O][F] & Elaborazione messaggi rimozione riga da mappa NINJA & Capitolato
			& Gestire i messaggi di rimozione riga ricevuti da STOREKEEPER  \\
			\hline
			R[4.1.3][O][F] & Elaborazione messaggi aggiornamento riga su mappa NINJA & Capitolato
			& Gestire i messaggi di aggiornamento riga ricevuti da STOREKEEPER  \\
			\hline
		R[4.2][O][F] & Invio messaggi NINJA & Capitolato
		& Creazione messaggi da inviare da NINJA \\
		\hline		
			R[4.2.1][O][F] & Invio messaggi conferma a STOREKEEPER & Capitolato
			& Creazione messaggi di conferma da inviare da STOREKEEPER \\
			\hline
		R[4.3][O][F] & Cambio interfaccia NINJA & Capitolato
		& Implementare la possibilità di mutare a STOREKEEPER quando necessario \\			
		\hline
	R[5][O][F] & MANAGER & Capitolato
	& Creazione actor MANAGER per gestire le proprietà degli attori STOREKEEPER \\
	\hline
		R[5.1][O][F] & Elaborazione messaggi ricevuti da MANAGER & Capitolato
		& Gestire i messaggi ricevuti da MANAGER \\
		\hline		
			R[5.1.1][O][F] & Elaborazione messaggi contenenti grandezza massima mappa per STOREKEEPER da MAIN & Capitolato
			& Gestione messaggi per impostare grandezza massima della mappa interna degli attori STOREKEEPER \\
			\hline
		R[5.2][O][F] & Invio messaggi MANAGER & Capitolato
		& Creazione messaggi da inviare da MANAGER \\
		\hline		
			R[5.2.1][O][F] & Invio messaggi contenenti le proprietà a STOREKEEPER & Capitolato
			& Creazione messaggi di risposta contenenti le proprietà da inviare da STOREKEEPER \\
			\hline
		R[5.3][O][F] & Creazione nuovi STOREKEEPER & Capitolato
		& Implementare la creazione di nuovi STOREKEEPER in caso di bisogno \\			
		\hline
			R[5.3.1][O][F] & Quando viene creata una nuova mappa & Capitolato
			& Creare un nuovo STOREKEEPER quando viene richiesta la creazione di una nuova mappa \\
			\hline
			R[5.3.2][O][F] & Quando la grandezza di una mappa supera la capienza dello STOREKEEPER & Capitolato
			& Creare un nuovo STOREKEEPER quando la capienza dalla mappa interna di tutti gli STOREKEEPER è raggiunta \\
			\hline
			R[6][N][F] & Programma di setup & UC1
			 & Implementare un programma di setup che permetta di installare il prodotto sulla propria macchina
			  \\
			\hline
			R[6.1][N][F] & Programma di setup - Installazione completa & UC1
			 & Implementare una funzione del programma di setup che permetta di installare completamente il
			  prodotto sulla propria macchina \\
			\hline
			R[6.2][D][F] & Programma di setup - Installazione server & UC1
			 & Implementare una funzione del programma di setup che permetta di installare solamente la parte
			 server del prodotto sulla propria macchina \\
			\hline
			R[6.1][D][F] & Programma di setup - Installazione client & UC1
			 & Implementare una funzione del programma di setup che permetta di installare solamente la parte
			 client del prodotto sulla propria macchina \\
			\hline
			R[7][O][F] & Implementazione driver & Capitolato
			 & Implementare un driver JDBC per l'interfacciamento al database \\
			\hline
			R[7.1][O][F] & Implementazione driver Scala & Capitolato
			 & Implementare un driver Scala per l'interfacciamento al database direttamente da programmi 
			 scritti nel suddetto linguaggio \\
			\hline
			R[7.2][O][F] & Implementazione driver Java & Capitolato
			 & Implementare un driver Java per l'interfacciamento al database direttamente da programmi 
			 scritti nel suddetto linguaggio \\
			\hline
			R[8][N][F] & Definizione di un DSL (\emph{Domain specific language}) & Capitolato, UC3, UC4, UC5
			 & Definire un DSL per svolgere le operazioni sul database \\
			\hline
			R[8.1][N][F] & Comandi a livello server/database & UC3
			 & Definire dei comandi per poter gestire i database su un server \\
			\hline
			R[8.1.1][N][F] & Comando di selezione di un database & UC3.1
			 & Definire un comando che permetta di selezionare uno specifico database \\
			\hline
			R[8.1.2][N][F] & Comando di creazione di un database & UC3.2
			 & Definire un comando che permetta di creare un nuovo database \\
			\hline
			R[8.1.3][N][F] & Comando di rimozione di un database & UC3.5
			 & Definire un comando che permetta di rimuovere uno specifico database \\
			\hline
			R[8.2][N][F] & Comandi a livello di mappe & UC4
			 & Definire dei comandi per poter gestire le mappe di un database \\
			\hline
			R[8.2.1][N][F] & Comando di creazione di una mappa & UC4.1
			 & Definire un comando che permetta di creare una nuova mappa all'interno di un database, 
			 dandole un nome \\
			\hline
			R[8.2.2][N][F] & Comando di modifica del nome di una mappa & UC4.2
			 & Definire un comando che permetta di modificare il nome di una mappa preesistente \\
			\hline
			R[8.2.3][N][F] & Comando di rimozione di una mappa & UC4.3
			 & Definire un comando che permetta di rimuovere una mappa da un database \\
			\hline
			R[8.3][N][F] & Comandi a livello di righe & UC5
			 & Definire dei comandi per poter gestire le righe di una mappa \\
			\hline
			R[8.3.1][N][F] & Comando di inserimento di una riga & UC5.1
			 & Definire un comando per inserire una riga (coppia chiave-valore) all'interno di una mappa \\
			\hline
			R[8.3.2][N][F] & Comando di ricerca di una chiave & UC5.2
			 & Definire un comando per ricercare una riga avente una determinata chiave all'interno di una mappa
			  \\
			\hline
			R[8.3.3][N][F] & Comando di rimozione di una riga & UC5.2
			 & Definire un comando per rimuovere una riga all'interno di una mappa utilizzando la sua chiave\\
			\hline
			R[10][N][F] & Implementazione di un'interfaccia da riga di comando & Capitolato, UC2, UC3,
			UC4, UC5 & Implementare un'interfaccia da riga di comando che permetta di interagire con il
			database \\
			\hline
			R[10.1][N][F] & Interfaccia per connettersi ed autenticarsi a un server & UC2 &
			Implementare un'interfaccia da riga di comando che permetta connettersi ad un server inserendone
			l'indirizzo e che permetta di accedervi con username e password \\
			\hline
			R[10.2][N][F] & Interfaccia amministratore & UC3, UC3.6 & Implementare un'interfaccia 	da riga
			di comando che permetta all'amministratore del server di gestire i database su esso presenti e di
			gestire gli utenti che accedono al server definendone i permessi \\
			\hline
			R[10.2.1][N][F] & Interfaccia per inserire un utente & UC3.6.1 &
			Implementare la possibilità di aggiungere un utente con username e password alla lista di utenti
			che accedono al server da riga di comando\\
			\hline
			R[10.2.2][N][F] & Interfaccia per modificare i permessi utente & UC3.6.2 &
			Implementare la possibilità di modificare i permessi utente da riga di comando\\
			\hline
			R[10.2.3][N][F] & Interfaccia per rimuovere un utente & UC3.6.3 &
			Implementare la possibilità di rimuovere un utente dalla lista degli utenti che hanno accesso al
			server da riga di comando \\
			\hline
			R[10.3][N][F] & Interfaccia per operazioni su database, mappe, righe & UC3,
			UC4, UC5 & Implementare un'interfaccia da riga di comando basata sull'utilizzo dei comandi DSL
			definti, che permetta di interagire con i database effettuando operazioni sui database stessi, sulle
			mappe e sulle righe\\
\bottomrule
\caption{Requisiti funzionali}
\end{longtable}   