
\newcolumntype{s}{>{\hsize=.37\hsize}X}
\newcolumntype{f}{>{\hsize=.42\hsize}X}
\newcolumntype{m}{>{\hsize=.21\hsize}X}

\begin{longtable}{s f m X}  
			\rowcolor{orange!85}Codice & Nome & Fonte & Descrizione \\
\endhead
R[1][N][F] & Implementazione del sistema lato server & Capitolato
	& Implementazione della parte server di \emph{Actorbase} \\
	\hline
	R[1.1][N][F] & Implementazione interfaccia CLI lato server & Capitolato
	& Implementazione dell'interfaccia da riga di comando per gestire le operazioni lato server \\
	\hline
		R[1.1.1][N][F] & Controllo presenza configurazioni & UC11
	& Implementazione controllo sulla presenza di una configurazione \emph{Actorbase} sul server all'avvio dell'interfaccia server \\
	\hline
	R[1.1.2][N][F] & Configurazione iniziale impostazioni server & UC11
	& Implementazione della modalità di configurazione iniziale di \emph{Actorbase} sul server su cui è stata avviato l'applicativo server nel caso in cui non fosse presente una configurazione \\
	\hline
	R[1.1.2.1][N][F] & Scelta porta connessione al server & UC11
	& Implementazione della scelta della porta da utilizzare come ingresso alla connessione \\
	\hline
	R[1.1.2.2][N][F] & Scelta username amministratore server & UC11
	& Implementazione della scelta dell'username con cui accedere al sistema come amministratore \\
	\hline
	R[1.1.2.3][N][F] & Scelta password amministratore server & UC11
	& Implementazione della scelta della password con cui accedere al sistema come amministratore \\
	\hline
	R[1.1.2.4][N][F] & Messaggio inizializzazione server & UC11
	& Fornire all'utente un messaggio informativo sul processo di inizializzazione della configurazione \emph{Actorbase sul server} \\
	\hline
	R[1.1.2.5][N][F] & Elaborazione dati di configurazione server inseriti & UC11
	& Gestire i dati di configurazione server inseriti dall'utente durante il processo di configurazione \\
	\hline
	R[1.1.2.5.1][N][F] & Configurazione server effettuata con successo & UC11
	& Nel caso in cui i dati inseriti siano validi configurare una nuova istanza di \emph{Actorbase} sul server e informare l'utente del successo dell'operazione di configurazione\\
	\hline
	R[1.1.2.5.2][N][F] & Configurazione server fallita & UC13
	& Nel caso in cui i dati inseriti non siano validi informare l'utente con un messaggio di errore esplicativo\\
	\hline
	R[1.1.2.5.2.1][N][F] & Porta già utilizzata da un altro servizio & UC13
	& Nel caso in cui la porta richiesta durante la configurazione del server fosse già in uso da un altro servizio, o in generale non disponibile, implementare un messaggio di errore informativo\\
	\hline
	R[1.1.2.5.2.2][N][F] & Impostazione username amministratore fallita & UC13
	& Nel caso in cui l'username inserito durante la configurazione del server fosse  non disponibile o contenesse caratteri non validi, implementare un messaggio di errore informativo\\
	\hline
	R[1.1.2.5.2.3][N][F] & Impostazione password amministratore fallita & UC13
	& Nel caso in cui la password inserita durante la configurazione del server contenesse caratteri non validi, implementare un messaggio di errore informativo\\
	\hline
	R[1.1.3][N][F] & Arresto del server & UC12
	& Implementare l'arresto del server dall'interfaccia CLI lato server attraverso un apposito comando\\
	\hline
	R[1.1.4][N][F] & Riapertura configurazione esistente & UC11
	& Implementare la riapertura automatica della configurazione presente nel caso sul server fosse già stata configurata un'istanza di \emph{Actorbase} e si avviasse l'interfaccia lato server\\
	\hline
	R[1.2][N][F] & Elaborazione richiesta connessione & UC1
	& Implementare l'elaborazione di una richiesta di connessione da parte di un client\\
	\hline
	R[1.2.1][N][F] & Elaborazione comando connessione & UC1.1
	& Implementare l'elaborazione del comando DSL di richiesta di connessione\\
	\hline
	R[1.2.2][N][F] & Elaborazione nome utente & UC1.2
	& Implementare l'elaborazione del nome utente che intende connettersi\\
	\hline
	R[1.2.3][N][F] & Elaborazione password utente & UC1.3
	& Implementare l'elaborazione della password immessa dall'utente che intende connettersi\\
	\hline
	R[1.2.4][N][F] & Conferma connessione & UC1
	& Confermare al client che ha richiesto la connessione il successo della stessa\\
	\hline
	R[1.2.4][N][F] & Connessione fallita & UC7
	& Informare il client del fallimento del tentativo di connessione\\
	\hline
	R[1.3][N][F] & Operazioni a livello server & UC3
	& Implementare le operazioni eseguibili a livello server definite in UC3\\
	\hline
	R[1.3.1][N][F] & Elaborazione richiesta visualizzazione lista database & UC3.1
	& Implementare l'elaborazione di una richiesta di visualizzazione della lista dei database presenti da parte di un client\\
	\hline
	R[1.3.1.1][N][F] & Elaborazione comando visualizzazione lista database & UC3.1
	& Implementare il riconoscimento del comando di visualizzazione lista database all'interno di un messaggio ricevuto da un client\\
	\hline
	R[1.3.1.2][N][F] & Selezione in base ai permessi dell'utente & UC3.1
	& Implementare un filtro selettivo sulla lista dei database da fornire al client basato sui permessi dell'utente che ha richiesto la lista\\
	\hline
	R[1.3.1.3][N][F] & Invio lista database visualizzabili al client & UC3.1
	& Implementare l'invio di un messaggio di risposta al client, contenente la lista dei database da esso visualizzabili\\
	\hline
	R[1.3.2][N][F] & Elaborazione richiesta di esportazione database & UC3.2
	& Implementare l'elaborazione di una richiesta di esportazione database presenti da parte di un client\\
	\hline
	R[1.3.2.1][N][F] & Elaborazione comando di esportazione singolo & UC3.2.2
	& Implementare il riconoscimento e l'elaborazione del comando di esportazione di un singolo database\\
	\hline
	R[1.3.2.1.1][N][F] & Messaggio database da esportare non presente & UC3.2.2
	& Controllare la presenza del database richiesto e inviare al client un messaggio informativo nel caso in cui il database per cui è stata richiesta l'esportazione non fosse presente\\
	\hline
	R[1.3.2.2][N][F] & Elaborazione comando di esportazione completa & UC3.2.1
	& Implementare il riconoscimento e l'elaborazione del comando di esportazione di tutti i database\\
	\hline
	R[1.3.2.3][N][F] & Filtro database da esportare in base ai permessi utente & UC3.2
	& Implementare un filtro selettivo sui database da esportare in base ai permessi di lettura dell'utente che ha richiesto l'operazione\\
	\hline
	R[1.3.2.4][N][F] & Creazione file esportazione & UC3.2
	& Creare dei file di esportazione per i diversi database che si devono esportare sul client\\
	\hline
	R[1.3.2.5][N][F] & Invio file di esportazione al client & UC3.2
	& Inviare i file di esportazione precedentemente generati al client che li ha richiesti\\
	\hline
	R[1.3.3][N][F] & Elaborazione richiesta di importazione database & UC3.3
	& Implementare l'elaborazione di una richiesta di importazione di un database da parte di un client\\
	\hline
	R[1.3.3.1][N][F] & Elaborazione comando di importazione database & UC3.3.1
	& Implementare il riconoscimento del comando di importazione database\\
	\hline
	R[1.3.3.2][N][F] & Gestione del contenuto del file di importazione & UC3.3
	& Gestire il contenuto del file di importazione inviato dal client\\
	\hline
	R[1.3.3.2.1][N][F] & Messaggio file di importazione illeggibile & UC3.3
	& Implementare un controllo di integrità e un parse sul file di importazione inviato dal client, nel caso in cui questo controllo non andasse a buon fine annullare l'importazione e informare il client con un messaggio\\
	\hline
	R[1.3.3.2.2][N][F] & Messaggio database da importare già presente & UC3.3
	& Nel caso in cui il database da importare fosse già presente sul server, annullare l'importazione e informare il client che ha richiesto l'importazione con un messaggio\\
	\hline
	R[1.3.3.3][N][F] & Creazione database contenuto nel file di importazione & UC3.3
	& Creare un database utilizzando i dati estratti dal file di importazione\\
	\hline
	R[1.3.3.4][N][F] & Messaggio importazione avvenuta con successo & UC3.3
	& Informare il client con un messaggio del successo dell'operazione di importazione\\
	\hline
	R[1.3.4][N][F] & Elaborazione richiesta di creazione database & UC3.4
	& Implementare l'elaborazione di una richiesta di creazione di un database da parte di un client\\
	\hline
	R[1.3.4.1][N][F] & Elaborazione comando di creazione database & UC3.4.1
	& Implementare il riconoscimento del comando di creazione database\\
	\hline
	R[1.3.4.2][N][F] & Lettura nome database da creare & UC3.4.2
	& Implementare il riconoscimento del nome del database da creare\\
	\hline
	R[1.3.4.3][N][F] & Creazione di un nuovo database sul server & UC3.4
	& Creare un nuovo database sul server con il nome inviato dal client assieme al comando\\
	\hline
	R[1.3.4.4][N][F] & Messaggio creazione database avvenuta con successo & UC3.4
	& Informare un client con un messaggio informativo nel caso in cui il database richiesto fosse stato creato con successo sul server\\
	\hline
	R[1.3.4.5][N][F] & Messaggio creazione database fallita & UC3.4
	& Informare un client con un messaggio informativo nel caso in cui la creazione richiesta fosse fallita\\
	\hline
	R[1.3.5][N][F] & Elaborazione richiesta di eliminazione database & UC3.5
	& Implementare l'elaborazione di una richiesta di eliminazione di un database da parte di un client\\
	\hline
	R[1.3.5.1][N][F] & Elaborazione comando di creazione database & UC3.5.1
	& Implementare il riconoscimento del comando di eliminazione database\\
	\hline
	R[1.3.5.2][N][F] & Lettura nome database da eliminare & UC3.5.2
	& Implementare il riconoscimento del nome del database da eliminare \\
	\hline
	R[1.3.5.3][N][F] & Eliminazione del database dal server & UC3.5
	& Eliminare il database col nome inviato dal client assieme al comando di eliminazione database\\
	\hline
	R[1.3.5.4][N][F] & Messaggio eliminazione database avvenuta con successo & UC3.5
	& Informare un client con un messaggio informativo nel caso in cui il database richiesto fosse stato eliminato con successo dal server\\
	\hline
	R[1.3.5.5][N][F] & Messaggio eliminazione database fallita & UC3.4
	& Informare un client con un messaggio informativo nel caso in cui la eliminazione  richiesta fosse fallita\\
	\hline
	R[1.3.6][N][F] & Elaborazione richiesta di rinominazione database & UC3.6
	& Implementare l'elaborazione di una richiesta di rinominazione di un database da parte di un client\\
	\hline
	R[1.3.6.1][N][F] & Elaborazione comando di rinominazione database & UC3.6.1
	& Implementare il riconoscimento del comando di rinominazione database\\
	\hline
	R[1.3.6.2][N][F] & Lettura nome database da rinominare & UC3.6.2
	& Implementare il riconoscimento del nome del database da rinominare \\
	\hline
	R[1.3.6.3][N][F] & Lettura nuovo nome database da rinominare & UC3.6.3
	& Implementare il riconoscimento del nuovo nome da assegnare al database da rinominare\\
	\hline
	R[1.3.6.4][N][F] & Rinomina del database & UC3.6
	& Implementare la rinominazione del database\\
	\hline
	R[1.3.6.5][N][F] & Messaggio rinominazione database avvenuta con successo & UC3.6
	& Informare un client con un messaggio informativo nel caso in cui il database richiesto fosse stato rinominato con successo sul server\\
	\hline
	R[1.3.6.6][N][F] & Messaggio creazione database fallita & UC3.6
	& Informare un client con un messaggio informativo nel caso in cui la rinominazione richiesta fosse fallita\\
	\hline
	R[1.3.7][N][F] & Elaborazione richiesta di selezione database & UC3.7
	& Implementare l'elaborazione di una richiesta di selezione di un database da parte di un client\\
	\hline
	R[1.3.7.1][N][F] & Elaborazione comando di selezione database & UC3.7.1
	& Implementare il riconoscimento del comando di selezione database\\
	\hline
	R[1.3.7.2][N][F] & Lettura nome database da selezionare & UC3.7.2
	& Implementare il riconoscimento del nome del database da selezionare \\
	\hline
	R[1.3.7.3][N][F] & Selezione del database richiesto & UC3.7
	& Selezionare il database richiesto dal client\\
	\hline
	R[1.3.7.4][N][F] & Messaggio selezione database avvenuta con successo & UC3.7
	& Informare un client con un messaggio informativo nel caso in cui il database richiesto fosse stato selezionato con successo\\
	\hline
	R[1.3.7.5][N][F] & Messaggio selezione database fallita & UC3.7
	& Informare un client con un messaggio informativo nel caso in cui la selezione  richiesta fosse fallita\\
	\hline
	R[1.4][N][F] & Operazioni a livello database & UC4
	& Implementare le operazioni eseguibili a livello database definite in UC4\\
	\hline
	R[1.4.1][N][F] & Elaborazione richiesta visualizzazione lista mappe & UC4.1
	& Implementare l'elaborazione di una richiesta di visualizzazione della lista delle mappe presenti nel database selezionato dal client\\
	\hline
	R[1.4.1.1][N][F] & Elaborazione comando visualizzazione lista mappe & UC4.1
	& Implementare il riconoscimento del comando di visualizzazione lista mappe\\
	\hline
	R[1.4.1.2][N][F] & Invio lista mappe al client & UC4.1
	& Inviare al client la lista delle mappe che compongono il database da esso selezionato\\
	\hline
	R[1.4.2][N][F] & Elaborazione richiesta creazione mappa & UC4.2
	& Implementare l'elaborazione di una richiesta di creazione di una nuova mappa nel database selezionato dal client\\
	\hline
	R[1.4.2.1][N][F] & Elaborazione comando creazione mappa & UC4.2.1
	& Implementare il riconoscimento del comando di creazione di una nuova mappa\\
	\hline
	R[1.4.2.2][N][F] & Lettura nome mappa da creare & UC4.2.2
	& Implementare il riconoscimento del nome della mappa da creare\\
	\hline
	
	
	
	
	
	
	
R[3][N][F] & Sistema ad attori & Capitolato
	& Implementazione della logica di gestione dei dati basata sul sistema ad attori descritto dal capitolato \\
	\hline
R[3.1][N][F] & Implementazione dell'attore Storekeeper & Capitolato
	& Implementazione dell'attore Storekeeper per mantenere i dati su memoria principale \\
	\hline
		R[3.1.1][N][F] & Gestione messaggi ricevuti da Storekeeper & Capitolato
		& Implementare la gestione dei messaggi ricevuti da attori di tipo Storekeeper \\
		\hline
			R[3.1.1.1][N][F] & Gestione messaggi aggiunta item & Capitolato \newline UC5.3
			& Aggiunge una item con la chiave ed il valore presenti nel messaggio \\
			\hline
			R[3.1.1.2][N][F] & Gestione messaggi rimozione item & Capitolato \newline UC5.6
			& Rimuove la item con chiave uguale a quella presente nel messaggio\\
			\hline
			R[3.1.1.3][N][F] & Gestione messaggi aggiornamento item & Capitolato \newline UC5.4 \newline UC5.5
			& Aggiorna il valore della item con chiave uguale a quella presente nel messaggio con il valore presente nel messaggio\\
			\hline
			R[3.1.1.4][N][F] & Gestione messaggi lettura item & Capitolato \newline UC5 \newline UC5.2
			& Se presente restituisce il valore della item con chiave uguale a quella presente nel messaggio \\
			\hline
			R[3.1.1.5][O][F] & Gestione messaggi di risposta da attori Manager & Capitolato
			& Gestione da parte degli attori Storekeeper dei messaggi ricevuti da attori Manager in risposta alla richieste inviate \\
			\hline
		R[3.1.2][N][F] & Messaggi inviabili da Storekeeper & Capitolato
		& Implementazione dei messaggi che un attore Storekeeper deve poter inviare \\
		\hline
			R[3.1.2.1][N][F] & Messaggi inviabili da Storekeeper a Warehouseman & Capitolato
			& Implementazione dei messaggi che un attore Storekeeper deve poter inviare a un attore  Warehouseman \\
			\hline
				R[3.1.2.1.1][N][F] & Invio messaggi aggiunta da Storekeeper a Warehouseman & Capitolato \newline UC5.3
				& Implementazione messaggi di aggiunta item da inviare a Warehouseman \\
				\hline
				R[3.1.2.1.2][N][F] & Invio messaggi rimozione da Storekeeper a Warehouseman & Capitolato \newline UC5.6
				& Implementazione messaggi di rimozione item da inviare a Warehouseman \\
				\hline
				R[3.1.2.1.3][N][F] & Invio messaggi aggiornamento da Storekeeper a Warehouseman & Capitolato \newline UC5.4 \newline UC5.5
				& Implementazione messaggi di aggiornamento item da inviare a Warehouseman \\
				\hline
			R[3.1.2.2][N][F] & Messaggi inviabili da Storekeeper a Storefinder & Capitolato
			& Implementazione messaggi di risposta al comando da inviare a attori di tipo Storefinder \\
			\hline
			R[3.1.2.3][O][F] & Messaggi inviabili da Storekeeper a Manager & Capitolato
			& Implementazione messaggi di richiesta proprietà interne a attori di tipo Manager \\
			\hline
			R[3.1.2.4][O][F] & Messaggi inviabili da Storekeeper a Ninja & Capitolato
			& Implementazione messaggi da inviare a attori di tipo Ninja \\
			\hline
				R[3.1.2.4.1][O][F] & Invio messaggi aggiunta da Storekeeper a Ninja & Capitolato \newline UC5.3
				& Creazione messaggi di aggiunta item da inviare a Ninja \\
				\hline
				R[3.1.2.4.2][O][F] & Invio messaggi rimozione da Storekeeper a Ninja & Capitolato \newline UC.6
				& Creazione messaggi di rimozione item da inviare a Ninja \\
				\hline
				R[3.1.2.4.3][O][F] & Invio messaggi aggiornamento da Storekeeper a Ninja & Capitolato \newline UC5.4 \newline UC5.5
				& Creazione messaggi di aggiornamento item da inviare a Ninja \\
				\hline
		R[3.1.3][O][F] & Gestione memoria principale & Decisione interna
		& Implementazione gestione controllata dell'uso della memoria principale \\
		\hline
			R[3.1.3.1][O][F] & Impostazione memoria massima utilizzabile dal sistema & Decisione interna
			& Implementare la possibilità di impostare un valore massimo di memoria occupabile dal sistema \\
			\hline
			R[3.1.3.2][O][F] & Rilevazione memoria totale ed in uso & Decisione interna
			& Implementare la rilevazione della memoria totale del sistema e di quella utilizzata \\
			\hline
			R[3.1.3.3][O][F] & Valutazione svuotamento memoria & Decisione interna
			& Implementare un sistema che valuti quando è necessario svuotare la memoria interna \\
			\hline
			R[3.1.3.4][O][F] & Selezionare elementi da rimuovere dalla memoria & Decisione interna
			& Implementare la selezione degli elementi che possono essere rimossi dalla memoria \\
			\hline
			R[3.1.3.5][O][F] & Eliminare elementi selezionati	 & Decisione interna
			& Implementare l'eliminazione degli elementi selezionati dalla memoria interna \\
			\hline
			R[3.1.3.6][O][F] & Recuperare elementi da disco se necessario & Decisione interna
			& Implementare il recupero da disco degli elementi rimossi dalla memoria interna \\
			\hline
		R[3.1.4][O][F] & Creazione attori Ninja da parte di Storekeeper & Capitolato 
		& Implementare la creazione di nuovi attori Ninja da parte dell'attore Storekeeper \\
		\hline
	R[3.2][N][F] & Implementazione dell'attore Storefinder & Capitolato
	& Implementazione dell'actor Storefinder per gestire le richieste \\
	\hline
		R[3.2.1][N][F] & Implementazione dell'attore Storefinder Main & Capitolato
		& Implementazione dell'actor Main, sottotipo di Storefinder\\
		\hline		
			R[3.2.1.1][N][F] & Gestione connessioni da parte dell'attore Main & Capitolato \newline UC1
			& Implementare la connessione via rete al server \\
			\hline
			R[3.2.1.3][D][F] & Gestione database tramite l'attore Main & Decisione interna \newline UC3
			& Implementare la gestione di più database all'interno dello stesso server \\
			\hline	
			R[3.2.1.4][N][F] & Creazione attori da parte dell'attore Main & Capitolato
			& Implementare e gestire la creazione di nuovi attori da parte degli attori Main \\
			\hline
				R[3.2.1.4.1][N][F] & Creazione attori Main ad ogni nuova connessione & Capitolato \newline UC1
				& Permettere all'actor Main principale di creare nuovi Main per gestire molteplici connessione da parte di diversi client \\
				\hline
				R[3.2.1.4.2][N][F] & Creazione attori Warehouseman da parte di attori Main& Capitolato
				& Permettere all'actor Main di creare attori Warehouseman per leggere i dati presenti su disco all'avvio \\
				\hline
				R[3.2.1.4.3][N][F] & Creazione attori Storekeeper da parte di attori Main & Capitolato \newline UC4
				& Permettere all'actor Main di creare attori Storekeeper quando una nuova mappa viene creata dall'utente o quando viene letta da disco \\
				\hline
				R[3.2.1.4.4][N][F] & Creazione attori Storefinder da parte di attori Main & Capitolato
				& Permettere all'actor Main di creare attori Storefinder per gestire le richieste interne \\
				\hline
			R[3.2.1.5][O][F] & Gestione richieste esterne da parte dell'attore Main & Capitolato
			& Implementare il riconoscimento dei comandi DSL da parte dell'attore Main\\
			\hline
				R[3.2.1.5.1][N][F] & Parsing stringa comando & Capitolato
				& Implementare la gestione dei comandi inseriti dall'utente \\
				\hline	
					R[3.2.1.5.1.1][N][F] & Invio messaggi in base al comando & Capitolato
					& Inviare messaggi agli attori adatti dopo un comando utente \\
					\hline			
			R[3.2.1.6][O][F] & Implementazione messaggi tra attori Storefinder e attori Main & Capitolato
			& Gestire le comunicazioni tra attori Main e Storefinder \\
			\hline
				R[3.2.1.6.1][N][F] & Messaggi da Main a Storefinder & Capitolato
				& Creazione messaggi da inviare a Storefinder \\
				\hline
				R[3.2.1.6.2][N][F] & Messaggi da Storefinder a Main & Capitolato
				& Gestire i messaggi di risposta ricevuti da Storefinder \\
				\hline
				R[3.2.1.6.3][N][F] & Gestione numero massimo di Storefinder & Capitolato
				& Gestire le impostazioni per il numero massimo di Storefinder possibili \\
				\hline
			R[3.2.1.7][O][F] & Gestione dell'importazione di database da attore Main & Decisione interna \newline UC3.3
			& Implementare la gestione dell'importazione di database attraverso un attore di tipo Main\\
			\hline
			R[3.2.1.8][O][F] & Gestione dell'esportazione di database da attore Main & Decisione interna \newline UC3.2
			& Implementare la gestione dell'esportazione di database attraverso un attore di tipo Main \\
			\hline
		R[3.2.2][N][F] & Gestione messaggi ricevuti da attori Storefinder & Capitolato
		& Implementare la gestione dei messaggi che un attore di tipo Storefinder può ricevere\\
		\hline		
			R[3.2.2.1][N][F] & Gestione messaggi di aggiunta item da Main & Capitolato \newline UC5.3
			& Gestire i messaggi ricevuti di aggiunta item da Main \\
			\hline
			R[3.2.2.2][N][F] & Gestione messaggi di rimozione item da Main & Capitolato \newline UC5.6
			& Gestire i messaggi ricevuti di rimozione item da Main \\
			\hline
			R[3.2.2.3][N][F] & Gestione messaggi di aggiornamento item da Main & Capitolato \newline UC5.5 \newline UC5.4
			& Gestire i messaggi ricevuti di aggiornamento item da Main \\
			\hline
			R[3.2.2.4][N][F] & Gestione messaggi ricevuti da Storekeeper & Capitolato
			& Gestire i messaggi di risposta ricevuti da attori di tipo Storekeeper \\
			\hline
		R[3.2.3][N][F] & Messaggi inviabili da Storefinder & Capitolato
		& Implementazione dei messaggi inviabili da  attori di tipo Storefinder \\
		\hline		
			R[3.2.3.1][N][F] & Messaggio lettura item a Storekeeper & Capitolato
			& Creazione messaggi di lettura item da inviare a Storekeeper\\
			\hline
			R[3.2.3.2][N][F] & Messaggio aggiunta item a Storekeeper & Capitolato
			& Creazione messaggi di aggiunta item da inviare a Storekeeper\\
			\hline
			R[3.2.3.3][N][F] & Messaggio rimozione item a Storekeeper & Capitolato
			& Creazione messaggi di rimozione item da inviare a Storekeeper\\
			\hline
			R[3.2.3.4][N][F] & Messaggio aggiornamento item a Storekeeper & Capitolato
			& Creazione messaggi di aggiornamento item da inviare a Storekeeper\\
			\hline
	R[3.3][N][F] & Implementazione dell'attore Warehouseman & Capitolato
		& Creazione dell'actor Warehouseman per l'interazione con il filesystem \\
		\hline
		R[3.3.1][N][F] & Gestione file tramite Warehouseman & Capitolato
		& Implementare l'integrazione tra Warehouseman e filesystem \\
		\hline		
			R[3.3.1.1][N][F] & Creazione file tramite Warehouseman & Capitolato
			& Implementare la creazione di file su disco \\
			\hline
			R[3.3.1.2][N][F] & Rimozione file tramite Warehouseman & Capitolato
			& Implementare la rimozione di file su disco \\
			\hline
			R[3.3.1.3][N][F] & Modifica file tramite Warehouseman & Capitolato
			& Implementare la modifica di file su disco \\
			\hline
			R[3.3.1.4][N][F] & Rinomina file tramite Warehouseman & Capitolato
			& Implementare la rinomina dei file su disco \\
			\hline		
		R[3.3.2][N][F] & Gestione messaggi ricevuti da Warehouseman & Capitolato
		& Gestire i messaggi ricevuti da Warehouseman \\
		\hline		
			R[3.3.2.1][N][F] & Elaborazione messaggi aggiunta item su file & Capitolato
			& Gestire i messaggi di aggiunta item su file ricevuti da Storekeeper  \\
			\hline
			R[3.3.2.2][N][F] & Elaborazione messaggi rimozione item da file & Capitolato
			& Gestire i messaggi di rimozione item su file ricevuti da Storekeeper  \\
			\hline
			R[3.3.2.3][N][F] & Elaborazione messaggi aggiornamento item su file & Capitolato
			& Gestire i messaggi di aggiornamento item su file ricevuti da Storekeeper  \\
			\hline
			R[3.3.2.4][N][F] & Elaborazione messaggi rimozione mappa & Capitolato
			& Gestire i messaggi di rimozione mappa ricevuti da Storekeeper  \\
			\hline
			R[3.3.2.4][N][F] & Elaborazione messaggi lettura mappe da disco & Capitolato
			& Gestire i messaggi di richiesta di lettura filesystem ricevuti da Main  \\
			\hline
		R[3.3.3][N][F] & Messaggi inviabili da Warehouseman & Capitolato
		& Creazione messaggi da inviare da Warehouseman \\
		\hline		
			R[3.3.3.1][N][F] & Invio messaggi conferma a Storekeeper & Capitolato
			& Creazione messaggi di risposta da inviare da Storekeeper \\
			\hline
			R[3.3.3.2][N][F] & Invio messaggi risposta a Main & Capitolato
			& Creazione messaggi di risposta da inviare da Main \\
			\hline
	R[3.4][O][F] & Implementazione dell'attore Ninja & Capitolato
		& Creazione dell'actor Ninja per il recupero dei dati \\
		\hline
		R[3.4.1][O][F] & Gestione messaggi ricevuti da Ninja & Capitolato
		& Gestire i messaggi ricevuti da Ninja  \\
		\hline		
			R[3.4.1.1][O][F] & Gestione messaggi aggiunta item a mappa Ninja & Capitolato
			& Gestire i messaggi di aggiunta item ricevuti da Storekeeper  \\
			\hline
			R[3.4.1.2][O][F] & Gestione messaggi rimozione item da mappa Ninja & Capitolato
			& Gestire i messaggi di rimozione item ricevuti da Storekeeper  \\
			\hline
			R[3.4.1.3][O][F] & Gestione messaggi aggiornamento item su mappa Ninja & Capitolato
			& Gestire i messaggi di aggiornamento item ricevuti da Storekeeper  \\
			\hline
		R[3.4.2][O][F] & Messaggi inviabili da attori Ninja & Capitolato
		& Creazione messaggi da inviare da Ninja \\
		\hline		
			R[3.4.2.1][O][F] & Invio messaggi conferma a Storekeeper & Capitolato
			& Creazione messaggi di conferma da inviare da Storekeeper \\
			\hline
		R[3.4.3][O][F] & Trasformazione di un attore Ninja in Storekeeper & Capitolato
		& Implementare la possibilità di mutare a Storekeeper quando necessario \\			
		\hline
	R[3.5][O][F] & Imlpementazione dell'attore Manager & Capitolato
	& Creazione actor Manager per gestire le proprietà degli attori Storekeeper \\
	\hline
		R[3.5.1][O][F] & Gestione messaggi ricevuti da Manager & Capitolato
		& Gestire i messaggi ricevuti da Manager \\
		\hline		
			R[3.5.1.1][O][F] & Gestione messaggi contenenti grandezza massima mappa per Storekeeper da Main & Capitolato
			& Gestione messaggi per impostare grandezza massima della mappa interna degli attori Storekeeper \\
			\hline
		R[3.5.2][O][F] & Messaggi inviabili da attori Manager & Capitolato
		& Creazione messaggi da inviare da Manager \\
		\hline		
			R[3.5.2.1][O][F] & Messaggio contenente le proprietà a Storekeeper & Capitolato
			& Creazione messaggi di risposta contenenti le proprietà da inviare da Storekeeper \\
			\hline
		R[3.5.3][O][F] & Creazione nuovi attori Storekeeper & Capitolato
		& Implementare la creazione di nuovi Storekeeper in caso di bisogno \\			
		\hline
			R[4][N][F] & Definizione di un DSL (\emph{Domain specific language}) & Capitolato \newline UC1 \newline UC2 \newline UC3 \newline UC4 \newline UC5
			 & Definire un DSL per svolgere le operazioni sul database \\
			\hline
\bottomrule
\caption{Requisiti funzionali}
\end{longtable}   