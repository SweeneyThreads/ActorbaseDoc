%Document-Author: Bonato Paolo + Bortolazzo Matteo + Maino Elia
%Document-Date: 2016/01/13
%Document-Description: Documento di Analisi dei requisiti del gruppo SWEeneyThreads 

\documentclass[a4paper]{article}
\usepackage[english, italian]{babel}
\usepackage[T1]{fontenc}
\usepackage[utf8]{inputenc}
\usepackage{url}
\usepackage{graphicx}
\usepackage[hidelinks]{hyperref}
\usepackage{booktabs}
\usepackage{eurosym}
\usepackage{tabularx}
\usepackage{pifont}
\usepackage[table]{xcolor}
\usepackage{float}
\usepackage[]{appendix}
\usepackage{ltxtable} 
\usepackage{geometry}
\geometry{margin=1in}
\usepackage{longtable}
\usepackage{multirow}

\graphicspath{{Immagini/}}

\newcolumntype{Y}{>{\centering\arraybackslash}X}
\newcolumntype{s}{>{\hsize=.21\hsize}X}
\newcolumntype{f}{>{\hsize=.37\hsize}X}
\newcolumntype{m}{>{\hsize=.42\hsize}X}
\newcolumntype{t}{>{\hsize=.1\hsize}X}
\newcolumntype{r}{>{\hsize=.3\hsize}X}
\newcolumntype{k}{>{\hsize=.4\hsize}X}

\renewcommand{\abstractname}{Tabella contenuti}

\begin{document}
	
	\begin{titlepage}
		% Defines a new command for the horizontal lines, change thickness here
		\newcommand{\HRule}{\rule{\linewidth}{0.5mm}} 
		\center  
		
		% HEADING SECTION
		\textsc{\LARGE SweeneyThreads}\\[1.5cm] 
		\textsc{\Large Actorbase}\\[0.5cm] 
		\textsc{\large a NoSQL DB based on the Actor model}\\[0.5cm]
		
		
		% TITLE SECTION
		\HRule \\[0.4cm]
		{ \huge \bfseries Analisi dei requisiti}\\[0.4cm] 
		\HRule \\[1.5cm]
		
		% AUTHOR SECTION
		\begin{minipage}{0.4\textwidth}
			\begin{flushleft} \large
				\emph{Redattori:}\\
				Bonato Paolo \\
				Bortolazzo Matteo \\
				Maino Elia
			\end{flushleft}
		\end{minipage}
		~
		\begin{minipage}{0.4\textwidth}
			\begin{flushright} \large
				\emph{Approvazione:} \\
				Padovan Tommaso \\
				\emph{Verifica:} \\
				Biggeri Mattia \\
				Tommasin Davide 
			\end{flushright}
		\end{minipage}
		
		%immagine
		\begin{figure}[H]
			\centering
			\includegraphics[scale=0.8]{sweeney.png}
		\end{figure}
		\begin{center}
			Versione 1.2.2
		\end{center}
		% Date, change the \today to a set date if you want to be precise
		{\large \today}\\[3cm] 
		% Fill the rest of the page with whitespace
		\vfill  
	\end{titlepage}
	
	
	\tableofcontents
	
	\newpage 
	\section*{Diario delle modifiche}
		\begin{table}[H]
			\begin{tabularx}{\textwidth}{s f m X}
				\noalign{\hrule height 1.5pt}
				\rowcolor{orange!85} Versione & Data & Autore & Descrizione \\
				\noalign{\hrule height 0.5pt}
				1.3.1 & 2016-03-06 & \emph{Analista} \newline Maino Elia & Ridefinizione dei casi d'uso: stesura di diagrammi e descrizioni da UC 3.2 a UC 3.11
				\\
				\noalign{\hrule height 0.5pt}
				1.3.0 & 2016-03-05 & \emph{Analista} \newline Maino Elia & Ridefinizione dei casi d'uso: stesura di diagrammi e descrizioni fino a UC 3.1
				 \\
				\noalign{\hrule height 0.5pt}
				1.2.0 & 2016-01-21 & \emph{Responsabile} \newline Padovan Tommaso  & Approvazione documento
				 \\
				 \noalign{\hrule height 0.5pt}
				1.1.1 & 2016-01-21 & \emph{Analisti} \newline Bonato Paolo \newline Bortolazzo Matteo\newline Maino Elia  & 
				Correzioni degli errori individuati nella verifica
				 \\
				 \noalign{\hrule height 0.5pt}
				1.1.0 & 2016-01-21 & \emph{Verificatori} \newline Biggeri Mattia \newline Tommasin Davide  & Verifica del documento \\
				 \noalign{\hrule height 0.5pt}
				1.0.5 & 2016-01-20 & \emph{Analista} \newline Bortolazzo Matteo  & Stesura lista dei requisiti dedotti dal capitolato e da analisi interne
				 \\
				 \noalign{\hrule height 0.5pt}
				1.0.4 & 2016-01-19 & \emph{Analisti} \newline Bonato Paolo \newline Maino Elia & Stesura lista dei requisiti dedotti dai casi d'uso
				 \\
				\noalign{\hrule height 0.5pt}
				1.0.3 & 2016-01-18 & \emph{Analista} \newline Maino Elia & Stesura dei casi d'uso (da 4.x a 5.x) \\
				\noalign{\hrule height 0.5pt}
				1.0.2 & 2016-01-17 & \emph{Analista} \newline Maino Elia & Stesura dei casi d'uso (da 1.x a 3.x) \\
				\noalign{\hrule height 0.5pt}
				1.0.1 & 2016-01-17 & \emph{Analista} \newline Maino Elia & Stesura caratteristiche generali del prodotto \\
				\noalign{\hrule height 0.5pt}
				1.0.0 & 2016-01-17 & \emph{Analista} \newline Maino Elia & Creazione scheletro documento e stesura introduzione \\
				\noalign{\hrule height 1.5pt}
			\end{tabularx}
			\caption{Diario delle modifiche \label{tab:table_label}}
		\end{table}
	



	\newpage \section{Introduzione}
	\subsection{Scopo del documento}
		Lo scopo del seguente documento è presentare l'insieme di requisiti individuati dal gruppo 
		SWEeneyThreads durante l'analisi del \emph{Capitolato C1}, gli incontri con il committente 
		\emph{Cardin Riccardo} e l'analisi dei casi d'uso. 
		
		Si intende fornire una rappresentazione ordinata delle funzionalità che il prodotto \emph{Actorbase} 
		offrirà al momento del rilascio.
	\subsection{Scopo del prodotto}
		Lo scopo del progetto è la realizzazione di un DataBase NoSQL key-value basato sul modello ad 
		Attori con l'obiettivo di fornire una tecnologia adatta allo sviluppo di moderne 
		applicazioni che richiedono brevissimi tempi di risposta e che elaborano enormi quantità 
		di dati. Lo sviluppo porterà al rilascio del software sotto licenza MIT.
	\subsection{Glossario}
		Con lo scopo di evitare ambiguità di linguaggio e di massimizzare la comprensione dei documenti, il 
      gruppo ha steso un documento interno che è il \emph{Glossario v1.0.3}. In esso saranno definiti, in modo
      chiaro e conciso i termini che possono causare ambiguità o incomprensione del testo.
	\subsection{Riferimenti}
	\subsubsection{Informativi}	
		\begin{itemize}
			\item \textbf{Slide dell'insegnamento Ingegneria del software mod.A:} \\
			\url{http://www.math.unipd.it/~tullio/IS-1/2015/Dispense/E02.pdf}
		\end{itemize}
	\subsubsection{Normativi}
		\begin{itemize}
			\item \textbf{Norme di progetto:} \emph{Norme di progetto v1.1.2}
			\item \textbf{Capitolato d'appalto Actorbase (C1):} \\ 
			\url{http://www.math.unipd.it/~tullio/IS-1/2015/Progetto/C1p.pdf}
		\end{itemize}
		
		
	\newpage 
	\section{Caratteristiche generali del prodotto}
	\subsection{Obiettivi del prodotto}
		Il prodotto si pone l'obiettivo di fornire un database NoSQL basato sul modello ad attori che possa 
		essere utilizzato con successo dalle cosiddette \emph{Reactive Applications}, ovvero applicazioni
		orientate agli eventi, scalabili, resilienti e responsive. Per tali applicazioni il paradigma di accesso 
		ai dati basato su database di tipo relazionale risulta non applicabile, poichè troppo limitante, sono 
		dunque necessarie nuove forme di gestione dell'informazione, \emph{Actorbase} intende proporsi 
		come una valida opzione.
	\subsection{Funzioni principali del prodotto}
		\emph{Actorbase} prevede un'interazione con l'utente da riga di comando attraverso l'uso di un 
		\emph{Domain Specific Language (DSL)}. 
		\\ \\
		Una volta effettuata la connessione al server, l'utente potrà:
		\begin{itemize}
			\item Visualizzare i database presenti sul server
			\item Creare un nuovo database
			\item Selezionare un database da utilizzare
			\item Rinominare un database già presente
			\item Eliminare un database
			\item Esportare i database (backup)
		\end{itemize}
		La selezione di un database permetterà all'utente di effettuare una serie di operazioni di interrogazione e modifica di esso:
		\begin{itemize}
			\item Creazione di mappe
			\item Selezione di una mappa
			\item Rimozione di una mappa
			\item Visualizzazione delle mappe presenti
		\end{itemize}
		La selezione di una mappa permetterà all'utente di accedere alle operazioni di interrogazione e modifica a tale livello:
		\begin{itemize}
			\item Inserimento di un item
			\item Ricerca di un item per chiave
			\item Cancellazione di un item
			\item Aggiornamento di un item già presente
		\end{itemize}
	\subsection{Target d'utenza}
		Il prodotto si rivolge a sviluppatori di applicazioni moderne, che trattano enormi moli di dati (nell'
		ordine dei Petabyte), che richiedono brevissimi tempi di risposta e che necessitano di un uptime 
		del 100\%.
	\subsection{Vincoli per l'utilizzo}
		L'esecuzione di \emph{Actorbase} avviene attraverso la JVM (Java Virtual Machine), il corretto funzionamento di tutte le caratteristiche del prodotto è garantito su macchine che utilizzano la versione 8 o successive della JVM. 
	
	\newpage \section{Casi d'uso}
		Di seguito viene riportata la descrizione accurata di tutti i casi d'uso individuati dal gruppo a seguito
		delle seguenti attività:
		\begin{itemize}
			\item Analisi del capitolato C1 Actorbase
			\item Confronti con il committente \emph{Cardin Riccardo}
			\item Riunioni e discussioni interne al gruppo
			\item Analisi della struttura e delle funzionalità di altri database non realzionali
		\end{itemize}
		La struttura di un caso d'uso è definita nel documento \emph{Norme di progetto v1.1.1 sez 2.1.3}.
		\subsection{Attori}
		L'interazione degli utenti con l'applicativo avverrà tramite una CLI (Command Line Interface).
		Sono state individuate due tipologie principali di attori che interagiscono con il sistema:
		\begin{itemize}
			\item \textbf{Utente non autenticato:} è l'utente che ha avviato l'applicativo e interagisce con la CLI senza aver ancora effettuato con successo l'operazione di connessione al server;
			\item \textbf{Utente autenticato:} è l'utente che ha già effettuato con successo l'operazione di connessione ad un server che utilizza \emph{Actorbase}. 
		\end{itemize}
		\emph{Actorbase} non prevede un meccanismo di permessi per quanto riguarda i dati presenti su un server. Una volta che l'utente si è autenticato con successo ha la possibilità di effettuare tutte le operazioni disponibili. Per tale motivo non è presente un utente amministratore, in quanto l'amministrazione del server è già una funzionalità garantita all'utente autenticato.
		\subsection{Visione ad alto livello delle interazioni tramite CLI}
		 	\begin{figure}[H]
				\centering
				\includegraphics[scale=0.2]{UC/"Actorbase".png}
				\caption{Interazioni tramite CLI}
			\end{figure}
			Il diagramma in figura illustra le principali operazioni che un utente (autenticato e non) può
			 effettuare sul sistema tramite l'interfaccia da riga di comando. 
			\begin{itemize}
				\item Connessione al server
				\item Visualizzazione di un aiuto
				\item Operazioni a livello di database
				\item Operazioni a livello di mappa
				\item Operazioni a livello di riga
				\item Disconnessione dal server
			\end{itemize}
	In ogni momento un utente può inserire un comando inesistente o non utilizzabile in quel determinato momento. In tal caso nessuna azione viene eseguita dal programma e l'utente riceve un messaggio di errore esplicativo seguito dalla possibilità di inserire un nuovo comando.	 
	 
	 
	 
	 \subsection{UC 1: Connessione al server}
	 \begin{figure}[H]
				\centering
				\includegraphics[scale=0.3]{UC/"UC 1 Connessione".png}
				\caption{Diagramma di UC1: Connessione al server}
			\end{figure}
	\textbf{Descrizione} 
	\\ \\
	L'utente ha appena avviato l'interfaccia CLI dell'applicativo e intende connettersi ad un server contenente i database utilizzando il comando \texttt{CONNECT} (UC 1.1). L'utente deve inserire l'indirizzo del server (UC 1.2) e successivamente l'username e la password necessari per l'accesso (UC 1.3), nel caso in cui il tentativo di connessione dovesse fallire, l'utente riceve un messaggio di errore informativo.
	\begin{table}[H]
			\begin{tabularx}{\textwidth}{r X}
				\textbf{Codice gerarchico} & UC1 \\
				\noalign{\hrule height 0.5pt}
				\textbf{Nome sintetico} & Connessione al server \\
				\noalign{\hrule height 0.5pt}
				\textbf{Attore principale} & Utente non autenticato\\
				\noalign{\hrule height 0.5pt}
				\textbf{Attori secondari} & Nessuno \\
				\noalign{\hrule height 0.5pt}
				\textbf{Pre-condizione} & L'utente ha avviato l'interfaccia CLI, non è autenticato e intende connettersi ad un server\\
				\noalign{\hrule height 0.5pt}
				\textbf{Post-condizione} & L'utente risulta connesso al server \\
				\noalign{\hrule height 0.5pt}
				\textbf{Flusso eventi} & \begin{enumerate}
				\item L'utente inserisce il comando \texttt{CONNECT} (UC 1.1)
				\item L'utente inserisce l'indirizzo del server (UC 1.2)
				\item L'utente inserisce username e password (UC 1.3) e preme invio
				\end{enumerate} \\
				\noalign{\hrule height 0.5pt}
				\textbf{Scenari alternativi} & Nessuno \\
				\noalign{\hrule height 0.5pt}
				\textbf{Lista requisiti\newline dedotti} & \begin{itemize}
				\item ...
				\end{itemize} 
			\end{tabularx}
			\caption{Caso d'uso UC 1 - Connessione al server}
		 \end{table} 
	 
	 
	\subsection{UC 1.1: Inserimento comando di connessione}
	 \textbf{Descrizione}
	 \\ \\
	 L'utente intende effettuare una connessione al server, deve avviare la procedura di connessione inserendo il comando \texttt{CONNECT} da terminale.
	\begin{table}[H]
			\begin{tabularx}{\textwidth}{r  X}
				\textbf{Codice gerarchico} & UC1.1 \\
				\noalign{\hrule height 0.5pt}
				\textbf{Nome sintetico} & Inserimento comando di connessione\\
				\noalign{\hrule height 0.5pt}
				\textbf{Attore principale} & Utente non autenticato\\
				\noalign{\hrule height 0.5pt}
				\textbf{Attori secondari} & Nessuno \\
				\noalign{\hrule height 0.5pt}
				\textbf{Pre-condizione} & L'utente intende avviare la procedura di connessione al server\\
				\noalign{\hrule height 0.5pt}
				\textbf{Post-condizione} & L'utente ha inserito con successo il comando di connessione\\
				\noalign{\hrule height 0.5pt}
				\textbf{Flusso eventi} & \begin{enumerate}
				\item L'utente scrive sul terminale il comando di connessione
				\end{enumerate} \\
				\noalign{\hrule height 0.5pt}
				\textbf{Scenari alternativi} & Nessuno \\
				\noalign{\hrule height 0.5pt}
				\textbf{Lista requisiti\newline dedotti} & \begin{itemize}
				\item ...
				\end{itemize} 
			\end{tabularx}
			\caption{Caso d'uso UC 1.1 - Inserimento comando di connessione}
		 \end{table} 	 
	 
	 
	 \subsection{UC 1.2: Inserimento indirizzo server }
	 \textbf{Descrizione}
	 \\ \\
	 L'utente sta effettuando l'operazione di connessione al server e deve inserire l'indirizzo del server a cui vuole connettersi.
	\begin{table}[H]
			\begin{tabularx}{\textwidth}{r  X}
				\textbf{Codice gerarchico} & UC1.2 \\
				\noalign{\hrule height 0.5pt}
				\textbf{Nome sintetico} & Inserimento indirizzo server \\
				\noalign{\hrule height 0.5pt}
				\textbf{Attore principale} & Utente non autenticato\\
				\noalign{\hrule height 0.5pt}
				\textbf{Attori secondari} & Nessuno \\
				\noalign{\hrule height 0.5pt}
				\textbf{Pre-condizione} & L'utente ha inserito il comando \texttt{CONNECT}\\
				\noalign{\hrule height 0.5pt}
				\textbf{Post-condizione} & L'utente ha inserito con successo l'indirizzo del server \\
				\noalign{\hrule height 0.5pt}
				\textbf{Flusso eventi} & \begin{enumerate}
				\item L'utente scrive sul terminale l'indirizzo del server
				\end{enumerate} \\
				\noalign{\hrule height 0.5pt}
				\textbf{Scenari alternativi} & Nessuno \\
				\noalign{\hrule height 0.5pt}
				\textbf{Lista requisiti\newline dedotti} & \begin{itemize}
				\item ...
				\end{itemize} 
			\end{tabularx}
			\caption{Caso d'uso UC 1.2 - Inserimento indirizzo server}
		 \end{table} 
		 
		 
		 \subsection{UC 1.3: Inserimento username e password}
	 \textbf{Descrizione}
	 \\ \\
	 L'utente sta effettuando l'operazione di connessione al server e deve inserire il proprio username e la password di accesso al server.
	\begin{table}[H]
			\begin{tabularx}{\textwidth}{r  X}
				\textbf{Codice gerarchico} & UC1.3 \\
				\noalign{\hrule height 0.5pt}
				\textbf{Nome sintetico} & Inserimento username e password \\
				\noalign{\hrule height 0.5pt}
				\textbf{Attore principale} & Utente non autenticato\\
				\noalign{\hrule height 0.5pt}
				\textbf{Attori secondari} & Nessuno \\
				\noalign{\hrule height 0.5pt}
				\textbf{Pre-condizione} & L'utente ha inserito l'indirizzo del server\\
				\noalign{\hrule height 0.5pt}
				\textbf{Post-condizione} & L'utente ha inserito con successo username e password \\
				\noalign{\hrule height 0.5pt}
				\textbf{Flusso eventi} & \begin{enumerate}
				\item L'utente scrive su terminale il proprio username
				\item L'utente scrive su terminale la password 
				\end{enumerate} \\
				\noalign{\hrule height 0.5pt}
				\textbf{Scenari alternativi} & Nessuno \\
				\noalign{\hrule height 0.5pt}
				\textbf{Lista requisiti\newline dedotti} & \begin{itemize}
				\item ...
				\end{itemize} 
			\end{tabularx}
			\caption{Caso d'uso UC 1.3 - Inserimento username e password}
		 \end{table} 	
		 
		  
	 \subsection{UC 2: Visualizzazione aiuto}
	 \begin{figure}[H]
				\centering
				\includegraphics[scale=0.3]{UC/"UC 2 Visualizzazione aiuto".png}
				\caption{Diagramma di UC2: Visualizzazione aiuto}
			\end{figure}
	\textbf{Descrizione} 
	\\ \\
	L'utente (autenticato o non) intende ricevere un aiuto su come utilizzare i comandi di \emph{Actorbase}. Ha a disposizione due possibilità: 
	\begin{itemize}
	\item Aiuto semplice/generale
	\item Aiuto specifico
	\end{itemize}
	La prima modalità stampa sul terminale la lista completa dei comandi esistenti con relativa nota esplicativa sul loro comportamento. \\
	La modalità di aiuto specifica richiede l'inserimento del nome del comando per cui si richiedono le informazioni, stampa sul terminale la descrizione di tale comando.
	\begin{table}[H]
			\begin{tabularx}{\textwidth}{r X}
				\textbf{Codice gerarchico} & UC2 \\
				\noalign{\hrule height 0.5pt}
				\textbf{Nome sintetico} & Visualizzazione aiuto \\
				\noalign{\hrule height 0.5pt}
				\textbf{Attore principale} & Utente non autenticato e Utente autenticato\\
				\noalign{\hrule height 0.5pt}
				\textbf{Attori secondari} & Nessuno \\
				\noalign{\hrule height 0.5pt}
				\textbf{Pre-condizione} & L'utente ha avviato l'interfaccia CLI e intende visualizzare le informazioni di aiuto\\
				\noalign{\hrule height 0.5pt}
				\textbf{Post-condizione} & L'utente ha visualizzato le informazioni di aiuto richieste\\
				\noalign{\hrule height 0.5pt}
				\textbf{Flusso eventi} & \begin{enumerate}
				\item L'utente inserisce il comando relativo all'aiuto richiesto e preme invio
				\item L'utente riceve le informazioni di aiuto stampate sul terminale
				\end{enumerate} \\
				\noalign{\hrule height 0.5pt}
				\textbf{Scenari alternativi} & Nessuno \\
				\noalign{\hrule height 0.5pt}
				\textbf{Lista requisiti\newline dedotti} & \begin{itemize}
				\item ...
				\end{itemize} 
			\end{tabularx}
			\caption{Caso d'uso UC 2 - Visualizzazione aiuto}
		 \end{table} 
		 
		 
		 \subsection{UC 2.1: Aiuto semplice}
	 \textbf{Descrizione}
	 \\ \\
	 L'utente sta effettuando l'operazione di visualizzazione dell'aiuto e intende inserire il comando di aiuto semplice.
	\begin{table}[H]
			\begin{tabularx}{\textwidth}{r  X}
				\textbf{Codice gerarchico} & UC2.1 \\
				\noalign{\hrule height 0.5pt}
				\textbf{Nome sintetico} & Aiuto semplice\\
				\noalign{\hrule height 0.5pt}
				\textbf{Attore principale} & Utente non autenticato e Utente autenticato\\
				\noalign{\hrule height 0.5pt}
				\textbf{Attori secondari} & Nessuno \\
				\noalign{\hrule height 0.5pt}
				\textbf{Pre-condizione} & L'utente intende ricevere un aiuto generale\\
				\noalign{\hrule height 0.5pt}
				\textbf{Post-condizione} & L'utente ha inserito correttamente il comando di aiuto semplice/generale\\
				\noalign{\hrule height 0.5pt}
				\textbf{Flusso eventi} & \begin{enumerate}
				\item L'utente inserisce il comando \texttt{HELP}
				\end{enumerate} \\
				\noalign{\hrule height 0.5pt}
				\textbf{Scenari alternativi} & Nessuno \\
				\noalign{\hrule height 0.5pt}
				\textbf{Lista requisiti\newline dedotti} & \begin{itemize}
				\item ...
				\end{itemize} 
			\end{tabularx}
			\caption{Caso d'uso UC 2.1 - Aiuto semplice}
		 \end{table} 	
	 
	 
	 \subsection{UC 2.2: Aiuto specifico}
	 \textbf{Descrizione}
	 \\ \\
	 L'utente sta effettuando l'operazione di visualizzazione dell'aiuto e intende inserire il comando di aiuto specifico.
	\begin{table}[H]
			\begin{tabularx}{\textwidth}{r  X}
				\textbf{Codice gerarchico} & UC2.2 \\
				\noalign{\hrule height 0.5pt}
				\textbf{Nome sintetico} & Aiuto specifico \\
				\noalign{\hrule height 0.5pt}
				\textbf{Attore principale} & Utente non autenticato e Utente autenticato\\
				\noalign{\hrule height 0.5pt}
				\textbf{Attori secondari} & Nessuno \\
				\noalign{\hrule height 0.5pt}
				\textbf{Pre-condizione} & L'utente intende ricevere un aiuto specifico\\
				\noalign{\hrule height 0.5pt}
				\textbf{Post-condizione} & L'utente ha inserito correttamente il comando di aiuto specifico\\
				\noalign{\hrule height 0.5pt}
				\textbf{Flusso eventi} & \begin{enumerate}
				\item L'utente inserisce il comando \texttt{HELP} seguito dal nome del comando per cui vuole ricevere l'aiuto
				\end{enumerate} \\
				\noalign{\hrule height 0.5pt}
				\textbf{Scenari alternativi} & Nessuno \\
				\noalign{\hrule height 0.5pt}
				\textbf{Lista requisiti\newline dedotti} & \begin{itemize}
				\item ...
				\end{itemize} 
			\end{tabularx}
			\caption{Caso d'uso UC 2.2 - Aiuto specifico}
		 \end{table} 	
	 
	 
	 \subsection{UC 3: Operazioni sui database}
	 \begin{figure}[H]
				\centering
				\includegraphics[scale=0.2]{UC/"UC 3 Operazioni sui database".png}
				\caption{Diagramma di UC3: Operazioni sui database}
			\end{figure}
	\textbf{Descrizione} 
	\\ \\
	L'utente ha effettuato correttamente la connessione, ha dunque accesso ai database presenti sul server. Su tali database può eseguire le seguenti operazioni:
	\begin{itemize}
	\item Visualizzare la lista dei database presenti
	\item Esportare uno o più database (backup)
	\item Importare un database
	\item Creare un database
	\item Eliminare un database
	\item Rinominare un database presente
	\item Selezionare un database per effettuare operazioni a livello di mappa su di esso
	\end{itemize}
	Queste operazioni possono portare a diverse situazioni di errore. In tal caso non viene eseguita alcuna operazione e l'utente riceve un messaggio di errore esplicativo.
	\begin{table}[H]
			\begin{tabularx}{\textwidth}{r X}
				\textbf{Codice gerarchico} & UC3 \\
				\noalign{\hrule height 0.5pt}
				\textbf{Nome sintetico} & Operazioni sui database\\
				\noalign{\hrule height 0.5pt}
				\textbf{Attore principale} & Utente autenticato\\
				\noalign{\hrule height 0.5pt}
				\textbf{Attori secondari} & Nessuno \\
				\noalign{\hrule height 0.5pt}
				\textbf{Pre-condizione} & L'utente si è connesso con successo ad un server\\
				\noalign{\hrule height 0.5pt}
				\textbf{Post-condizione} & L'operazione sui database selezionata è stata eseguita con successo\\
				\noalign{\hrule height 0.5pt}
				\textbf{Flusso eventi} & \begin{enumerate}
				\item L'utente inserisce il comando e le informazioni necessarie per l'operazione richiesta e preme invio
				\end{enumerate} \\
				\noalign{\hrule height 0.5pt}
				\textbf{Scenari alternativi} & \begin{enumerate}
				\item L'utente ha richiesto un'operazione di esportazione e questa è fallita, l'operazione non viene eseguita e l'utente riceve un messaggio di errore (UC 3.8)
				\item L'utente ha richiesto un'operazione di importazione e questa è fallita, l'operazione non viene eseguita e l'utente riceve un messaggio di errore (UC 3.9)
				\item L'utente ha richiesto la creazione di un nuovo database sul server, la creazione è fallita e l'utente riceve un messaggio di errore (UC 3.10)
				\item L'utente ha tentato di accedere ad un database non presente sul server, l'accesso è fallito e l'utente riceve un messaggio di errore (UC 3.11)
\end{enumerate}				 \\
				\noalign{\hrule height 0.5pt}
				\textbf{Lista requisiti\newline dedotti} & \begin{itemize}
				\item ...
				\end{itemize} 
			\end{tabularx}
			\caption{Caso d'uso UC 3 - Operazioni sui database}
		 \end{table} 
	 
	 
	\subsection{UC 3.1: Visualizzazione lista dei database}
	\textbf{Descrizione} 
	\\ \\
	L'utente intende visualizzare la lista dei database presenti sul server a cui è connesso. Inserisce quindi il comando \texttt{SHOWDB} e il sistema stampa sul terminale la lista.
	\begin{table}[H]
			\begin{tabularx}{\textwidth}{r X}
				\textbf{Codice gerarchico} & UC3.1 \\
				\noalign{\hrule height 0.5pt}
				\textbf{Nome sintetico} & Visualizzazione lista dei database\\
				\noalign{\hrule height 0.5pt}
				\textbf{Attore principale} & Utente autenticato\\
				\noalign{\hrule height 0.5pt}
				\textbf{Attori secondari} & Nessuno \\
				\noalign{\hrule height 0.5pt}
				\textbf{Pre-condizione} & L'utente intende visualizzare la lista dei database presenti\\
				\noalign{\hrule height 0.5pt}
				\textbf{Post-condizione} & L'utente riceve la lista dei database presenti stampata a video\\
				\noalign{\hrule height 0.5pt}
				\textbf{Flusso eventi} & \begin{enumerate}
				\item L'utente inserisce il comando \texttt{SHOWDB} e preme invio
				\end{enumerate} \\
				\noalign{\hrule height 0.5pt}
				\textbf{Scenari alternativi} & Nessuno \\
				\noalign{\hrule height 0.5pt}
				\textbf{Lista requisiti\newline dedotti} & \begin{itemize}
				\item ...
				\end{itemize} 
			\end{tabularx}
			\caption{Caso d'uso UC 3.1 - Visualizzazione lista dei database}
		 \end{table} 	 
	 
	 
	 \subsection{UC 3.2: Esportazione di database}
	 \begin{figure}[H]
				\centering
				\includegraphics[scale=0.25]{UC/"UC 3-2 Esportazione di database".png}
				\caption{Diagramma di UC3.2: Esportazione di database}
			\end{figure}
	\textbf{Descrizione} 
	\\ \\
	L'utente intende effettuare l'esportazione di tutti i database presenti o di un database specifico, utilizzando il comando \texttt{EXPORT} seguito dal nome del database da esportare (opzionale) e dal percorso locale in cui si vuole esportare il database.
	\begin{table}[H]
			\begin{tabularx}{\textwidth}{r X}
				\textbf{Codice gerarchico} & UC3.2 \\
				\noalign{\hrule height 0.5pt}
				\textbf{Nome sintetico} & Esportazione di database\\
				\noalign{\hrule height 0.5pt}
				\textbf{Attore principale} & Utente autenticato\\
				\noalign{\hrule height 0.5pt}
				\textbf{Attori secondari} & Nessuno \\
				\noalign{\hrule height 0.5pt}
				\textbf{Pre-condizione} & L'utente intende esportare dei database\\
				\noalign{\hrule height 0.5pt}
				\textbf{Post-condizione} & L'utente ha esportato con successo i database selezionati\\
				\noalign{\hrule height 0.5pt}
				\textbf{Flusso eventi} & \begin{enumerate}
				\item L'utente inserisce il comando \texttt{EXPORT} (UC 3.2.1)
				\item L'utente inserisce le informazioni di esportazione (UC 3.2.2 o UC 3.2.3) e preme invio
				\end{enumerate} \\
				\noalign{\hrule height 0.5pt}
				\textbf{Scenari alternativi} & Nessuno \\
				\noalign{\hrule height 0.5pt}
				\textbf{Lista requisiti\newline dedotti} & \begin{itemize}
				\item ...
				\end{itemize} 
			\end{tabularx}
			\caption{Caso d'uso UC 3.2 - Esportazione di database}
		 \end{table}
		 
		 
		  \subsection{UC 3.2.1: Inserimento comando di esportazione}
	\textbf{Descrizione} 
	\\ \\
	L'utente intende effettuare un'esportazione di database. Per avviare la procedura di esportazione deve inserire il relativo comando \texttt{EXPORT}.
	\begin{table}[H]
			\begin{tabularx}{\textwidth}{r X}
				\textbf{Codice gerarchico} & UC3.2.1 \\
				\noalign{\hrule height 0.5pt}
				\textbf{Nome sintetico} & Inserimento comando di esportazione\\
				\noalign{\hrule height 0.5pt}
				\textbf{Attore principale} & Utente autenticato\\
				\noalign{\hrule height 0.5pt}
				\textbf{Attori secondari} & Nessuno \\
				\noalign{\hrule height 0.5pt}
				\textbf{Pre-condizione} & L'utente intende esportare dei database\\
				\noalign{\hrule height 0.5pt}
				\textbf{Post-condizione} & L'utente ha inserito con successo il comando di esportazione\\
				\noalign{\hrule height 0.5pt}
				\textbf{Flusso eventi} & \begin{enumerate}
				\item L'utente scrive su terminale il comando \texttt{EXPORT}
				\end{enumerate} \\
				\noalign{\hrule height 0.5pt}
				\textbf{Scenari alternativi} & Nessuno \\
				\noalign{\hrule height 0.5pt}
				\textbf{Lista requisiti\newline dedotti} & \begin{itemize}
				\item ...
				\end{itemize} 
			\end{tabularx}
			\caption{Caso d'uso UC 3.2.1 - Inserimento comando di esportazione}
		 \end{table} 	 
		 
		 \subsection{UC 3.2.2: Esportazione di tutti i database}
	\textbf{Descrizione} 
	\\ \\
	L'utente intende effettuare un'esportazione di tutti i database. Ha inserito il comando di connessione e ora deve inserire il percorso locale in cui vuole esportare i database.
	\begin{table}[H]
			\begin{tabularx}{\textwidth}{r X}
				\textbf{Codice gerarchico} & UC3.2.2 \\
				\noalign{\hrule height 0.5pt}
				\textbf{Nome sintetico} & Esportazione di tutti i database\\
				\noalign{\hrule height 0.5pt}
				\textbf{Attore principale} & Utente autenticato\\
				\noalign{\hrule height 0.5pt}
				\textbf{Attori secondari} & Nessuno \\
				\noalign{\hrule height 0.5pt}
				\textbf{Pre-condizione} & L'utente intende esportare tutti i database e ha inserito il comando di esportazione\\
				\noalign{\hrule height 0.5pt}
				\textbf{Post-condizione} & L'utente ha inserito correttamente i dati per esportare tutti i database\\
				\noalign{\hrule height 0.5pt}
				\textbf{Flusso eventi} & \begin{enumerate}
				\item L'utente scrive su terminale il percorso locale in cui vuole esportare i database 
				\end{enumerate} \\
				\noalign{\hrule height 0.5pt}
				\textbf{Scenari alternativi} & Nessuno \\
				\noalign{\hrule height 0.5pt}
				\textbf{Lista requisiti\newline dedotti} & \begin{itemize}
				\item ...
				\end{itemize} 
			\end{tabularx}
			\caption{Caso d'uso UC 3.2.2 - Esportazione di tutti i database}
		 \end{table} 
		 
		 \subsection{UC 3.2.3: Esportazione di un singolo database}
	\textbf{Descrizione} 
	\\ \\
	L'utente intende effettuare un'esportazione di un singolo database. Ha inserito il comando di connessione e ora deve inserire il nome del database da esportare seguito dal percorso locale in cui vuole esportare i database.
	\begin{table}[H]
			\begin{tabularx}{\textwidth}{r X}
				\textbf{Codice gerarchico} & UC3.2.3 \\
				\noalign{\hrule height 0.5pt}
				\textbf{Nome sintetico} & Esportazione di un singolo database\\
				\noalign{\hrule height 0.5pt}
				\textbf{Attore principale} & Utente autenticato\\
				\noalign{\hrule height 0.5pt}
				\textbf{Attori secondari} & Nessuno \\
				\noalign{\hrule height 0.5pt}
				\textbf{Pre-condizione} & L'utente intende esportare un singolo database e ha inserito il comando di esportazione\\
				\noalign{\hrule height 0.5pt}
				\textbf{Post-condizione} & L'utente ha inserito correttamente i dati per esportare un singolo database\\
				\noalign{\hrule height 0.5pt}
				\textbf{Flusso eventi} & \begin{enumerate}
				\item L'utente scrive su terminale il nome del database che vuole esportare
				\item L'utente scrive su terminale il percorso locale in cui vuole esportare il database 
				\end{enumerate} \\
				\noalign{\hrule height 0.5pt}
				\textbf{Scenari alternativi} & Nessuno \\
				\noalign{\hrule height 0.5pt}
				\textbf{Lista requisiti\newline dedotti} & \begin{itemize}
				\item ...
				\end{itemize} 
			\end{tabularx}
			\caption{Caso d'uso UC 3.2.3 - Esportazione di un singolo database}
		 \end{table} 		 	
	 
	 
	 
	 \subsection{UC 3.3: Importazione di database}
	 \begin{figure}[H]
				\centering
				\includegraphics[scale=0.25]{UC/"UC 3-3 Importazione di database".png}
				\caption{Diagramma di UC3.3: Importazione di database}
			\end{figure}
	\textbf{Descrizione} 
	\\ \\
	L'utente intende effettuare l'importazione di un database utilizzando il comando \texttt{IMPORT} seguito dal percorso locale da cui si vuole importare il database.
	\begin{table}[H]
			\begin{tabularx}{\textwidth}{r X}
				\textbf{Codice gerarchico} & UC3.3 \\
				\noalign{\hrule height 0.5pt}
				\textbf{Nome sintetico} & Importazione di database\\
				\noalign{\hrule height 0.5pt}
				\textbf{Attore principale} & Utente autenticato\\
				\noalign{\hrule height 0.5pt}
				\textbf{Attori secondari} & Nessuno \\
				\noalign{\hrule height 0.5pt}
				\textbf{Pre-condizione} & L'utente intende importare un database\\
				\noalign{\hrule height 0.5pt}
				\textbf{Post-condizione} & L'utente ha importato con successo il database selezionato\\
				\noalign{\hrule height 0.5pt}
				\textbf{Flusso eventi} & \begin{enumerate}
				\item L'utente inserisce il comando \texttt{IMPORT} (UC 3.3.1)
				\item L'utente inserisce il percorso di importazione (UC 3.3.2) e preme invio
				\end{enumerate} \\
				\noalign{\hrule height 0.5pt}
				\textbf{Scenari alternativi} & Nessuno \\
				\noalign{\hrule height 0.5pt}
				\textbf{Lista requisiti\newline dedotti} & \begin{itemize}
				\item ...
				\end{itemize} 
			\end{tabularx}
			\caption{Caso d'uso UC 3.3 - Importazione di database}
		 \end{table}
	 
	 
	 \subsection{UC 3.3.1: Inserimento comando di importazione}
	\textbf{Descrizione} 
	\\ \\
	L'utente intende effettuare l'esportazione di un database. Per avviare la procedura di importazione deve inserire il relativo comando \texttt{IMPORT}.
	\begin{table}[H]
			\begin{tabularx}{\textwidth}{r X}
				\textbf{Codice gerarchico} & UC3.3.1 \\
				\noalign{\hrule height 0.5pt}
				\textbf{Nome sintetico} & Inserimento comando di importazione\\
				\noalign{\hrule height 0.5pt}
				\textbf{Attore principale} & Utente autenticato\\
				\noalign{\hrule height 0.5pt}
				\textbf{Attori secondari} & Nessuno \\
				\noalign{\hrule height 0.5pt}
				\textbf{Pre-condizione} & L'utente intende importare un database\\
				\noalign{\hrule height 0.5pt}
				\textbf{Post-condizione} & L'utente ha inserito con successo il comando di importazione\\
				\noalign{\hrule height 0.5pt}
				\textbf{Flusso eventi} & \begin{enumerate}
				\item L'utente scrive su terminale il comando \texttt{IMPORT}
				\end{enumerate} \\
				\noalign{\hrule height 0.5pt}
				\textbf{Scenari alternativi} & Nessuno \\
				\noalign{\hrule height 0.5pt}
				\textbf{Lista requisiti\newline dedotti} & \begin{itemize}
				\item ...
				\end{itemize} 
			\end{tabularx}
			\caption{Caso d'uso UC 3.3.1 - Inserimento comando di importazione}
		 \end{table} 	 
		 
		 \subsection{UC 3.3.2: Inserimento percorso di importazione}
	\textbf{Descrizione} 
	\\ \\
	L'utente intende effettuare l'esportazione di un database. Ha inserito il comando di importazione e ora deve inserire il percorso da cui importare i dati.
	\begin{table}[H]
			\begin{tabularx}{\textwidth}{r X}
				\textbf{Codice gerarchico} & UC3.3.2 \\
				\noalign{\hrule height 0.5pt}
				\textbf{Nome sintetico} & Inserimento percorso di importazione\\
				\noalign{\hrule height 0.5pt}
				\textbf{Attore principale} & Utente autenticato\\
				\noalign{\hrule height 0.5pt}
				\textbf{Attori secondari} & Nessuno \\
				\noalign{\hrule height 0.5pt}
				\textbf{Pre-condizione} & L'utente ha inserito il comando di importazione\\
				\noalign{\hrule height 0.5pt}
				\textbf{Post-condizione} & L'utente ha inserito con successo il percorso di importazione\\
				\noalign{\hrule height 0.5pt}
				\textbf{Flusso eventi} & \begin{enumerate}
				\item L'utente scrive su terminale il percorso di importazione
				\end{enumerate} \\
				\noalign{\hrule height 0.5pt}
				\textbf{Scenari alternativi} & Nessuno \\
				\noalign{\hrule height 0.5pt}
				\textbf{Lista requisiti\newline dedotti} & \begin{itemize}
				\item ...
				\end{itemize} 
			\end{tabularx}
			\caption{Caso d'uso UC 3.3.2 - Inserimento percorso di importazione}
		 \end{table} 	 
		 
		 
		 \subsection{UC 3.4: Creazione database}
	 \begin{figure}[H]
				\centering
				\includegraphics[scale=0.25]{UC/"UC 3-4 Creazione database".png}
				\caption{Diagramma di UC3.4: Creazione database}
			\end{figure}
	\textbf{Descrizione} 
	\\ \\
	L'utente intende creare un nuovo database inserendo il comando di creazione seguito dal nome del database da creare.
	\begin{table}[H]
			\begin{tabularx}{\textwidth}{r X}
				\textbf{Codice gerarchico} & UC3.4 \\
				\noalign{\hrule height 0.5pt}
				\textbf{Nome sintetico} & Creazione database\\
				\noalign{\hrule height 0.5pt}
				\textbf{Attore principale} & Utente autenticato\\
				\noalign{\hrule height 0.5pt}
				\textbf{Attori secondari} & Nessuno \\
				\noalign{\hrule height 0.5pt}
				\textbf{Pre-condizione} & L'utente intende creare un database\\
				\noalign{\hrule height 0.5pt}
				\textbf{Post-condizione} & L'utente ha creato con successo un database con il nome inserito\\
				\noalign{\hrule height 0.5pt}
				\textbf{Flusso eventi} & \begin{enumerate}
				\item L'utente inserisce il comando di creazione database (UC 3.4.1)
				\item L'utente inserisce il nome del database da creare (UC 3.4.2) e preme invio
				\end{enumerate} \\
				\noalign{\hrule height 0.5pt}
				\textbf{Scenari alternativi} & Nessuno \\
				\noalign{\hrule height 0.5pt}
				\textbf{Lista requisiti\newline dedotti} & \begin{itemize}
				\item ...
				\end{itemize} 
			\end{tabularx}
			\caption{Caso d'uso UC 3.4 - Creazione database}
		 \end{table}
		 
		 
		  \subsection{UC 3.4.1: Inserimento comando di creazione database}
	\textbf{Descrizione} 
	\\ \\
	L'utente intende creare un nuovo database, a tal fine deve come prima cosa inserire il comando di creazione di database.
	\begin{table}[H]
			\begin{tabularx}{\textwidth}{r X}
				\textbf{Codice gerarchico} & UC3.4.1 \\
				\noalign{\hrule height 0.5pt}
				\textbf{Nome sintetico} & Inserimento comando di creazione database\\
				\noalign{\hrule height 0.5pt}
				\textbf{Attore principale} & Utente autenticato\\
				\noalign{\hrule height 0.5pt}
				\textbf{Attori secondari} & Nessuno \\
				\noalign{\hrule height 0.5pt}
				\textbf{Pre-condizione} & L'utente intende creare un database\\
				\noalign{\hrule height 0.5pt}
				\textbf{Post-condizione} & L'utente ha inserito con successo il comando di creazione database\\
				\noalign{\hrule height 0.5pt}
				\textbf{Flusso eventi} & \begin{enumerate}
				\item L'utente scrive su terminale il comando di creazione database
				\end{enumerate} \\
				\noalign{\hrule height 0.5pt}
				\textbf{Scenari alternativi} & Nessuno \\
				\noalign{\hrule height 0.5pt}
				\textbf{Lista requisiti\newline dedotti} & \begin{itemize}
				\item ...
				\end{itemize} 
			\end{tabularx}
			\caption{Caso d'uso UC 3.4.1 - Inserimento comando di creazione database}
		 \end{table}
		 
		 \subsection{UC 3.4.2: Inserimento nome database da creare}
	\textbf{Descrizione} 
	\\ \\
	L'utente intende creare un nuovo database, ha inserito il comando di creazione database e ora deve inserire il nome che intende dare al nuovo database.
	\begin{table}[H]
			\begin{tabularx}{\textwidth}{r X}
				\textbf{Codice gerarchico} & UC3.4.2 \\
				\noalign{\hrule height 0.5pt}
				\textbf{Nome sintetico} & Inserimento nome database da creare\\
				\noalign{\hrule height 0.5pt}
				\textbf{Attore principale} & Utente autenticato\\
				\noalign{\hrule height 0.5pt}
				\textbf{Attori secondari} & Nessuno \\
				\noalign{\hrule height 0.5pt}
				\textbf{Pre-condizione} & L'utente ha inserito il comando di creazione database\\
				\noalign{\hrule height 0.5pt}
				\textbf{Post-condizione} & L'utente ha inserito con successo il nome del database da creare\\
				\noalign{\hrule height 0.5pt}
				\textbf{Flusso eventi} & \begin{enumerate}
				\item L'utente scrive su terminale il nome del database da creare
				\end{enumerate} \\
				\noalign{\hrule height 0.5pt}
				\textbf{Scenari alternativi} & Nessuno \\
				\noalign{\hrule height 0.5pt}
				\textbf{Lista requisiti\newline dedotti} & \begin{itemize}
				\item ...
				\end{itemize} 
			\end{tabularx}
			\caption{Caso d'uso UC 3.4.2 - Inserimento nome database da creare}
		 \end{table}
		 
		 
		 \subsection{UC 3.5: Eliminazione database}
	 \begin{figure}[H]
				\centering
				\includegraphics[scale=0.25]{UC/"UC 3-5 Eliminazione database".png}
				\caption{Diagramma di UC3.5: Eliminazione database}
			\end{figure}
	\textbf{Descrizione} 
	\\ \\
	L'utente intende eliminare un database presente sul server inserendo il comando per eliminare un database, seguito dal nome del database da eliminare.
	\begin{table}[H]
			\begin{tabularx}{\textwidth}{r X}
				\textbf{Codice gerarchico} & UC3.5 \\
				\noalign{\hrule height 0.5pt}
				\textbf{Nome sintetico} & Eliminazione database\\
				\noalign{\hrule height 0.5pt}
				\textbf{Attore principale} & Utente autenticato\\
				\noalign{\hrule height 0.5pt}
				\textbf{Attori secondari} & Nessuno \\
				\noalign{\hrule height 0.5pt}
				\textbf{Pre-condizione} & L'utente intende eliminare un database\\
				\noalign{\hrule height 0.5pt}
				\textbf{Post-condizione} & L'utente ha eliminato con successo il database selezionato\\
				\noalign{\hrule height 0.5pt}
				\textbf{Flusso eventi} & \begin{enumerate}
				\item L'utente inserisce il comando di eliminazione database (UC 3.5.1)
				\item L'utente inserisce il nome del database da eliminare (UC 3.5.2) e preme invio
				\end{enumerate} \\
				\noalign{\hrule height 0.5pt}
				\textbf{Scenari alternativi} & Nessuno \\
				\noalign{\hrule height 0.5pt}
				\textbf{Lista requisiti\newline dedotti} & \begin{itemize}
				\item ...
				\end{itemize} 
			\end{tabularx}
			\caption{Caso d'uso UC 3.5 - Eliminazione database}
		 \end{table}
		 
		 
		\subsection{UC 3.5.1: Inserimento comando di eliminazione database}
	\textbf{Descrizione} 
	\\ \\
	L'utente intende eliminare un database presente sul server, a tal fine deve per prima cosa scrivere su terminale il comando di eliminazione database.
	\begin{table}[H]
			\begin{tabularx}{\textwidth}{r X}
				\textbf{Codice gerarchico} & UC3.5.1 \\
				\noalign{\hrule height 0.5pt}
				\textbf{Nome sintetico} & Inserimento comando di eliminazione database\\
				\noalign{\hrule height 0.5pt}
				\textbf{Attore principale} & Utente autenticato\\
				\noalign{\hrule height 0.5pt}
				\textbf{Attori secondari} & Nessuno \\
				\noalign{\hrule height 0.5pt}
				\textbf{Pre-condizione} & L'utente intende eliminare un database\\
				\noalign{\hrule height 0.5pt}
				\textbf{Post-condizione} & L'utente ha inserito con successo il comando di eliminazione database\\
				\noalign{\hrule height 0.5pt}
				\textbf{Flusso eventi} & \begin{enumerate}
				\item L'utente scrive su terminale il comando di eliminazione database
				\end{enumerate} \\
				\noalign{\hrule height 0.5pt}
				\textbf{Scenari alternativi} & Nessuno \\
				\noalign{\hrule height 0.5pt}
				\textbf{Lista requisiti\newline dedotti} & \begin{itemize}
				\item ...
				\end{itemize} 
			\end{tabularx}
			\caption{Caso d'uso UC 3.5.1 - Inserimento comando di eliminazione database}
		 \end{table}		 
		  
		\subsection{UC 3.5.2: Inserimento nome database da eliminare}
	\textbf{Descrizione} 
	\\ \\
	L'utente intende eliminare un database presente sul server, ha inserito il comando di eliminazione database e ora deve inserire il nome del database da eliminare.
	\begin{table}[H]
			\begin{tabularx}{\textwidth}{r X}
				\textbf{Codice gerarchico} & UC3.5.2 \\
				\noalign{\hrule height 0.5pt}
				\textbf{Nome sintetico} & Inserimento nome database da eliminare\\
				\noalign{\hrule height 0.5pt}
				\textbf{Attore principale} & Utente autenticato\\
				\noalign{\hrule height 0.5pt}
				\textbf{Attori secondari} & Nessuno \\
				\noalign{\hrule height 0.5pt}
				\textbf{Pre-condizione} & L'utente intende eliminare un database e ha inserito il comando di eliminazione database\\
				\noalign{\hrule height 0.5pt}
				\textbf{Post-condizione} & L'utente ha inserito con successo il nome del database da eliminare\\
				\noalign{\hrule height 0.5pt}
				\textbf{Flusso eventi} & \begin{enumerate}
				\item L'utente scrive su terminale il nome del database da eliminare
				\end{enumerate} \\
				\noalign{\hrule height 0.5pt}
				\textbf{Scenari alternativi} & Nessuno \\
				\noalign{\hrule height 0.5pt}
				\textbf{Lista requisiti\newline dedotti} & \begin{itemize}
				\item ...
				\end{itemize} 
			\end{tabularx}
			\caption{Caso d'uso UC 3.5.2 - Inserimento nome database da eliminare}
		 \end{table}		 
		 
		\subsection{UC 3.6: Rinominazione database}
	 \begin{figure}[H]
				\centering
				\includegraphics[scale=0.25]{UC/"UC 3-6 Rinominazione database".png}
				\caption{Diagramma di UC3.6: Rinominazione database}
			\end{figure}
	\textbf{Descrizione} 
	\\ \\
	L'utente intende rinominare un database presente sul server inserendo il comando per rinominare un database, seguito dal nome del database da rinominare e il nuovo nome.
	\begin{table}[H]
			\begin{tabularx}{\textwidth}{r X}
				\textbf{Codice gerarchico} & UC3.6 \\
				\noalign{\hrule height 0.5pt}
				\textbf{Nome sintetico} & Rinominazione database\\
				\noalign{\hrule height 0.5pt}
				\textbf{Attore principale} & Utente autenticato\\
				\noalign{\hrule height 0.5pt}
				\textbf{Attori secondari} & Nessuno \\
				\noalign{\hrule height 0.5pt}
				\textbf{Pre-condizione} & L'utente intende rinominare un database\\
				\noalign{\hrule height 0.5pt}
				\textbf{Post-condizione} & L'utente ha rinominato con successo il database selezionato\\
				\noalign{\hrule height 0.5pt}
				\textbf{Flusso eventi} & \begin{enumerate}
				\item L'utente inserisce il comando di rinominazione database (UC 3.6.1)
				\item L'utente inserisce il nome del database da rinominare (UC 3.6.2)
				\item L'utente inserisce il nuovo nome per il database (UC 3.6.3) e preme invio
				\end{enumerate} \\
				\noalign{\hrule height 0.5pt}
				\textbf{Scenari alternativi} & Nessuno \\
				\noalign{\hrule height 0.5pt}
				\textbf{Lista requisiti\newline dedotti} & \begin{itemize}
				\item ...
				\end{itemize} 
			\end{tabularx}
			\caption{Caso d'uso UC 3.6 - Rinominazione database}
		 \end{table}		 
		 
		 \subsection{UC 3.6.1: Inserimento comando di rinominazione database}
	\textbf{Descrizione} 
	\\ \\
	L'utente intende rinominare un database presente sul server, a tal fine deve per prima cosa inserire il comando di rinominazione database.
	\begin{table}[H]
			\begin{tabularx}{\textwidth}{r X}
				\textbf{Codice gerarchico} & UC3.6.1 \\
				\noalign{\hrule height 0.5pt}
				\textbf{Nome sintetico} & Inserimento comando di rinominazione database\\
				\noalign{\hrule height 0.5pt}
				\textbf{Attore principale} & Utente autenticato\\
				\noalign{\hrule height 0.5pt}
				\textbf{Attori secondari} & Nessuno \\
				\noalign{\hrule height 0.5pt}
				\textbf{Pre-condizione} & L'utente intende rinominare un database\\
				\noalign{\hrule height 0.5pt}
				\textbf{Post-condizione} & L'utente ha inserito con successo il comando di rinominazione database\\
				\noalign{\hrule height 0.5pt}
				\textbf{Flusso eventi} & \begin{enumerate}
				\item L'utente scrive su terminale il comando di rinominazione database
				\end{enumerate} \\
				\noalign{\hrule height 0.5pt}
				\textbf{Scenari alternativi} & Nessuno \\
				\noalign{\hrule height 0.5pt}
				\textbf{Lista requisiti\newline dedotti} & \begin{itemize}
				\item ...
				\end{itemize} 
			\end{tabularx}
			\caption{Caso d'uso UC 3.6.1 - Inserimento comando di rinominazione database}
		 \end{table}		 
		 
		 \subsection{UC 3.6.2: Inserimento nome database da rinominare}
	\textbf{Descrizione} 
	\\ \\
	L'utente intende rinominare un database presente sul server, ha inserito il comando per rinominare un database, ora deve inserire il nome del database che vuole rinominare.
	\begin{table}[H]
			\begin{tabularx}{\textwidth}{r X}
				\textbf{Codice gerarchico} & UC3.6.2 \\
				\noalign{\hrule height 0.5pt}
				\textbf{Nome sintetico} & Inserimento nome database da rinominare\\
				\noalign{\hrule height 0.5pt}
				\textbf{Attore principale} & Utente autenticato\\
				\noalign{\hrule height 0.5pt}
				\textbf{Attori secondari} & Nessuno \\
				\noalign{\hrule height 0.5pt}
				\textbf{Pre-condizione} & L'utente intende rinominare un database e ha inserito il comando di rinominazione database\\
				\noalign{\hrule height 0.5pt}
				\textbf{Post-condizione} & L'utente ha inserito con successo il nome del database da rinominare\\
				\noalign{\hrule height 0.5pt}
				\textbf{Flusso eventi} & \begin{enumerate}
				\item L'utente scrive su terminale il nome del database da rinominare
				\end{enumerate} \\
				\noalign{\hrule height 0.5pt}
				\textbf{Scenari alternativi} & Nessuno \\
				\noalign{\hrule height 0.5pt}
				\textbf{Lista requisiti\newline dedotti} & \begin{itemize}
				\item ...
				\end{itemize} 
			\end{tabularx}
			\caption{Caso d'uso UC 3.6.2 - Inserimento nome database da rinominare}
		 \end{table}		 
		 
		 \subsection{UC 3.6.3: Inserimento nuovo nome database}
	\textbf{Descrizione} 
	\\ \\
	L'utente intende rinominare un database presente sul server, ha inserito il comando per rinominare un database e il nome del database che vuole rinominare, ora deve inserire il nuovo nome per il database.
	\begin{table}[H]
			\begin{tabularx}{\textwidth}{r X}
				\textbf{Codice gerarchico} & UC3.6.3 \\
				\noalign{\hrule height 0.5pt}
				\textbf{Nome sintetico} & Inserimento nuovo nome database\\
				\noalign{\hrule height 0.5pt}
				\textbf{Attore principale} & Utente autenticato\\
				\noalign{\hrule height 0.5pt}
				\textbf{Attori secondari} & Nessuno \\
				\noalign{\hrule height 0.5pt}
				\textbf{Pre-condizione} & L'utente intende rinominare un database e ha inserito il comando di rinominazione database seguito dal nome del database da rinominare\\
				\noalign{\hrule height 0.5pt}
				\textbf{Post-condizione} & L'utente ha inserito con successo il nuovo nome per il database\\
				\noalign{\hrule height 0.5pt}
				\textbf{Flusso eventi} & \begin{enumerate}
				\item L'utente scrive su terminale il nuovo nome per il database
				\end{enumerate} \\
				\noalign{\hrule height 0.5pt}
				\textbf{Scenari alternativi} & Nessuno \\
				\noalign{\hrule height 0.5pt}
				\textbf{Lista requisiti\newline dedotti} & \begin{itemize}
				\item ...
				\end{itemize} 
			\end{tabularx}
			\caption{Caso d'uso UC 3.6.3 - Inserimento nuovo nome database}
		 \end{table}		 
		 
		 
		 \subsection{UC 3.7: Selezione di un database}
	 \begin{figure}[H]
				\centering
				\includegraphics[scale=0.25]{UC/"UC 3-7 Selezione di un database".png}
				\caption{Diagramma di UC3.7: Selezione di un database}
			\end{figure}
	\textbf{Descrizione} 
	\\ \\
	L'utente intende selezionare uno dei database presenti sul server al fine di effettuare operazioni a livello di mappa o item su di esso. Per selezionare un database l'utente deve inserire l'apposito comando da terminale, seguito dal nome del database che vuole selezionare.
	\begin{table}[H]
			\begin{tabularx}{\textwidth}{r X}
				\textbf{Codice gerarchico} & UC3.7 \\
				\noalign{\hrule height 0.5pt}
				\textbf{Nome sintetico} & Selezione di un database\\
				\noalign{\hrule height 0.5pt}
				\textbf{Attore principale} & Utente autenticato\\
				\noalign{\hrule height 0.5pt}
				\textbf{Attori secondari} & Nessuno \\
				\noalign{\hrule height 0.5pt}
				\textbf{Pre-condizione} & L'utente intende selezionare un database\\
				\noalign{\hrule height 0.5pt}
				\textbf{Post-condizione} & L'utente ha selezionato con successo il database richiesto\\
				\noalign{\hrule height 0.5pt}
				\textbf{Flusso eventi} & \begin{enumerate}
				\item L'utente inserisce il comando di selezione database (UC 3.7.1)
				\item L'utente inserisce il nome del database da selezionare (UC 3.7.2) e preme invio
				\end{enumerate} \\
				\noalign{\hrule height 0.5pt}
				\textbf{Scenari alternativi} & Nessuno \\
				\noalign{\hrule height 0.5pt}
				\textbf{Lista requisiti\newline dedotti} & \begin{itemize}
				\item ...
				\end{itemize} 
			\end{tabularx}
			\caption{Caso d'uso UC 3.7 - Selezione di un database}
		 \end{table}	
		 
		 
		 \subsection{UC 3.7.1: Inserimento comando di selezione database}
	\textbf{Descrizione} 
	\\ \\
	L'utente intende selezionare uno dei database presenti sul server al fine di effettuare operazioni a livello di mappa o item su di esso. Per effettuare l'operazione di selezione deve in primo luogo inserire il comando di selezione database da terminale.
	\begin{table}[H]
			\begin{tabularx}{\textwidth}{r X}
				\textbf{Codice gerarchico} & UC3.7.1 \\
				\noalign{\hrule height 0.5pt}
				\textbf{Nome sintetico} & Inserimento comando di selezione database\\
				\noalign{\hrule height 0.5pt}
				\textbf{Attore principale} & Utente autenticato\\
				\noalign{\hrule height 0.5pt}
				\textbf{Attori secondari} & Nessuno \\
				\noalign{\hrule height 0.5pt}
				\textbf{Pre-condizione} & L'utente intende selezionare un database\\
				\noalign{\hrule height 0.5pt}
				\textbf{Post-condizione} & L'utente ha inserito con successo il comando di selezione database\\
				\noalign{\hrule height 0.5pt}
				\textbf{Flusso eventi} & \begin{enumerate}
				\item L'utente scrive su terminale il comando di selezione database 
				\end{enumerate} \\
				\noalign{\hrule height 0.5pt}
				\textbf{Scenari alternativi} & Nessuno \\
				\noalign{\hrule height 0.5pt}
				\textbf{Lista requisiti\newline dedotti} & \begin{itemize}
				\item ...
				\end{itemize} 
			\end{tabularx}
			\caption{Caso d'uso UC 3.7.1 - Inserimento comando di selezione database}
		 \end{table}		 
		 
		 \subsection{UC 3.7.2: Inserimento nome database da selezionare}
	\textbf{Descrizione} 
	\\ \\
	L'utente ha inserito il comando per selezionare un database, ora deve inserire il nome del database da selezionare.
	\begin{table}[H]
			\begin{tabularx}{\textwidth}{r X}
				\textbf{Codice gerarchico} & UC3.7.2 \\
				\noalign{\hrule height 0.5pt}
				\textbf{Nome sintetico} & Inserimento nome database da selezionare\\
				\noalign{\hrule height 0.5pt}
				\textbf{Attore principale} & Utente autenticato\\
				\noalign{\hrule height 0.5pt}
				\textbf{Attori secondari} & Nessuno \\
				\noalign{\hrule height 0.5pt}
				\textbf{Pre-condizione} & L'utente intende selezionare un database e ha inserito il comando di selezione database\\
				\noalign{\hrule height 0.5pt}
				\textbf{Post-condizione} & L'utente ha inserito con successo il nome del database da selezionare\\
				\noalign{\hrule height 0.5pt}
				\textbf{Flusso eventi} & \begin{enumerate}
				\item L'utente scrive su terminale il nome del database da selezionare 
				\end{enumerate} \\
				\noalign{\hrule height 0.5pt}
				\textbf{Scenari alternativi} & Nessuno \\
				\noalign{\hrule height 0.5pt}
				\textbf{Lista requisiti\newline dedotti} & \begin{itemize}
				\item ...
				\end{itemize} 
			\end{tabularx}
			\caption{Caso d'uso UC 3.7.2 - Inserimento nome database da selezionare}
		 \end{table}		 	
		 
		 \subsection{UC 3.8: Errore esportazione fallita}
	\textbf{Descrizione} 
	\\ \\
	L'utente ha richiesto l'esportazione di uno o più database, l'operazione è fallita dunque l'utente riceve un messaggio di errore informativo e la possibilità di richiedere una nuova operazione.
	\begin{table}[H]
			\begin{tabularx}{\textwidth}{r X}
				\textbf{Codice gerarchico} & UC3.8 \\
				\noalign{\hrule height 0.5pt}
				\textbf{Nome sintetico} & Errore esportazione fallita\\
				\noalign{\hrule height 0.5pt}
				\textbf{Attore principale} & Utente autenticato\\
				\noalign{\hrule height 0.5pt}
				\textbf{Attori secondari} & Nessuno \\
				\noalign{\hrule height 0.5pt}
				\textbf{Pre-condizione} & L'utente ha richiesto un'operazione di esportazione database\\
				\noalign{\hrule height 0.5pt}
				\textbf{Post-condizione} & L'operazione richiesta non è stata eseguita con successo, nessun database è stato esportato, l'utente ha ricevuto un messaggio di errore e ora può richiedere una nuova operazione\\
				\noalign{\hrule height 0.5pt}
				\textbf{Flusso eventi} & \begin{enumerate}
				\item L'utente riceve un messaggio di errore informativo sul terminale
				\item L'utente riceve la possibilità di inserire un nuovo comando
				\end{enumerate} \\
				\noalign{\hrule height 0.5pt}
				\textbf{Scenari alternativi} & Nessuno \\
				\noalign{\hrule height 0.5pt}
				\textbf{Lista requisiti\newline dedotti} & \begin{itemize}
				\item ...
				\end{itemize} 
			\end{tabularx}
			\caption{Caso d'uso UC 3.8 - Errore esportazione fallita}
		 \end{table}		 	 	 	 
		 
		 
		 \subsection{UC 3.9: Errore importazione fallita}
	\textbf{Descrizione} 
	\\ \\
	L'utente ha richiesto l'importazione di un database, l'operazione è fallita dunque l'utente riceve un messaggio di errore informativo e la possibilità di richiedere una nuova operazione.
	\begin{table}[H]
			\begin{tabularx}{\textwidth}{r X}
				\textbf{Codice gerarchico} & UC3.9 \\
				\noalign{\hrule height 0.5pt}
				\textbf{Nome sintetico} & Errore importazione fallita\\
				\noalign{\hrule height 0.5pt}
				\textbf{Attore principale} & Utente autenticato\\
				\noalign{\hrule height 0.5pt}
				\textbf{Attori secondari} & Nessuno \\
				\noalign{\hrule height 0.5pt}
				\textbf{Pre-condizione} & L'utente ha richiesto un'operazione di importazione database\\
				\noalign{\hrule height 0.5pt}
				\textbf{Post-condizione} & L'operazione richiesta non è stata eseguita con successo, nessun database è stato importato, l'utente ha ricevuto un messaggio di errore e ora può richiedere una nuova operazione\\
				\noalign{\hrule height 0.5pt}
				\textbf{Flusso eventi} & \begin{enumerate}
				\item L'utente riceve un messaggio di errore informativo sul terminale
				\item L'utente riceve la possibilità di inserire un nuovo comando
				\end{enumerate} \\
				\noalign{\hrule height 0.5pt}
				\textbf{Scenari alternativi} & Nessuno \\
				\noalign{\hrule height 0.5pt}
				\textbf{Lista requisiti\newline dedotti} & \begin{itemize}
				\item ...
				\end{itemize} 
			\end{tabularx}
			\caption{Caso d'uso UC 3.9 - Errore importazione fallita}
		 \end{table}	
		 
		 \subsection{UC 3.10: Errore creazione fallita}
	\textbf{Descrizione} 
	\\ \\
	L'utente ha richiesto la creazione di un database, l'operazione è fallita (per nome inserito errato o per problemi di creazione) dunque l'utente riceve un messaggio di errore informativo e la possibilità di richiedere una nuova operazione.
	\begin{table}[H]
			\begin{tabularx}{\textwidth}{r X}
				\textbf{Codice gerarchico} & UC3.10 \\
				\noalign{\hrule height 0.5pt}
				\textbf{Nome sintetico} & Errore creazione fallita\\
				\noalign{\hrule height 0.5pt}
				\textbf{Attore principale} & Utente autenticato\\
				\noalign{\hrule height 0.5pt}
				\textbf{Attori secondari} & Nessuno \\
				\noalign{\hrule height 0.5pt}
				\textbf{Pre-condizione} & L'utente ha richiesto un'operazione di creazione database\\
				\noalign{\hrule height 0.5pt}
				\textbf{Post-condizione} & L'operazione richiesta non è stata eseguita con successo, nessun database è stato creato, l'utente ha ricevuto un messaggio di errore e ora può richiedere una nuova operazione\\
				\noalign{\hrule height 0.5pt}
				\textbf{Flusso eventi} & \begin{enumerate}
				\item L'utente riceve un messaggio di errore informativo sul terminale
				\item L'utente riceve la possibilità di inserire un nuovo comando
				\end{enumerate} \\
				\noalign{\hrule height 0.5pt}
				\textbf{Scenari alternativi} & Nessuno \\
				\noalign{\hrule height 0.5pt}
				\textbf{Lista requisiti\newline dedotti} & \begin{itemize}
				\item ...
				\end{itemize} 
			\end{tabularx}
			\caption{Caso d'uso UC 3.10 - Errore creazione fallita}
		 \end{table}			
		 
		 \subsection{UC 3.11: Errore database inesistente}
	\textbf{Descrizione} 
	\\ \\
	L'utente ha tentato di accedere ad un database non presente sul server. L'accesso al database richiesto è fallito, l'utente riceve un messaggio informativo e la possibilità di richiedere una nuova operazione.
	\begin{table}[H]
			\begin{tabularx}{\textwidth}{r X}
				\textbf{Codice gerarchico} & UC3.11 \\
				\noalign{\hrule height 0.5pt}
				\textbf{Nome sintetico} & Errore database inesistente\\
				\noalign{\hrule height 0.5pt}
				\textbf{Attore principale} & Utente autenticato\\
				\noalign{\hrule height 0.5pt}
				\textbf{Attori secondari} & Nessuno \\
				\noalign{\hrule height 0.5pt}
				\textbf{Pre-condizione} & L'utente ha richiesto l'accesso ad un determinato database, inserendone il nome\\
				\noalign{\hrule height 0.5pt}
				\textbf{Post-condizione} & Il database richiesto non è presente sul server, l'utente ha ricevuto un messaggio di errore e la possibilità di inserire un nuovo comando\\
				\noalign{\hrule height 0.5pt}
				\textbf{Flusso eventi} & \begin{enumerate}
				\item L'utente riceve un messaggio di errore informativo sul terminale
				\item L'utente riceve la possibilità di inserire un nuovo comando
				\end{enumerate} \\
				\noalign{\hrule height 0.5pt}
				\textbf{Scenari alternativi} & Nessuno \\
				\noalign{\hrule height 0.5pt}
				\textbf{Lista requisiti\newline dedotti} & \begin{itemize}
				\item ...
				\end{itemize} 
			\end{tabularx}
			\caption{Caso d'uso UC 3.11 - Errore database inesistente}
		 \end{table}			
		 
		 
		 
		 
		 
		 
		 
		 
		 
		 
		 
		 
	 
	\newpage \section{Requisiti}
		Di seguito vengono riportati tutti i requisiti individuati dal gruppo SWEeneyThreads. \\Tali requisiti
		derivano dai casi d’uso, dall'analisi del capitolato, dagli scambi di informazioni avvenuti con il
		\emph{proponente Cardin Riccardo}, oppure da necessità interne. \\
		La struttura di un requisito è definita nel documento \emph{Norme di progetto v1.1.1 sez 2.1.2}.
		I requisiti saranno elencati secondo un ordine. Ogni requisito seguirà la seguente codifica: \\
		\begin{center}
			R[Codice][Importanza][Tipo]
		\end{center}
		\textbf{Codice} \\ \\ Un codice univoco ed espresso in modo gerarchico;\\ \\
		\textbf{Importanza} \\ \\Può assumere i seguenti valori:
		\begin{itemize}
			\item \textbf{N:} necessario;
			\item \textbf{D:} desiderabile;
			\item \textbf{O:} opzionale.
		\end{itemize}
		\textbf{Tipo} \\ \\Può assumere i seguenti valori:
		\begin{itemize}
			\item \textbf{F:} funzionale;
			\item \textbf{Q:} di qualità;
			\item \textbf{P:} prestazionale;
			\item \textbf{V:} vincolo.
		\end{itemize}
	
	\subsection{Requisiti funzionali}
	I requisiti funzionali riguardano le funzioni vere e proprie del prodotto. Il soddisfacimento di uno di essi
	equivale all'implementazione di una funzionalità.
	\LTXtable{\textwidth}{tabelle_requisiti/funzionali.tex}
	\subsection{Requisiti di vincolo}
	I requisiti di vincolo rappresentano i vincoli che devono essere soddisfatti dal prodotto e che esulano dalle
	sue caratteristiche funzionali.
	\LTXtable{\textwidth}{tabelle_requisiti/vincolo.tex}
	\subsection{Requisiti di qualità}
	I requisiti di qualità specificano le operazioni da compiere per far raggiungere al prodotto il livello 
	qualitativo richiesto.
	\LTXtable{\textwidth}{tabelle_requisiti/qualita.tex}
	\subsection{Requisiti prestazionali}
	I requisiti prestazionali consentono, al loro soddisfacimento, di far raggiungere al prodotto un determinato
	livello di prestazioni, non si traducono in nuove funzionaltà per l'utente ma intendono migliorare 
	la stabilità e la velocità del programma.
	\LTXtable{\textwidth}{tabelle_requisiti/prestazionali.tex}
	
	\subsection{Tracciamento fonti-requisiti}
	Di seguito si riporta in forma tabellare l'elenco di requisiti ricavati dalle diverse fonti.
	\LTXtable{\textwidth}{tabelle_requisiti/fontirequisiti.tex}
	
	\cleardoublepage
	\addcontentsline{toc}{section}{\listfigurename}
	\listoffigures
	
	\cleardoublepage
	\addcontentsline{toc}{section}{\listtablename}
	\listoftables
		
\end{document}
	
