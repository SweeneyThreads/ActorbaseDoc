%Document-Author: Bonato Paolo + Bortolazzo Matteo + Maino Elia
%Document-Date: 2016-01-16
%Document-Description: Documento di analisi dei requisiti


\documentclass[a4paper]{report}
\usepackage[english, italian]{babel}
\usepackage[T1]{fontenc}
\usepackage[utf8]{inputenc}
\usepackage{url}
\usepackage{graphicx}
\graphicspath{{../Immagini/}}
\usepackage[hidelinks]{hyperref}
\usepackage{booktabs}
\usepackage{tabularx}
\usepackage{pifont}
\usepackage[table]{xcolor}
\usepackage{float}

\newcolumntype{s}{>{\hsize=.21\hsize}X}
\newcolumntype{f}{>{\hsize=.37\hsize}X}
\newcolumntype{m}{>{\hsize=.42\hsize}X}

\newcommand{\mychapter}[2]{
	\setcounter{chapter}{#1}
	\setcounter{section}{0}
	\setcounter{subsection}{1}
	\chapter*{#2}
	\addcontentsline{toc}{chapter}{#2}
}

\renewcommand{\abstractname}{Tabella contenuti}

\begin{document}
	
	\begin{titlepage}
		% Defines a new command for the horizontal lines, change thickness here
		\newcommand{\HRule}{\rule{\linewidth}{0.5mm}} 
		\center  
		
		% HEADING SECTION
		\textsc{\LARGE SweeneyThreads}\\[1.5cm] 
		\textsc{\Large Actorbase}\\[0.5cm] 
		\textsc{\large a NoSQL DB based on the Actor model}\\[0.5cm]
		
		
		% TITLE SECTION
		\HRule \\[0.4cm]
		{ \huge \bfseries Analisi dei requisiti}\\[0.4cm] 
		\HRule \\[1.5cm]
		
		% AUTHOR SECTION
		\begin{minipage}{0.4\textwidth}
			\begin{flushleft} \large
				\emph{Redattori:}\\
				Bonato Paolo \\
				Bortolazzo Matteo \\
				Maino Elia
			\end{flushleft}
		\end{minipage}
		~
		\begin{minipage}{0.4\textwidth}
			\begin{flushright} \large
				\emph{Approvazione:} \\
				\dots \\
				\emph{Verifica:} \\
				\dots
			\end{flushright}
		\end{minipage}
		
		%immagine
		\begin{figure}[H]
			\centering
			\includegraphics[scale=0.8]{sweeney.png}
		\end{figure}
		\begin{center}
			Versione 1.0.3
		\end{center}
		% Date, change the \today to a set date if you want to be precise
		{\large \today}\\[3cm] 
		% Fill the rest of the page with whitespace
		\vfill  
	\end{titlepage}
	
	
	\tableofcontents
	
	\mychapter{0}{Diario delle modifiche}
		\begin{table}[H]
			\begin{tabularx}{\textwidth}{s f m X}
				\noalign{\hrule height 1.5pt}
				\rowcolor{orange!85} Versione & Data & Autore & Descrizione \\
				1.0.3 & 2016-01-18 & \emph{Analista} Maino Elia & Stesura dei casi d'uso (da 4.x a 5.x) \\
				\noalign{\hrule height 0.5pt}
				1.0.2 & 2016-01-17 & \emph{Analista} Maino Elia & Stesura dei casi d'uso (da 1.x a 3.x) \\
				\noalign{\hrule height 0.5pt}
				1.0.1 & 2016-01-17 & \emph{Analista} Maino Elia & Stesura caratteristiche generali del prodotto \\
				\noalign{\hrule height 0.5pt}
				1.0.0 & 2016-01-17 & \emph{Analista} Maino Elia & Creazione scheletro documento e stesura introduzione \\
				\noalign{\hrule height 1.5pt}
			\end{tabularx}
			\caption{Diario delle modifiche \label{tab:table_label}}
		\end{table}

	\mychapter{1}{Introduzione}
	\section{Scopo del documento}
		Lo scopo del seguente documento è presentare l'insieme di requisiti individuati dal gruppo 
		SWEeneyThreads durante l'analisi del \emph{Capitolato C1}, gli incontri con il committente 
		\emph{Cardin Riccardo} e l'analisi dei casi d'uso. 
		
		Si intende fornire una rappresentazione ordinata delle funzionalità che il prodotto \emph{Actorbase} 
		offrirà al momento del rilascio.
	\section{Scopo del prodotto}
		Lo scopo del progetto è la realizzazione di un DataBase NoSQL key-value basato sul modello ad 
		Attori\ped{\textit{G}} con l'obiettivo di fornire una tecnologia adatta allo sviluppo di moderne 
		applicazioni che richiedono brevissimi tempi di risposta e che elaborano enormi quantità 
		di dati. Lo sviluppo porterà al rilascio del software sotto licenza MIT.
	\section{Glossario}
		Con lo scopo di evitare ambiguità di linguaggio e di massimizzare la comprensione dei documenti, il 
		gruppo ha steso un documento interno che è il \emph{Glossario v1.0.0}. La prima occorrenza
		di ogni termine termine contenuto nel \emph{Glossario} e presente in questo documento verrà 
		marcato con una "\textit{G}" maiuscola in pedice.
	\section{Riferimenti}
	\subsection{Informativi}	
		\dots
	\subsection{Normativi}
		\begin{itemize}
			\item Capitolato d'appalto Actorbase (C1): \\ 
			\url{http://www.math.unipd.it/~tullio/IS-1/2015/Progetto/C1p.pdf}
		\end{itemize}
	\mychapter{2}{Caratteristiche generali del prodotto}
	\section{Obiettivi del prodotto}
		Il prodotto si pone l'obiettivo di fornire un database NoSQL basato sul modello ad attori che possa 
		essere utilizzato con successo dalle cosiddette \emph{Reactive Applications}, ovvero applicazioni
		orientate agli eventi, scalabili, resilienti e responsive. Per tali applicazioni il paradigma di accesso 
		ai dati basato su database di tipo relazionale risulta non applicabile, poichè troppo limitante, sono 
		dunque necessarie nuove forme di gestione dell'informazione, \emph{Actorbase} intende proporsi 
		come una valida opzione.
	\section{Funzioni principali del prodotto}
		\emph{Actorbase} prevede un'interazione con l'utente da riga di comando attraverso l'uso di un 
		\emph{Domain Specific Language (DSL)}. 
		Oltre alla creazione di uno schema sarà possibile effettuare le seguenti operazioni:
		\begin{enumerate}
			\item Inserimento
			\item Cancellazione 
			\item Aggiornamento (caso particolare di un inserimento con chiave già presente)
		\end{enumerate}
	\section{Target d'utenza}
		Il prodotto si rivolge a sviluppatori di applicazioni moderne, che trattano enormi moli di dati (nell'
		ordine dei Petabyte), che richiedono brevissimi tempi di risposta e che necessitano di un uptime 
		del 100\%.
	\section{Vincoli per l'utilizzo}
		L'utilizzo di \emph{Actorbase} non richiede hardware o software particolare, sebbene l'esecuzione 
		su macchine non recenti possa influenzare le prestazioni.	
	
	\mychapter{3}{Casi d'uso}
		Di seguito viene riportata la descrizione accurata di tutti i casi d'uso individuati dal gruppo a seguito
		delle seguenti attività:
		\begin{itemize}
			\item Analisi del capitolato C1 Actorbase
			\item Confronti con il committente \emph{Cardin Riccardo}
			\item Riunioni e discussioni interne al gruppo
			\item Analisi della struttura e delle funzionalità di altri database non realzionali
		\end{itemize}
		La struttura di un caso d'uso è definita nel documento \emph{Norme di progetto v1.1.1 sez 2.1.3}.
		\newpage
		\section{Visione ad alto livello delle operazioni sul sistema}
		 	\begin{figure}[H]
				\centering
				\includegraphics[scale=0.35]{"UC0".png}
				\caption{Operazioni principali ad alto livello}
			\end{figure}
			Il diagramma in \emph{figura 3.1} illustra le principali operazioni che un utente esterno può
			 effettuare sul sistema:
			\begin{itemize}
				\item Installazione dell'applicativo
				\item Connessione ad un server e autenticazione allo stesso
				\item Gestione dei database presenti sul server
				\item Operazioni sui singoli database
				\item Operazioni sulle mappe dei database
			\end{itemize}
		\section{Caso d'uso UC1: Installazione}
			\begin{figure}[H]
				\centering
				\includegraphics[scale=0.3]{"UC1".png}
				\caption{Caso d'uso UC1}
			\end{figure}
		 \textbf{Descrizione} \\ \\
		 L'utente è in possesso dei file necessari a far partire il programma di setup. Una volta avviato
		  l'utente può scegliere tra tre modalità d'installazione: 
		 \begin{itemize}
		 	\item Installazione completa
		 	\item Installazione server
		 	\item Installazione client
		 \end{itemize}
		 A seconda dell'opzione selezionata il programma di setup installa le relative componenti. Durante 
		 l'installazione possono occorrere errori che ne impediscano il completamento, in tal caso il setup
		 informa l'utente e interrompe l'installazione.
			\begin{table}[H]
			\begin{tabularx}{\textwidth}{X | X}\toprule
				\rowcolor{orange!65}Codice gerarchico & UC1 \\
				Nome sintetico & Installazione \\
				\rowcolor{orange!65}Attore principale & Utente generico\\
				Attori secondari & Nessuno \\
				\rowcolor{orange!65}Pre-condizione & L'utente è in possesso del programma \\
				Post-condizione & Le componenti selezionate risultano installate sulla
				 macchina dell'utente \\
				\rowcolor{orange!65}Flusso eventi & \begin{enumerate}
				\item L'utente lancia il programma di setup
				\item L'utente seleziona le componenti da installare
				\item Il programma di setup installa le componenti selezionate
				\end{enumerate} \\
				Scenari alternativi & \begin{enumerate}
				\item Errore durante l'installazione, l'utente riceve un messaggio
				\end{enumerate} \\
				\rowcolor{orange!65}Lista requisiti dedotti & \\
				\bottomrule
			\end{tabularx}
			\caption{Caso d'uso UC1}
		 \end{table}
	\section{Caso d'uso UC2: Connessione ed autenticazione}
		 	\begin{figure}[H]
				\centering
				\includegraphics[scale=0.3]{"UC2".png}
				\caption{Caso d'uso UC2}
			\end{figure}
		 \textbf{Descrizione} \\ \\
		 L'utente ha appena avviato l'applicativo e intende accedere al sistema. L'utente 
		 inserisce il nome del server e le proprie credenziali di accesso. Il sistema tenta di connettersi a tale
		 server. Se la connessione non risulta possibile l'utente riceve un messaggio di errore che lo informa
		 del fallimento.
		 Una volta stabilita la connessione il sistema verifica le credenziali di accesso, se le credenziali sono
		 corrette l'utente risulta connesso a tutti gli effetti, se sono errate riceve un relativo messaggio di 
		 errore.
			\begin{table}[H]
			\begin{tabularx}{\textwidth}{X | X}\toprule
				\rowcolor{orange!65}Codice gerarchico & UC2 \\
				Nome sintetico & Connessione ad un server \\
				\rowcolor{orange!65}Attore principale & Utente generico\\
				Attori secondari & Nessuno \\
				\rowcolor{orange!65}Pre-condizione & L'utente ha avviato il programma e vuole accedere al
				 sistema\\
				Post-condizione & L'utente risulta connesso al server selezionato \\
				\rowcolor{orange!65}Flusso eventi & \begin{enumerate}
				\item L'utente inserisce il nome del server a cui intende connettersi
				\item L'utente inserisce le proprie credenziali
				\item La connessione riesce
				\item Il sistema verifica le credenziali inserite
				\item L'utente risulta connesso
				\end{enumerate} \\
				Scenari alternativi & \begin{enumerate}
				\item Errore durante la connessione, l'utente riceve un messaggio
				\item Credenziali errate, l'utente riceve un messaggio
				\end{enumerate} \\
				\rowcolor{orange!65}Lista requisiti dedotti & \\
				\bottomrule
			\end{tabularx}
			\caption{Caso d'uso UC2}
		 \end{table}
	 \section{Caso d'uso UC3: Operazioni sul server}
	 	\begin{figure}[H]
			\centering
			\includegraphics[scale=0.3]{"UC3".png}
			\caption{Caso d'uso UC3}
		\end{figure}
	 \textbf{Descrizione} \\ \\
	 L'utente ha effettuato l'autenticazione e risulta correttamente connesso al server. Le credenziali di
	 accesso determinano i permessi di accesso (stabiliti in un file di configurazione interno), e differenziano
	 gli utenti tra \emph{utenti standard} e \emph{amministratori}. \\
	 Un \emph{utente standard} può solo selezionare uno dei database di cui dispone dei permessi di
	  accesso e su di esso eseguire operazioni di modifica.
	 Un \emph{amministratore} oltre a disporre degli stessi permessi degli utenti standard, può effettuare
	 le seguenti operazioni a livello di server:
	 \begin{itemize}
	 	\item Creare un nuovo database
	 	\item Importare un database da file
	 	\item Esportare un database su file
	 	\item Rimuovere un database dal server
	 	\item Gestire i permessi di accesso al server e ai singoli database
	 \end{itemize}
	 Poichè \emph{Actorbase} fornisce un'interfaccia da riga di comando, anche un utente standard può
	 richiedere un'operazione a livello amministratore, in tal caso l'operazione non sarà eseguita e il 
	 sistema informerà l'utente con un messaggio di errore che lo informa di non avere i permessi 
	 necessari. Un'operazione legittima può inoltre non andare a buon fine, anche in quel caso l'utente 
	 riceverà un relativo messaggio di errore.
		\begin{table}[H]
		\begin{tabularx}{\textwidth}{X | X}\toprule
			\rowcolor{orange!65}Codice gerarchico & UC3 \\
			Nome sintetico & Operazioni sul server \\
			\rowcolor{orange!65}Attore principale & Utente standard, Amministratore\\
			Attori secondari & Sistema di I/O \\
			\rowcolor{orange!65}Pre-condizione & L'utente è autenticato\\
			Post-condizione & L'operazione selezionata risulta effettuata\\
			\rowcolor{orange!65}Flusso eventi & \begin{enumerate}
			\item L'utente richiede una delle operazioni di modifica
			\item L'operazione richiesta viene effettuata dal sistema
			\end{enumerate} \\
			Scenari alternativi & \begin{enumerate}
			\item L'utente non dispone dei permessi necessari ad effettuare l'operazione richiesta, l'utente
			riceve un messaggio di errore
			\item L'operazione richiesta non è stata effettuata, l'utente riceve un messaggio di errore
			\end{enumerate} \\
			\rowcolor{orange!65}Lista requisiti dedotti & \\
			\bottomrule
		\end{tabularx}
		\caption{Caso d'uso UC3}
	 \end{table}
	 \section{Caso d'uso UC3.1: Selezione di un database}
	 	\begin{figure}[H]
			\centering
			\includegraphics[scale=0.3]{"UC3-1".png}
			\caption{Caso d'uso UC3.1}
		\end{figure}
	 \textbf{Descrizione} \\ \\
	 L'utente inserisce il nome del database che vuole utilizzare, se tale database esiste sul server e se 
	 l'utente dispone dei permessi necessari ad accedervi il database risulta selezionato, altrimenti l'utente
	 riceve un messaggio d'errore esplicativo.
		\begin{table}[H]
		\begin{tabularx}{\textwidth}{X | X}\toprule
			\rowcolor{orange!65}Codice gerarchico & UC3.1 \\
			Nome sintetico & Selezione di un database \\
			\rowcolor{orange!65}Attore principale & Utente generico\\
			Attori secondari & Nessuno \\
			\rowcolor{orange!65}Pre-condizione & L'utente è autenticato\\
			Post-condizione & Il database risulta selezionato, l'utente può ora effettuare operazioni su tale
			database \\
			\rowcolor{orange!65}Flusso eventi & \begin{enumerate}
			\item L'utente inserisce il nome del database da selezionare
			\item Il database viene selezionato
			\end{enumerate} \\
			Scenari alternativi &  \begin{enumerate}
			\item Il database selezionato non è presente sul server, l'utente riceve un messaggio di errore
			\item L'utente non dispone dei permessi necessari ad accedere al database selezionato, l'utente
			riceve un messaggio di errore
			\end{enumerate}			 \\
			\rowcolor{orange!65}Lista requisiti dedotti & \\
			\bottomrule
		\end{tabularx}
		\caption{Caso d'uso UC3.1}
	 \end{table}
	 \section{Caso d'uso UC3.2: Creazione di un database}
	 	\begin{figure}[H]
			\centering
			\includegraphics[scale=0.3]{"UC3-2".png}
			\caption{Caso d'uso UC3.2}
		\end{figure}
	 \textbf{Descrizione} \\ \\
	 L'utente è autenticato come \emph{amministratore} e desidera creare un nuovo database sul server. 
	 \\ L'utente inserisce un nome per il nuovo database, se tale nome risulta valido il database viene
	  creato, altrimenti l'utente riceve un messaggio di errore. \\
		\begin{table}[H]
		\begin{tabularx}{\textwidth}{X | X}\toprule
			\rowcolor{orange!65}Codice gerarchico & UC3.2 \\
			Nome sintetico & Creazione di un database \\
			\rowcolor{orange!65}Attore principale & Amministratore\\
			Attori secondari & Nessuno \\
			\rowcolor{orange!65}Pre-condizione & L'utente è autenticato come amministratore del server\\
			Post-condizione & Viene creato un database il cui nome è quello inserito dall'utente \\
			\rowcolor{orange!65}Flusso eventi & \begin{enumerate}
			\item L'utente inserisce un nome per il database
			\item Il sistema verifica il nome inserito
			\item Viene creato un database con il nome inserito
			\end{enumerate} \\
			Scenari alternativi & \begin{enumerate}
			\item Il nome inserito non risulta valido, l'utente riceve un messaggio di errore e il database non
			 viene creato
			\end{enumerate} \\
			\rowcolor{orange!65}Lista requisiti dedotti & \\
			\bottomrule
		\end{tabularx}
		\caption{Caso d'uso UC3.2}
	 \end{table}
	 \section{Caso d'uso UC3.3: Importazione di un database}
	 	\begin{figure}[H]
			\centering
			\includegraphics[scale=0.3]{"UC3-3".png}
			\caption{Caso d'uso UC3.3}
		\end{figure}
	 \textbf{Descrizione} \\ \\
	 L'utente desidera creare un nuovo schema importandolo da un file presente sul computer. L'utente
	  deve specificare il percorso del file. Il sistema ricerca il file nel percorso dato, se lo trova lo apre e ne 
	  legge il contenuto. In base al contenuto il sistema crea un nuovo schema rendendone amministratore
	  l'utente.
		\begin{table}[H]
		\begin{tabularx}{\textwidth}{X | X}\toprule
			\rowcolor{orange!65}Codice gerarchico & UC3.3 \\
			Nome sintetico & Importazione di un database \\
			\rowcolor{orange!65}Attore principale & Amministratore\\
			Attori secondari & Sistema di I/O \\
			\rowcolor{orange!65}Pre-condizione & L'utente è autenticato come amministratore del server\\
			Post-condizione & \'E stato creato un nuovo database sul server con i dati presenti nel file
			selezionato \\
			\rowcolor{orange!65}Flusso eventi & \begin{enumerate}
			\item L'utente inserisce il percorso del file da cui importare il database
			\item Il sistema di I/O ottiene il file
			\item Il sistema verifica il contenuto del file
			\item Il sistema crea un database basato sui dati contenuti nel file
			\end{enumerate} \\
			Scenari alternativi & \begin{enumerate}
			\item Il file non viene trovato dal sistema di I/O, l'utente riceve un messaggio di errore
			\item Il contenuto del file risulta illeggibile, l'utente riceve un messaggio di errore
			\item Il database contenuto nel file non può essere importato perchè viola dei vincoli del sistema 
			(es. esiste già un altro database con lo stesso nome), l'utente riceve un messaggio di errore
			\end{enumerate} \\
			\rowcolor{orange!65}Lista requisiti dedotti & \\
			\bottomrule
		\end{tabularx}
		\caption{Caso d'uso UC3.3}
	 \end{table}
	 \section{Caso d'uso UC3.4: Esportazione di un database}
	 	\begin{figure}[H]
			\centering
			\includegraphics[scale=0.28]{"UC3-4".png}
			\caption{Caso d'uso UC3.4}
		\end{figure}
	 \textbf{Descrizione} \\ \\
	 L'utente inserisce il nome del database che vuole esportare, se sul server non esiste nessun database 
	 con il nome inserito, l'utente riceve un messaggio di errore. \\
	 Se il database è presente l'utente deve specificare il percorso in cui il file verrà creato, il file avrà come
	 nome il nome del database. Se la creazione del file non va a buon fine il sistema di I/O informa il 
	 sistema e l'utente riceve un messaggio di errore.
		\begin{table}[H]
		\begin{tabularx}{\textwidth}{X | X}\toprule
			\rowcolor{orange!65}Codice gerarchico & UC3.4 \\
			Nome sintetico & Esportazione di un database \\
			\rowcolor{orange!65}Attore principale & Amministratore\\
			Attori secondari & Sistema di I/O \\
			\rowcolor{orange!65}Pre-condizione & L'utente è autenticato come amministratore del server\\
			Post-condizione & Il database selezionato viene esportato su file \\
			\rowcolor{orange!65}Flusso eventi & \begin{enumerate}
			\item L'utente inserisce il nome del database da esportare
			\item L'utente specifica il percorso del file da creare
			\item Il sistema crea il file contenente i dati del database
			\end{enumerate} \\
			Scenari alternativi & \begin{enumerate}
			\item Il database selezionato non esiste, l'utente riceve un messaggio di errore
			\item La creazione del file non ha successo, l'utente riceve un messaggio di errore
			\end{enumerate} \\
			\rowcolor{orange!65}Lista requisiti dedotti & \\
			\bottomrule
		\end{tabularx}
		\caption{Caso d'uso UC4}
	 \end{table}
	\section{Caso d'uso UC3.5: Rimozione di un database}
	 	\begin{figure}[H]
			\centering
			\includegraphics[scale=0.3]{"UC3-5".png}
			\caption{Caso d'uso UC3.5}
		\end{figure}
	 \textbf{Descrizione} \\ \\
	 L'utente inserisce il nome del database che vuole eliminare dal server, se sul server non è 
	 presente nessun database con quel nome l'utente riceve un messaggio di errore, altrimenti il database
	 selezionato viene eliminato.
		\begin{table}[H]
		\begin{tabularx}{\textwidth}{X | X}\toprule
			\rowcolor{orange!65}Codice gerarchico & UC3.5 \\
			Nome sintetico & Rimozione schema \\
			\rowcolor{orange!65}Attore principale & Amministratore\\
			Attori secondari & Nessuno \\
			\rowcolor{orange!65}Pre-condizione & L'utente è autenticato come amministratore del server\\
			Post-condizione & Il database selezionato è stato rimosso \\
			\rowcolor{orange!65}Flusso eventi & \begin{enumerate}
			\item L'utente inserisce il nome del database da eliminare
			\item Il sistema rimuove il database selezionato
			\end{enumerate} \\
			Scenari alternativi & \begin{enumerate}
			\item Il database selezionato non esiste, l'utente riceve un messaggio di errore
			\end{enumerate} \\
			\rowcolor{orange!65}Lista requisiti dedotti & \\
			\bottomrule
		\end{tabularx}
		\caption{Caso d'uso UC4.5}
	 \end{table}
	 \section{Caso d'uso UC3.6: Gestione dei permessi}
	 	\begin{figure}[H]
			\centering
			\includegraphics[scale=0.3]{"UC3-6".png}
			\caption{Caso d'uso UC3.6}
		\end{figure}
	 \textbf{Descrizione} \\ \\
	 L'\emph{amministratore} può modificare i permessi di accesso di altri utenti al server e al singolo
	  database, in particolare può:
	 \begin{itemize}
	 	\item Aggiungere un nuovo utente al server, generando un username e una password
	 	\item Modificare i permessi di un qualsiasi utente
	 	\item Rimuovere un utente dalla lista degli utenti del server
	 \end{itemize}
	 Ognuna di queste operazioni può avere esito negativo, in tal caso l'amministratore dello schema che ha
	 richiesto l'operazione riceve un messaggio d'errore esplicativo.
		\begin{table}[H]
		\begin{tabularx}{\textwidth}{X | X}\toprule
			\rowcolor{orange!65}Codice gerarchico & UC3.6 \\
			Nome sintetico & Gestione dei permessi \\
			\rowcolor{orange!65}Attore principale & Amministratore\\
			Attori secondari & Nessuno \\
			\rowcolor{orange!65}Pre-condizione & L'utente è autenticato come amministratore del server\\
			Post-condizione & L'operazione di modifica dei permessi è eseguita correttamente \\
			\rowcolor{orange!65}Flusso eventi & \begin{enumerate}
			\item L'utente seleziona l'operazione che vuole effettuare
			\item L'operazione viene effettuata dal sistema
			\end{enumerate} \\
			Scenari alternativi & \begin{enumerate}
			\item \'E impossibile completare l'operazione richiesta, l'utente riceve un messaggio di errore
			\end{enumerate} \\
			\rowcolor{orange!65}Lista requisiti dedotti & \\
			\bottomrule
		\end{tabularx}
		\caption{Caso d'uso UC3.6}
	 \end{table}
	 \section{Caso d'uso UC3.6.1: Aggiunta di un utente al server}
	 \begin{figure}[H]
			\centering
			\includegraphics[scale=0.3]{"UC3-6-1".png}
			\caption{Caso d'uso UC3.6.1}
		\end{figure}
	 \textbf{Descrizione} \\ \\
	 L'\emph{amministratore} inserisce un nuovo utente generando un username e una password, se 
	 l'username inserito risulta già in uso l'inserimento non avviene e l'\emph{amministratore} riceve un
	 messaggio di errore. Altrimenti l'utente risulta correttamente inserito ed è possibile accedere al server
	 con relativi username e password. \\
	 Un utente appena inserito è di tipo standard e non ha accesso a nessun database del server.
		\begin{table}[H]
		\begin{tabularx}{\textwidth}{X | X}\toprule
			\rowcolor{orange!65}Codice gerarchico & UC4.6.1 \\
			Nome sintetico & Aggiunta di un utente al server \\
			\rowcolor{orange!65}Attore principale & Amministratore\\
			Attori secondari & Nessuno \\
			\rowcolor{orange!65}Pre-condizione & L'utente è autenticato come amministratore del server ed
			 ha selezionato l'opzione di aggiunta di un utente\\
			Post-condizione & \'E stato aggiunto un nuovo utente con l'username e la password inseriti\\
			\rowcolor{orange!65}Flusso eventi & \begin{enumerate}
			\item L'amministratore inserisce l'username dell'utente da creare
			\item Il sistema verifica che l'username sia disponibile
			\item L'amministratore inserisce la password dell'utente da creare
			\end{enumerate} \\
			Scenari alternativi & \begin{enumerate}
			\item L'username inserito non è disponibile, l'utente riceve un messaggio di errore
			\end{enumerate} \\
			\rowcolor{orange!65}Lista requisiti dedotti & \\
			\bottomrule
		\end{tabularx}
		\caption{Caso d'uso UC3.6.1}
	 \end{table}
	 \section{Caso d'uso UC3.6.2: Modifica dei permessi utente}
	 \begin{figure}[H]
			\centering
			\includegraphics[scale=0.3]{"UC3-6-2".png}
			\caption{Caso d'uso UC3.6.2}
		\end{figure}
	 \textbf{Descrizione} \\ \\
	 L'\emph{amministratore} inserisce l'username dell'utente che vuole modificare, nel caso non esistesse
	  un utente con tale username riceve un messaggio di errore. \\
	 I permessi modificabili riguardano:
	 \begin{itemize}
	 	\item Il livello di accesso al server: \emph{standard} o \emph{amministratore}
	 	\item L'accesso ai diversi database
	 \end{itemize}
		\begin{table}[H]
		\begin{tabularx}{\textwidth}{X | X}\toprule
			\rowcolor{orange!65}Codice gerarchico & UC3.6.2 \\
			Nome sintetico & Modifica dei permessi utente \\
			\rowcolor{orange!65}Attore principale & Amministratore\\
			Attori secondari & Nessuno \\
			\rowcolor{orange!65}Pre-condizione & L'utente è autenticato come amministratore del server ed
			 ha selezionato l'opzione di modifica dei permessi\\
			Post-condizione & I permessi dell'utente selezionato sono stati modificati correttamente \\
			\rowcolor{orange!65}Flusso eventi & \begin{enumerate}
			\item L'amministratore inserisce un username
			\item Il sistema verifica che esista un utente con tale username
			\item L'amministratore decide il livello di accesso al server
			\item L'amministratore decide il livello di accesso ai singoli database
			\end{enumerate} \\
			Scenari alternativi & \begin{enumerate}
			\item L'utente inserito non esiste, l'amministratore riceve un messaggio di errore
			\end{enumerate} \\
			\rowcolor{orange!65}Lista requisiti dedotti & \\
			\bottomrule
		\end{tabularx}
		\caption{Caso d'uso UC3.6.2}
	 \end{table}
	 \section{Caso d'uso UC3.6.3: Rimozione di un utente dal server}
	 \begin{figure}[H]
			\centering
			\includegraphics[scale=0.3]{"UC3-6-3".png}
			\caption{Caso d'uso UC3.6.3}
		\end{figure}
	 \textbf{Descrizione} \\ \\
	 L'\emph{amministratore} inserisce l'username dell'utente che vuole rimuovere, nel caso non esistesse
	  un utente con tale username riceve un messaggio di errore. \\
		\begin{table}[H]
		\begin{tabularx}{\textwidth}{X | X}\toprule
			\rowcolor{orange!65}Codice gerarchico & UC3.6.3 \\
			Nome sintetico & Rimozione di un utente dal server \\
			\rowcolor{orange!65}Attore principale & Amministratore\\
			Attori secondari & Nessuno \\
			\rowcolor{orange!65}Pre-condizione & L'utente è autenticato come amministratore del server ed
			 ha selezionato l'opzione di rimozione di un utente\\
			Post-condizione & L'utente selezionato è stato rimosso dal server\\
			\rowcolor{orange!65}Flusso eventi & \begin{enumerate}
			\item L'amministratore inserisce un username
			\item Il sistema verifica che esista un utente con tale username
			\end{enumerate} \\
			Scenari alternativi & \begin{enumerate}
			\item L'utente inserito non esiste, l'amministratore riceve un messaggio di errore
			\end{enumerate} \\
			\rowcolor{orange!65}Lista requisiti dedotti & \\
			\bottomrule
		\end{tabularx}
		\caption{Caso d'uso UC3.6.3}
	 \end{table}
	 \section{Caso d'uso UC4: Operazioni sui database}
	 \begin{figure}[H]
			\centering
			\includegraphics[scale=0.3]{"UC4".png}
			\caption{Caso d'uso UC4}
		\end{figure}
	 \textbf{Descrizione} \\ \\
	 L'utente effettua una delle seguenti operazioni:
	 \begin{itemize}
	 	\item Aggiunta di una nuova mappa
	 	\item Modifica del nome di una mappa preesistente
	 	\item Rimozione di una mappa
	 \end{itemize}
		\begin{table}[H]
		\begin{tabularx}{\textwidth}{X | X}\toprule
			\rowcolor{orange!65}Codice gerarchico & UC4 \\
			Nome sintetico & Operazioni sui database \\
			\rowcolor{orange!65}Attore principale & Utente generico\\
			Attori secondari & Nessuno \\
			\rowcolor{orange!65}Pre-condizione & L'utente è autenticato e dispone dei permessi di accesso
			al database\\
			Post-condizione & L'operazione selezionata è stata effettuata\\
			\rowcolor{orange!65}Flusso eventi & \begin{enumerate}
			\item L'utente seleziona una delle operazioni
			\item Il sistema esegue l'operazione
			\end{enumerate} \\
			Scenari alternativi & \begin{enumerate}
			\item L'operazione ha avuto esito negativo, l'utente riceve un messaggio di errore
			\end{enumerate} \\
			\rowcolor{orange!65}Lista requisiti dedotti & \\
			\bottomrule
		\end{tabularx}
		\caption{Caso d'uso UC4}
	 \end{table}
	 \section{Caso d'uso UC4.1: Aggiunta di una nuova mappa}
	 \begin{figure}[H]
			\centering
			\includegraphics[scale=0.3]{"UC4-1".png}
			\caption{Caso d'uso UC4.1}
		\end{figure}
	 \textbf{Descrizione} \\ \\
	 L'utente aggiunge una nuova mappa al database definendone il nome, se tale nome è già associato ad 
	 un'altra mappa dello stesso database allora la mappa non viene creata e l'utente riceve un messaggio
	 di errore.
		\begin{table}[H]
		\begin{tabularx}{\textwidth}{X | X}\toprule
			\rowcolor{orange!65}Codice gerarchico & UC4.1 \\
			Nome sintetico & Aggiunta di una nuova mappa \\
			\rowcolor{orange!65}Attore principale & Utente generico\\
			Attori secondari & Nessuno \\
			\rowcolor{orange!65}Pre-condizione & L'utente è autenticato e dispone dei permessi di accesso
			al database\\
			Post-condizione & \'E stata aggiunta al database una mappa con il nome inserito\\
			\rowcolor{orange!65}Flusso eventi & \begin{enumerate}
			\item L'utente inserisce il nome della mappa
			\end{enumerate} \\
			Scenari alternativi & \begin{enumerate}
			\item Il nome inserito è già in uso da un'altra mappa, l'utente riceve un messaggio di errore
			\end{enumerate} \\
			\rowcolor{orange!65}Lista requisiti dedotti & \\
			\bottomrule
		\end{tabularx}
		\caption{Caso d'uso UC4.1}
	 \end{table}
	  \section{Caso d'uso UC4.2: Modifica del nome di una mappa preesistente}
	 \begin{figure}[H]
			\centering
			\includegraphics[scale=0.3]{"UC4-2".png}
			\caption{Caso d'uso UC4.2}
		\end{figure}
	 \textbf{Descrizione} \\ \\
	 L'utente modifica il nome di una mappa preesistente,se il nome inserito non corrisponde a nessuna
	 mappa del database o se il nome inserito corrisponde al nome di un'altra mappa del database la
	 modifica viene impedita e l'utente riceve un messaggio di errore.
		\begin{table}[H]
		\begin{tabularx}{\textwidth}{X | X}\toprule
			\rowcolor{orange!65}Codice gerarchico & UC4.2 \\
			Nome sintetico & Modifica del nome di una mappa preesistente \\
			\rowcolor{orange!65}Attore principale & Utente generico\\
			Attori secondari & Nessuno \\
			\rowcolor{orange!65}Pre-condizione & L'utente è autenticato e dispone dei permessi di accesso
			al database\\
			Post-condizione & Il nome della mappa selezionata è stato modificato\\
			\rowcolor{orange!65}Flusso eventi & \begin{enumerate}
			\item L'utente inserisce il nome della mappa da modificare
			\item L'utente inserisce il nuovo nome della mappa
			\end{enumerate} \\
			Scenari alternativi & \begin{enumerate}
			\item Non esiste una mappa con il nome inserito, l'utente riceve un messaggio di errore
			\item Il nome inserito è già in uso da un'altra mappa, l'utente riceve un messaggio di errore
			\end{enumerate} \\
			\rowcolor{orange!65}Lista requisiti dedotti & \\
			\bottomrule
		\end{tabularx}
		\caption{Caso d'uso UC4.2}
	 \end{table}
	 \section{Caso d'uso UC4.3: Rimozione di una mappa}
	 \begin{figure}[H]
			\centering
			\includegraphics[scale=0.3]{"UC4-3".png}
			\caption{Caso d'uso UC4.3}
		\end{figure}
	 \textbf{Descrizione} \\ \\
	 L'utente inserisce il nome della mappa che vuole eliminare, se nel database non esiste una mappa con
	 quel nome l'utente riceve un messaggio di errore.
		\begin{table}[H]
		\begin{tabularx}{\textwidth}{X | X}\toprule
			\rowcolor{orange!65}Codice gerarchico & UC4.3\\
			Nome sintetico & Rimozione di una mappa\\
			\rowcolor{orange!65}Attore principale & Utente generico\\
			Attori secondari & Nessuno \\
			\rowcolor{orange!65}Pre-condizione & L'utente è autenticato e dispone dei permessi di accesso
			al database\\
			Post-condizione & La mappa selezionata è stata rimossa\\
			\rowcolor{orange!65}Flusso eventi & \begin{enumerate}
			\item L'utente inserisce il nome della mappa da rimuovere
			\end{enumerate} \\
			Scenari alternativi & \begin{enumerate}
			\item Non esiste una mappa con il nome inserito, l'utente riceve un messaggio di errore
			\end{enumerate} \\
			\rowcolor{orange!65}Lista requisiti dedotti & \\
			\bottomrule
		\end{tabularx}
		\caption{Caso d'uso UC4.3}
	 \end{table}
	 \section{Caso d'uso UC5: Operazioni sulle mappe}
	 \begin{figure}[H]
			\centering
			\includegraphics[scale=0.3]{"UC5".png}
			\caption{Caso d'uso UC5}
		\end{figure}
	 \textbf{Descrizione} \\ \\
	 L'utente esegue una delle seguenti operazioni:
	 \begin{itemize}
	 	\item Inserimento di un item
	 	\item Cancellazione di un item
	 	\item Aggiornamento di un item
	 \end{itemize}
	 Se l'operazione richiesta ha avuto esito negativo l'utente riceve un messaggio di errore.
		\begin{table}[H]
		\begin{tabularx}{\textwidth}{X | X}\toprule
			\rowcolor{orange!65}Codice gerarchico & UC5\\
			Nome sintetico & Operazioni sulle mappe\\
			\rowcolor{orange!65}Attore principale & Utente generico\\
			Attori secondari & Nessuno \\
			\rowcolor{orange!65}Pre-condizione & L'utente è autenticato e dispone dei permessi di accesso
			al database\\
			Post-condizione & L'operazione selezionata è stata eseguita\\
			\rowcolor{orange!65}Flusso eventi & \begin{enumerate}
			\item L'utente seleziona una delle operazioni
			\item Il sistema esegue l'operazione
			\end{enumerate} \\
			Scenari alternativi & \begin{enumerate}
			\item L'operazione ha avuto esito negativo, l'utente riceve un messaggio di errore
			\end{enumerate} \\
			\rowcolor{orange!65}Lista requisiti dedotti & \\
			\bottomrule
		\end{tabularx}
		\caption{Caso d'uso UC5}
	 \end{table}
	\mychapter{4}{Requisiti}
	\cleardoublepage
	\addcontentsline{toc}{chapter}{\listfigurename}
	\listoffigures
	
	\cleardoublepage
	\addcontentsline{toc}{chapter}{\listtablename}
	\listoftables
		
\end{document}
	
