%Document-Author: Bonato Paolo + Bortolazzo Matteo + Maino Elia
%Document-Date: 2016-01-16
%Document-Description: Documento di analisi dei requisiti


\documentclass[a4paper]{report}
\usepackage[english, italian]{babel}
\usepackage[T1]{fontenc}
\usepackage[utf8]{inputenc}
\usepackage{url}
\usepackage{graphicx}
\graphicspath{{Immagini/}}
\usepackage[hidelinks]{hyperref}
\usepackage{booktabs}
\usepackage{tabularx}
\usepackage{pifont}
\usepackage[table]{xcolor}
\usepackage{float}

\newcolumntype{s}{>{\hsize=.21\hsize}X}
\newcolumntype{f}{>{\hsize=.37\hsize}X}
\newcolumntype{m}{>{\hsize=.42\hsize}X}

\newcommand{\mychapter}[2]{
	\setcounter{chapter}{#1}
	\setcounter{section}{0}
	\setcounter{subsection}{1}
	\chapter*{#2}
	\addcontentsline{toc}{chapter}{#2}
}

\renewcommand{\abstractname}{Tabella contenuti}

\begin{document}
	
	\begin{titlepage}
		% Defines a new command for the horizontal lines, change thickness here
		\newcommand{\HRule}{\rule{\linewidth}{0.5mm}} 
		\center  
		
		% HEADING SECTION
		\textsc{\LARGE SweeneyThreads}\\[1.5cm] 
		\textsc{\Large Actorbase}\\[0.5cm] 
		\textsc{\large a NoSQL DB based on the Actor model}\\[0.5cm]
		
		
		% TITLE SECTION
		\HRule \\[0.4cm]
		{ \huge \bfseries Analisi dei requisiti}\\[0.4cm] 
		\HRule \\[1.5cm]
		
		% AUTHOR SECTION
		\begin{minipage}{0.4\textwidth}
			\begin{flushleft} \large
				\emph{Redattori:}\\
				Bonato Paolo \\
				Bortolazzo Matteo \\
				Maino Elia
			\end{flushleft}
		\end{minipage}
		~
		\begin{minipage}{0.4\textwidth}
			\begin{flushright} \large
				\emph{Approvazione:} \\
				\dots \\
				\emph{Verifica:} \\
				\dots
			\end{flushright}
		\end{minipage}
		
		%immagine
		\begin{figure}[H]
			\centering
			\includegraphics[scale=0.8]{sweeney.png}
		\end{figure}
		\begin{center}
			Versione 1.0.0
		\end{center}
		% Date, change the \today to a set date if you want to be precise
		{\large \today}\\[3cm] 
		% Fill the rest of the page with whitespace
		\vfill  
	\end{titlepage}
	
	
	\tableofcontents
	
	\mychapter{0}{Diario delle modifiche}
		\begin{table}[H]
			\begin{tabularx}{\textwidth}{s f m X}
				\noalign{\hrule height 1.5pt}
				\rowcolor{orange!85} Versione & Data & Autore & Descrizione \\
				1.0.2 & 2016-01-17 & \emph{Analista} Maino Elia & Stesura primo insiem di casi d'uso (da 1.x a 4.x) \\
				\noalign{\hrule height 0.5pt}
				1.0.1 & 2016-01-17 & \emph{Analista} Maino Elia & Stesura caratteristiche generali del prodotto \\
				\noalign{\hrule height 0.5pt}
				1.0.0 & 2016-01-17 & \emph{Analista} Maino Elia & Creazione scheletro documento e stesura introduzione \\
				\noalign{\hrule height 1.5pt}
			\end{tabularx}
			\caption{Diario delle modifiche \label{tab:table_label}}
		\end{table}

	\mychapter{1}{Introduzione}
	\section{Scopo del documento}
		Lo scopo del seguente documento è presentare l'insieme di requisiti individuati dal gruppo 
		SWEeneyThreads durante l'analisi del \emph{Capitolato C1}, gli incontri con il committente 
		\emph{Cardin Riccardo} e l'analisi dei casi d'uso. 
		
		Si intende fornire una rappresentazione ordinata delle funzionalità che il prodotto \emph{Actorbase} 
		offrirà al momento del rilascio.
	\section{Scopo del prodotto}
		Lo scopo del progetto è la realizzazione di un DataBase NoSQL key-value basato sul modello ad 
		Attori\ped{\textit{G}} con l'obiettivo di fornire una tecnologia adatta allo sviluppo di moderne 
		applicazioni che richiedono brevissimi tempi di risposta e che elaborano enormi quantità 
		di dati. Lo sviluppo porterà al rilascio del software sotto licenza MIT.
	\section{Glossario}
		Con lo scopo di evitare ambiguità di linguaggio e di massimizzare la comprensione dei documenti, il 
		gruppo ha steso un documento interno che è il \emph{Glossario v1.0.0}. La prima occorrenza
		di ogni termine termine contenuto nel \emph{Glossario} e presente in questo documento verrà 
		marcato con una "\textit{G}" maiuscola in pedice.
	\section{Riferimenti}
	\subsection{Informativi}	
		\dots
	\subsection{Normativi}
		\begin{itemize}
			\item Capitolato d'appalto Actorbase (C1): \\ 
			\url{http://www.math.unipd.it/~tullio/IS-1/2015/Progetto/C1p.pdf}
		\end{itemize}
	\mychapter{2}{Caratteristiche generali del prodotto}
	\section{Obiettivi del prodotto}
		Il prodotto si pone l'obiettivo di fornire un database NoSQL basato sul modello ad attori che possa 
		essere utilizzato con successo dalle cosiddette \emph{Reactive Applications}, ovvero applicazioni
		orientate agli eventi, scalabili, resilienti e responsive. Per tali applicazioni il paradigma di accesso 
		ai dati basato su database di tipo relazionale risulta non applicabile, poichè troppo limitante, sono 
		dunque necessarie nuove forme di gestione dell'informazione, \emph{Actorbase} intende proporsi 
		come una valida opzione.
	\section{Funzioni principali del prodotto}
		\emph{Actorbase} prevede un'interazione con l'utente da riga di comando attraverso l'uso di un 
		\emph{Domain Specific Language (DSL)}. 
		Oltre alla creazione di uno schema sarà possibile effettuare le seguenti operazioni:
		\begin{enumerate}
			\item Inserimento
			\item Cancellazione 
			\item Aggiornamento (caso particolare di un inserimento con chiave già presente)
		\end{enumerate}
	\section{Target d'utenza}
		Il prodotto si rivolge a sviluppatori di applicazioni moderne, che trattano enormi moli di dati (nell'
		ordine dei Petabyte), che richiedono brevissimi tempi di risposta e che necessitano di un uptime 
		del 100\%.
	\section{Vincoli per l'utilizzo}
		L'utilizzo di \emph{Actorbase} non richiede hardware o software particolare, sebbene l'esecuzione 
		su macchine non recenti possa influenzare le prestazioni.	
	
	\mychapter{3}{Casi d'uso}
		Di seguito viene riportata la descrizione accurata di tutti i casi d'uso individuati dal gruppo a seguito
		delle seguenti attività:
		\begin{itemize}
			\item Analisi del capitolato C1 Actorbase
			\item Confronti con il committente \emph{Cardin Riccardo}
			\item Riunioni e discussioni interne al gruppo
			\item Analisi della struttura e delle funzionalità di altri database non realzionali
		\end{itemize}
		La struttura di un caso d'uso è definita nel documento \emph{Norme di progetto v1.1.1 sez 2.1.3}.
		\newpage
		\section{Visione ad alto livello delle operazioni sul sistema}
		 	\begin{figure}[H]
				\centering
				\includegraphics[scale=0.35]{"UC0".png}
				\caption{Operazioni principali ad alto livello}
			\end{figure}
			Il diagramma in \emph{figura 3.1} illustra le principali operazioni che un utente esterno può
			 effettuare sul sistema:
			\begin{itemize}
				\item Installazione dell'applicativo
				\item Connessione ad un server e autenticazione allo stesso
				\item Modifica delle credenziali di accesso
				\item Operazioni sugli schemi del database
				\item Operazioni sulle tabelle del database
				\item Operazioni sui dati del database
			\end{itemize}
		\section{Caso d'uso UC1: Installazione}
			\begin{figure}[H]
				\centering
				\includegraphics[scale=0.3]{"UC1".png}
				\caption{Caso d'uso UC1}
			\end{figure}
		 \textbf{Descrizione} \\ \\
		 L'utente è in possesso dei file necessari a far partire il programma di setup. Una volta avviato
		  l'utente può scegliere tra tre modalità d'installazione: 
		 \begin{itemize}
		 	\item Installazione completa
		 	\item Installazione server
		 	\item Installazione client
		 \end{itemize}
		 A seconda dell'opzione selezionata il programma di setup installa le relative componenti. Durante 
		 l'installazione possono occorrere errori che ne impediscano il completamento, in tal caso il setup
		 informa l'utente e interrompe l'installazione.
			\begin{table}[H]
			\begin{tabularx}{\textwidth}{X | X}\toprule
				\rowcolor{orange!65}Codice gerarchico & UC1 \\
				Nome sintetico & Installazione \\
				\rowcolor{orange!65}Attore principale & Utente generico\\
				Attori secondari & Nessuno \\
				\rowcolor{orange!65}Pre-condizione & L'utente è in possesso del programma \\
				Post-condizione & Le componenti selezionate risultano installate sulla
				 macchina dell'utente \\
				\rowcolor{orange!65}Flusso eventi & \begin{enumerate}
				\item L'utente lancia il programma di setup
				\item L'utente seleziona le componenti da installare
				\item Il programma di setup installa le componenti selezionate
				\end{enumerate} \\
				Scenari alternativi & \begin{enumerate}
				\item Errore durante l'installazione, l'utente riceve un messaggio
				\end{enumerate} \\
				\rowcolor{orange!65}Lista requisiti dedotti & \\
				\bottomrule
			\end{tabularx}
			\caption{Caso d'uso UC1}
		 \end{table}
	\section{Caso d'uso UC2: Connessione ed autenticazione}
		 	\begin{figure}[H]
				\centering
				\includegraphics[scale=0.3]{"UC2".png}
				\caption{Caso d'uso UC2}
			\end{figure}
		 \textbf{Descrizione} \\ \\
		 L'utente ha appena avviato l'applicativo e intende accedere al sistema. L'utente 
		 inserisce il nome del server e le proprie credenziali di accesso. Il sistema tenta di connettersi a tale
		 server. Se la connessione va a buon fine l'utente si trova effettivamente connesso al server
		 desiderato, altrimenti riceve un messaggio di errore che lo informa del fallimento della connessione.
		 Una volta stabilita la connessione il sistema verifica le credenziali di accesso, se le credenziali sono
		 corrette l'utente risulta connesso a tutti gli effetti, se sono errate riceve un relativo messaggio di 
		 errore.
			\begin{table}[H]
			\begin{tabularx}{\textwidth}{X | X}\toprule
				\rowcolor{orange!65}Codice gerarchico & UC2 \\
				Nome sintetico & Connessione ad un server \\
				\rowcolor{orange!65}Attore principale & Utente generico\\
				Attori secondari & Nessuno \\
				\rowcolor{orange!65}Pre-condizione & L'utente ha avviato il programma e vuole accedere al
				 sistema\\
				Post-condizione & L'utente risulta connesso al server selezionato \\
				\rowcolor{orange!65}Flusso eventi & \begin{enumerate}
				\item L'utente inserisce il nome del server a cui intende connettersi
				\item La connessione riesce
				\item L'utente inserisce le proprie credenziali
				\item Il sistema verifica le credenziali inserite
				\item L'utente risulta connesso
				\end{enumerate} \\
				Scenari alternativi & \begin{enumerate}
				\item Errore durante la connessione, l'utente riceve un messaggio
				\item Credenziali errate, l'utente riceve un messaggio
				\end{enumerate} \\
				\rowcolor{orange!65}Lista requisiti dedotti & \\
				\bottomrule
			\end{tabularx}
			\caption{Caso d'uso UC2}
		 \end{table}
	\section{Caso d'uso UC3: Modifica delle credenziali}
	 	\begin{figure}[H]
			\centering
			\includegraphics[scale=0.4]{"UC3".png}
			\caption{Caso d'uso UC3}
		\end{figure}
	 \textbf{Descrizione} \\ \\
	 L'utente ha già eseguito l'autenticazione e ora desidera modificare le credenziali con cui effettuare 
	 l'accesso al database. Le credenziali modificabili sono il nome utente e la password. Al termine delle
	 modifiche le credenziali di accesso contengono i valori aggiornati. Se la modifica non è andata a buon
	 fine l'utente riceve un messaggio di errore e le credenziali di accesso mantengono i valori precedenti.
		\begin{table}[H]
		\begin{tabularx}{\textwidth}{X | X}\toprule
			\rowcolor{orange!65}Codice gerarchico & UC3 \\
			Nome sintetico & Modifica delle credenziali \\
			\rowcolor{orange!65}Attore principale & Utente generico\\
			Attori secondari & Nessuno \\
			\rowcolor{orange!65}Pre-condizione & L'utente è autenticato \\
			Post-condizione & Le credenziali dell'utente risultano modificate \\
			\rowcolor{orange!65}Flusso eventi & \begin{enumerate}
			\item L'utente fornisce le nuove credenziali
			\item Le credenziali vengono aggiornate ai nuovi valori
			\end{enumerate} \\
			Scenari alternativi & \begin{enumerate}
			\item Le modifiche non vengono applicate, l'utente riceve un messaggio di errore
			\end{enumerate} \\
			\rowcolor{orange!65}Lista requisiti dedotti & \\
			\bottomrule
		\end{tabularx}
		\caption{Caso d'uso UC3}
	 \end{table}
	 \section{Caso d'uso UC3.1: Modifica username}
	 	\begin{figure}[H]
			\centering
			\includegraphics[scale=0.4]{"UC3-1".png}
			\caption{Caso d'uso UC3.1}
		\end{figure}
	 \textbf{Descrizione} \\ \\
	 L'utente ha selezionato l'opzione "modifica username" dal menu di modifica, inserisce un nuovo valore
	 per l'username, tale valore viene analizzato (può contenere caratteri non validi o può essere già 
	 in uso). Se il nuovo valore risulta valido le modifiche vengono completate altrimenti l'utente riceve 
	 un messaggio di errore che lo informa che i dati da lui inseriti non sono validi e dunque le modifiche non
	 sono state effettuate.
		\begin{table}[H]
		\begin{tabularx}{\textwidth}{X | X}\toprule
			\rowcolor{orange!65}Codice gerarchico & UC3.1 \\
			Nome sintetico & Modifica username \\
			\rowcolor{orange!65}Attore principale & Utente generico\\
			Attori secondari & Nessuno \\
			\rowcolor{orange!65}Pre-condizione & L'utente è autenticato ed ha selezionato "modifica
			 username" dalle opzioni di modifica\\
			Post-condizione & L'username dell'utente viene modificato \\
			\rowcolor{orange!65}Flusso eventi & \begin{enumerate}
			\item L'utente inserisce il nuovo valore per l'username
			\item Il nuovo valore viene verificato dal sistema
			\item L'username contiene il nuovo valore
			\end{enumerate} \\
			Scenari alternativi & \begin{enumerate}
			\item Il nuovo valore risulta non valido, l'utente riceve un messaggio di errore
			\end{enumerate} \\
			\rowcolor{orange!65}Lista requisiti dedotti & \\
			\bottomrule
		\end{tabularx}
		\caption{Caso d'uso UC3.1}
	 \end{table}
	 \section{Caso d'uso UC3.2: Modifica password}
	 	\begin{figure}[H]
			\centering
			\includegraphics[scale=0.4]{"UC3-2".png}
			\caption{Caso d'uso UC3.2}
		\end{figure}
	 \textbf{Descrizione} \\ \\
	 L'utente ha selezionato l'opzione "modifica password" dal menu di modifica, inserisce un nuovo valore
	 per la password, tale valore viene analizzato (può contenere caratteri non validi). 
	 Se il nuovo valore risulta valido le modifiche vengono completate altrimenti l'utente riceve 
	 un messaggio di errore che lo informa che i dati da lui inseriti non sono validi e dunque le modifiche non
	 sono state effettuate.
		\begin{table}[H]
		\begin{tabularx}{\textwidth}{X | X}\toprule
			\rowcolor{orange!65}Codice gerarchico & UC3.1 \\
			Nome sintetico & Modifica password \\
			\rowcolor{orange!65}Attore principale & Utente generico\\
			Attori secondari & Nessuno \\
			\rowcolor{orange!65}Pre-condizione & L'utente è autenticato ed ha selezionato "modifica
			 password" dalle opzioni di modifica\\
			Post-condizione & La password dell'utente viene modificata \\
			\rowcolor{orange!65}Flusso eventi & \begin{enumerate}
			\item L'utente inserisce il nuovo valore per la password
			\item Il nuovo valore viene verificato dal sistema
			\item La password contiene il nuovo valore
			\end{enumerate} \\
			Scenari alternativi & \begin{enumerate}
			\item Il nuovo valore risulta non valido, l'utente riceve un messaggio di errore
			\end{enumerate} \\
			\rowcolor{orange!65}Lista requisiti dedotti & \\
			\bottomrule
		\end{tabularx}
		\caption{Caso d'uso UC3.2}
	 \end{table}
	 \section{Caso d'uso UC4: Operazioni sugli schemi}
	 	\begin{figure}[H]
			\centering
			\includegraphics[scale=0.3]{"UC4".png}
			\caption{Caso d'uso UC4}
		\end{figure}
	 \textbf{Descrizione} \\ \\
	 L'utente è autenticato e desidera effettuare delle operazioni di modifica sugli schemi del database, 
	 alcune operazioni sono permesse solo all'utente amministratore dunque nel caso un utente standard
	 richiedesse al sistema di eseguire una di queste operazioni riceverebbe un messaggio di errore.
	 Un'operazione può avere due esiti: successo o fallimento, nel secondo caso l'utente riceve un
	  messaggio d'errore che spiega il perchè del fallimento.
		\begin{table}[H]
		\begin{tabularx}{\textwidth}{X | X}\toprule
			\rowcolor{orange!65}Codice gerarchico & UC4 \\
			Nome sintetico & Operazioni sugli schemi \\
			\rowcolor{orange!65}Attore principale & Utente generico, Utente amministratore\\
			Attori secondari & Nessuno \\
			\rowcolor{orange!65}Pre-condizione & L'utente è autenticato\\
			Post-condizione & Le operazioni selezionate risultano effettuate \\
			\rowcolor{orange!65}Flusso eventi & \begin{enumerate}
			\item L'utente rischiede una delle operazioni di modifica
			\item L'operazione rischiesta viene effettuata dal sistema
			\end{enumerate} \\
			Scenari alternativi & \begin{enumerate}
			\item L'utente non dispone dei permessi necessari ad effettuare le modifiche richieste, l'utente
			riceve un messaggio di errore
			\item L'operazione richiesta non è stata effettuata, l'utente riceve un messaggio di errore
			\end{enumerate} \\
			\rowcolor{orange!65}Lista requisiti dedotti & \\
			\bottomrule
		\end{tabularx}
		\caption{Caso d'uso UC4}
	 \end{table}
	 \section{Caso d'uso UC4.1: Selezione schema}
	 	\begin{figure}[H]
			\centering
			\includegraphics[scale=0.3]{"UC4-1".png}
			\caption{Caso d'uso UC4.1}
		\end{figure}
	 \textbf{Descrizione} \\ \\
	 L'utente seleziona uno schema da una lista di schemi a cui ha accesso, lo schema scelto risulta
	  selezionato.
		\begin{table}[H]
		\begin{tabularx}{\textwidth}{X | X}\toprule
			\rowcolor{orange!65}Codice gerarchico & UC4.1 \\
			Nome sintetico & Operazioni sugli schemi \\
			\rowcolor{orange!65}Attore principale & Utente generico\\
			Attori secondari & Nessuno \\
			\rowcolor{orange!65}Pre-condizione & L'utente è autenticato\\
			Post-condizione & Lo schema risulta selezionato \\
			\rowcolor{orange!65}Flusso eventi & \begin{enumerate}
			\item L'utente seleziona uno schema tra quelli a cui ha accesso
			\item Lo schema viene selezionato
			\end{enumerate} \\
			Scenari alternativi &  Nessuno \\
			\rowcolor{orange!65}Lista requisiti dedotti & \\
			\bottomrule
		\end{tabularx}
		\caption{Caso d'uso UC4.1}
	 \end{table}
	 \section{Caso d'uso UC4.2: Creazione schema}
	 	\begin{figure}[H]
			\centering
			\includegraphics[scale=0.3]{"UC4-2".png}
			\caption{Caso d'uso UC4.2}
		\end{figure}
	 \textbf{Descrizione} \\ \\
	 L'utente è autenticato e desidera creare uno schema da zero, inserendo un nome. L'utente inserisce il
	 nome dello schema che vuole creare, il sistema verifica che il nome inserito non violi vincoli di validità e
	 successivamente crea lo schema rendendone amministratore l'utente.
		\begin{table}[H]
		\begin{tabularx}{\textwidth}{X | X}\toprule
			\rowcolor{orange!65}Codice gerarchico & UC4.2 \\
			Nome sintetico & Creazione schema \\
			\rowcolor{orange!65}Attore principale & Utente generico\\
			Attori secondari & Nessuno \\
			\rowcolor{orange!65}Pre-condizione & L'utente è autenticato\\
			Post-condizione & Viene creato uno schema il cui nome è quello inserito dall'utente, l'utente
			risulta amministratore dello schema \\
			\rowcolor{orange!65}Flusso eventi & \begin{enumerate}
			\item L'utente inserisce un nome per lo schema
			\item Il sistema verifica il nome inserito
			\item Viene creato uno schema con il nome inserito di cui l'utente è ora amministratore
			\end{enumerate} \\
			Scenari alternativi & \begin{enumerate}
			\item Il nome inserito non risulta valido, l'utente riceve un messaggio di errore e lo schema non
			 viene creato
			\end{enumerate} \\
			\rowcolor{orange!65}Lista requisiti dedotti & \\
			\bottomrule
		\end{tabularx}
		\caption{Caso d'uso UC4}
	 \end{table}
	 \section{Caso d'uso UC4.3: Importazione schema}
	 	\begin{figure}[H]
			\centering
			\includegraphics[scale=0.3]{"UC4-3".png}
			\caption{Caso d'uso UC4.3}
		\end{figure}
	 \textbf{Descrizione} \\ \\
	 L'utente desidera creare un nuovo schema importandolo da un file presente sul computer. L'utente
	  deve specificare il percorso del file. Il sistema ricerca il file nel percorso dato, se lo trova lo apre e ne 
	  legge il contenuto. In base al contenuto il sistema crea un nuovo schema rendendone amministratore
	  l'utente.
		\begin{table}[H]
		\begin{tabularx}{\textwidth}{X | X}\toprule
			\rowcolor{orange!65}Codice gerarchico & UC4.3 \\
			Nome sintetico & Importazione schema \\
			\rowcolor{orange!65}Attore principale & Utente generico\\
			Attori secondari & Sistema di I/O \\
			\rowcolor{orange!65}Pre-condizione & L'utente è autenticato\\
			Post-condizione & \'E stato importato un nuovo schema uguale a quello presente sul file di
			 importazione di cui l'utente è ora amministratore \\
			\rowcolor{orange!65}Flusso eventi & \begin{enumerate}
			\item L'utente inserisce il percorso del file da cui importare lo schema
			\item Il sistema verifica il contenuto del file
			\item Il sistema crea uno schema basato sui dati contenuti nel file
			\end{enumerate} \\
			Scenari alternativi & \begin{enumerate}
			\item Il file non viene trovato
			\item Il contenuto del file risulta illeggibile
			\item Lo schema contenuto nel file non può essere creato perchè viola i vincoli del database
			\end{enumerate} \\
			\rowcolor{orange!65}Lista requisiti dedotti & \\
			\bottomrule
		\end{tabularx}
		\caption{Caso d'uso UC4.3}
	 \end{table}
	 \section{Caso d'uso UC4.4: Esportazione schema}
	 	\begin{figure}[H]
			\centering
			\includegraphics[scale=0.28]{"UC4-4".png}
			\caption{Caso d'uso UC4.4}
		\end{figure}
	 \textbf{Descrizione} \\ \\
	 L'utente desidera esportare uno schema presente sul database su un file. L'utente seleziona lo schema
	 che vuole esportare, specifica il percorso dove il file verrà creato e il nome del file. Il sistema crea un
	 file contenente i dati dello schema selezionato. Se la scrittura del file non avviene con successo l'utente
	 riceve un messaggio di errore.
		\begin{table}[H]
		\begin{tabularx}{\textwidth}{X | X}\toprule
			\rowcolor{orange!65}Codice gerarchico & UC4.4 \\
			Nome sintetico & Esportazione schema \\
			\rowcolor{orange!65}Attore principale & Utente generico\\
			Attori secondari & Sistema di I/O \\
			\rowcolor{orange!65}Pre-condizione & L'utente è autenticato\\
			Post-condizione & Lo schema selezionato viene esportato su un file nella macchina dell'utente \\
			\rowcolor{orange!65}Flusso eventi & \begin{enumerate}
			\item L'utente seleziona lo schema
			\item L'utente specifica il percorso del file
			\item L'utente inserisce il nome del file
			\item Il sistema crea il file contenente i dati dello schema
			\end{enumerate} \\
			Scenari alternativi & \begin{enumerate}
			\item La creazione del file non ha successo
			\end{enumerate} \\
			\rowcolor{orange!65}Lista requisiti dedotti & \\
			\bottomrule
		\end{tabularx}
		\caption{Caso d'uso UC4}
	 \end{table}
	\section{Caso d'uso UC4.5: Rimozione schema}
	 	\begin{figure}[H]
			\centering
			\includegraphics[scale=0.3]{"UC4-5".png}
			\caption{Caso d'uso UC4.5}
		\end{figure}
	 \textbf{Descrizione} \\ \\
	 L'amministratore seleziona uno degli schemi disponibili e ne richiede l'eliminazione. Il sistema elimina
	 lo schema selezionato, se l'eliminazione non risulta possibile l'utente riceve un relativo messaggio di 
	 errore.
		\begin{table}[H]
		\begin{tabularx}{\textwidth}{X | X}\toprule
			\rowcolor{orange!65}Codice gerarchico & UC4.5 \\
			Nome sintetico & Rimozione schema \\
			\rowcolor{orange!65}Attore principale & Utente amministratore\\
			Attori secondari & Nessuno \\
			\rowcolor{orange!65}Pre-condizione & L'utente è autenticato\\
			Post-condizione & Lo schema selezionato è stato rimosso \\
			\rowcolor{orange!65}Flusso eventi & \begin{enumerate}
			\item L'utente seleziona uno degli schemi di cui è amministratore
			\item Il sistema rimuove lo schema selezionato
			\end{enumerate} \\
			Scenari alternativi & \begin{enumerate}
			\item \'E impossibile rimuovere lo schema selezionato, l'utente riceve un messaggio di errore
			\end{enumerate} \\
			\rowcolor{orange!65}Lista requisiti dedotti & \\
			\bottomrule
		\end{tabularx}
		\caption{Caso d'uso UC4.5}
	 \end{table}
	 \section{Caso d'uso UC4.6: Gestione permessi schema}
	 	\begin{figure}[H]
			\centering
			\includegraphics[scale=0.3]{"UC4-6".png}
			\caption{Caso d'uso UC4.6}
		\end{figure}
	 \textbf{Descrizione} \\ \\
	 L'amministratore seleziona uno degli schemi disponibili e può:
	 \begin{itemize}
	 	\item Aggiungere un nuovo utente amministratore
	 	\item Aggiungere un nuovo utente collaboratore
	 	\item Modificare i permessi degli utenti che possono già accedere allo schema
	 	\item Rimuovere il permesso di accesso di un utente allo schema
	 \end{itemize}
	 Ognuna di queste operazioni può avere esito negativo, in tal caso l'amministratore dello schema che ha
	 richiesto l'operazione riceve un messaggio d'errore esplicativo.
		\begin{table}[H]
		\begin{tabularx}{\textwidth}{X | X}\toprule
			\rowcolor{orange!65}Codice gerarchico & UC4.6 \\
			Nome sintetico & Gestione permessi schema \\
			\rowcolor{orange!65}Attore principale & Utente amministratore\\
			Attori secondari & Nessuno \\
			\rowcolor{orange!65}Pre-condizione & L'utente è autenticato\\
			Post-condizione & L'operazione di modifica dei permessi è eseguita correttamente \\
			\rowcolor{orange!65}Flusso eventi & \begin{enumerate}
			\item L'utente seleziona uno degli schemi di cui è amministratore
			\item L'utente seleziona l'operazione che vuole effettuare
			\item L'operazione viene effettuata dal sistema
			\end{enumerate} \\
			Scenari alternativi & \begin{enumerate}
			\item \'E impossibile completare l'operazione richiesta, l'utente riceve un messaggio di errore
			\end{enumerate} \\
			\rowcolor{orange!65}Lista requisiti dedotti & \\
			\bottomrule
		\end{tabularx}
		\caption{Caso d'uso UC4.6}
	 \end{table}
	 \section{Caso d'uso UC4.6.1: Aggiunta di un amministratore allo schema}
	 \textbf{Descrizione} \\ \\
	 L'amministratore seleziona un utente e lo rende amministratore dello schema. Nel caso l'utente 
	 selezionato non risultasse valido l'operazione non viene eseguita e l'amministratore riceve un 
	 messaggio di errore.
		\begin{table}[H]
		\begin{tabularx}{\textwidth}{X | X}\toprule
			\rowcolor{orange!65}Codice gerarchico & UC4.6.1 \\
			Nome sintetico & Aggiunta di un amministratore allo schema \\
			\rowcolor{orange!65}Attore principale & Utente amministratore\\
			Attori secondari & Nessuno \\
			\rowcolor{orange!65}Pre-condizione & L'utente è amministratore dello schema \\
			Post-condizione & L'utente selezionato viene aggiunto come amministratore allo schema \\
			\rowcolor{orange!65}Flusso eventi & \begin{enumerate}
			\item L'utente seleziona un utente
			\item Il sistema verifica che l'utente sia valido
			\item L'utente selezionato viene aggiunto come amministratore
			\end{enumerate} \\
			Scenari alternativi & \begin{enumerate}
			\item L'utente selezionato non risulta valido, l'utente riceve un messaggio di errore
			\end{enumerate} \\
			\rowcolor{orange!65}Lista requisiti dedotti & \\
			\bottomrule
		\end{tabularx}
		\caption{Caso d'uso UC4.6.1}
	 \end{table}
	 \section{Caso d'uso UC4.6.2: Aggiunta di un collaboratore allo schema}
	 \textbf{Descrizione} \\ \\
	 L'amministratore seleziona un utente e lo rende collaboratore dello schema. Nel caso l'utente 
	 selezionato non risultasse valido l'operazione non viene eseguita e l'amministratore riceve un 
	 messaggio di errore.
		\begin{table}[H]
		\begin{tabularx}{\textwidth}{X | X}\toprule
			\rowcolor{orange!65}Codice gerarchico & UC4.6.1 \\
			Nome sintetico & Aggiunta di un collaboratore allo schema \\
			\rowcolor{orange!65}Attore principale & Utente amministratore\\
			Attori secondari & Nessuno \\
			\rowcolor{orange!65}Pre-condizione & L'utente è amministratore dello schema \\
			Post-condizione & L'utente selezionato viene aggiunto come collaboratore allo schema \\
			\rowcolor{orange!65}Flusso eventi & \begin{enumerate}
			\item L'utente seleziona un utente
			\item Il sistema verifica che l'utente sia valido
			\item L'utente selezionato viene aggiunto come collaboratore
			\end{enumerate} \\
			Scenari alternativi & \begin{enumerate}
			\item L'utente selezionato non risulta valido, l'utente riceve un messaggio di errore
			\end{enumerate} \\
			\rowcolor{orange!65}Lista requisiti dedotti & \\
			\bottomrule
		\end{tabularx}
		\caption{Caso d'uso UC4.6.2}
	 \end{table}
	 \section{Caso d'uso UC4.6.3: Modifica dei privilegi di accesso}
	 \textbf{Descrizione} \\ \\
	 L'amministratore seleziona un utente che ha già accesso allo schema e ne modifica i privilegi. 
	 Nel caso l'utente selezionato non risultasse valido l'operazione non viene eseguita e l'amministratore
	  riceve un messaggio di errore.
		\begin{table}[H]
		\begin{tabularx}{\textwidth}{X | X}\toprule
			\rowcolor{orange!65}Codice gerarchico & UC4.6.3 \\
			Nome sintetico & Modifica dei privilegi di accesso \\
			\rowcolor{orange!65}Attore principale & Utente amministratore\\
			Attori secondari & Nessuno \\
			\rowcolor{orange!65}Pre-condizione & L'utente è amministratore dello schema \\
			Post-condizione & I privilegi di accesso allo schema dell'utente selezionato vengono modificati \\
			\rowcolor{orange!65}Flusso eventi & \begin{enumerate}
			\item L'amministratore seleziona un utente
			\item Il sistema verifica che l'utente sia valido
			\item L'amministratore sceglie i privilegi da dare all'utente
			\item L'utente ottiene i privilegi selezionati
			\end{enumerate} \\
			Scenari alternativi & \begin{enumerate}
			\item L'utente selezionato non risulta avere già accesso allo schema, l'utente riceve un messaggio
			 di errore e l'operazione non viene eseguita
			 \item L'utente selezionato è già amministratore dello schema, l'operazione non viene eseguita e
			 L'amministratore che ha richiesto la modifica riceve un messaggio di errore
			\end{enumerate} \\
			\rowcolor{orange!65}Lista requisiti dedotti & \\
			\bottomrule
		\end{tabularx}
		\caption{Caso d'uso UC4.6.3}
	 \end{table}
	 \section{Caso d'uso UC4.6.4: Rimozione di un utente dallo schema}
	 \textbf{Descrizione} \\ \\
	 L'amministratore seleziona un utente che ha già accesso allo schema e lo rimuove dallo schema. 
	 Nel caso l'utente selezionato non risultasse valido l'operazione non viene eseguita e l'amministratore
	  riceve un messaggio di errore.
		\begin{table}[H]
		\begin{tabularx}{\textwidth}{X | X}\toprule
			\rowcolor{orange!65}Codice gerarchico & UC4.6.4 \\
			Nome sintetico & Rimozione di un utente dallo schema \\
			\rowcolor{orange!65}Attore principale & Utente amministratore\\
			Attori secondari & Nessuno \\
			\rowcolor{orange!65}Pre-condizione & L'utente è amministratore dello schema \\
			Post-condizione & L'utente selezionato viene rimosso dallo schema\\
			\rowcolor{orange!65}Flusso eventi & \begin{enumerate}
			\item L'amministratore seleziona un utente
			\item Il sistema verifica che l'utente sia valido
			\item L'utente selezionato viene rimosso dallo schema
			\end{enumerate} \\
			Scenari alternativi & \begin{enumerate}
			\item L'utente selezionato non risulta avere già accesso allo schema, l'utente riceve un messaggio
			 di errore e l'operazione non viene eseguita
			 \item L'utente selezionato è già amministratore dello schema, l'operazione non viene eseguita e
			 L'amministratore che ha richiesto la modifica riceve un messaggio di errore
			\end{enumerate} \\
			\rowcolor{orange!65}Lista requisiti dedotti & \\
			\bottomrule
		\end{tabularx}
		\caption{Caso d'uso UC4.6.4}
	 \end{table}
	\mychapter{4}{Requisiti}
	\cleardoublepage
	\addcontentsline{toc}{chapter}{\listfigurename}
	\listoffigures
	
	\cleardoublepage
	\addcontentsline{toc}{chapter}{\listtablename}
	\listoftables
		
\end{document}
	
