%Document-Author: Bonato Paolo + Bortolazzo Matteo + Maino Elia
%Document-Date: 2016/01/13
%Document-Description: Documento di Analisi dei requisiti del gruppo SWEeneyThreads 

\documentclass[a4paper]{article}
\usepackage[english, italian]{babel}
\usepackage[T1]{fontenc}
\usepackage[utf8]{inputenc}
\usepackage{url}
\usepackage{graphicx}
\usepackage[hidelinks]{hyperref}
\usepackage{booktabs}
\usepackage{eurosym}
\usepackage{tabularx}
\usepackage{pifont}
\usepackage[table]{xcolor}
\usepackage{float}
\usepackage[]{appendix}
\usepackage{ltxtable} 
\usepackage{geometry}
\geometry{margin=1in}
\usepackage{longtable}
\usepackage{multirow}

\graphicspath{{Immagini/}}

\newcolumntype{Y}{>{\centering\arraybackslash}X}
\newcolumntype{s}{>{\hsize=.21\hsize}X}
\newcolumntype{f}{>{\hsize=.37\hsize}X}
\newcolumntype{m}{>{\hsize=.42\hsize}X}
\newcolumntype{t}{>{\hsize=.1\hsize}X}
\newcolumntype{r}{>{\hsize=.3\hsize}X}
\newcolumntype{k}{>{\hsize=.4\hsize}X}

\renewcommand{\abstractname}{Tabella contenuti}

\begin{document}
	
	\begin{titlepage}
		% Defines a new command for the horizontal lines, change thickness here
		\newcommand{\HRule}{\rule{\linewidth}{0.5mm}} 
		\center  
		
		% HEADING SECTION
		\textsc{\LARGE SWEeneyThreads}\\[1.5cm] 
		\textsc{\Large Actorbase}\\[0.5cm] 
		\textsc{\large a NoSQL DB based on the Actor model}\\[0.5cm]
		
		
		% TITLE SECTION
		\HRule \\[0.4cm]
		{ \huge \bfseries Analisi dei requisiti}\\[0.4cm] 
		\HRule \\[1.5cm]
		
		% AUTHOR SECTION
		\begin{minipage}{0.4\textwidth}
			\begin{flushleft} \large
				\emph{Redattori:}\\
				Maino Elia \\
				Tommasin Davide
			\end{flushleft}
		\end{minipage}
		~
		\begin{minipage}{0.4\textwidth}
			\begin{flushright} \large
				\emph{Approvazione:} \\
				Biggeri Mattia \\
				\emph{Verifica:} \\
				Bonato Paolo 
			\end{flushright}
		\end{minipage}
		
		%immagine
		\begin{figure}[H]
			\centering
			\includegraphics[scale=0.8]{sweeney.png}
		\end{figure}
		\begin{center}
			Versione 1.3.4
		\end{center}
		% Date, change the \today to a set date if you want to be precise
		{\large \today}\\[3cm] 
		% Fill the rest of the page with whitespace
		\vfill  
	\end{titlepage}
	
	
	\tableofcontents
	
	\newpage 
	\section*{Diario delle modifiche}
		\begin{table}[H]
			\begin{tabularx}{\textwidth}{s f m X}
				\noalign{\hrule height 1.5pt}
				\rowcolor{orange!85} Versione & Data & Autore & Descrizione \\
				\noalign{\hrule height 0.5pt}
				1.3.4 & 2016-03-17 & \emph{Progettista} \newline Maino Elia & Ridefinizione dei casi d'uso relativi alle operazioni ad alto livello lato client e lato server
				\\
				\noalign{\hrule height 0.5pt}
				1.3.3 & 2016-03-15 & \emph{Progettista} \newline Maino Elia & Ridefinizione dei casi d'uso: stesura di diagrammi e descrizioni da UC 4.12 a UC 7
				\\
				\noalign{\hrule height 0.5pt}
				1.3.2 & 2016-03-14 & \emph{Progettista} \newline Maino Elia \newline \emph{Analista} \newline Tommasin Davide & Ridefinizione dei casi d'uso: stesura di diagrammi e descrizioni da UC 3.12 a UC 4.12
				\\
				\noalign{\hrule height 0.5pt}
				1.3.1 & 2016-03-06 & \emph{Analista} \newline Maino Elia & Ridefinizione dei casi d'uso: stesura di diagrammi e descrizioni da UC 3.2 a UC 3.11
				\\
				\noalign{\hrule height 0.5pt}
				1.3.0 & 2016-03-05 & \emph{Analista} \newline Maino Elia & Ridefinizione dei casi d'uso: stesura di diagrammi e descrizioni fino a UC 3.1
				 \\
				\noalign{\hrule height 0.5pt}
				1.2.0 & 2016-01-21 & \emph{Responsabile} \newline Padovan Tommaso  & Approvazione documento
				 \\
				 \noalign{\hrule height 0.5pt}
				1.1.1 & 2016-01-21 & \emph{Analisti} \newline Bonato Paolo \newline Bortolazzo Matteo\newline Maino Elia  & 
				Correzioni degli errori individuati nella verifica
				 \\
				 \noalign{\hrule height 0.5pt}
				1.1.0 & 2016-01-21 & \emph{Verificatori} \newline Biggeri Mattia \newline Tommasin Davide  & Verifica del documento \\
				 \noalign{\hrule height 0.5pt}
				1.0.5 & 2016-01-20 & \emph{Analista} \newline Bortolazzo Matteo  & Stesura lista dei requisiti dedotti dal capitolato e da analisi interne
				 \\
				 \noalign{\hrule height 0.5pt}
				1.0.4 & 2016-01-19 & \emph{Analisti} \newline Bonato Paolo \newline Maino Elia & Stesura lista dei requisiti dedotti dai casi d'uso
				 \\
				\noalign{\hrule height 0.5pt}
				1.0.3 & 2016-01-18 & \emph{Analista} \newline Maino Elia & Stesura dei casi d'uso (da 4.x a 5.x) \\
				\noalign{\hrule height 0.5pt}
				1.0.2 & 2016-01-17 & \emph{Analista} \newline Maino Elia & Stesura dei casi d'uso (da 1.x a 3.x) \\
				\noalign{\hrule height 0.5pt}
				1.0.1 & 2016-01-17 & \emph{Analista} \newline Maino Elia & Stesura caratteristiche generali del prodotto \\
				\noalign{\hrule height 0.5pt}
				1.0.0 & 2016-01-17 & \emph{Analista} \newline Maino Elia & Creazione scheletro documento e stesura introduzione \\
				\noalign{\hrule height 1.5pt}
			\end{tabularx}
			\caption{Diario delle modifiche \label{tab:table_label}}
		\end{table}
	



	\newpage \section{Introduzione}
	\subsection{Scopo del documento}
		Lo scopo del seguente documento è presentare l'insieme di requisiti individuati dal gruppo 
		SWEeneyThreads durante l'analisi del \emph{Capitolato C1}, gli incontri con il committente 
		\emph{Cardin Riccardo} e l'analisi dei casi d'uso. 
		
		Si intende fornire una rappresentazione ordinata delle funzionalità che il prodotto \emph{Actorbase} 
		offrirà al momento del rilascio.
	\subsection{Scopo del prodotto}
		Lo scopo del progetto è la realizzazione di un DataBase NoSQL key-value basato sul modello ad 
		Attori con l'obiettivo di fornire una tecnologia adatta allo sviluppo di moderne 
		applicazioni che richiedono brevissimi tempi di risposta e che elaborano enormi quantità 
		di dati. Lo sviluppo porterà al rilascio del software sotto licenza MIT.
	\subsection{Glossario}
		Con lo scopo di evitare ambiguità di linguaggio e di massimizzare la comprensione dei documenti, il 
      gruppo ha steso un documento interno che è il \emph{Glossario v1.0.3}. In esso saranno definiti, in modo
      chiaro e conciso i termini che possono causare ambiguità o incomprensione del testo.
	\subsection{Riferimenti}
	\subsubsection{Informativi}	
		\begin{itemize}
			\item \textbf{Slide dell'insegnamento Ingegneria del software mod.A:} \\
			\url{http://www.math.unipd.it/~tullio/IS-1/2015/Dispense/E02.pdf}
		\end{itemize}
	\subsubsection{Normativi}
		\begin{itemize}
			\item \textbf{Norme di progetto:} \emph{Norme di progetto v1.1.2}
			\item \textbf{Capitolato d'appalto Actorbase (C1):} \\ 
			\url{http://www.math.unipd.it/~tullio/IS-1/2015/Progetto/C1p.pdf}
		\end{itemize}
		
		
	\newpage 
	\section{Caratteristiche generali del prodotto}
	\subsection{Obiettivi del prodotto}
		Il prodotto si pone l'obiettivo di fornire un database NoSQL basato sul modello ad attori che possa 
		essere utilizzato con successo dalle cosiddette \emph{Reactive Applications}, ovvero applicazioni
		orientate agli eventi, scalabili, resilienti e responsive. Per tali applicazioni il paradigma di accesso 
		ai dati basato su database di tipo relazionale risulta non applicabile, poiché troppo limitante, sono 
		dunque necessarie nuove forme di gestione dell'informazione, \emph{Actorbase} intende proporsi 
		come una valida opzione.
	\subsection{Funzioni principali del prodotto}
		\emph{Actorbase} prevede un'interazione con l'utente da riga di comando attraverso l'uso di un 
		\emph{Domain Specific Language (DSL)}. 
		\\ \\
		Una volta effettuata la connessione al server, l'utente potrà:
		\begin{itemize}
			\item Visualizzare i database presenti sul server
			\item Creare un nuovo database
			\item Selezionare un database da utilizzare
			\item Rinominare un database già presente
			\item Eliminare un database
			\item Esportare i database (backup)
		\end{itemize}
		La selezione di un database permetterà all'utente di effettuare una serie di operazioni di interrogazione e modifica di esso:
		\begin{itemize}
			\item Creazione di mappe
			\item Selezione di una mappa
			\item Rimozione di una mappa
			\item Visualizzazione delle mappe presenti
		\end{itemize}
		La selezione di una mappa permetterà all'utente di accedere alle operazioni di interrogazione e modifica a tale livello:
		\begin{itemize}
			\item Inserimento di un item
			\item Ricerca di un item per chiave
			\item Cancellazione di un item
			\item Aggiornamento di un item già presente
		\end{itemize}
	\subsection{Target d'utenza}
		Il prodotto si rivolge a sviluppatori di applicazioni moderne, che trattano enormi moli di dati (nell'
		ordine dei Petabyte), che richiedono brevissimi tempi di risposta e che necessitano di un uptime 
		del 100\%.
	\subsection{Vincoli per l'utilizzo}
		L'esecuzione di \emph{Actorbase} avviene attraverso la JVM (Java Virtual Machine), il corretto funzionamento di tutte le caratteristiche del prodotto è garantito su macchine che utilizzano la versione 8 o successive della JVM. 
	
	\newpage \section{Casi d'uso}
		Di seguito viene riportata la descrizione accurata di tutti i casi d'uso individuati dal gruppo a seguito
		delle seguenti attività:
		\begin{itemize}
			\item Analisi del capitolato C1 Actorbase
			\item Confronti con il committente \emph{Cardin Riccardo}
			\item Riunioni e discussioni interne al gruppo
			\item Analisi della struttura e delle funzionalità di altri database non relazionali
		\end{itemize}
		La struttura di un caso d'uso è definita nel documento \emph{Norme di progetto v1.1.1 sez 2.1.3}.
		\subsection{Attori}
		L'interazione degli utenti con l'applicativo avverrà tramite una CLI (Command Line Interface).
		Sono state individuate tre tipologie principali di attori che interagiscono con il sistema:
		\begin{itemize}
			\item \textbf{Utente non autenticato:} è l'utente che ha avviato l'applicativo e interagisce con la CLI senza aver ancora effettuato con successo l'operazione di connessione al server;
			\item \textbf{Utente autenticato:} è l'utente che ha già effettuato con successo l'operazione di connessione ad un server che utilizza \emph{Actorbase}. 
			\item \textbf{Amministratore server:} è l'amministratore del server su cui è in funzione \emph{Actorbase}, è un \emph{utente autenticato} che dispone di funzionalità di gestione avanzate.
		\end{itemize}
		\emph{Actorbase} prevede un meccanismo di permessi per controllare l'accesso ai diversi database presenti sul server. Quando un utente crea un nuovo database riceve i permessi di amministrazione per esso: oltre a poter leggere e scrivere i dati, egli può cambiare i permessi di accesso a tale database per gli altri utenti del server. \\ 
		Ricapitolando esistono tre permessi di accesso ad un singolo database (elencati in ordine di privilegi garantiti):
		\begin{itemize}
		\item Amministrazione
		\item Scrittura
		\item Lettura
		\end{itemize}
		Il sistema di permessi è applicato solamente al livello database, non ci sono permessi specifici per mappe o altri elementi.
		\subsection{Visione ad alto livello delle interazioni tramite interfaccia client}
		 	\begin{figure}[H]
				\centering
				\includegraphics[scale=0.2]{UC/"Actorbase client".png}
				\caption{Interazioni tramite interfaccia client}
			\end{figure}
			Il diagramma in figura illustra le principali operazioni che un utente (autenticato e non) può
			 effettuare sul sistema tramite l'interfaccia client da riga di comando. 
			\begin{itemize}
				\item Connessione al server
				\item Visualizzazione di un aiuto
				\item Operazioni a livello server
				\item Operazioni a livello database
				\item Operazioni a livello mappa
				\item Disconnessione dal server
			\end{itemize}
			Se l'utente autenticato è l'amministratore del server, egli ha a disposizione delle funzionalità di gestione aggiuntive, quali:
			\begin{itemize}
			\item Gestione degli utenti che hanno accesso al server
			\item Gestione dei server su cui effettuare la distribuzione del database
			\item Gestione delle impostazioni del server
			\end{itemize}
	In ogni momento un utente può inserire un comando inesistente o non utilizzabile in quel determinato momento. In tal caso nessuna azione viene eseguita dal programma e l'utente riceve un messaggio di errore esplicativo seguito dalla possibilità di inserire un nuovo comando.	 
	 
	 
	 
	 \subsection{UC 1: Connessione al server}
	 \begin{figure}[H]
				\centering
				\includegraphics[scale=0.3]{UC/"UC 1 Connessione".png}
				\caption{Diagramma di UC1: Connessione al server}
			\end{figure}
	\textbf{Descrizione} 
	\\ \\
	L'utente ha appena avviato l'interfaccia CLI dell'applicativo e intende connettersi ad un server contenente i database utilizzando il comando di connessione (UC 1.1). L'utente deve inserire l'indirizzo del server (UC 1.2) e successivamente l'username e la password necessari per l'accesso (UC 1.3), nel caso in cui il tentativo di connessione dovesse fallire, l'utente riceve un messaggio di errore informativo.
	\begin{table}[H]
			\begin{tabularx}{\textwidth}{r X}
				\textbf{Codice gerarchico} & UC1 \\
				\noalign{\hrule height 0.5pt}
				\textbf{Nome sintetico} & Connessione al server \\
				\noalign{\hrule height 0.5pt}
				\textbf{Attore principale} & Utente non autenticato\\
				\noalign{\hrule height 0.5pt}
				\textbf{Attori secondari} & Nessuno \\
				\noalign{\hrule height 0.5pt}
				\textbf{Pre-condizione} & L'utente ha avviato l'interfaccia CLI, non è autenticato e intende connettersi ad un server\\
				\noalign{\hrule height 0.5pt}
				\textbf{Post-condizione} & L'utente risulta connesso al server \\
				\noalign{\hrule height 0.5pt}
				\textbf{Flusso eventi} & \begin{enumerate}
				\item L'utente inserisce il comando di connessione (UC 1.1)
				\item L'utente inserisce l'indirizzo del server (UC 1.2)
				\item L'utente inserisce username e password (UC 1.3) e preme invio
				\end{enumerate} \\
				\noalign{\hrule height 0.5pt}
				\textbf{Scenari alternativi} & Nessuno \\
				\noalign{\hrule height 0.5pt}
				\textbf{Lista requisiti\newline dedotti} & \begin{itemize}
				\item ...
				\end{itemize} 
			\end{tabularx}
			\caption{Caso d'uso UC 1 - Connessione al server}
		 \end{table} 
	 
	 
	\subsection{UC 1.1: Inserimento comando di connessione}
	 \textbf{Descrizione}
	 \\ \\
	 L'utente intende effettuare una connessione al server, deve avviare la procedura di connessione inserendo il comando di connessione da terminale.
	\begin{table}[H]
			\begin{tabularx}{\textwidth}{r  X}
				\textbf{Codice gerarchico} & UC1.1 \\
				\noalign{\hrule height 0.5pt}
				\textbf{Nome sintetico} & Inserimento comando di connessione\\
				\noalign{\hrule height 0.5pt}
				\textbf{Attore principale} & Utente non autenticato\\
				\noalign{\hrule height 0.5pt}
				\textbf{Attori secondari} & Nessuno \\
				\noalign{\hrule height 0.5pt}
				\textbf{Pre-condizione} & L'utente intende avviare la procedura di connessione al server\\
				\noalign{\hrule height 0.5pt}
				\textbf{Post-condizione} & L'utente ha inserito con successo il comando di connessione\\
				\noalign{\hrule height 0.5pt}
				\textbf{Flusso eventi} & \begin{enumerate}
				\item L'utente scrive sul terminale il comando di connessione
				\end{enumerate} \\
				\noalign{\hrule height 0.5pt}
				\textbf{Scenari alternativi} & Nessuno \\
				\noalign{\hrule height 0.5pt}
				\textbf{Lista requisiti\newline dedotti} & \begin{itemize}
				\item ...
				\end{itemize} 
			\end{tabularx}
			\caption{Caso d'uso UC 1.1 - Inserimento comando di connessione}
		 \end{table} 	 
	 
	 
	 \subsection{UC 1.2: Inserimento indirizzo server }
	 \textbf{Descrizione}
	 \\ \\
	 L'utente sta effettuando l'operazione di connessione al server e deve inserire l'indirizzo del server a cui vuole connettersi.
	\begin{table}[H]
			\begin{tabularx}{\textwidth}{r  X}
				\textbf{Codice gerarchico} & UC1.2 \\
				\noalign{\hrule height 0.5pt}
				\textbf{Nome sintetico} & Inserimento indirizzo server \\
				\noalign{\hrule height 0.5pt}
				\textbf{Attore principale} & Utente non autenticato\\
				\noalign{\hrule height 0.5pt}
				\textbf{Attori secondari} & Nessuno \\
				\noalign{\hrule height 0.5pt}
				\textbf{Pre-condizione} & L'utente ha inserito il comando di connessione\\
				\noalign{\hrule height 0.5pt}
				\textbf{Post-condizione} & L'utente ha inserito con successo l'indirizzo del server \\
				\noalign{\hrule height 0.5pt}
				\textbf{Flusso eventi} & \begin{enumerate}
				\item L'utente scrive sul terminale l'indirizzo del server
				\end{enumerate} \\
				\noalign{\hrule height 0.5pt}
				\textbf{Scenari alternativi} & Nessuno \\
				\noalign{\hrule height 0.5pt}
				\textbf{Lista requisiti\newline dedotti} & \begin{itemize}
				\item ...
				\end{itemize} 
			\end{tabularx}
			\caption{Caso d'uso UC 1.2 - Inserimento indirizzo server}
		 \end{table} 
		 
		 
		 \subsection{UC 1.3: Inserimento username e password}
	 \textbf{Descrizione}
	 \\ \\
	 L'utente sta effettuando l'operazione di connessione al server e deve inserire il proprio username e la password di accesso al server.
	\begin{table}[H]
			\begin{tabularx}{\textwidth}{r  X}
				\textbf{Codice gerarchico} & UC1.3 \\
				\noalign{\hrule height 0.5pt}
				\textbf{Nome sintetico} & Inserimento username e password \\
				\noalign{\hrule height 0.5pt}
				\textbf{Attore principale} & Utente non autenticato\\
				\noalign{\hrule height 0.5pt}
				\textbf{Attori secondari} & Nessuno \\
				\noalign{\hrule height 0.5pt}
				\textbf{Pre-condizione} & L'utente ha inserito l'indirizzo del server\\
				\noalign{\hrule height 0.5pt}
				\textbf{Post-condizione} & L'utente ha inserito con successo username e password \\
				\noalign{\hrule height 0.5pt}
				\textbf{Flusso eventi} & \begin{enumerate}
				\item L'utente scrive su terminale il proprio username
				\item L'utente scrive su terminale la password 
				\end{enumerate} \\
				\noalign{\hrule height 0.5pt}
				\textbf{Scenari alternativi} & Nessuno \\
				\noalign{\hrule height 0.5pt}
				\textbf{Lista requisiti\newline dedotti} & \begin{itemize}
				\item ...
				\end{itemize} 
			\end{tabularx}
			\caption{Caso d'uso UC 1.3 - Inserimento username e password}
		 \end{table} 	
		 
		  
	 \subsection{UC 2: Visualizzazione aiuto}
	 \begin{figure}[H]
				\centering
				\includegraphics[scale=0.3]{UC/"UC 2 Visualizzazione aiuto".png}
				\caption{Diagramma di UC2: Visualizzazione aiuto}
			\end{figure}
	\textbf{Descrizione} 
	\\ \\
	L'utente (autenticato o non) intende ricevere un aiuto su come utilizzare i comandi di \emph{Actorbase}. Ha a disposizione due possibilità: 
	\begin{itemize}
	\item Aiuto generico
	\item Aiuto specifico
	\end{itemize}
	La prima modalità stampa sul terminale la lista completa dei comandi esistenti con relativa nota esplicativa sul loro comportamento. \\
	La modalità di aiuto specifica richiede l'inserimento del nome del comando per cui si richiedono le informazioni, stampa sul terminale la descrizione di tale comando.
	\begin{table}[H]
			\begin{tabularx}{\textwidth}{r X}
				\textbf{Codice gerarchico} & UC2 \\
				\noalign{\hrule height 0.5pt}
				\textbf{Nome sintetico} & Visualizzazione aiuto \\
				\noalign{\hrule height 0.5pt}
				\textbf{Attore principale} & Utente non autenticato e Utente autenticato\\
				\noalign{\hrule height 0.5pt}
				\textbf{Attori secondari} & Nessuno \\
				\noalign{\hrule height 0.5pt}
				\textbf{Pre-condizione} & L'utente ha avviato l'interfaccia CLI e intende visualizzare le informazioni di aiuto\\
				\noalign{\hrule height 0.5pt}
				\textbf{Post-condizione} & L'utente ha visualizzato le informazioni di aiuto richieste\\
				\noalign{\hrule height 0.5pt}
				\textbf{Flusso eventi} & \begin{enumerate}
				\item L'utente inserisce il comando relativo all'aiuto richiesto e preme invio
				\item L'utente riceve le informazioni di aiuto stampate sul terminale
				\end{enumerate} \\
				\noalign{\hrule height 0.5pt}
				\textbf{Scenari alternativi} & Nessuno \\
				\noalign{\hrule height 0.5pt}
				\textbf{Lista requisiti\newline dedotti} & \begin{itemize}
				\item ...
				\end{itemize} 
			\end{tabularx}
			\caption{Caso d'uso UC 2 - Visualizzazione aiuto}
		 \end{table} 
		 
		 
		 \subsection{UC 2.1: Aiuto generico}
	 \textbf{Descrizione}
	 \\ \\
	 L'utente sta effettuando l'operazione di visualizzazione dell'aiuto e intende inserire il comando di aiuto semplice.
	\begin{table}[H]
			\begin{tabularx}{\textwidth}{r  X}
				\textbf{Codice gerarchico} & UC2.1 \\
				\noalign{\hrule height 0.5pt}
				\textbf{Nome sintetico} & Aiuto generico \\
				\noalign{\hrule height 0.5pt}
				\textbf{Attore principale} & Utente non autenticato e Utente autenticato\\
				\noalign{\hrule height 0.5pt}
				\textbf{Attori secondari} & Nessuno \\
				\noalign{\hrule height 0.5pt}
				\textbf{Pre-condizione} & L'utente intende ricevere un aiuto generico \\
				\noalign{\hrule height 0.5pt}
				\textbf{Post-condizione} & L'utente ha inserito correttamente il comando di aiuto generico\\
				\noalign{\hrule height 0.5pt}
				\textbf{Flusso eventi} & \begin{enumerate}
				\item L'utente inserisce il comando \texttt{HELP}
				\end{enumerate} \\
				\noalign{\hrule height 0.5pt}
				\textbf{Scenari alternativi} & Nessuno \\
				\noalign{\hrule height 0.5pt}
				\textbf{Lista requisiti\newline dedotti} & \begin{itemize}
				\item ...
				\end{itemize} 
			\end{tabularx}
			\caption{Caso d'uso UC 2.1 - Aiuto generico}
		 \end{table} 	
	 
	 
	 \subsection{UC 2.2: Aiuto specifico}
	 \textbf{Descrizione}
	 \\ \\
	 L'utente sta effettuando l'operazione di visualizzazione dell'aiuto e intende inserire il comando di aiuto specifico.
	\begin{table}[H]
			\begin{tabularx}{\textwidth}{r  X}
				\textbf{Codice gerarchico} & UC2.2 \\
				\noalign{\hrule height 0.5pt}
				\textbf{Nome sintetico} & Aiuto specifico \\
				\noalign{\hrule height 0.5pt}
				\textbf{Attore principale} & Utente non autenticato e Utente autenticato\\
				\noalign{\hrule height 0.5pt}
				\textbf{Attori secondari} & Nessuno \\
				\noalign{\hrule height 0.5pt}
				\textbf{Pre-condizione} & L'utente intende ricevere un aiuto specifico\\
				\noalign{\hrule height 0.5pt}
				\textbf{Post-condizione} & L'utente ha inserito correttamente il comando di aiuto specifico\\
				\noalign{\hrule height 0.5pt}
				\textbf{Flusso eventi} & \begin{enumerate}
				\item L'utente inserisce il comando \texttt{HELP} seguito dal nome del comando per cui vuole ricevere l'aiuto
				\end{enumerate} \\
				\noalign{\hrule height 0.5pt}
				\textbf{Scenari alternativi} & Nessuno \\
				\noalign{\hrule height 0.5pt}
				\textbf{Lista requisiti\newline dedotti} & \begin{itemize}
				\item ...
				\end{itemize} 
			\end{tabularx}
			\caption{Caso d'uso UC 2.2 - Aiuto specifico}
		 \end{table} 	
	 
	 
	 \subsection{UC 3: Operazioni a livello server}
	 \begin{figure}[H]
				\centering
				\includegraphics[scale=0.2]{UC/"UC 3 Operazioni a livello server".png}
				\caption{Diagramma di UC3: Operazioni a livello server}
			\end{figure}
	\textbf{Descrizione} 
	\\ \\
	L'utente ha effettuato correttamente la connessione, ha dunque accesso ai database presenti sul server. Su tali database può eseguire le seguenti operazioni:
	\begin{itemize}
	\item Visualizzare la lista dei database presenti
	\item Esportare uno o più database (backup)
	\item Importare un database
	\item Creare un database
	\item Eliminare un database
	\item Rinominare un database presente
	\item Selezionare un database per effettuare operazioni a livello di mappa su di esso
	\end{itemize}
	Queste operazioni possono portare a diverse situazioni di errore. In tal caso non viene eseguita alcuna operazione e l'utente riceve un messaggio di errore esplicativo.
	\begin{table}[H]
			\begin{tabularx}{\textwidth}{r X}
				\textbf{Codice gerarchico} & UC3 \\
				\noalign{\hrule height 0.5pt}
				\textbf{Nome sintetico} & Operazioni sui database\\
				\noalign{\hrule height 0.5pt}
				\textbf{Attore principale} & Utente autenticato\\
				\noalign{\hrule height 0.5pt}
				\textbf{Attori secondari} & Nessuno \\
				\noalign{\hrule height 0.5pt}
				\textbf{Pre-condizione} & L'utente si è connesso con successo ad un server\\
				\noalign{\hrule height 0.5pt}
				\textbf{Post-condizione} & L'operazione sui database selezionata è stata eseguita con successo\\
				\noalign{\hrule height 0.5pt}
				\textbf{Flusso eventi} & \begin{enumerate}
				\item L'utente inserisce il comando e le informazioni necessarie per l'operazione richiesta e preme invio
				\end{enumerate} \\
				\noalign{\hrule height 0.5pt}
				\textbf{Scenari alternativi} & \begin{enumerate}
				\item L'utente ha richiesto un'operazione di esportazione e questa è fallita, l'operazione non viene eseguita e l'utente riceve un messaggio di errore (UC 3.8)
				\item L'utente ha richiesto un'operazione di importazione e questa è fallita, l'operazione non viene eseguita e l'utente riceve un messaggio di errore (UC 3.9)
				\item L'utente ha richiesto la creazione di un nuovo database sul server, la creazione è fallita e l'utente riceve un messaggio di errore (UC 3.10)
				\item L'utente ha tentato di accedere ad un database non presente sul server, l'accesso è fallito e l'utente riceve un messaggio di errore (UC 3.11)
				\item L'utente ha richiesto un'operazione di selezione di un database non disponendo dei permessi di lettura per tale database, la selezione non viene effettuata e l'utente riceve un messaggio di errore (UC 3.12)
\end{enumerate}				 \\
				\noalign{\hrule height 0.5pt}
				\textbf{Lista requisiti\newline dedotti} & \begin{itemize}
				\item ...
				\end{itemize} 
			\end{tabularx}
			\caption{Caso d'uso UC 3 - Operazioni a livello server}
		 \end{table} 
	 
	 
	\subsection{UC 3.1: Visualizzazione lista dei database}
	\textbf{Descrizione} 
	\\ \\
	L'utente intende visualizzare la lista dei database presenti sul server a cui è connesso. Inserisce quindi il comando \texttt{SHOWDB} e il sistema stampa sul terminale la lista.
	\begin{table}[H]
			\begin{tabularx}{\textwidth}{r X}
				\textbf{Codice gerarchico} & UC3.1 \\
				\noalign{\hrule height 0.5pt}
				\textbf{Nome sintetico} & Visualizzazione lista dei database\\
				\noalign{\hrule height 0.5pt}
				\textbf{Attore principale} & Utente autenticato\\
				\noalign{\hrule height 0.5pt}
				\textbf{Attori secondari} & Nessuno \\
				\noalign{\hrule height 0.5pt}
				\textbf{Pre-condizione} & L'utente intende visualizzare la lista dei database presenti\\
				\noalign{\hrule height 0.5pt}
				\textbf{Post-condizione} & L'utente riceve la lista dei database presenti stampata a video\\
				\noalign{\hrule height 0.5pt}
				\textbf{Flusso eventi} & \begin{enumerate}
				\item L'utente inserisce il comando \texttt{SHOWDB} e preme invio
				\end{enumerate} \\
				\noalign{\hrule height 0.5pt}
				\textbf{Scenari alternativi} & Nessuno \\
				\noalign{\hrule height 0.5pt}
				\textbf{Lista requisiti\newline dedotti} & \begin{itemize}
				\item ...
				\end{itemize} 
			\end{tabularx}
			\caption{Caso d'uso UC 3.1 - Visualizzazione lista dei database}
		 \end{table} 	 
	 
	 
	 \subsection{UC 3.2: Esportazione di database}
	 \begin{figure}[H]
				\centering
				\includegraphics[scale=0.25]{UC/"UC 3-2 Esportazione di database".png}
				\caption{Diagramma di UC3.2: Esportazione di database}
			\end{figure}
	\textbf{Descrizione} 
	\\ \\
	L'utente intende effettuare l'esportazione di tutti i database presenti o di un database specifico, utilizzando il comando \texttt{EXPORT} seguito dal nome del database da esportare (opzionale) e dal percorso locale in cui si vuole esportare il database.
	\begin{table}[H]
			\begin{tabularx}{\textwidth}{r X}
				\textbf{Codice gerarchico} & UC3.2 \\
				\noalign{\hrule height 0.5pt}
				\textbf{Nome sintetico} & Esportazione di database\\
				\noalign{\hrule height 0.5pt}
				\textbf{Attore principale} & Utente autenticato\\
				\noalign{\hrule height 0.5pt}
				\textbf{Attori secondari} & Nessuno \\
				\noalign{\hrule height 0.5pt}
				\textbf{Pre-condizione} & L'utente intende esportare dei database\\
				\noalign{\hrule height 0.5pt}
				\textbf{Post-condizione} & L'utente ha esportato con successo i database selezionati\\
				\noalign{\hrule height 0.5pt}
				\textbf{Flusso eventi} & \begin{enumerate}
				\item L'utente inserisce le informazioni di esportazione (UC 3.2.1 o UC 3.2.2) e preme invio
				\end{enumerate} \\
				\noalign{\hrule height 0.5pt}
				\textbf{Scenari alternativi} & Nessuno \\
				\noalign{\hrule height 0.5pt}
				\textbf{Lista requisiti\newline dedotti} & \begin{itemize}
				\item ...
				\end{itemize} 
			\end{tabularx}
			\caption{Caso d'uso UC 3.2 - Esportazione di database}
		 \end{table}
		 
		 
		 \subsection{UC 3.2.1: Esportazione di tutti i database}
	\textbf{Descrizione} 
	\\ \\
	L'utente intende effettuare un'esportazione di tutti i database. Inserisce il comando di esportazione e il percorso locale in cui vuole esportare i database.
	\begin{table}[H]
			\begin{tabularx}{\textwidth}{r X}
				\textbf{Codice gerarchico} & UC3.2.1 \\
				\noalign{\hrule height 0.5pt}
				\textbf{Nome sintetico} & Esportazione di tutti i database\\
				\noalign{\hrule height 0.5pt}
				\textbf{Attore principale} & Utente autenticato\\
				\noalign{\hrule height 0.5pt}
				\textbf{Attori secondari} & Nessuno \\
				\noalign{\hrule height 0.5pt}
				\textbf{Pre-condizione} & L'utente intende esportare tutti i database e ha inserito il comando di esportazione\\
				\noalign{\hrule height 0.5pt}
				\textbf{Post-condizione} & L'utente ha inserito correttamente i dati per esportare tutti i database\\
				\noalign{\hrule height 0.5pt}
				\textbf{Flusso eventi} & \begin{enumerate}
				\item L'utente scrive su terminale il comando di esportazione
				\item L'utente scrive su terminale il percorso locale in cui vuole esportare i database 
				\end{enumerate} \\
				\noalign{\hrule height 0.5pt}
				\textbf{Scenari alternativi} & Nessuno \\
				\noalign{\hrule height 0.5pt}
				\textbf{Lista requisiti\newline dedotti} & \begin{itemize}
				\item ...
				\end{itemize} 
			\end{tabularx}
			\caption{Caso d'uso UC 3.2.1 - Esportazione di tutti i database}
		 \end{table} 
		 
		 \subsection{UC 3.2.2: Esportazione di un singolo database}
	\textbf{Descrizione} 
	\\ \\
	L'utente intende effettuare un'esportazione di un singolo database. Inserisce il comando di esportazione e il nome del database da esportare seguito dal percorso locale in cui vuole esportare i database.
	\begin{table}[H]
			\begin{tabularx}{\textwidth}{r X}
				\textbf{Codice gerarchico} & UC3.2.2 \\
				\noalign{\hrule height 0.5pt}
				\textbf{Nome sintetico} & Esportazione di un singolo database\\
				\noalign{\hrule height 0.5pt}
				\textbf{Attore principale} & Utente autenticato\\
				\noalign{\hrule height 0.5pt}
				\textbf{Attori secondari} & Nessuno \\
				\noalign{\hrule height 0.5pt}
				\textbf{Pre-condizione} & L'utente intende esportare un singolo database e ha inserito il comando di esportazione\\
				\noalign{\hrule height 0.5pt}
				\textbf{Post-condizione} & L'utente ha inserito correttamente i dati per esportare un singolo database\\
				\noalign{\hrule height 0.5pt}
				\textbf{Flusso eventi} & \begin{enumerate}
				\item L'utente scrive su terminale il comando di esportazione
				\item L'utente scrive su terminale il nome del database che vuole esportare
				\item L'utente scrive su terminale il percorso locale in cui vuole esportare il database 
				\end{enumerate} \\
				\noalign{\hrule height 0.5pt}
				\textbf{Scenari alternativi} & Nessuno \\
				\noalign{\hrule height 0.5pt}
				\textbf{Lista requisiti\newline dedotti} & \begin{itemize}
				\item ...
				\end{itemize} 
			\end{tabularx}
			\caption{Caso d'uso UC 3.2.2 - Esportazione di un singolo database}
		 \end{table} 		 	
	 
	 
	 
	 \subsection{UC 3.3: Importazione di database}
	 \begin{figure}[H]
				\centering
				\includegraphics[scale=0.25]{UC/"UC 3-3 Importazione di database".png}
				\caption{Diagramma di UC3.3: Importazione di database}
			\end{figure}
	\textbf{Descrizione} 
	\\ \\
	L'utente intende effettuare l'importazione di un database utilizzando il comando \texttt{IMPORT} seguito dal percorso locale da cui si vuole importare il database.
	\begin{table}[H]
			\begin{tabularx}{\textwidth}{r X}
				\textbf{Codice gerarchico} & UC3.3 \\
				\noalign{\hrule height 0.5pt}
				\textbf{Nome sintetico} & Importazione di database\\
				\noalign{\hrule height 0.5pt}
				\textbf{Attore principale} & Utente autenticato\\
				\noalign{\hrule height 0.5pt}
				\textbf{Attori secondari} & Nessuno \\
				\noalign{\hrule height 0.5pt}
				\textbf{Pre-condizione} & L'utente intende importare un database\\
				\noalign{\hrule height 0.5pt}
				\textbf{Post-condizione} & L'utente ha importato con successo il database selezionato\\
				\noalign{\hrule height 0.5pt}
				\textbf{Flusso eventi} & \begin{enumerate}
				\item L'utente inserisce il comando \texttt{IMPORT} (UC 3.3.1)
				\item L'utente inserisce il percorso di importazione (UC 3.3.2) e preme invio
				\end{enumerate} \\
				\noalign{\hrule height 0.5pt}
				\textbf{Scenari alternativi} & Nessuno \\
				\noalign{\hrule height 0.5pt}
				\textbf{Lista requisiti\newline dedotti} & \begin{itemize}
				\item ...
				\end{itemize} 
			\end{tabularx}
			\caption{Caso d'uso UC 3.3 - Importazione di database}
		 \end{table}
	 
	 
	 \subsection{UC 3.3.1: Inserimento comando di importazione}
	\textbf{Descrizione} 
	\\ \\
	L'utente intende effettuare l'importazione di un database. Per avviare la procedura di importazione deve inserire il relativo comando \texttt{IMPORT}.
	\begin{table}[H]
			\begin{tabularx}{\textwidth}{r X}
				\textbf{Codice gerarchico} & UC3.3.1 \\
				\noalign{\hrule height 0.5pt}
				\textbf{Nome sintetico} & Inserimento comando di importazione\\
				\noalign{\hrule height 0.5pt}
				\textbf{Attore principale} & Utente autenticato\\
				\noalign{\hrule height 0.5pt}
				\textbf{Attori secondari} & Nessuno \\
				\noalign{\hrule height 0.5pt}
				\textbf{Pre-condizione} & L'utente intende importare un database\\
				\noalign{\hrule height 0.5pt}
				\textbf{Post-condizione} & L'utente ha inserito con successo il comando di importazione\\
				\noalign{\hrule height 0.5pt}
				\textbf{Flusso eventi} & \begin{enumerate}
				\item L'utente scrive su terminale il comando \texttt{IMPORT}
				\end{enumerate} \\
				\noalign{\hrule height 0.5pt}
				\textbf{Scenari alternativi} & Nessuno \\
				\noalign{\hrule height 0.5pt}
				\textbf{Lista requisiti\newline dedotti} & \begin{itemize}
				\item ...
				\end{itemize} 
			\end{tabularx}
			\caption{Caso d'uso UC 3.3.1 - Inserimento comando di importazione}
		 \end{table} 	 
		 
		 \subsection{UC 3.3.2: Inserimento percorso di importazione}
	\textbf{Descrizione} 
	\\ \\
	L'utente intende effettuare l'importazione di un database. Ha inserito il comando di importazione e ora deve inserire il percorso da cui importare i dati.
	\begin{table}[H]
			\begin{tabularx}{\textwidth}{r X}
				\textbf{Codice gerarchico} & UC3.3.2 \\
				\noalign{\hrule height 0.5pt}
				\textbf{Nome sintetico} & Inserimento percorso di importazione\\
				\noalign{\hrule height 0.5pt}
				\textbf{Attore principale} & Utente autenticato\\
				\noalign{\hrule height 0.5pt}
				\textbf{Attori secondari} & Nessuno \\
				\noalign{\hrule height 0.5pt}
				\textbf{Pre-condizione} & L'utente ha inserito il comando di importazione\\
				\noalign{\hrule height 0.5pt}
				\textbf{Post-condizione} & L'utente ha inserito con successo il percorso di importazione\\
				\noalign{\hrule height 0.5pt}
				\textbf{Flusso eventi} & \begin{enumerate}
				\item L'utente scrive su terminale il percorso di importazione
				\end{enumerate} \\
				\noalign{\hrule height 0.5pt}
				\textbf{Scenari alternativi} & Nessuno \\
				\noalign{\hrule height 0.5pt}
				\textbf{Lista requisiti\newline dedotti} & \begin{itemize}
				\item ...
				\end{itemize} 
			\end{tabularx}
			\caption{Caso d'uso UC 3.3.2 - Inserimento percorso di importazione}
		 \end{table} 	 
		 
		 
		 \subsection{UC 3.4: Creazione database}
	 \begin{figure}[H]
				\centering
				\includegraphics[scale=0.25]{UC/"UC 3-4 Creazione database".png}
				\caption{Diagramma di UC3.4: Creazione database}
			\end{figure}
	\textbf{Descrizione} 
	\\ \\
	L'utente intende creare un nuovo database inserendo il comando di creazione seguito dal nome del database da creare.
	\begin{table}[H]
			\begin{tabularx}{\textwidth}{r X}
				\textbf{Codice gerarchico} & UC3.4 \\
				\noalign{\hrule height 0.5pt}
				\textbf{Nome sintetico} & Creazione database\\
				\noalign{\hrule height 0.5pt}
				\textbf{Attore principale} & Utente autenticato\\
				\noalign{\hrule height 0.5pt}
				\textbf{Attori secondari} & Nessuno \\
				\noalign{\hrule height 0.5pt}
				\textbf{Pre-condizione} & L'utente intende creare un database\\
				\noalign{\hrule height 0.5pt}
				\textbf{Post-condizione} & L'utente ha creato con successo un database con il nome inserito\\
				\noalign{\hrule height 0.5pt}
				\textbf{Flusso eventi} & \begin{enumerate}
				\item L'utente inserisce il comando di creazione database (UC 3.4.1)
				\item L'utente inserisce il nome del database da creare (UC 3.4.2) e preme invio
				\end{enumerate} \\
				\noalign{\hrule height 0.5pt}
				\textbf{Scenari alternativi} & Nessuno \\
				\noalign{\hrule height 0.5pt}
				\textbf{Lista requisiti\newline dedotti} & \begin{itemize}
				\item ...
				\end{itemize} 
			\end{tabularx}
			\caption{Caso d'uso UC 3.4 - Creazione database}
		 \end{table}
		 
		 
		  \subsection{UC 3.4.1: Inserimento comando di creazione database}
	\textbf{Descrizione} 
	\\ \\
	L'utente intende creare un nuovo database, a tal fine deve come prima cosa inserire il comando di creazione di database.
	\begin{table}[H]
			\begin{tabularx}{\textwidth}{r X}
				\textbf{Codice gerarchico} & UC3.4.1 \\
				\noalign{\hrule height 0.5pt}
				\textbf{Nome sintetico} & Inserimento comando di creazione database\\
				\noalign{\hrule height 0.5pt}
				\textbf{Attore principale} & Utente autenticato\\
				\noalign{\hrule height 0.5pt}
				\textbf{Attori secondari} & Nessuno \\
				\noalign{\hrule height 0.5pt}
				\textbf{Pre-condizione} & L'utente intende creare un database\\
				\noalign{\hrule height 0.5pt}
				\textbf{Post-condizione} & L'utente ha inserito con successo il comando di creazione database\\
				\noalign{\hrule height 0.5pt}
				\textbf{Flusso eventi} & \begin{enumerate}
				\item L'utente scrive su terminale il comando di creazione database
				\end{enumerate} \\
				\noalign{\hrule height 0.5pt}
				\textbf{Scenari alternativi} & Nessuno \\
				\noalign{\hrule height 0.5pt}
				\textbf{Lista requisiti\newline dedotti} & \begin{itemize}
				\item ...
				\end{itemize} 
			\end{tabularx}
			\caption{Caso d'uso UC 3.4.1 - Inserimento comando di creazione database}
		 \end{table}
		 
		 \subsection{UC 3.4.2: Inserimento nome database da creare}
	\textbf{Descrizione} 
	\\ \\
	L'utente intende creare un nuovo database, ha inserito il comando di creazione database e ora deve inserire il nome che intende dare al nuovo database.
	\begin{table}[H]
			\begin{tabularx}{\textwidth}{r X}
				\textbf{Codice gerarchico} & UC3.4.2 \\
				\noalign{\hrule height 0.5pt}
				\textbf{Nome sintetico} & Inserimento nome database da creare\\
				\noalign{\hrule height 0.5pt}
				\textbf{Attore principale} & Utente autenticato\\
				\noalign{\hrule height 0.5pt}
				\textbf{Attori secondari} & Nessuno \\
				\noalign{\hrule height 0.5pt}
				\textbf{Pre-condizione} & L'utente ha inserito il comando di creazione database\\
				\noalign{\hrule height 0.5pt}
				\textbf{Post-condizione} & L'utente ha inserito con successo il nome del database da creare\\
				\noalign{\hrule height 0.5pt}
				\textbf{Flusso eventi} & \begin{enumerate}
				\item L'utente scrive su terminale il nome del database da creare
				\end{enumerate} \\
				\noalign{\hrule height 0.5pt}
				\textbf{Scenari alternativi} & Nessuno \\
				\noalign{\hrule height 0.5pt}
				\textbf{Lista requisiti\newline dedotti} & \begin{itemize}
				\item ...
				\end{itemize} 
			\end{tabularx}
			\caption{Caso d'uso UC 3.4.2 - Inserimento nome database da creare}
		 \end{table}
		 
		 
		 \subsection{UC 3.5: Eliminazione database}
	 \begin{figure}[H]
				\centering
				\includegraphics[scale=0.25]{UC/"UC 3-5 Eliminazione database".png}
				\caption{Diagramma di UC3.5: Eliminazione database}
			\end{figure}
	\textbf{Descrizione} 
	\\ \\
	L'utente intende eliminare un database presente sul server inserendo il comando per eliminare un database, seguito dal nome del database da eliminare.
	\begin{table}[H]
			\begin{tabularx}{\textwidth}{r X}
				\textbf{Codice gerarchico} & UC3.5 \\
				\noalign{\hrule height 0.5pt}
				\textbf{Nome sintetico} & Eliminazione database\\
				\noalign{\hrule height 0.5pt}
				\textbf{Attore principale} & Utente autenticato\\
				\noalign{\hrule height 0.5pt}
				\textbf{Attori secondari} & Nessuno \\
				\noalign{\hrule height 0.5pt}
				\textbf{Pre-condizione} & L'utente intende eliminare un database\\
				\noalign{\hrule height 0.5pt}
				\textbf{Post-condizione} & L'utente ha eliminato con successo il database selezionato\\
				\noalign{\hrule height 0.5pt}
				\textbf{Flusso eventi} & \begin{enumerate}
				\item L'utente inserisce il comando di eliminazione database (UC 3.5.1)
				\item L'utente inserisce il nome del database da eliminare (UC 3.5.2) e preme invio
				\end{enumerate} \\
				\noalign{\hrule height 0.5pt}
				\textbf{Scenari alternativi} & Nessuno \\
				\noalign{\hrule height 0.5pt}
				\textbf{Lista requisiti\newline dedotti} & \begin{itemize}
				\item ...
				\end{itemize} 
			\end{tabularx}
			\caption{Caso d'uso UC 3.5 - Eliminazione database}
		 \end{table}
		 
		 
		\subsection{UC 3.5.1: Inserimento comando di eliminazione database}
	\textbf{Descrizione} 
	\\ \\
	L'utente intende eliminare un database presente sul server, a tal fine deve per prima cosa scrivere su terminale il comando di eliminazione database.
	\begin{table}[H]
			\begin{tabularx}{\textwidth}{r X}
				\textbf{Codice gerarchico} & UC3.5.1 \\
				\noalign{\hrule height 0.5pt}
				\textbf{Nome sintetico} & Inserimento comando di eliminazione database\\
				\noalign{\hrule height 0.5pt}
				\textbf{Attore principale} & Utente autenticato\\
				\noalign{\hrule height 0.5pt}
				\textbf{Attori secondari} & Nessuno \\
				\noalign{\hrule height 0.5pt}
				\textbf{Pre-condizione} & L'utente intende eliminare un database\\
				\noalign{\hrule height 0.5pt}
				\textbf{Post-condizione} & L'utente ha inserito con successo il comando di eliminazione database\\
				\noalign{\hrule height 0.5pt}
				\textbf{Flusso eventi} & \begin{enumerate}
				\item L'utente scrive su terminale il comando di eliminazione database
				\end{enumerate} \\
				\noalign{\hrule height 0.5pt}
				\textbf{Scenari alternativi} & Nessuno \\
				\noalign{\hrule height 0.5pt}
				\textbf{Lista requisiti\newline dedotti} & \begin{itemize}
				\item ...
				\end{itemize} 
			\end{tabularx}
			\caption{Caso d'uso UC 3.5.1 - Inserimento comando di eliminazione database}
		 \end{table}		 
		  
		\subsection{UC 3.5.2: Inserimento nome database da eliminare}
	\textbf{Descrizione} 
	\\ \\
	L'utente intende eliminare un database presente sul server, ha inserito il comando di eliminazione database e ora deve inserire il nome del database da eliminare.
	\begin{table}[H]
			\begin{tabularx}{\textwidth}{r X}
				\textbf{Codice gerarchico} & UC3.5.2 \\
				\noalign{\hrule height 0.5pt}
				\textbf{Nome sintetico} & Inserimento nome database da eliminare\\
				\noalign{\hrule height 0.5pt}
				\textbf{Attore principale} & Utente autenticato\\
				\noalign{\hrule height 0.5pt}
				\textbf{Attori secondari} & Nessuno \\
				\noalign{\hrule height 0.5pt}
				\textbf{Pre-condizione} & L'utente intende eliminare un database e ha inserito il comando di eliminazione database\\
				\noalign{\hrule height 0.5pt}
				\textbf{Post-condizione} & L'utente ha inserito con successo il nome del database da eliminare\\
				\noalign{\hrule height 0.5pt}
				\textbf{Flusso eventi} & \begin{enumerate}
				\item L'utente scrive su terminale il nome del database da eliminare
				\end{enumerate} \\
				\noalign{\hrule height 0.5pt}
				\textbf{Scenari alternativi} & Nessuno \\
				\noalign{\hrule height 0.5pt}
				\textbf{Lista requisiti\newline dedotti} & \begin{itemize}
				\item ...
				\end{itemize} 
			\end{tabularx}
			\caption{Caso d'uso UC 3.5.2 - Inserimento nome database da eliminare}
		 \end{table}		 
		 
		\subsection{UC 3.6: Rinominazione database}
	 \begin{figure}[H]
				\centering
				\includegraphics[scale=0.25]{UC/"UC 3-6 Rinominazione database".png}
				\caption{Diagramma di UC3.6: Rinominazione database}
			\end{figure}
	\textbf{Descrizione} 
	\\ \\
	L'utente intende rinominare un database presente sul server inserendo il comando per rinominare un database, seguito dal nome del database da rinominare e il nuovo nome.
	\begin{table}[H]
			\begin{tabularx}{\textwidth}{r X}
				\textbf{Codice gerarchico} & UC3.6 \\
				\noalign{\hrule height 0.5pt}
				\textbf{Nome sintetico} & Rinominazione database\\
				\noalign{\hrule height 0.5pt}
				\textbf{Attore principale} & Utente autenticato\\
				\noalign{\hrule height 0.5pt}
				\textbf{Attori secondari} & Nessuno \\
				\noalign{\hrule height 0.5pt}
				\textbf{Pre-condizione} & L'utente intende rinominare un database\\
				\noalign{\hrule height 0.5pt}
				\textbf{Post-condizione} & L'utente ha rinominato con successo il database selezionato\\
				\noalign{\hrule height 0.5pt}
				\textbf{Flusso eventi} & \begin{enumerate}
				\item L'utente inserisce il comando di rinominazione database (UC 3.6.1)
				\item L'utente inserisce il nome del database da rinominare (UC 3.6.2)
				\item L'utente inserisce il nuovo nome per il database (UC 3.6.3) e preme invio
				\end{enumerate} \\
				\noalign{\hrule height 0.5pt}
				\textbf{Scenari alternativi} & Nessuno \\
				\noalign{\hrule height 0.5pt}
				\textbf{Lista requisiti\newline dedotti} & \begin{itemize}
				\item ...
				\end{itemize} 
			\end{tabularx}
			\caption{Caso d'uso UC 3.6 - Rinominazione database}
		 \end{table}		 
		 
		 \subsection{UC 3.6.1: Inserimento comando di rinominazione database}
	\textbf{Descrizione} 
	\\ \\
	L'utente intende rinominare un database presente sul server, a tal fine deve per prima cosa inserire il comando di rinominazione database.
	\begin{table}[H]
			\begin{tabularx}{\textwidth}{r X}
				\textbf{Codice gerarchico} & UC3.6.1 \\
				\noalign{\hrule height 0.5pt}
				\textbf{Nome sintetico} & Inserimento comando di rinominazione database\\
				\noalign{\hrule height 0.5pt}
				\textbf{Attore principale} & Utente autenticato\\
				\noalign{\hrule height 0.5pt}
				\textbf{Attori secondari} & Nessuno \\
				\noalign{\hrule height 0.5pt}
				\textbf{Pre-condizione} & L'utente intende rinominare un database\\
				\noalign{\hrule height 0.5pt}
				\textbf{Post-condizione} & L'utente ha inserito con successo il comando di rinominazione database\\
				\noalign{\hrule height 0.5pt}
				\textbf{Flusso eventi} & \begin{enumerate}
				\item L'utente scrive su terminale il comando di rinominazione database
				\end{enumerate} \\
				\noalign{\hrule height 0.5pt}
				\textbf{Scenari alternativi} & Nessuno \\
				\noalign{\hrule height 0.5pt}
				\textbf{Lista requisiti\newline dedotti} & \begin{itemize}
				\item ...
				\end{itemize} 
			\end{tabularx}
			\caption{Caso d'uso UC 3.6.1 - Inserimento comando di rinominazione database}
		 \end{table}		 
		 
		 \subsection{UC 3.6.2: Inserimento nome database da rinominare}
	\textbf{Descrizione} 
	\\ \\
	L'utente intende rinominare un database presente sul server, ha inserito il comando per rinominare un database, ora deve inserire il nome del database che vuole rinominare.
	\begin{table}[H]
			\begin{tabularx}{\textwidth}{r X}
				\textbf{Codice gerarchico} & UC3.6.2 \\
				\noalign{\hrule height 0.5pt}
				\textbf{Nome sintetico} & Inserimento nome database da rinominare\\
				\noalign{\hrule height 0.5pt}
				\textbf{Attore principale} & Utente autenticato\\
				\noalign{\hrule height 0.5pt}
				\textbf{Attori secondari} & Nessuno \\
				\noalign{\hrule height 0.5pt}
				\textbf{Pre-condizione} & L'utente intende rinominare un database e ha inserito il comando di rinominazione database\\
				\noalign{\hrule height 0.5pt}
				\textbf{Post-condizione} & L'utente ha inserito con successo il nome del database da rinominare\\
				\noalign{\hrule height 0.5pt}
				\textbf{Flusso eventi} & \begin{enumerate}
				\item L'utente scrive su terminale il nome del database da rinominare
				\end{enumerate} \\
				\noalign{\hrule height 0.5pt}
				\textbf{Scenari alternativi} & Nessuno \\
				\noalign{\hrule height 0.5pt}
				\textbf{Lista requisiti\newline dedotti} & \begin{itemize}
				\item ...
				\end{itemize} 
			\end{tabularx}
			\caption{Caso d'uso UC 3.6.2 - Inserimento nome database da rinominare}
		 \end{table}		 
		 
		 \subsection{UC 3.6.3: Inserimento nuovo nome database}
	\textbf{Descrizione} 
	\\ \\
	L'utente intende rinominare un database presente sul server, ha inserito il comando per rinominare un database e il nome del database che vuole rinominare, ora deve inserire il nuovo nome per il database.
	\begin{table}[H]
			\begin{tabularx}{\textwidth}{r X}
				\textbf{Codice gerarchico} & UC3.6.3 \\
				\noalign{\hrule height 0.5pt}
				\textbf{Nome sintetico} & Inserimento nuovo nome database\\
				\noalign{\hrule height 0.5pt}
				\textbf{Attore principale} & Utente autenticato\\
				\noalign{\hrule height 0.5pt}
				\textbf{Attori secondari} & Nessuno \\
				\noalign{\hrule height 0.5pt}
				\textbf{Pre-condizione} & L'utente intende rinominare un database e ha inserito il comando di rinominazione database seguito dal nome del database da rinominare\\
				\noalign{\hrule height 0.5pt}
				\textbf{Post-condizione} & L'utente ha inserito con successo il nuovo nome per il database\\
				\noalign{\hrule height 0.5pt}
				\textbf{Flusso eventi} & \begin{enumerate}
				\item L'utente scrive su terminale il nuovo nome per il database
				\end{enumerate} \\
				\noalign{\hrule height 0.5pt}
				\textbf{Scenari alternativi} & Nessuno \\
				\noalign{\hrule height 0.5pt}
				\textbf{Lista requisiti\newline dedotti} & \begin{itemize}
				\item ...
				\end{itemize} 
			\end{tabularx}
			\caption{Caso d'uso UC 3.6.3 - Inserimento nuovo nome database}
		 \end{table}		 
		 
		 
		 \subsection{UC 3.7: Selezione di un database}
	 \begin{figure}[H]
				\centering
				\includegraphics[scale=0.25]{UC/"UC 3-7 Selezione di un database".png}
				\caption{Diagramma di UC3.7: Selezione di un database}
			\end{figure}
	\textbf{Descrizione} 
	\\ \\
	L'utente intende selezionare uno dei database presenti sul server al fine di effettuare operazioni a livello di mappa o item su di esso. Per selezionare un database l'utente deve inserire l'apposito comando da terminale, seguito dal nome del database che vuole selezionare.
	\begin{table}[H]
			\begin{tabularx}{\textwidth}{r X}
				\textbf{Codice gerarchico} & UC3.7 \\
				\noalign{\hrule height 0.5pt}
				\textbf{Nome sintetico} & Selezione di un database\\
				\noalign{\hrule height 0.5pt}
				\textbf{Attore principale} & Utente autenticato\\
				\noalign{\hrule height 0.5pt}
				\textbf{Attori secondari} & Nessuno \\
				\noalign{\hrule height 0.5pt}
				\textbf{Pre-condizione} & L'utente intende selezionare un database\\
				\noalign{\hrule height 0.5pt}
				\textbf{Post-condizione} & L'utente ha selezionato con successo il database richiesto\\
				\noalign{\hrule height 0.5pt}
				\textbf{Flusso eventi} & \begin{enumerate}
				\item L'utente inserisce il comando di selezione database (UC 3.7.1)
				\item L'utente inserisce il nome del database da selezionare (UC 3.7.2) e preme invio
				\end{enumerate} \\
				\noalign{\hrule height 0.5pt}
				\textbf{Scenari alternativi} & Nessuno \\
				\noalign{\hrule height 0.5pt}
				\textbf{Lista requisiti\newline dedotti} & \begin{itemize}
				\item ...
				\end{itemize} 
			\end{tabularx}
			\caption{Caso d'uso UC 3.7 - Selezione di un database}
		 \end{table}	
		 
		 
		 \subsection{UC 3.7.1: Inserimento comando di selezione database}
	\textbf{Descrizione} 
	\\ \\
	L'utente intende selezionare uno dei database presenti sul server al fine di effettuare operazioni a livello di mappa o item su di esso. Per effettuare l'operazione di selezione deve in primo luogo inserire il comando di selezione database da terminale.
	\begin{table}[H]
			\begin{tabularx}{\textwidth}{r X}
				\textbf{Codice gerarchico} & UC3.7.1 \\
				\noalign{\hrule height 0.5pt}
				\textbf{Nome sintetico} & Inserimento comando di selezione database\\
				\noalign{\hrule height 0.5pt}
				\textbf{Attore principale} & Utente autenticato\\
				\noalign{\hrule height 0.5pt}
				\textbf{Attori secondari} & Nessuno \\
				\noalign{\hrule height 0.5pt}
				\textbf{Pre-condizione} & L'utente intende selezionare un database\\
				\noalign{\hrule height 0.5pt}
				\textbf{Post-condizione} & L'utente ha inserito con successo il comando di selezione database\\
				\noalign{\hrule height 0.5pt}
				\textbf{Flusso eventi} & \begin{enumerate}
				\item L'utente scrive su terminale il comando di selezione database 
				\end{enumerate} \\
				\noalign{\hrule height 0.5pt}
				\textbf{Scenari alternativi} & Nessuno \\
				\noalign{\hrule height 0.5pt}
				\textbf{Lista requisiti\newline dedotti} & \begin{itemize}
				\item ...
				\end{itemize} 
			\end{tabularx}
			\caption{Caso d'uso UC 3.7.1 - Inserimento comando di selezione database}
		 \end{table}		 
		 
		 \subsection{UC 3.7.2: Inserimento nome database da selezionare}
	\textbf{Descrizione} 
	\\ \\
	L'utente ha inserito il comando per selezionare un database, ora deve inserire il nome del database da selezionare.
	\begin{table}[H]
			\begin{tabularx}{\textwidth}{r X}
				\textbf{Codice gerarchico} & UC3.7.2 \\
				\noalign{\hrule height 0.5pt}
				\textbf{Nome sintetico} & Inserimento nome database da selezionare\\
				\noalign{\hrule height 0.5pt}
				\textbf{Attore principale} & Utente autenticato\\
				\noalign{\hrule height 0.5pt}
				\textbf{Attori secondari} & Nessuno \\
				\noalign{\hrule height 0.5pt}
				\textbf{Pre-condizione} & L'utente intende selezionare un database e ha inserito il comando di selezione database\\
				\noalign{\hrule height 0.5pt}
				\textbf{Post-condizione} & L'utente ha inserito con successo il nome del database da selezionare\\
				\noalign{\hrule height 0.5pt}
				\textbf{Flusso eventi} & \begin{enumerate}
				\item L'utente scrive su terminale il nome del database da selezionare 
				\end{enumerate} \\
				\noalign{\hrule height 0.5pt}
				\textbf{Scenari alternativi} & Nessuno \\
				\noalign{\hrule height 0.5pt}
				\textbf{Lista requisiti\newline dedotti} & \begin{itemize}
				\item ...
				\end{itemize} 
			\end{tabularx}
			\caption{Caso d'uso UC 3.7.2 - Inserimento nome database da selezionare}
		 \end{table}		 	
		 
		 \subsection{UC 3.8: Errore esportazione fallita}
	\textbf{Descrizione} 
	\\ \\
	L'utente ha richiesto l'esportazione di uno o più database, l'operazione è fallita dunque l'utente riceve un messaggio di errore informativo e la possibilità di richiedere una nuova operazione.
	\begin{table}[H]
			\begin{tabularx}{\textwidth}{r X}
				\textbf{Codice gerarchico} & UC3.8 \\
				\noalign{\hrule height 0.5pt}
				\textbf{Nome sintetico} & Errore esportazione fallita\\
				\noalign{\hrule height 0.5pt}
				\textbf{Attore principale} & Utente autenticato\\
				\noalign{\hrule height 0.5pt}
				\textbf{Attori secondari} & Nessuno \\
				\noalign{\hrule height 0.5pt}
				\textbf{Pre-condizione} & L'utente ha richiesto un'operazione di esportazione database\\
				\noalign{\hrule height 0.5pt}
				\textbf{Post-condizione} & L'operazione richiesta non è stata eseguita con successo, nessun database è stato esportato, l'utente ha ricevuto un messaggio di errore e ora può richiedere una nuova operazione\\
				\noalign{\hrule height 0.5pt}
				\textbf{Flusso eventi} & \begin{enumerate}
				\item L'utente riceve un messaggio di errore informativo sul terminale
				\item L'utente riceve la possibilità di inserire un nuovo comando
				\end{enumerate} \\
				\noalign{\hrule height 0.5pt}
				\textbf{Scenari alternativi} & Nessuno \\
				\noalign{\hrule height 0.5pt}
				\textbf{Lista requisiti\newline dedotti} & \begin{itemize}
				\item ...
				\end{itemize} 
			\end{tabularx}
			\caption{Caso d'uso UC 3.8 - Errore esportazione fallita}
		 \end{table}		 	 	 	 
		 
		 
		 \subsection{UC 3.9: Errore importazione fallita}
	\textbf{Descrizione} 
	\\ \\
	L'utente ha richiesto l'importazione di un database, l'operazione è fallita dunque l'utente riceve un messaggio di errore informativo e la possibilità di richiedere una nuova operazione.
	\begin{table}[H]
			\begin{tabularx}{\textwidth}{r X}
				\textbf{Codice gerarchico} & UC3.9 \\
				\noalign{\hrule height 0.5pt}
				\textbf{Nome sintetico} & Errore importazione fallita\\
				\noalign{\hrule height 0.5pt}
				\textbf{Attore principale} & Utente autenticato\\
				\noalign{\hrule height 0.5pt}
				\textbf{Attori secondari} & Nessuno \\
				\noalign{\hrule height 0.5pt}
				\textbf{Pre-condizione} & L'utente ha richiesto un'operazione di importazione database\\
				\noalign{\hrule height 0.5pt}
				\textbf{Post-condizione} & L'operazione richiesta non è stata eseguita con successo, nessun database è stato importato, l'utente ha ricevuto un messaggio di errore e ora può richiedere una nuova operazione\\
				\noalign{\hrule height 0.5pt}
				\textbf{Flusso eventi} & \begin{enumerate}
				\item L'utente riceve un messaggio di errore informativo sul terminale
				\item L'utente riceve la possibilità di inserire un nuovo comando
				\end{enumerate} \\
				\noalign{\hrule height 0.5pt}
				\textbf{Scenari alternativi} & Nessuno \\
				\noalign{\hrule height 0.5pt}
				\textbf{Lista requisiti\newline dedotti} & \begin{itemize}
				\item ...
				\end{itemize} 
			\end{tabularx}
			\caption{Caso d'uso UC 3.9 - Errore importazione fallita}
		 \end{table}	
		 
		 \subsection{UC 3.10: Errore creazione fallita}
	\textbf{Descrizione} 
	\\ \\
	L'utente ha richiesto la creazione di un database, l'operazione è fallita dunque l'utente riceve un messaggio di errore informativo e la possibilità di richiedere una nuova operazione.
	\begin{table}[H]
			\begin{tabularx}{\textwidth}{r X}
				\textbf{Codice gerarchico} & UC3.10 \\
				\noalign{\hrule height 0.5pt}
				\textbf{Nome sintetico} & Errore creazione fallita\\
				\noalign{\hrule height 0.5pt}
				\textbf{Attore principale} & Utente autenticato\\
				\noalign{\hrule height 0.5pt}
				\textbf{Attori secondari} & Nessuno \\
				\noalign{\hrule height 0.5pt}
				\textbf{Pre-condizione} & L'utente ha richiesto un'operazione di creazione database\\
				\noalign{\hrule height 0.5pt}
				\textbf{Post-condizione} & L'operazione richiesta non è stata eseguita con successo, nessun database è stato creato, l'utente ha ricevuto un messaggio di errore e ora può richiedere una nuova operazione\\
				\noalign{\hrule height 0.5pt}
				\textbf{Flusso eventi} & \begin{enumerate}
				\item L'utente riceve un messaggio di errore informativo sul terminale
				\item L'utente riceve la possibilità di inserire un nuovo comando
				\end{enumerate} \\
				\noalign{\hrule height 0.5pt}
				\textbf{Scenari alternativi} & Nessuno \\
				\noalign{\hrule height 0.5pt}
				\textbf{Lista requisiti\newline dedotti} & \begin{itemize}
				\item ...
				\end{itemize} 
			\end{tabularx}
			\caption{Caso d'uso UC 3.10 - Errore creazione fallita}
		 \end{table}			
		 
		 \subsection{UC 3.11: Errore database inesistente}
	\textbf{Descrizione} 
	\\ \\
	L'utente ha tentato di accedere ad un database non presente sul server. L'accesso al database richiesto è fallito, l'utente riceve un messaggio informativo e la possibilità di richiedere una nuova operazione.
	\begin{table}[H]
			\begin{tabularx}{\textwidth}{r X}
				\textbf{Codice gerarchico} & UC3.11 \\
				\noalign{\hrule height 0.5pt}
				\textbf{Nome sintetico} & Errore database inesistente\\
				\noalign{\hrule height 0.5pt}
				\textbf{Attore principale} & Utente autenticato\\
				\noalign{\hrule height 0.5pt}
				\textbf{Attori secondari} & Nessuno \\
				\noalign{\hrule height 0.5pt}
				\textbf{Pre-condizione} & L'utente ha richiesto l'accesso ad un determinato database, inserendone il nome\\
				\noalign{\hrule height 0.5pt}
				\textbf{Post-condizione} & Il database richiesto non è presente sul server, l'utente ha ricevuto un messaggio di errore e la possibilità di inserire un nuovo comando\\
				\noalign{\hrule height 0.5pt}
				\textbf{Flusso eventi} & \begin{enumerate}
				\item L'utente riceve un messaggio di errore informativo sul terminale
				\item L'utente riceve la possibilità di inserire un nuovo comando
				\end{enumerate} \\
				\noalign{\hrule height 0.5pt}
				\textbf{Scenari alternativi} & Nessuno \\
				\noalign{\hrule height 0.5pt}
				\textbf{Lista requisiti\newline dedotti} & \begin{itemize}
				\item ...
				\end{itemize} 
			\end{tabularx}
			\caption{Caso d'uso UC 3.11 - Errore database inesistente}
		 \end{table}		
		 
		 \subsection{UC 3.12: Errore permessi lettura insufficienti}
	\textbf{Descrizione} 
	\\ \\
	L'utente ha tentato di accedere ad un database esistente, non avendo i permessi necessari per accedere al database. L'accesso al database richiesto è fallito, l'utente riceve un messaggio informativo e la possibilità di richiedere una nuova operazione.
	\begin{table}[H]
			\begin{tabularx}{\textwidth}{r X}
				\textbf{Codice gerarchico} & UC3.12 \\
				\noalign{\hrule height 0.5pt}
				\textbf{Nome sintetico} & Errore permessi lettura insufficienti\\
				\noalign{\hrule height 0.5pt}
				\textbf{Attore principale} & Utente autenticato\\
				\noalign{\hrule height 0.5pt}
				\textbf{Attori secondari} & Nessuno \\
				\noalign{\hrule height 0.5pt}
				\textbf{Pre-condizione} & L'utente ha richiesto l'accesso ad un determinato database, inserendone il nome\\
				\noalign{\hrule height 0.5pt}
				\textbf{Post-condizione} & Il database richiesto è presente sul server ma l'utente non ha i permessi per accedervi, l'utente ha ricevuto un messaggio di errore e la possibilità di inserire un nuovo comando\\
				\noalign{\hrule height 0.5pt}
				\textbf{Flusso eventi} & \begin{enumerate}
				\item L'utente riceve un messaggio di errore informativo sul terminale
				\item L'utente riceve la possibilità di inserire un nuovo comando
				\end{enumerate} \\
				\noalign{\hrule height 0.5pt}
				\textbf{Scenari alternativi} & Nessuno \\
				\noalign{\hrule height 0.5pt}
				\textbf{Lista requisiti\newline dedotti} & \begin{itemize}
				\item ...
				\end{itemize} 
			\end{tabularx}
			\caption{Caso d'uso UC 3.12 - Errore permessi lettura insufficienti}
		 \end{table}		
		 
		 
		 
		 \subsection{UC 4: Operazioni a livello database}
	 \begin{figure}[H]
				\centering
				\includegraphics[scale=0.2]{UC/"UC 4 Operazioni a livello database".png}
				\caption{Diagramma di UC4: Operazioni a livello database}
			\end{figure}
	\textbf{Descrizione} 
	\\ \\
	L'utente ha effettuato correttamente la connessione e ha selezionato un database di cui dispone dei permessi di accesso. Su tale database può effettuare diverse operazioni a livello database:
	\begin{itemize}
	\item Visualizzare la lista delle mappe presenti
	\item Creare una mappa
	\item Eliminare una mappa
	\item Rinominare una mappa presente
	\item Selezionare una mappa per effettuare operazioni su di essa
	\item Visualizzare i permessi di accesso al database
	\item Modificare i permessi di accesso al database
	\end{itemize}
	Queste operazioni possono portare a diverse situazioni di errore. In tal caso non viene eseguita alcuna operazione e l'utente riceve un messaggio di errore esplicativo. 
	\begin{table}[H]
			\begin{tabularx}{\textwidth}{r X}
				\textbf{Codice gerarchico} & UC4 \\
				\noalign{\hrule height 0.5pt}
				\textbf{Nome sintetico} & Operazioni a livello database\\
				\noalign{\hrule height 0.5pt}
				\textbf{Attore principale} & Utente autenticato\\
				\noalign{\hrule height 0.5pt}
				\textbf{Attori secondari} & Nessuno \\
				\noalign{\hrule height 0.5pt}
				\textbf{Pre-condizione} & L'utente ha selezionato correttamente un database\\
				\noalign{\hrule height 0.5pt}
				\textbf{Post-condizione} & L'operazione sul database selezionata è stata eseguita correttamente\\
				\noalign{\hrule height 0.5pt}
				\textbf{Flusso eventi} & \begin{enumerate}
				\item L'utente inserisce il comando e le informazioni necessarie per l'operazione richiesta e preme invio
				\end{enumerate} \\
				\noalign{\hrule height 0.5pt}
				\textbf{Scenari alternativi} & \begin{enumerate}
				\item L'utente ha richiesto la creazione di una nuova mappa, la creazione è fallita e l'utente riceve un messaggio di errore (UC 4.8)
				\item L'utente ha richiesto un'operazione di modifica del database ma non dispone dei permessi di scrittura per tale database, l'operazione non viene effettuata e l'utente riceve un messaggio di errore (UC 4.9)
				\item L'utente ha tentato di accedere ad una mappa non presente, l'accesso è fallito e l'utente riceve un messaggio di errore (UC 4.10)
				\item L'utente ha richiesto un'operazione relativa ai permessi di accesso al database non disponendo dei permessi di amministrazione necessari, l'operazione viene rifiutata e l'utente riceve un messaggio di errore (UC 4.11)
\end{enumerate}				 \\
				\noalign{\hrule height 0.5pt}
				\textbf{Lista requisiti\newline dedotti} & \begin{itemize}
				\item ...
				\end{itemize} 
			\end{tabularx}
			\caption{Caso d'uso UC 4 - Operazioni a livello database}
		 \end{table} 
		 
		 
		 \subsection{UC 4.1: Visualizzazione lista delle mappe}
	\textbf{Descrizione} 
	\\ \\
	L'utente intende visualizzare la lista delle mappe che compongono il database su cui sta operando. Inserisce quindi il comando \texttt{SHOW} e il sistema stampa sul terminale la lista.
	\begin{table}[H]
			\begin{tabularx}{\textwidth}{r X}
				\textbf{Codice gerarchico} & UC4.1 \\
				\noalign{\hrule height 0.5pt}
				\textbf{Nome sintetico} & Visualizzazione lista delle mappe\\
				\noalign{\hrule height 0.5pt}
				\textbf{Attore principale} & Utente autenticato\\
				\noalign{\hrule height 0.5pt}
				\textbf{Attori secondari} & Nessuno \\
				\noalign{\hrule height 0.5pt}
				\textbf{Pre-condizione} & L'utente intende visualizzare la lista delle mappe che compongono il database selezionato\\
				\noalign{\hrule height 0.5pt}
				\textbf{Post-condizione} & L'utente riceve la lista delle mappe presenti stampata a video\\
				\noalign{\hrule height 0.5pt}
				\textbf{Flusso eventi} & \begin{enumerate}
				\item L'utente inserisce il comando \texttt{SHOW} e preme invio
				\end{enumerate} \\
				\noalign{\hrule height 0.5pt}
				\textbf{Scenari alternativi} & Nessuno \\
				\noalign{\hrule height 0.5pt}
				\textbf{Lista requisiti\newline dedotti} & \begin{itemize}
				\item ...
				\end{itemize} 
			\end{tabularx}
			\caption{Caso d'uso UC 4.1 - Visualizzazione lista delle mappe}
		 \end{table} 
		 
		 
		 \subsection{UC 4.2: Creazione mappa}
	 \begin{figure}[H]
				\centering
				\includegraphics[scale=0.25]{UC/"UC 4-2 Creazione mappa".png}
				\caption{Diagramma di UC4.2: Creazione mappa}
			\end{figure}
	\textbf{Descrizione} 
	\\ \\
	L'utente intende creare una nuova mappa inserendo il comando di creazione seguito dal nome della mappa da creare.
	\begin{table}[H]
			\begin{tabularx}{\textwidth}{r X}
				\textbf{Codice gerarchico} & UC4.2 \\
				\noalign{\hrule height 0.5pt}
				\textbf{Nome sintetico} & Creazione mappa\\
				\noalign{\hrule height 0.5pt}
				\textbf{Attore principale} & Utente autenticato\\
				\noalign{\hrule height 0.5pt}
				\textbf{Attori secondari} & Nessuno \\
				\noalign{\hrule height 0.5pt}
				\textbf{Pre-condizione} & L'utente intende creare una mappa\\
				\noalign{\hrule height 0.5pt}
				\textbf{Post-condizione} & L'utente ha creato con successo una mappa con il nome inserito\\
				\noalign{\hrule height 0.5pt}
				\textbf{Flusso eventi} & \begin{enumerate}
				\item L'utente inserisce il comando di creazione mappa(UC 4.2.1)
				\item L'utente inserisce il nome della mappa da creare (UC 4.2.2) e preme invio
				\end{enumerate} \\
				\noalign{\hrule height 0.5pt}
				\textbf{Scenari alternativi} & Nessuno \\
				\noalign{\hrule height 0.5pt}
				\textbf{Lista requisiti\newline dedotti} & \begin{itemize}
				\item ...
				\end{itemize} 
			\end{tabularx}
			\caption{Caso d'uso UC 4.2 - Creazione mappa}
		 \end{table}
		 
		 
		  \subsection{UC 4.2.1: Inserimento comando di creazione mappa}
	\textbf{Descrizione} 
	\\ \\
	L'utente intende creare una nuova mappa, a tal fine deve come prima cosa inserire il comando di creazione di una mappa.
	\begin{table}[H]
			\begin{tabularx}{\textwidth}{r X}
				\textbf{Codice gerarchico} & UC4.2.1 \\
				\noalign{\hrule height 0.5pt}
				\textbf{Nome sintetico} & Inserimento comando di creazione mappa\\
				\noalign{\hrule height 0.5pt}
				\textbf{Attore principale} & Utente autenticato\\
				\noalign{\hrule height 0.5pt}
				\textbf{Attori secondari} & Nessuno \\
				\noalign{\hrule height 0.5pt}
				\textbf{Pre-condizione} & L'utente intende creare una mappa\\
				\noalign{\hrule height 0.5pt}
				\textbf{Post-condizione} & L'utente ha inserito con successo il comando di creazione mappa\\
				\noalign{\hrule height 0.5pt}
				\textbf{Flusso eventi} & \begin{enumerate}
				\item L'utente scrive su terminale il comando di creazione mappa
				\end{enumerate} \\
				\noalign{\hrule height 0.5pt}
				\textbf{Scenari alternativi} & Nessuno \\
				\noalign{\hrule height 0.5pt}
				\textbf{Lista requisiti\newline dedotti} & \begin{itemize}
				\item ...
				\end{itemize} 
			\end{tabularx}
			\caption{Caso d'uso UC 4.2.1 - Inserimento comando di creazione mappa}
		 \end{table}
		 
		 \subsection{UC 4.2.2: Inserimento nome mappa da creare}
	\textbf{Descrizione} 
	\\ \\
	L'utente intende creare una nuova mappa, ha inserito il comando di creazione di una mappa e ora deve inserire il nome che intende dare alla nuova mappa.
	\begin{table}[H]
			\begin{tabularx}{\textwidth}{r X}
				\textbf{Codice gerarchico} & UC4.2.2 \\
				\noalign{\hrule height 0.5pt}
				\textbf{Nome sintetico} & Inserimento nome mappa da creare\\
				\noalign{\hrule height 0.5pt}
				\textbf{Attore principale} & Utente autenticato\\
				\noalign{\hrule height 0.5pt}
				\textbf{Attori secondari} & Nessuno \\
				\noalign{\hrule height 0.5pt}
				\textbf{Pre-condizione} & L'utente ha inserito il comando di creazione mappa\\
				\noalign{\hrule height 0.5pt}
				\textbf{Post-condizione} & L'utente ha inserito con successo il nome della mappa da creare\\
				\noalign{\hrule height 0.5pt}
				\textbf{Flusso eventi} & \begin{enumerate}
				\item L'utente scrive su terminale il nome della mappa da creare
				\end{enumerate} \\
				\noalign{\hrule height 0.5pt}
				\textbf{Scenari alternativi} & Nessuno \\
				\noalign{\hrule height 0.5pt}
				\textbf{Lista requisiti\newline dedotti} & \begin{itemize}
				\item ...
				\end{itemize} 
			\end{tabularx}
			\caption{Caso d'uso UC 4.2.2 - Inserimento nome mappa da creare}
		 \end{table}
		 
		 
		 \subsection{UC 4.3: Eliminazione mappa}
	 \begin{figure}[H]
				\centering
				\includegraphics[scale=0.25]{UC/"UC 4-3 Eliminazione mappa".png}
				\caption{Diagramma di UC4.3: Eliminazione mappa}
			\end{figure}
	\textbf{Descrizione} 
	\\ \\
	L'utente intende eliminare una mappa presente nel database selezionato in precedenza, inserendo il comando per eliminare una mappa, seguito dal nome della mappa da eliminare.
	\begin{table}[H]
			\begin{tabularx}{\textwidth}{r X}
				\textbf{Codice gerarchico} & UC4.3 \\
				\noalign{\hrule height 0.5pt}
				\textbf{Nome sintetico} & Eliminazione mappa\\
				\noalign{\hrule height 0.5pt}
				\textbf{Attore principale} & Utente autenticato\\
				\noalign{\hrule height 0.5pt}
				\textbf{Attori secondari} & Nessuno \\
				\noalign{\hrule height 0.5pt}
				\textbf{Pre-condizione} & L'utente intende eliminare una mappa\\
				\noalign{\hrule height 0.5pt}
				\textbf{Post-condizione} & L'utente ha eliminato con successo la mappa selezionata\\
				\noalign{\hrule height 0.5pt}
				\textbf{Flusso eventi} & \begin{enumerate}
				\item L'utente inserisce il comando di eliminazione mappa (UC 4.3.1)
				\item L'utente inserisce il nome della mappa da eliminare (UC 4.3.2) e preme invio
				\end{enumerate} \\
				\noalign{\hrule height 0.5pt}
				\textbf{Scenari alternativi} & Nessuno \\
				\noalign{\hrule height 0.5pt}
				\textbf{Lista requisiti\newline dedotti} & \begin{itemize}
				\item ...
				\end{itemize} 
			\end{tabularx}
			\caption{Caso d'uso UC 4.3 - Eliminazione mappa}
		 \end{table}
		 
		 
		\subsection{UC 4.3.1: Inserimento comando di eliminazione mappa}
	\textbf{Descrizione} 
	\\ \\
	L'utente intende eliminare una mappa, a tal fine deve per prima cosa scrivere su terminale il comando di eliminazione di una mappa.
	\begin{table}[H]
			\begin{tabularx}{\textwidth}{r X}
				\textbf{Codice gerarchico} & UC4.3.1 \\
				\noalign{\hrule height 0.5pt}
				\textbf{Nome sintetico} & Inserimento comando di eliminazione mappa\\
				\noalign{\hrule height 0.5pt}
				\textbf{Attore principale} & Utente autenticato\\
				\noalign{\hrule height 0.5pt}
				\textbf{Attori secondari} & Nessuno \\
				\noalign{\hrule height 0.5pt}
				\textbf{Pre-condizione} & L'utente intende eliminare una mappa\\
				\noalign{\hrule height 0.5pt}
				\textbf{Post-condizione} & L'utente ha inserito con successo il comando di eliminazione di una mappa\\
				\noalign{\hrule height 0.5pt}
				\textbf{Flusso eventi} & \begin{enumerate}
				\item L'utente scrive su terminale il comando di eliminazione mappa
				\end{enumerate} \\
				\noalign{\hrule height 0.5pt}
				\textbf{Scenari alternativi} & Nessuno \\
				\noalign{\hrule height 0.5pt}
				\textbf{Lista requisiti\newline dedotti} & \begin{itemize}
				\item ...
				\end{itemize} 
			\end{tabularx}
			\caption{Caso d'uso UC 4.3.1 - Inserimento comando di eliminazione mappa}
		 \end{table}		 
		  
		\subsection{UC 4.3.2: Inserimento nome mappa da eliminare}
	\textbf{Descrizione} 
	\\ \\
	L'utente intende eliminare una mappa, ha inserito il comando di eliminazione di una mappa e ora deve inserire il nome della mappa da eliminare.
	\begin{table}[H]
			\begin{tabularx}{\textwidth}{r X}
				\textbf{Codice gerarchico} & UC4.3.2 \\
				\noalign{\hrule height 0.5pt}
				\textbf{Nome sintetico} & Inserimento nome mappa da eliminare\\
				\noalign{\hrule height 0.5pt}
				\textbf{Attore principale} & Utente autenticato\\
				\noalign{\hrule height 0.5pt}
				\textbf{Attori secondari} & Nessuno \\
				\noalign{\hrule height 0.5pt}
				\textbf{Pre-condizione} & L'utente intende eliminare una mappa e ha inserito il comando di eliminazione di una mappa\\
				\noalign{\hrule height 0.5pt}
				\textbf{Post-condizione} & L'utente ha inserito con successo il nome della mappa da eliminare\\
				\noalign{\hrule height 0.5pt}
				\textbf{Flusso eventi} & \begin{enumerate}
				\item L'utente scrive su terminale il nome della mappa da eliminare
				\end{enumerate} \\
				\noalign{\hrule height 0.5pt}
				\textbf{Scenari alternativi} & Nessuno \\
				\noalign{\hrule height 0.5pt}
				\textbf{Lista requisiti\newline dedotti} & \begin{itemize}
				\item ...
				\end{itemize} 
			\end{tabularx}
			\caption{Caso d'uso UC 4.3.2 - Inserimento nome mappa da eliminare}
		 \end{table}		 
		 
		\subsection{UC 4.4: Rinominazione mappa}
	 \begin{figure}[H]
				\centering
				\includegraphics[scale=0.25]{UC/"UC 4-4 Rinominazione mappa".png}
				\caption{Diagramma di UC4.4: Rinominazione mappa}
			\end{figure}
	\textbf{Descrizione} 
	\\ \\
	L'utente intende rinominare una mappa inserendo il comando per rinominare una mappa, seguito dal nome della mappa da rinominare e il nuovo nome.
	\begin{table}[H]
			\begin{tabularx}{\textwidth}{r X}
				\textbf{Codice gerarchico} & UC4.4 \\
				\noalign{\hrule height 0.5pt}
				\textbf{Nome sintetico} & Rinominazione mappa\\
				\noalign{\hrule height 0.5pt}
				\textbf{Attore principale} & Utente autenticato\\
				\noalign{\hrule height 0.5pt}
				\textbf{Attori secondari} & Nessuno \\
				\noalign{\hrule height 0.5pt}
				\textbf{Pre-condizione} & L'utente intende rinominare una mappa\\
				\noalign{\hrule height 0.5pt}
				\textbf{Post-condizione} & L'utente ha rinominato con successo la mappa selezionata\\
				\noalign{\hrule height 0.5pt}
				\textbf{Flusso eventi} & \begin{enumerate}
				\item L'utente inserisce il comando di rinominazione mappa (UC 4.4.1)
				\item L'utente inserisce il nome della mappa da rinominare (UC 4.4.2)
				\item L'utente inserisce il nuovo nome per la mappa (UC 4.4.3) e preme invio
				\end{enumerate} \\
				\noalign{\hrule height 0.5pt}
				\textbf{Scenari alternativi} & Nessuno \\
				\noalign{\hrule height 0.5pt}
				\textbf{Lista requisiti\newline dedotti} & \begin{itemize}
				\item ...
				\end{itemize} 
			\end{tabularx}
			\caption{Caso d'uso UC 4.4 - Rinominazione mappa}
		 \end{table}		 
		 
		 \subsection{UC 4.4.1: Inserimento comando di rinominazione mappa}
	\textbf{Descrizione} 
	\\ \\
	L'utente intende rinominare una mappa, a tal fine deve per prima cosa inserire il comando di rinominazione mappa.
	\begin{table}[H]
			\begin{tabularx}{\textwidth}{r X}
				\textbf{Codice gerarchico} & UC4.4.1 \\
				\noalign{\hrule height 0.5pt}
				\textbf{Nome sintetico} & Inserimento comando di rinominazione mappa\\
				\noalign{\hrule height 0.5pt}
				\textbf{Attore principale} & Utente autenticato\\
				\noalign{\hrule height 0.5pt}
				\textbf{Attori secondari} & Nessuno \\
				\noalign{\hrule height 0.5pt}
				\textbf{Pre-condizione} & L'utente intende rinominare una mappa\\
				\noalign{\hrule height 0.5pt}
				\textbf{Post-condizione} & L'utente ha inserito con successo il comando di rinominazione mappa\\
				\noalign{\hrule height 0.5pt}
				\textbf{Flusso eventi} & \begin{enumerate}
				\item L'utente scrive su terminale il comando di rinominazione mappa
				\end{enumerate} \\
				\noalign{\hrule height 0.5pt}
				\textbf{Scenari alternativi} & Nessuno \\
				\noalign{\hrule height 0.5pt}
				\textbf{Lista requisiti\newline dedotti} & \begin{itemize}
				\item ...
				\end{itemize} 
			\end{tabularx}
			\caption{Caso d'uso UC 4.4.1 - Inserimento comando di rinominazione mappa}
		 \end{table}		 
		 
		 \subsection{UC 4.4.2: Inserimento nome mappa da rinominare}
	\textbf{Descrizione} 
	\\ \\
	L'utente intende rinominare una mappa, ha inserito il comando per rinominare una mappa, ora deve inserire il nome della mappa che vuole rinominare.
	\begin{table}[H]
			\begin{tabularx}{\textwidth}{r X}
				\textbf{Codice gerarchico} & UC4.4.2 \\
				\noalign{\hrule height 0.5pt}
				\textbf{Nome sintetico} & Inserimento nome mappa da rinominare\\
				\noalign{\hrule height 0.5pt}
				\textbf{Attore principale} & Utente autenticato\\
				\noalign{\hrule height 0.5pt}
				\textbf{Attori secondari} & Nessuno \\
				\noalign{\hrule height 0.5pt}
				\textbf{Pre-condizione} & L'utente intende rinominare una mappa e ha inserito il comando di rinominazione mappa\\
				\noalign{\hrule height 0.5pt}
				\textbf{Post-condizione} & L'utente ha inserito con successo il nome della mappa da rinominare\\
				\noalign{\hrule height 0.5pt}
				\textbf{Flusso eventi} & \begin{enumerate}
				\item L'utente scrive su terminale il nome della mappa da rinominare
				\end{enumerate} \\
				\noalign{\hrule height 0.5pt}
				\textbf{Scenari alternativi} & Nessuno \\
				\noalign{\hrule height 0.5pt}
				\textbf{Lista requisiti\newline dedotti} & \begin{itemize}
				\item ...
				\end{itemize} 
			\end{tabularx}
			\caption{Caso d'uso UC 4.4.2 - Inserimento nome mappa da rinominare}
		 \end{table}		 
		 
		 \subsection{UC 4.4.3: Inserimento nuovo nome mappa}
	\textbf{Descrizione} 
	\\ \\
	L'utente intende rinominare una mappa, ha inserito il comando per rinominare una mappa e il nome della mappa che vuole rinominare, ora deve inserire il nuovo nome per la mappa.
	\begin{table}[H]
			\begin{tabularx}{\textwidth}{r X}
				\textbf{Codice gerarchico} & UC4.4.3 \\
				\noalign{\hrule height 0.5pt}
				\textbf{Nome sintetico} & Inserimento nuovo nome mappa\\
				\noalign{\hrule height 0.5pt}
				\textbf{Attore principale} & Utente autenticato\\
				\noalign{\hrule height 0.5pt}
				\textbf{Attori secondari} & Nessuno \\
				\noalign{\hrule height 0.5pt}
				\textbf{Pre-condizione} & L'utente intende rinominare una mappa e ha inserito il comando di rinominazione mappa seguito dal nome della mappa da rinominare\\
				\noalign{\hrule height 0.5pt}
				\textbf{Post-condizione} & L'utente ha inserito con successo il nuovo nome per la mappa\\
				\noalign{\hrule height 0.5pt}
				\textbf{Flusso eventi} & \begin{enumerate}
				\item L'utente scrive su terminale il nuovo nome per la mappa
				\end{enumerate} \\
				\noalign{\hrule height 0.5pt}
				\textbf{Scenari alternativi} & Nessuno \\
				\noalign{\hrule height 0.5pt}
				\textbf{Lista requisiti\newline dedotti} & \begin{itemize}
				\item ...
				\end{itemize} 
			\end{tabularx}
			\caption{Caso d'uso UC 4.4.3 - Inserimento nuovo nome mappa}
		 \end{table}		 
		 
		 
		 \subsection{UC 4.5: Selezione mappa}
	 \begin{figure}[H]
				\centering
				\includegraphics[scale=0.25]{UC/"UC 4-5 Selezione mappa".png}
				\caption{Diagramma di UC4.5: Selezione mappa}
			\end{figure}
	\textbf{Descrizione} 
	\\ \\
	L'utente intende selezionare una delle mappe che compongono il database su cui sta operando al fine di effettuare operazioni a livello di item su di essa. Per selezionare una mappa l'utente deve inserire l'apposito comando da terminale, seguito dal nome della mappa che vuole selezionare.
	\begin{table}[H]
			\begin{tabularx}{\textwidth}{r X}
				\textbf{Codice gerarchico} & UC4.5 \\
				\noalign{\hrule height 0.5pt}
				\textbf{Nome sintetico} & Selezione mappa\\
				\noalign{\hrule height 0.5pt}
				\textbf{Attore principale} & Utente autenticato\\
				\noalign{\hrule height 0.5pt}
				\textbf{Attori secondari} & Nessuno \\
				\noalign{\hrule height 0.5pt}
				\textbf{Pre-condizione} & L'utente intende selezionare una mappa\\
				\noalign{\hrule height 0.5pt}
				\textbf{Post-condizione} & L'utente ha selezionato con successo la mappa richiesta\\
				\noalign{\hrule height 0.5pt}
				\textbf{Flusso eventi} & \begin{enumerate}
				\item L'utente inserisce il comando di selezione mappa (UC 4.5.1)
				\item L'utente inserisce il nome della mappa da selezionare (UC 4.5.2) e preme invio
				\end{enumerate} \\
				\noalign{\hrule height 0.5pt}
				\textbf{Scenari alternativi} & Nessuno \\
				\noalign{\hrule height 0.5pt}
				\textbf{Lista requisiti\newline dedotti} & \begin{itemize}
				\item ...
				\end{itemize} 
			\end{tabularx}
			\caption{Caso d'uso UC 4.5 - Selezione mappa}
		 \end{table}	
		 
		 
		 \subsection{UC 4.5.1: Inserimento comando di selezione mappa}
	\textbf{Descrizione} 
	\\ \\
	L'utente intende selezionare una delle mappe che compongono il database. Per effettuare l'operazione di selezione deve in primo luogo inserire il comando di selezione mappa da terminale.
	\begin{table}[H]
			\begin{tabularx}{\textwidth}{r X}
				\textbf{Codice gerarchico} & UC4.5.1 \\
				\noalign{\hrule height 0.5pt}
				\textbf{Nome sintetico} & Inserimento comando di selezione mappa\\
				\noalign{\hrule height 0.5pt}
				\textbf{Attore principale} & Utente autenticato\\
				\noalign{\hrule height 0.5pt}
				\textbf{Attori secondari} & Nessuno \\
				\noalign{\hrule height 0.5pt}
				\textbf{Pre-condizione} & L'utente intende selezionare una mappa\\
				\noalign{\hrule height 0.5pt}
				\textbf{Post-condizione} & L'utente ha inserito con successo il comando di selezione mappa \\
				\noalign{\hrule height 0.5pt}
				\textbf{Flusso eventi} & \begin{enumerate}
				\item L'utente scrive su terminale il comando di selezione mappa 
				\end{enumerate} \\
				\noalign{\hrule height 0.5pt}
				\textbf{Scenari alternativi} & Nessuno \\
				\noalign{\hrule height 0.5pt}
				\textbf{Lista requisiti\newline dedotti} & \begin{itemize}
				\item ...
				\end{itemize} 
			\end{tabularx}
			\caption{Caso d'uso UC 4.5.1 - Inserimento comando di selezione mappa}
		 \end{table}		 
		 
		 \subsection{UC 4.5.2: Inserimento nome mappa da selezionare}
	\textbf{Descrizione} 
	\\ \\
	L'utente ha inserito il comando per selezionare una mappa, ora deve inserire il nome della mappa da selezionare.
	\begin{table}[H]
			\begin{tabularx}{\textwidth}{r X}
				\textbf{Codice gerarchico} & UC4.5.2 \\
				\noalign{\hrule height 0.5pt}
				\textbf{Nome sintetico} & Inserimento nome mappa da selezionare\\
				\noalign{\hrule height 0.5pt}
				\textbf{Attore principale} & Utente autenticato\\
				\noalign{\hrule height 0.5pt}
				\textbf{Attori secondari} & Nessuno \\
				\noalign{\hrule height 0.5pt}
				\textbf{Pre-condizione} & L'utente intende selezionare una mappa e ha inserito il comando di selezione mappa \\
				\noalign{\hrule height 0.5pt}
				\textbf{Post-condizione} & L'utente ha inserito con successo il nome della mappa da selezionare\\
				\noalign{\hrule height 0.5pt}
				\textbf{Flusso eventi} & \begin{enumerate}
				\item L'utente scrive su terminale il nome della mappa da selezionare 
				\end{enumerate} \\
				\noalign{\hrule height 0.5pt}
				\textbf{Scenari alternativi} & Nessuno \\
				\noalign{\hrule height 0.5pt}
				\textbf{Lista requisiti\newline dedotti} & \begin{itemize}
				\item ...
				\end{itemize} 
			\end{tabularx}
			\caption{Caso d'uso UC 4.5.2 - Inserimento nome mappa da selezionare}
		 \end{table}		
		 
		 \subsection{UC 4.6: Visualizzazione permessi accesso}
	 \begin{figure}[H]
				\centering
				\includegraphics[scale=0.3]{UC/"UC 4-6 Visualizzazione permessi".png}
				\caption{Diagramma di UC4.6: Visualizzazione aiuto}
			\end{figure}
	\textbf{Descrizione} 
	\\ \\
L'utente avente i permessi di amministratore, vuole visualizzare i permessi di altri utenti. Ha a disposizione due possibilità: 
	\begin{itemize}
		\item Visualizzare la lista dei permessi relativi all'intero database
		\item Visualizzare i permessi di accesso al database per un singolo utente
	\end{itemize}
	La prima modalità stampa sul terminale la lista completa degli utenti che hanno un qualche tipo di accesso al database selezionato, con relativi permessi di amministrazione, scrittura e lettura. \\
	La modalità di visualizzazione permessi utente singolo richiede l'inserimento del nome dell'utente per cui si richiedono le informazioni, stampa sul terminale il nome dell'utente seguito dai suoi permessi.
	\begin{table}[H]
			\begin{tabularx}{\textwidth}{r X}
				\textbf{Codice gerarchico} & UC4.6 \\
				\noalign{\hrule height 0.5pt}
				\textbf{Nome sintetico} & Visualizzazione permessi accesso \\
				\noalign{\hrule height 0.5pt}
				\textbf{Attore principale} & Utente autenticato\\
				\noalign{\hrule height 0.5pt}
				\textbf{Attori secondari} & Nessuno \\
				\noalign{\hrule height 0.5pt}
				\textbf{Pre-condizione} & L'utente intende visualizzare i permessi di uno o tutti gli utenti\\
				\noalign{\hrule height 0.5pt}
				\textbf{Post-condizione} & L'utente ha visualizzato i permessi di uno o tutti gli utenti\\
				\noalign{\hrule height 0.5pt}
				\textbf{Flusso eventi} & \begin{enumerate}
				\item L'utente inserisce il comando relativo alla visualizzazione richiesta e preme invio
				\item L'utente riceve le informazioni di visualizzazione permessi stampate sul terminale
				\end{enumerate} \\
				\noalign{\hrule height 0.5pt}
				\textbf{Scenari alternativi} & Nessuno \\
				\noalign{\hrule height 0.5pt}
				\textbf{Lista requisiti\newline dedotti} & \begin{itemize}
				\item ...
				\end{itemize} 
			\end{tabularx}
			\caption{Caso d'uso UC 4.6 - Visualizzazione permessi accesso}
		 \end{table} 
		 
		 
		 \subsection{UC 4.6.1: Visualizzazione lista permessi}
	 \textbf{Descrizione}
	 \\ \\
	 L'utente sta effettuando l'operazione di visualizzazione dei permessi e intende inserire il comando di visualizzazione lista permessi.
	\begin{table}[H]
			\begin{tabularx}{\textwidth}{r  X}
				\textbf{Codice gerarchico} & UC4.6.1 \\
				\noalign{\hrule height 0.5pt}
				\textbf{Nome sintetico} & Visualizzazione lista permessi \\
				\noalign{\hrule height 0.5pt}
				\textbf{Attore principale} & Utente autenticato\\
				\noalign{\hrule height 0.5pt}
				\textbf{Attori secondari} & Nessuno \\
				\noalign{\hrule height 0.5pt}
				\textbf{Pre-condizione} & L'utente vuole visualizzare i permessi di tutti gli utenti che hanno un qualche tipo di accesso al database selezionato\\
				\noalign{\hrule height 0.5pt}
				\textbf{Post-condizione} & L'utente ha inserito correttamente il comando di visualizzazione lista permessi\\
				\noalign{\hrule height 0.5pt}
				\textbf{Flusso eventi} & \begin{enumerate}
				\item L'utente inserisce il comando di visualizzazione lista permessi
				\end{enumerate} \\
				\noalign{\hrule height 0.5pt}
				\textbf{Scenari alternativi} & Nessuno \\
				\noalign{\hrule height 0.5pt}
				\textbf{Lista requisiti\newline dedotti} & \begin{itemize}
				\item ...
				\end{itemize} 
			\end{tabularx}
			\caption{Caso d'uso UC 4.6.1 - Visualizzazione lista permessi }
		 \end{table} 	
	 
	 
	 \subsection{UC 4.6.2: Visualizzazione permessi utente singolo}
	 \textbf{Descrizione}
	 \\ \\
	 L'utente sta effettuando l'operazione di visualizzazione dei permessi e intende visualizzare i permessi di un singolo utente.
	\begin{table}[H]
			\begin{tabularx}{\textwidth}{r  X}
				\textbf{Codice gerarchico} & UC4.6.2 \\
				\noalign{\hrule height 0.5pt}
				\textbf{Nome sintetico} & Visualizzazione permessi utente singolo \\
				\noalign{\hrule height 0.5pt}
				\textbf{Attore principale} & Utente autenticato\\
				\noalign{\hrule height 0.5pt}
				\textbf{Attori secondari} & Nessuno \\
				\noalign{\hrule height 0.5pt}
				\textbf{Pre-condizione} & L'utente vuole visualizzare i permessi di uno specifico utente\\
				\noalign{\hrule height 0.5pt}
				\textbf{Post-condizione} & L'utente ha inserito correttamente il comando visualizzazione permessi utente singolo\\
				\noalign{\hrule height 0.5pt}
				\textbf{Flusso eventi} & \begin{enumerate}
				\item L'utente scrive su terminale il comando di visualizzazione permessi seguito dal nome dell'utente di cui vuole visualizzare i permessi
				\end{enumerate} \\
				\noalign{\hrule height 0.5pt}
				\textbf{Scenari alternativi} & Nessuno \\
				\noalign{\hrule height 0.5pt}
				\textbf{Lista requisiti\newline dedotti} & \begin{itemize}
				\item ...
				\end{itemize} 
			\end{tabularx}
			\caption{Caso d'uso UC 4.6.2 - Visualizzazione permessi utente singolo}
		 \end{table}
 		 
		 
		 
		 \subsection{UC 4.7: Modifica permessi accesso}
	 \begin{figure}[H]
				\centering
				\includegraphics[scale=0.25]{UC/"UC 4-7 Modifica permessi".png}
				\caption{Diagramma di UC 4.7: Modifica permessi accesso}
			\end{figure}
	\textbf{Descrizione} 
	\\ \\
L'utente avente i permessi di amministratore, vuole modificare i permessi di amministrazione, scrittura, lettura di un utente.
	\begin{table}[H]
			\begin{tabularx}{\textwidth}{r X}
				\textbf{Codice gerarchico} & UC4.7 \\
				\noalign{\hrule height 0.5pt}
				\textbf{Nome sintetico} & Modifica permessi accesso\\
				\noalign{\hrule height 0.5pt}
				\textbf{Attore principale} & Utente autenticato\\
				\noalign{\hrule height 0.5pt}
				\textbf{Attori secondari} & Nessuno \\
				\noalign{\hrule height 0.5pt}
				\textbf{Pre-condizione} & L'utente intende modificare i permessi di un determinato utente\\
				\noalign{\hrule height 0.5pt}
				\textbf{Post-condizione} & L'utente ha modificato con successo i permessi di un determinato utente\\
				\noalign{\hrule height 0.5pt}
				\textbf{Flusso eventi} & \begin{enumerate}
				\item L'utente inserisce il comando di modifica permessi (UC 4.7.1)
				\item L'utente inserisce il nome dell'utente a cui verranno modificati i permessi (UC 4.7.2)
				\item L'utente inserisce i valori dei nuovi permessi (UC 4.7.3) e preme invio
				\end{enumerate} \\
				\noalign{\hrule height 0.5pt}
				\textbf{Scenari alternativi} & Nessuno \\
				\noalign{\hrule height 0.5pt}
				\textbf{Lista requisiti\newline dedotti} & \begin{itemize}
				\item ...																
				\end{itemize} 
			\end{tabularx}
			\caption{Caso d'uso UC 4.7 - Modifica permessi}
		 \end{table}	
		 
		 
		 \subsection{UC 4.7.1: Inserimento comando modifica permessi}
	\textbf{Descrizione} 
	\\ \\
	L'utente avente permessi di amministratore, ha intenzione di modificare i permessi di un altro utente. Per effettuare l'operazione di modifica deve prima di tutto inserire il comando di modifica permessi da terminale.
	\begin{table}[H]
			\begin{tabularx}{\textwidth}{r X}
				\textbf{Codice gerarchico} & UC4.7.1 \\
				\noalign{\hrule height 0.5pt}
				\textbf{Nome sintetico} & Inserimento comando modifica permessi\\
				\noalign{\hrule height 0.5pt}
				\textbf{Attore principale} & Utente autenticato\\
				\noalign{\hrule height 0.5pt}
				\textbf{Attori secondari} & Nessuno \\
				\noalign{\hrule height 0.5pt}
				\textbf{Pre-condizione} & L'utente intende modificare i permessi di un determinato utente\\
				\noalign{\hrule height 0.5pt}
				\textbf{Post-condizione} & L'utente ha inserito con successo il comando di modifica permessi\\
				\noalign{\hrule height 0.5pt}
				\textbf{Flusso eventi} & \begin{enumerate}
				\item L'utente scrive su terminale il comando di modifica permessi
				\end{enumerate} \\
				\noalign{\hrule height 0.5pt}
				\textbf{Scenari alternativi} & Nessuno \\
				\noalign{\hrule height 0.5pt}
				\textbf{Lista requisiti\newline dedotti} & \begin{itemize}
				\item ...								
				\end{itemize} 
			\end{tabularx}
			\caption{Caso d'uso UC 4.7.1 - Inserimento comando modifica permessi}
		 \end{table}		 
		 
		 \subsection{UC 4.7.2: Inserimento nome utente da modificare}
	\textbf{Descrizione} 
	\\ \\
	L'utente ha inserito il comando per modificare i permessi, ora deve inserire il nome dell'utente a cui verranno modificati i permessi.
	\begin{table}[H]
			\begin{tabularx}{\textwidth}{r X}
				\textbf{Codice gerarchico} & UC4.7.2 \\
				\noalign{\hrule height 0.5pt}
				\textbf{Nome sintetico} & Inserimento nome utente da modificare\\
				\noalign{\hrule height 0.5pt}
				\textbf{Attore principale} & Utente autenticato\\
				\noalign{\hrule height 0.5pt}
				\textbf{Attori secondari} & Nessuno \\				
				\noalign{\hrule height 0.5pt}
				\textbf{Pre-condizione} &  L'utente intende modificare i permessi di un determinato utente e ha inserito il comando di modifica permessi\\
				\noalign{\hrule height 0.5pt}
				\textbf{Post-condizione} & L'utente ha inserito con successo il nome dell'utente a cui verranno modificati i permessi\\
				\noalign{\hrule height 0.5pt}
				\textbf{Flusso eventi} & \begin{enumerate}
				\item L'utente scrive su terminale il nome dell'utente a cui verranno modificati i permessi 
				\end{enumerate} \\
				\noalign{\hrule height 0.5pt}
				\textbf{Scenari alternativi} & Nessuno \\
				\noalign{\hrule height 0.5pt}
				\textbf{Lista requisiti\newline dedotti} & \begin{itemize}
				\item ...																
				\end{itemize} 
			\end{tabularx}
			\caption{Caso d'uso UC 4.7.2 - Inserimento nome utente da modificare}
		 \end{table}
		 
		 \subsection{UC 4.7.3: Inserimento nuovi permessi}
	\textbf{Descrizione} 
	\\ \\
	L'utente ha inserito il comando per modificare i permessi e il nome dell'utente a cui verranno modificati i permessi, ora deve inserire i valori dei nuovi permessi da assegnare all'utente.
	\begin{table}[H]
			\begin{tabularx}{\textwidth}{r X}
				\textbf{Codice gerarchico} & UC4.7.3 \\
				\noalign{\hrule height 0.5pt}
				\textbf{Nome sintetico} & Inserimento nuovi permessi\\
				\noalign{\hrule height 0.5pt}
				\textbf{Attore principale} & Utente autenticato\\
				\noalign{\hrule height 0.5pt}
				\textbf{Attori secondari} & Nessuno \\				
				\noalign{\hrule height 0.5pt}
				\textbf{Pre-condizione} &  L'utente intende modificare i permessi di un determinato utente, ha inserito il comando di modifica permessi e il nome utente\\
				\noalign{\hrule height 0.5pt}
				\textbf{Post-condizione} & L'utente ha inserito con successo i nuovi permessi\\
				\noalign{\hrule height 0.5pt}
				\textbf{Flusso eventi} & \begin{enumerate}
				\item L'utente scrive su terminale il nuovi permessi da applicare all'utente precedentemente selezionato
				\end{enumerate} \\
				\noalign{\hrule height 0.5pt}
				\textbf{Scenari alternativi} & Nessuno \\
				\noalign{\hrule height 0.5pt}
				\textbf{Lista requisiti\newline dedotti} & \begin{itemize}
				\item ...																
				\end{itemize} 
			\end{tabularx}
			\caption{Caso d'uso UC 4.7.3 - Inserimento nuovi permessi}
		 \end{table}
		 
		 
		 \subsection{UC 4.8: Errore creazione mappa fallita}
	\textbf{Descrizione} 
	\\ \\
	L'utente ha richiesto l'operazione di creazione di una nuova mappa ma questa è fallita, l'utente riceve un messaggio di errore informativo e la possibilità di inserire un nuovo comando.
	\begin{table}[H]
			\begin{tabularx}{\textwidth}{r X}
				\textbf{Codice gerarchico} & UC4.8 \\
				\noalign{\hrule height 0.5pt}
				\textbf{Nome sintetico} & Errore creazione mappa fallita\\
				\noalign{\hrule height 0.5pt}
				\textbf{Attore principale} & Utente autenticato\\
				\noalign{\hrule height 0.5pt}
				\textbf{Attori secondari} & Nessuno \\
				\noalign{\hrule height 0.5pt}
				\textbf{Pre-condizione} & L'utente ha richiesto un'operazione di creazione mappa sul database selezionato\\
				\noalign{\hrule height 0.5pt}
				\textbf{Post-condizione} & L'operazione richiesta non è stata eseguita con successo, nessuna mappa è stato creata, l'utente ha ricevuto un messaggio di errore e ora può richiedere una nuova operazione\\
				\noalign{\hrule height 0.5pt}
				\textbf{Flusso eventi} & \begin{enumerate}
				\item L'utente riceve un messaggio di errore informativo sul terminale
				\item L'utente riceve la possibilità di inserire un nuovo comando
				\end{enumerate} \\
				\noalign{\hrule height 0.5pt}
				\textbf{Scenari alternativi} & Nessuno \\
				\noalign{\hrule height 0.5pt}
				\textbf{Lista requisiti\newline dedotti} & \begin{itemize}
				\item ...
				\end{itemize} 
			\end{tabularx}
			\caption{Caso d'uso UC 4.8 - Errore creazione mappa fallita}
		 \end{table}	
		 
		  \subsection{UC 4.9: Errore permessi di scrittura insufficienti}
	\textbf{Descrizione} 
	\\ \\
	L'utente ha richiesto un'operazione che modifica il contenuto o la struttura del database selezionato. L'utente però non dispone dei permessi di scrittura necessari all'operazione, questa non viene effettuata e l'utente riceve un messaggio di errore informativo seguito dalla possibilità di inserire un nuovo comando.
	\begin{table}[H]
			\begin{tabularx}{\textwidth}{r X}
				\textbf{Codice gerarchico} & UC4.9 \\
				\noalign{\hrule height 0.5pt}
				\textbf{Nome sintetico} & Errore permessi di scrittura insufficienti\\
				\noalign{\hrule height 0.5pt}
				\textbf{Attore principale} & Utente autenticato\\
				\noalign{\hrule height 0.5pt}
				\textbf{Attori secondari} & Nessuno \\
				\noalign{\hrule height 0.5pt}
				\textbf{Pre-condizione} & L'utente ha richiesto un'operazione di modifica\\
				\noalign{\hrule height 0.5pt}
				\textbf{Post-condizione} & L'operazione richiesta non è stata eseguita con successo, nessuna modifica al database è stata effettuata, l'utente ha ricevuto un messaggio di errore e ora può richiedere una nuova operazione\\
				\noalign{\hrule height 0.5pt}
				\textbf{Flusso eventi} & \begin{enumerate}
				\item L'utente riceve un messaggio di errore informativo sul terminale
				\item L'utente riceve la possibilità di inserire un nuovo comando
				\end{enumerate} \\
				\noalign{\hrule height 0.5pt}
				\textbf{Scenari alternativi} & Nessuno \\
				\noalign{\hrule height 0.5pt}
				\textbf{Lista requisiti\newline dedotti} & \begin{itemize}
				\item ...
				\end{itemize} 
			\end{tabularx}
			\caption{Caso d'uso UC 4.9 - Errore permessi di scrittura insufficienti}
		 \end{table}	
		 
		 \subsection{UC 4.10: Errore mappa inesistente}
	\textbf{Descrizione} 
	\\ \\
	L'utente ha richiesto un accesso ad una mappa non presente nel database, l'accesso è fallito e l'utente riceve un messaggio di errore informativo e può inserire un nuovo comando.
	\begin{table}[H]
			\begin{tabularx}{\textwidth}{r X}
				\textbf{Codice gerarchico} & UC4.10 \\
				\noalign{\hrule height 0.5pt}
				\textbf{Nome sintetico} & Errore mappa inesistente\\
				\noalign{\hrule height 0.5pt}
				\textbf{Attore principale} & Utente autenticato\\
				\noalign{\hrule height 0.5pt}
				\textbf{Attori secondari} & Nessuno \\
				\noalign{\hrule height 0.5pt}
				\textbf{Pre-condizione} & L'utente ha richiesto un'operazione di accesso a una mappa\\
				\noalign{\hrule height 0.5pt}
				\textbf{Post-condizione} & L'operazione richiesta non è stata eseguita con successo, l'utente ha ricevuto un messaggio di errore e ora può richiedere una nuova operazione\\
				\noalign{\hrule height 0.5pt}
				\textbf{Flusso eventi} & \begin{enumerate}
				\item L'utente riceve un messaggio di errore informativo sul terminale
				\item L'utente riceve la possibilità di inserire un nuovo comando
				\end{enumerate} \\
				\noalign{\hrule height 0.5pt}
				\textbf{Scenari alternativi} & Nessuno \\
				\noalign{\hrule height 0.5pt}
				\textbf{Lista requisiti\newline dedotti} & \begin{itemize}
				\item ...
				\end{itemize} 
			\end{tabularx}
			\caption{Caso d'uso UC 4.10 - Errore mappa inesistente}
		 \end{table}							
		 
		 \subsection{UC 4.11: Errore permessi amministratore insufficienti}
	\textbf{Descrizione} 
	\\ \\
	L'utente ha richiesto un'operazione che accede ai permessi del database senza disporre dei permessi necessari di amministrazione, l'operazione è fallita dunque l'utente riceve un messaggio di errore informativo e la possibilità di richiedere una nuova operazione.
	\begin{table}[H]
			\begin{tabularx}{\textwidth}{r X}
				\textbf{Codice gerarchico} & UC4.11 \\
				\noalign{\hrule height 0.5pt}
				\textbf{Nome sintetico} & Errore permessi amministratore insufficienti\\
				\noalign{\hrule height 0.5pt}
				\textbf{Attore principale} & Utente autenticato\\
				\noalign{\hrule height 0.5pt}
				\textbf{Attori secondari} & Nessuno \\
				\noalign{\hrule height 0.5pt}
				\textbf{Pre-condizione} & L'utente ha richiesto un'operazione che accede ai permessi senza disporre dei permessi amministrativi necessari\\
				\noalign{\hrule height 0.5pt}
				\textbf{Post-condizione} & L'operazione richiesta non è stata eseguita con successo, l'utente ha ricevuto un messaggio di errore e ora può richiedere una nuova operazione\\
				\noalign{\hrule height 0.5pt}
				\textbf{Flusso eventi} & \begin{enumerate}
				\item L'utente riceve un messaggio di errore informativo sul terminale
				\item L'utente riceve la possibilità di inserire un nuovo comando
				\end{enumerate} \\
				\noalign{\hrule height 0.5pt}
				\textbf{Scenari alternativi} & Nessuno \\
				\noalign{\hrule height 0.5pt}
				\textbf{Lista requisiti\newline dedotti} & \begin{itemize}
				\item ...
				\end{itemize} 
			\end{tabularx}
			\caption{Caso d'uso UC 4.11 - Errore permessi amministratore insufficienti}
		 \end{table}		
		 
		 
		 \subsection{UC 4.12: Errore modifica permessi fallita}
	\textbf{Descrizione} 
	\\ \\
	L'utente ha richiesto di modificare i permessi di un determinato utente, l'operazione è fallita perché ha inserito valori di permessi non validi dunque l'utente riceve un messaggio di errore informativo e la possibilità di richiedere una nuova operazione.
	\begin{table}[H]
			\begin{tabularx}{\textwidth}{r X}
				\textbf{Codice gerarchico} & UC4.12 \\
				\noalign{\hrule height 0.5pt}
				\textbf{Nome sintetico} & Errore modifica permessi fallita\\
				\noalign{\hrule height 0.5pt}
				\textbf{Attore principale} & Utente autenticato\\
				\noalign{\hrule height 0.5pt}
				\textbf{Attori secondari} & Nessuno \\
				\noalign{\hrule height 0.5pt}
				\textbf{Pre-condizione} & L'utente ha richiesto un'operazione di modifica permessi\\
				\noalign{\hrule height 0.5pt}
				\textbf{Post-condizione} & L'operazione richiesta non è stata eseguita con successo, nessun permesso di un determinato utente è stato cambiato, l'utente ha ricevuto un messaggio di errore e ora può richiedere una nuova operazione\\
				\noalign{\hrule height 0.5pt}
				\textbf{Flusso eventi} & \begin{enumerate}
				\item L'utente riceve un messaggio di errore informativo sul terminale
				\item L'utente riceve la possibilità di inserire un nuovo comando
				\end{enumerate} \\
				\noalign{\hrule height 0.5pt}
				\textbf{Scenari alternativi} & Nessuno \\
				\noalign{\hrule height 0.5pt}
				\textbf{Lista requisiti\newline dedotti} & \begin{itemize}
				\item ...
				\end{itemize} 
			\end{tabularx}
			\caption{Caso d'uso UC 4.12 - Errore modifica permessi fallita}
		 \end{table}	
		 
		 \subsection{UC 5: Operazioni a livello mappa}
	 \begin{figure}[H]
				\centering
				\includegraphics[scale=0.2]{UC/"UC 5 Operazioni a livello mappa".png}
				\caption{Diagramma di UC5: Operazioni a livello mappa}
			\end{figure}
	\textbf{Descrizione} 
	\\ \\
	L'utente ha effettuato correttamente la connessione e ha selezionato un database di cui dispone dei permessi di accesso, ha successivamente selezionato una delle mappe che compongono database e su tale mappa intende effettuare una delle seguenti operazioni:
	\begin{itemize}
	\item Visualizzare la lista delle chiavi della mappa
	\item Ricercare un item per chiave
	\item Inserire un nuovo item
	\item Aggiornare un item già presente sostituendo il campo value o aggiungendo dati in coda
	\item Rimuovere un item
	\end{itemize}
	Queste operazioni possono portare a diverse situazioni di errore. In tal caso non viene eseguita alcuna operazione e l'utente riceve un messaggio di errore esplicativo. \\
	Inoltre le operazioni che modificano il contenuto della mappa, e quindi del database di cui essa fa parte, possono provocare un messaggio di errore nel caso in cui l'utente disponesse solo di permessi di lettura sul database attualmente selezionato (vedi UC 4.9).
	\begin{table}[H]
			\begin{tabularx}{\textwidth}{r X}
				\textbf{Codice gerarchico} & UC5 \\
				\noalign{\hrule height 0.5pt}
				\textbf{Nome sintetico} & Operazioni a livello mappa \\
				\noalign{\hrule height 0.5pt}
				\textbf{Attore principale} & Utente autenticato\\
				\noalign{\hrule height 0.5pt}
				\textbf{Attori secondari} & Nessuno \\
				\noalign{\hrule height 0.5pt}
				\textbf{Pre-condizione} & L'utente ha selezionato correttamente un database di cui dispone almeno dei permessi di lettura e una mappa al suo interno\\
				\noalign{\hrule height 0.5pt}
				\textbf{Post-condizione} & L'operazione sulla mappa selezionata è stata eseguita correttamente\\
				\noalign{\hrule height 0.5pt}
				\textbf{Flusso eventi} & \begin{enumerate}
				\item L'utente inserisce il comando e le informazioni necessarie per l'operazione richiesta e preme invio
				\end{enumerate} \\
				\noalign{\hrule height 0.5pt}
				\textbf{Scenari alternativi} & \begin{enumerate}
				\item L'utente ha richiesto l'inserimento di un nuovo item ma questo è fallito (ad esempio perché la chiave inserita è già presente). L'utente riceve un messaggio di errore informativo (UC 5.7)
				\item L'utente ha richiesto un'operazione che accede ad un determinato item tramite la chiave inserita dall'utente. Non è stato trovato nessun item con tale chiave dunque l'operazione è fallita e l'utente riceve un messaggio di errore informativo (UC 5.8)
				\item L'utente ha richiesto un'operazione di modifica di un item ma questa è fallita, l'utente riceve un messaggio di errore informativo (UC 5.9)
				\item L'utente ha richiesto la rimozione di un item ma questa è fallita, l'utente riceve un messaggio di errore informativo (UC 5.10)
\end{enumerate}				 \\
				\noalign{\hrule height 0.5pt}
				\textbf{Lista requisiti\newline dedotti} & \begin{itemize}
				\item ...
				\end{itemize} 
			\end{tabularx}
			\caption{Caso d'uso UC 5 - Operazioni a livello mappa}
		 \end{table} 
		 
		 
		 \subsection{UC 5.1: Visualizzazione chiavi della mappa}
	\textbf{Descrizione} 
	\\ \\
	L'utente intende visualizzare la lista delle chiavi degli item che compongono la mappa selezionata. L'utente inserisce il comando di visualizzazione lista chiavi e preme invio, il sistema stampa su terminale la lista delle chiavi convertite in stringa.
	\begin{table}[H]
			\begin{tabularx}{\textwidth}{r X}
				\textbf{Codice gerarchico} & UC5.1 \\
				\noalign{\hrule height 0.5pt}
				\textbf{Nome sintetico} & Visualizzazione chiavi della mappa \\
				\noalign{\hrule height 0.5pt}
				\textbf{Attore principale} & Utente autenticato\\
				\noalign{\hrule height 0.5pt}
				\textbf{Attori secondari} & Nessuno \\
				\noalign{\hrule height 0.5pt}
				\textbf{Pre-condizione} & L'utente intende visualizzare la lista delle chiavi degli item che compongono la mappa selezionata\\
				\noalign{\hrule height 0.5pt}
				\textbf{Post-condizione} & L'utente ha ricevuto la lista delle chiavi convertite a stringa stampata su terminale\\
				\noalign{\hrule height 0.5pt}
				\textbf{Flusso eventi} & \begin{enumerate}
				\item L'utente inserisce il comando di visualizzazione lista chiavi e preme invio
				\end{enumerate} \\
				\noalign{\hrule height 0.5pt}
				\textbf{Scenari alternativi} & Nessuno \\
				\noalign{\hrule height 0.5pt}
				\textbf{Lista requisiti\newline dedotti} & \begin{itemize}
				\item ...
				\end{itemize} 
			\end{tabularx}
			\caption{Caso d'uso UC 5.1 - Visualizzazione chiavi della mappa}
		 \end{table} 
		 
		 
		 \subsection{UC 5.2: Ricerca per chiave}
	 \begin{figure}[H]
				\centering
				\includegraphics[scale=0.2]{UC/"UC 5-2 Ricerca per chiave".png}
				\caption{Diagramma di UC5.2: Ricerca per chiave}
			\end{figure}
	\textbf{Descrizione} 
	\\ \\
	L'utente intende effettuare una query sulla mappa selezionata. La ricerca avviene inserendo il comando di ricerca seguito dal valore della chiave, tale valore può essere fornito in due modi:
	\begin{itemize}
	\item Con una stringa 
	\item Con un file
	\end{itemize}
	La ricerca restituisce il valore dell'item corrispondente alla chiave inserita, convertito in stringa e stampato su terminale.
	\begin{table}[H]
			\begin{tabularx}{\textwidth}{r X}
				\textbf{Codice gerarchico} & UC5.2 \\
				\noalign{\hrule height 0.5pt}
				\textbf{Nome sintetico} & Ricerca per chiave \\
				\noalign{\hrule height 0.5pt}
				\textbf{Attore principale} & Utente autenticato\\
				\noalign{\hrule height 0.5pt}
				\textbf{Attori secondari} & Nessuno \\
				\noalign{\hrule height 0.5pt}
				\textbf{Pre-condizione} & L'utente intende effettuare una query sulla mappa selezionata\\
				\noalign{\hrule height 0.5pt}
				\textbf{Post-condizione} & L'utente ha ricevuto il valore dell'item corrispondente alla chiave inserita stampato su terminale in formato stringa\\
				\noalign{\hrule height 0.5pt}
				\textbf{Flusso eventi} & \begin{enumerate}
				\item L'utente inserisce il comando di ricerca (UC 5.2.1)
				\item L'utente inserisce il valore della chiave come stringa (UC 5.2.2) o come file (UC 5.2.3) e preme invio
				\item L'utente riceve il valore dell'item stampato su terminale come stringa
				\end{enumerate} \\
				\noalign{\hrule height 0.5pt}
				\textbf{Scenari alternativi} & \begin{enumerate}
				\item L'utente ha scelto di usare un file come chiave di ricerca, l'accesso a tale file è fallito. L'operazione va ripetuta e l'utente riceve un messaggio di errore (UC 5.2.4)
				\end{enumerate} \\
				\noalign{\hrule height 0.5pt}
				\textbf{Lista requisiti\newline dedotti} & \begin{itemize}
				\item ...
				\end{itemize} 
			\end{tabularx}
			\caption{Caso d'uso UC 5.2 - Ricerca per chiave}
		 \end{table} 
		 
		 
		  \subsection{UC 5.2.1: Inserimento comando di ricerca}
	\textbf{Descrizione} 
	\\ \\
	L'utente intende effettuare una query sulla mappa selezionata. Per prima cosa l'utente deve scrivere su terminale l'apposito comando di ricerca.
	\begin{table}[H]
			\begin{tabularx}{\textwidth}{r X}
				\textbf{Codice gerarchico} & UC5.2.1 \\
				\noalign{\hrule height 0.5pt}
				\textbf{Nome sintetico} & Inserimento comando di ricerca \\
				\noalign{\hrule height 0.5pt}
				\textbf{Attore principale} & Utente autenticato\\
				\noalign{\hrule height 0.5pt}
				\textbf{Attori secondari} & Nessuno \\
				\noalign{\hrule height 0.5pt}
				\textbf{Pre-condizione} & L'utente intende effettuare una query sulla mappa selezionata\\
				\noalign{\hrule height 0.5pt}
				\textbf{Post-condizione} & L'utente ha inserito correttamente il comando di ricerca\\
				\noalign{\hrule height 0.5pt}
				\textbf{Flusso eventi} & \begin{enumerate}
				\item L'utente scrive su terminale il comando di ricerca item
				\end{enumerate} \\
				\noalign{\hrule height 0.5pt}
				\textbf{Scenari alternativi} & Nessuno\\
				\noalign{\hrule height 0.5pt}
				\textbf{Lista requisiti\newline dedotti} & \begin{itemize}
				\item ...
				\end{itemize} 
			\end{tabularx}
			\caption{Caso d'uso UC 5.2.1 - Inserimento comando di ricerca}
		 \end{table} 
		 
		 \subsection{UC 5.2.2: Utilizzo di un file come chiave}
	\textbf{Descrizione} 
	\\ \\
	L'utente intende effettuare una query sulla mappa selezionata. L'utente intende utilizzare come chiave di ricerca un file presente sul computer, per far questo deve inserire il percorso del file rispettando la sintassi del tipo \texttt{file(percorso)}.
	\begin{table}[H]
			\begin{tabularx}{\textwidth}{r X}
				\textbf{Codice gerarchico} & UC5.2.2 \\
				\noalign{\hrule height 0.5pt}
				\textbf{Nome sintetico} & Utilizzo di un file come chiave \\
				\noalign{\hrule height 0.5pt}
				\textbf{Attore principale} & Utente autenticato\\
				\noalign{\hrule height 0.5pt}
				\textbf{Attori secondari} & Nessuno \\
				\noalign{\hrule height 0.5pt}
				\textbf{Pre-condizione} & L'utente intende effettuare una query sulla mappa selezionata ed utilizzare come chiave di query un file\\
				\noalign{\hrule height 0.5pt}
				\textbf{Post-condizione} & L'utente ha inserito correttamente il comando per utilizzare un file come chiave, il contenuto del file viene usato come chiave\\
				\noalign{\hrule height 0.5pt}
				\textbf{Flusso eventi} & \begin{enumerate}
				\item L'utente scrive su terminale il comando per utilizzare un file, comprensivo del percorso del file
				\end{enumerate} \\
				\noalign{\hrule height 0.5pt}
				\textbf{Scenari alternativi} & Nessuno\\
				\noalign{\hrule height 0.5pt}
				\textbf{Lista requisiti\newline dedotti} & \begin{itemize}
				\item ...
				\end{itemize} 
			\end{tabularx}
			\caption{Caso d'uso UC 5.2.2 - Utilizzo di un file come chiave}
		 \end{table} 
		 
		 
		 \subsection{UC 5.2.3: Utilizzo di una stringa come chiave}
	\textbf{Descrizione} 
	\\ \\
	L'utente intende effettuare una query sulla mappa selezionata. L'utente intende utilizzare come chiave di ricerca una stringa, deve quindi inserire la stringa tra apici doppi.
	\begin{table}[H]
			\begin{tabularx}{\textwidth}{r X}
				\textbf{Codice gerarchico} & UC5.2.3 \\
				\noalign{\hrule height 0.5pt}
				\textbf{Nome sintetico} & Utilizzo di una stringa come chiave \\
				\noalign{\hrule height 0.5pt}
				\textbf{Attore principale} & Utente autenticato\\
				\noalign{\hrule height 0.5pt}
				\textbf{Attori secondari} & Nessuno \\
				\noalign{\hrule height 0.5pt}
				\textbf{Pre-condizione} & L'utente intende effettuare una query sulla mappa selezionata ed utilizzare come chiave di query una stringa\\
				\noalign{\hrule height 0.5pt}
				\textbf{Post-condizione} & L'utente ha inserito correttamente la stringa da usare come chiave\\
				\noalign{\hrule height 0.5pt}
				\textbf{Flusso eventi} & \begin{enumerate}
				\item L'utente scrive su terminale la stringa tra apici doppi
				\end{enumerate} \\
				\noalign{\hrule height 0.5pt}
				\textbf{Scenari alternativi} & Nessuno\\
				\noalign{\hrule height 0.5pt}
				\textbf{Lista requisiti\newline dedotti} & \begin{itemize}
				\item ...
				\end{itemize} 
			\end{tabularx}
			\caption{Caso d'uso UC 5.2.3 - Utilizzo di una stringa come chiave}
		 \end{table} 
		 
		 
		 \subsection{UC 5.2.4: Errore file non trovato/Illeggibile}
	\textbf{Descrizione} 
	\\ \\
	L'utente ha tentato di utilizzare un file come valore da utilizzare per l'operazione, l'accesso a tale file è fallito. Il file non è stato utilizzato e l'utente riceve un messaggio di errore informativo.
	\begin{table}[H]
			\begin{tabularx}{\textwidth}{r X}
				\textbf{Codice gerarchico} & UC5.2.4 \\
				\noalign{\hrule height 0.5pt}
				\textbf{Nome sintetico} & Errore file non trovato/Illeggibile \\
				\noalign{\hrule height 0.5pt}
				\textbf{Attore principale} & Utente autenticato\\
				\noalign{\hrule height 0.5pt}
				\textbf{Attori secondari} & Nessuno \\
				\noalign{\hrule height 0.5pt}
				\textbf{Pre-condizione} & L'utente ha tentato di utilizzare un file come valore da utilizzare per l'operazione, l'accesso a tale file è fallito\\
				\noalign{\hrule height 0.5pt}
				\textbf{Post-condizione} & L'utente ha ricevuto un messaggio di errore informativo, il file non è stato utilizzato e l'utente può inserire un nuovo comando\\
				\noalign{\hrule height 0.5pt}
				\textbf{Flusso eventi} & \begin{enumerate}
				\item L'utente riceve un messaggio di errore informativo sul terminale
				\item L'utente riceve la possibilità di inserire un nuovo comando
				\end{enumerate} \\
				\noalign{\hrule height 0.5pt}
				\textbf{Scenari alternativi} & Nessuno\\
				\noalign{\hrule height 0.5pt}
				\textbf{Lista requisiti\newline dedotti} & \begin{itemize}
				\item ...
				\end{itemize} 
			\end{tabularx}
			\caption{Caso d'uso UC 5.2.4 - Errore file non trovato/Illeggibile}
		 \end{table} 
		 
		 \subsection{UC 5.3: Inserimento di un item}
	 \begin{figure}[H]
				\centering
				\includegraphics[scale=0.2]{UC/"UC 5-3 Inserimento item".png}
				\caption{Diagramma di UC5.3: Inserimento di un item}
			\end{figure}
	\textbf{Descrizione} 
	\\ \\
	L'utente intende inserire un item (coppia chiave-valore) nella mappa selezionata. Deve inserire il comando di inserimento, il valore della chiave e il valore del valore e premere invio. L'item viene creato dal sistema ed è ora parte della mappa.
	\begin{table}[H]
			\begin{tabularx}{\textwidth}{r X}
				\textbf{Codice gerarchico} & UC5.3 \\
				\noalign{\hrule height 0.5pt}
				\textbf{Nome sintetico} & Inserimento di un item \\
				\noalign{\hrule height 0.5pt}
				\textbf{Attore principale} & Utente autenticato\\
				\noalign{\hrule height 0.5pt}
				\textbf{Attori secondari} & Nessuno \\
				\noalign{\hrule height 0.5pt}
				\textbf{Pre-condizione} & L'utente intende inserire un item nella mappa selezionata\\
				\noalign{\hrule height 0.5pt}
				\textbf{Post-condizione} & L'utente ha inserito con successo l'item\\
				\noalign{\hrule height 0.5pt}
				\textbf{Flusso eventi} & \begin{enumerate}
				\item L'utente inserisce il comando di inserimento item (UC 5.3.1)
				\item L'utente inserisce il valore della chiave (UC 5.3.2)
				\item L'utente inserisce il valore dell'item (UC 5.3.3)
				\end{enumerate} \\
				\noalign{\hrule height 0.5pt}
				\textbf{Scenari alternativi} & Nessuno \\
				\noalign{\hrule height 0.5pt}
				\textbf{Lista requisiti\newline dedotti} & \begin{itemize}
				\item ...
				\end{itemize} 
			\end{tabularx}
			\caption{Caso d'uso UC 5.3 - Inserimento di un item}
		 \end{table} 
		 
		 \subsection{UC 5.3.1: Inserimento comando di inserimento item}
	\textbf{Descrizione} 
	\\ \\
	L'utente intende inserire un item nella mappa selezionata. Per prima cosa l'utente deve scrivere su terminale l'apposito comando di inserimento item.
	\begin{table}[H]
			\begin{tabularx}{\textwidth}{r X}
				\textbf{Codice gerarchico} & UC5.3.1 \\
				\noalign{\hrule height 0.5pt}
				\textbf{Nome sintetico} & Inserimento comando di inserimento item \\
				\noalign{\hrule height 0.5pt}
				\textbf{Attore principale} & Utente autenticato\\
				\noalign{\hrule height 0.5pt}
				\textbf{Attori secondari} & Nessuno \\
				\noalign{\hrule height 0.5pt}
				\textbf{Pre-condizione} & L'utente intende inserire un item nella mappa selezionata\\
				\noalign{\hrule height 0.5pt}
				\textbf{Post-condizione} & L'utente ha inserito correttamente il comando di inserimento item\\
				\noalign{\hrule height 0.5pt}
				\textbf{Flusso eventi} & \begin{enumerate}
				\item L'utente scrive su terminale il comando di inserimento item
				\end{enumerate} \\
				\noalign{\hrule height 0.5pt}
				\textbf{Scenari alternativi} & Nessuno\\
				\noalign{\hrule height 0.5pt}
				\textbf{Lista requisiti\newline dedotti} & \begin{itemize}
				\item ...
				\end{itemize} 
			\end{tabularx}
			\caption{Caso d'uso UC 5.3.1 - Inserimento comando di inserimento item}
		 \end{table} 
		 
		 \subsection{UC 5.3.2: Inserimento chiave item da inserire}
	\textbf{Descrizione} 
	\\ \\
	L'utente intende inserire un item nella mappa selezionata. Ha scritto su terminale il comando di inserimento item, ora deve inserire una chiave per l'item. La chiave può essere inserita secondo le due modalità descritte in UC5.2.2 e UC5.2.3 ovvero come file o come stringa.
	\begin{table}[H]
			\begin{tabularx}{\textwidth}{r X}
				\textbf{Codice gerarchico} & UC5.3.2 \\
				\noalign{\hrule height 0.5pt}
				\textbf{Nome sintetico} & Inserimento chiave item da inserire\\
				\noalign{\hrule height 0.5pt}
				\textbf{Attore principale} & Utente autenticato\\
				\noalign{\hrule height 0.5pt}
				\textbf{Attori secondari} & Nessuno \\
				\noalign{\hrule height 0.5pt}
				\textbf{Pre-condizione} & L'utente intende inserire un item nella mappa selezionata e ha scritto su terminale il comando di inserimento item\\
				\noalign{\hrule height 0.5pt}
				\textbf{Post-condizione} & L'utente ha inserito correttamente la chiave per l'item da inserire\\
				\noalign{\hrule height 0.5pt}
				\textbf{Flusso eventi} & \begin{enumerate}
				\item L'utente inserisce la chiave come file o come stringa
				\end{enumerate} \\
				\noalign{\hrule height 0.5pt}
				\textbf{Scenari alternativi} & Nessuno\\
				\noalign{\hrule height 0.5pt}
				\textbf{Lista requisiti\newline dedotti} & \begin{itemize}
				\item ...
				\end{itemize} 
			\end{tabularx}
			\caption{Caso d'uso UC 5.3.2 - Inserimento chiave item da inserire}
		 \end{table} 
		 
		 \subsection{UC 5.3.3: Inserimento valore item da inserire}
	\textbf{Descrizione} 
	\\ \\
	L'utente intende inserire un item nella mappa selezionata. Ha scritto su terminale il comando di inserimento item seguito dalla chiave, ora deve inserire un valore per l'item. Il valore può essere inserito in due modi:
	\begin{itemize}
	\item Come file
	\item Come stringa
	\end{itemize}
	La sintassi relativa all'inserimento per file o stringa è la stessa descritta nei casi d'uso UC5.2.2 e UC5.2.3 che definiscono l'inserimento di un valore per la chiave di un item.
	\begin{table}[H]
			\begin{tabularx}{\textwidth}{r X}
				\textbf{Codice gerarchico} & UC5.3.3 \\
				\noalign{\hrule height 0.5pt}
				\textbf{Nome sintetico} & Inserimento valore item da inserire\\
				\noalign{\hrule height 0.5pt}
				\textbf{Attore principale} & Utente autenticato\\
				\noalign{\hrule height 0.5pt}
				\textbf{Attori secondari} & Nessuno \\
				\noalign{\hrule height 0.5pt}
				\textbf{Pre-condizione} & L'utente intende inserire un item nella mappa selezionata e ha scritto su terminale il comando di inserimento item seguito dalla chiave dell'item\\
				\noalign{\hrule height 0.5pt}
				\textbf{Post-condizione} & L'utente ha inserito correttamente il valore per l'item da inserire\\
				\noalign{\hrule height 0.5pt}
				\textbf{Flusso eventi} & \begin{enumerate}
				\item L'utente inserisce il valore come file o come stringa
				\end{enumerate} \\
				\noalign{\hrule height 0.5pt}
				\textbf{Scenari alternativi} & Nessuno\\
				\noalign{\hrule height 0.5pt}
				\textbf{Lista requisiti\newline dedotti} & \begin{itemize}
				\item ...
				\end{itemize} 
			\end{tabularx}
			\caption{Caso d'uso UC 5.3.3 - Inserimento valore item da inserire}
		 \end{table} 
		 
		 
		 \subsection{UC 5.4: Aggiornamento di un item}
	 \begin{figure}[H]
				\centering
				\includegraphics[scale=0.2]{UC/"UC 5-4 Aggiornamento item".png}
				\caption{Diagramma di UC5.4: Aggiornamento di un item}
			\end{figure}
	\textbf{Descrizione} 
	\\ \\
	L'utente intende aggiornare il valore di un item esistente. Per effettuare questa operazione deve selezionare l'item attraverso la sua chiave e inserire il nuovo valore.
	\begin{table}[H]
			\begin{tabularx}{\textwidth}{r X}
				\textbf{Codice gerarchico} & UC5.4 \\
				\noalign{\hrule height 0.5pt}
				\textbf{Nome sintetico} & Aggiornamento di un item \\
				\noalign{\hrule height 0.5pt}
				\textbf{Attore principale} & Utente autenticato\\
				\noalign{\hrule height 0.5pt}
				\textbf{Attori secondari} & Nessuno \\
				\noalign{\hrule height 0.5pt}
				\textbf{Pre-condizione} & L'utente intende aggiornare un item della mappa selezionata\\
				\noalign{\hrule height 0.5pt}
				\textbf{Post-condizione} & L'utente ha aggiornato con successo il valore dell'item\\
				\noalign{\hrule height 0.5pt}
				\textbf{Flusso eventi} & \begin{enumerate}
				\item L'utente inserisce il comando di aggiornamento item (UC 5.4.1)
				\item L'utente inserisce il valore della chiave dell'item da aggiornare (UC 5.4.2)
				\item L'utente inserisce il nuovo valore dell'item (UC 5.4.3)
				\end{enumerate} \\
				\noalign{\hrule height 0.5pt}
				\textbf{Scenari alternativi} & Nessuno \\
				\noalign{\hrule height 0.5pt}
				\textbf{Lista requisiti\newline dedotti} & \begin{itemize}
				\item ...
				\end{itemize} 
			\end{tabularx}
			\caption{Caso d'uso UC 5.4 - Aggiornamento di un item}
		 \end{table} 
		 
		 \subsection{UC 5.4.1: Inserimento comando di aggiornamento item}
	\textbf{Descrizione} 
	\\ \\
	L'utente intende aggiornare il valore di un item esistente. Per prima cosa l'utente deve scrivere su terminale l'apposito comando di aggiornamento item.
	\begin{table}[H]
			\begin{tabularx}{\textwidth}{r X}
				\textbf{Codice gerarchico} & UC5.4.1 \\
				\noalign{\hrule height 0.5pt}
				\textbf{Nome sintetico} & Inserimento comando di aggiornamento item \\
				\noalign{\hrule height 0.5pt}
				\textbf{Attore principale} & Utente autenticato\\
				\noalign{\hrule height 0.5pt}
				\textbf{Attori secondari} & Nessuno \\
				\noalign{\hrule height 0.5pt}
				\textbf{Pre-condizione} & L'utente intende aggiornare il valore di un item esistente\\
				\noalign{\hrule height 0.5pt}
				\textbf{Post-condizione} & L'utente ha inserito correttamente il comando di aggiornamento item\\
				\noalign{\hrule height 0.5pt}
				\textbf{Flusso eventi} & \begin{enumerate}
				\item L'utente scrive su terminale il comando di aggiornamento item
				\end{enumerate} \\
				\noalign{\hrule height 0.5pt}
				\textbf{Scenari alternativi} & Nessuno\\
				\noalign{\hrule height 0.5pt}
				\textbf{Lista requisiti\newline dedotti} & \begin{itemize}
				\item ...
				\end{itemize} 
			\end{tabularx}
			\caption{Caso d'uso UC 5.4.1 - Inserimento comando di aggiornamento item}
		 \end{table} 
		 
		 \subsection{UC 5.4.2: Inserimento chiave item da aggiornare}
	\textbf{Descrizione} 
	\\ \\
	L'utente intende aggiornare un item nella mappa selezionata. Ha scritto su terminale il comando di aggiornamento item, ora deve inserire una chiave per ottenere l'item. La chiave può essere inserita secondo le due modalità descritte in UC5.2.2 e UC5.2.3 ovvero come file o come stringa.
	\begin{table}[H]
			\begin{tabularx}{\textwidth}{r X}
				\textbf{Codice gerarchico} & UC5.4.2 \\
				\noalign{\hrule height 0.5pt}
				\textbf{Nome sintetico} & Inserimento chiave item da aggiornare \\
				\noalign{\hrule height 0.5pt}
				\textbf{Attore principale} & Utente autenticato\\
				\noalign{\hrule height 0.5pt}
				\textbf{Attori secondari} & Nessuno \\
				\noalign{\hrule height 0.5pt}
				\textbf{Pre-condizione} & L'utente intende aggiornare il valore di un item nella mappa selezionata e ha scritto su terminale il comando di aggiornamento item\\
				\noalign{\hrule height 0.5pt}
				\textbf{Post-condizione} & L'utente ha inserito correttamente la chiave per l'item che vuole aggiornare\\
				\noalign{\hrule height 0.5pt}
				\textbf{Flusso eventi} & \begin{enumerate}
				\item L'utente inserisce la chiave come file o come stringa
				\end{enumerate} \\
				\noalign{\hrule height 0.5pt}
				\textbf{Scenari alternativi} & Nessuno\\
				\noalign{\hrule height 0.5pt}
				\textbf{Lista requisiti\newline dedotti} & \begin{itemize}
				\item ...
				\end{itemize} 
			\end{tabularx}
			\caption{Caso d'uso UC 5.4.2 - Inserimento chiave item da aggiornare}
		 \end{table} 
		 
		 \subsection{UC 5.4.3: Inserimento valore item da aggiornare}
	\textbf{Descrizione} 
	\\ \\
	L'utente intende aggiornare il valore di un item della mappa selezionata. Ha scritto su terminale il comando di aggiornamento item seguito dalla chiave, ora deve inserire un nuovo valore per l'item. Il valore può essere inserito in due modi:
	\begin{itemize}
	\item Come file
	\item Come stringa
	\end{itemize}
	La sintassi relativa all'inserimento per file o stringa è la stessa descritta nei casi d'uso UC5.2.2 e UC5.2.3 che definiscono l'inserimento di un valore per la chiave di un item.
	\begin{table}[H]
			\begin{tabularx}{\textwidth}{r X}
				\textbf{Codice gerarchico} & UC5.4.3 \\
				\noalign{\hrule height 0.5pt}
				\textbf{Nome sintetico} & Inserimento valore item da aggiornare \\
				\noalign{\hrule height 0.5pt}
				\textbf{Attore principale} & Utente autenticato\\
				\noalign{\hrule height 0.5pt}
				\textbf{Attori secondari} & Nessuno \\
				\noalign{\hrule height 0.5pt}
				\textbf{Pre-condizione} & L'utente intende aggiornare il valore di un item della mappa selezionata e ha scritto su terminale il comando di aggiornamento item seguito dalla chiave dell'item\\
				\noalign{\hrule height 0.5pt}
				\textbf{Post-condizione} & L'utente ha inserito correttamente il nuovo valore per l'item da aggiornare \\
				\noalign{\hrule height 0.5pt}
				\textbf{Flusso eventi} & \begin{enumerate}
				\item L'utente inserisce il nuovo valore come file o come stringa
				\end{enumerate} \\
				\noalign{\hrule height 0.5pt}
				\textbf{Scenari alternativi} & Nessuno\\
				\noalign{\hrule height 0.5pt}
				\textbf{Lista requisiti\newline dedotti} & \begin{itemize}
				\item ...
				\end{itemize} 
			\end{tabularx}
			\caption{Caso d'uso UC 5.4.3 - Inserimento valore item da aggiornare}
		 \end{table} 
		 
		 
		 \subsection{UC 5.5: Aggiunta di un valore in coda a un item}
	 \begin{figure}[H]
				\centering
				\includegraphics[scale=0.2]{UC/"UC 5-5 Append item".png}
				\caption{Diagramma di UC5.5: Aggiunta di un valore in coda a un item}
			\end{figure}
	\textbf{Descrizione} 
	\\ \\
	L'utente intende aggiungere dei dati in coda al valore di un item. Deve accedere all'item tramite la sua chiave e poi inserire i dati che vuole aggiungere.
	\begin{table}[H]
			\begin{tabularx}{\textwidth}{r X}
				\textbf{Codice gerarchico} & UC5.5 \\
				\noalign{\hrule height 0.5pt}
				\textbf{Nome sintetico} & Aggiunta di un valore in coda a un item \\
				\noalign{\hrule height 0.5pt}
				\textbf{Attore principale} & Utente autenticato\\
				\noalign{\hrule height 0.5pt}
				\textbf{Attori secondari} & Nessuno \\
				\noalign{\hrule height 0.5pt}
				\textbf{Pre-condizione} & L'utente intende aggiungere dei dati in coda al valore di un item\\
				\noalign{\hrule height 0.5pt}
				\textbf{Post-condizione} & L'utente ha aggiunto con successo il valore inserito in coda all'item\\
				\noalign{\hrule height 0.5pt}
				\textbf{Flusso eventi} & \begin{enumerate}
				\item L'utente inserisce il comando di append (UC 5.5.1)
				\item L'utente inserisce il valore della chiave dell'item su cui effettuare append (UC 5.5.2)
				\item L'utente inserisce il valore da aggiungere in coda (UC 5.5.3) e preme invio
				\end{enumerate} \\
				\noalign{\hrule height 0.5pt}
				\textbf{Scenari alternativi} & Nessuno \\
				\noalign{\hrule height 0.5pt}
				\textbf{Lista requisiti\newline dedotti} & \begin{itemize}
				\item ...
				\end{itemize} 
			\end{tabularx}
			\caption{Caso d'uso UC 5.5 - Aggiunta di un valore in coda a un item}
		 \end{table} 
		 
		 \subsection{UC 5.5.1: Inserimento comando di append}
	\textbf{Descrizione} 
	\\ \\
	L'utente intende aggiungere dei dati in coda al valore di un item esistente. Per prima cosa l'utente deve scrivere su terminale l'apposito comando di append.
	\begin{table}[H]
			\begin{tabularx}{\textwidth}{r X}
				\textbf{Codice gerarchico} & UC5.5.1 \\
				\noalign{\hrule height 0.5pt}
				\textbf{Nome sintetico} & Inserimento comando di append \\
				\noalign{\hrule height 0.5pt}
				\textbf{Attore principale} & Utente autenticato\\
				\noalign{\hrule height 0.5pt}
				\textbf{Attori secondari} & Nessuno \\
				\noalign{\hrule height 0.5pt}
				\textbf{Pre-condizione} & L'utente intende aggiungere dei dati in coda al valore di un item\\
				\noalign{\hrule height 0.5pt}
				\textbf{Post-condizione} & L'utente ha inserito correttamente il comando di append\\
				\noalign{\hrule height 0.5pt}
				\textbf{Flusso eventi} & \begin{enumerate}
				\item L'utente scrive su terminale il comando di append
				\end{enumerate} \\
				\noalign{\hrule height 0.5pt}
				\textbf{Scenari alternativi} & Nessuno\\
				\noalign{\hrule height 0.5pt}
				\textbf{Lista requisiti\newline dedotti} & \begin{itemize}
				\item ...
				\end{itemize} 
			\end{tabularx}
			\caption{Caso d'uso UC 5.5.1 - Inserimento comando di append}
		 \end{table} 
		 
		 \subsection{UC 5.5.2: Inserimento chiave item su cui effettuare l'append}
	\textbf{Descrizione} 
	\\ \\
	L'utente intende aggiungere dei dati in coda al valore di un item esistente. Ha scritto su terminale il comando di append, ora deve inserire una chiave per ottenere l'item. La chiave può essere inserita secondo le due modalità descritte in UC5.2.2 e UC5.2.3 ovvero come file o come stringa.
	\begin{table}[H]
			\begin{tabularx}{\textwidth}{r X}
				\textbf{Codice gerarchico} & UC5.5.2 \\
				\noalign{\hrule height 0.5pt}
				\textbf{Nome sintetico} & Inserimento chiave item su cui effettuare l'append \\
				\noalign{\hrule height 0.5pt}
				\textbf{Attore principale} & Utente autenticato\\
				\noalign{\hrule height 0.5pt}
				\textbf{Attori secondari} & Nessuno \\
				\noalign{\hrule height 0.5pt}
				\textbf{Pre-condizione} & L'utente intende aggiungere dei dati in coda al valore di un item esistente nella mappa selezionata e ha scritto su terminale il comando di append\\
				\noalign{\hrule height 0.5pt}
				\textbf{Post-condizione} & L'utente ha inserito correttamente la chiave per l'item a cui vuole aggiungere un valore in coda\\
				\noalign{\hrule height 0.5pt}
				\textbf{Flusso eventi} & \begin{enumerate}
				\item L'utente inserisce la chiave come file o come stringa
				\end{enumerate} \\
				\noalign{\hrule height 0.5pt}
				\textbf{Scenari alternativi} & Nessuno\\
				\noalign{\hrule height 0.5pt}
				\textbf{Lista requisiti\newline dedotti} & \begin{itemize}
				\item ...
				\end{itemize} 
			\end{tabularx}
			\caption{Caso d'uso UC 5.5.2 - Inserimento chiave item su cui effettuare l'append}
		 \end{table} 
		 
		 \subsection{UC 5.5.3: Inserimento valore da aggiungere in coda}
	\textbf{Descrizione} 
	\\ \\
	L'utente intende aggiungere dei dati in coda al valore di un item esistente. Ha scritto su terminale il comando di append seguito dalla chiave, ora deve inserire il valore da aggiungere in coda. Il valore può essere inserito in due modi:
	\begin{itemize}
	\item Come file
	\item Come stringa
	\end{itemize}
	La sintassi relativa all'inserimento per file o stringa è la stessa descritta nei casi d'uso UC5.2.2 e UC5.2.3 che definiscono l'inserimento di un valore per la chiave di un item.
	\begin{table}[H]
			\begin{tabularx}{\textwidth}{r X}
				\textbf{Codice gerarchico} & UC5.5.3 \\
				\noalign{\hrule height 0.5pt}
				\textbf{Nome sintetico} & Inserimento valore da aggiungere in coda \\
				\noalign{\hrule height 0.5pt}
				\textbf{Attore principale} & Utente autenticato\\
				\noalign{\hrule height 0.5pt}
				\textbf{Attori secondari} & Nessuno \\
				\noalign{\hrule height 0.5pt}
				\textbf{Pre-condizione} & L'utente intende aggiungere dei dati in coda al valore di un item esistente. Ha scritto su terminale il comando di append seguito dalla chiave\\
				\noalign{\hrule height 0.5pt}
				\textbf{Post-condizione} & L'utente ha inserito correttamente il valore da inserire in coda \\
				\noalign{\hrule height 0.5pt}
				\textbf{Flusso eventi} & \begin{enumerate}
				\item L'utente inserisce il valore da inserire in coda come file o come stringa
				\end{enumerate} \\
				\noalign{\hrule height 0.5pt}
				\textbf{Scenari alternativi} & Nessuno\\
				\noalign{\hrule height 0.5pt}
				\textbf{Lista requisiti\newline dedotti} & \begin{itemize}
				\item ...
				\end{itemize} 
			\end{tabularx}
			\caption{Caso d'uso UC 5.5.3 - Inserimento valore da aggiungere in coda}
		 \end{table} 
		 
		 
		 \subsection{UC 5.6: Rimozione di un item}
	 \begin{figure}[H]
				\centering
				\includegraphics[scale=0.2]{UC/"UC 5-6 Rimozione item".png}
				\caption{Diagramma di UC5.6: Rimozione di un item}
			\end{figure}
	\textbf{Descrizione} 
	\\ \\
	L'utente intende rimuovere un item dalla mappa selezionata. Deve inserire l'apposito comando di rimozione item, seguito dalla chiave dell'item che vuole rimuovere.
	\begin{table}[H]
			\begin{tabularx}{\textwidth}{r X}
				\textbf{Codice gerarchico} & UC5.6 \\
				\noalign{\hrule height 0.5pt}
				\textbf{Nome sintetico} & Rimozione di un item \\
				\noalign{\hrule height 0.5pt}
				\textbf{Attore principale} & Utente autenticato\\
				\noalign{\hrule height 0.5pt}
				\textbf{Attori secondari} & Nessuno \\
				\noalign{\hrule height 0.5pt}
				\textbf{Pre-condizione} & L'utente intende rimuovere un item dalla mappa selezionata\\
				\noalign{\hrule height 0.5pt}
				\textbf{Post-condizione} & L'utente ha rimosso con successo l'item richiesto\\
				\noalign{\hrule height 0.5pt}
				\textbf{Flusso eventi} & \begin{enumerate}
				\item L'utente inserisce il comando di rimozione item (UC 5.6.1)
				\item L'utente inserisce il valore della chiave dell'item da rimuovere (UC 5.6.2)
				\end{enumerate} \\
				\noalign{\hrule height 0.5pt}
				\textbf{Scenari alternativi} & Nessuno \\
				\noalign{\hrule height 0.5pt}
				\textbf{Lista requisiti\newline dedotti} & \begin{itemize}
				\item ...
				\end{itemize} 
			\end{tabularx}
			\caption{Caso d'uso UC 5.6 - Rimozione di un item}
		 \end{table} 
		 
		 \subsection{UC 5.6.1: Inserimento comando di rimozione item}
	\textbf{Descrizione} 
	\\ \\
	L'utente intende rimuovere un item dalla mappa selezionata. Per prima cosa l'utente deve scrivere su terminale l'apposito comando di rimozione item.
	\begin{table}[H]
			\begin{tabularx}{\textwidth}{r X}
				\textbf{Codice gerarchico} & UC5.6.1 \\
				\noalign{\hrule height 0.5pt}
				\textbf{Nome sintetico} & Inserimento comando di rimozione item \\
				\noalign{\hrule height 0.5pt}
				\textbf{Attore principale} & Utente autenticato\\
				\noalign{\hrule height 0.5pt}
				\textbf{Attori secondari} & Nessuno \\
				\noalign{\hrule height 0.5pt}
				\textbf{Pre-condizione} & L'utente intende rimuovere un item dalla mappa selezionata\\
				\noalign{\hrule height 0.5pt}
				\textbf{Post-condizione} & L'utente ha inserito correttamente il comando di rimozione item \\
				\noalign{\hrule height 0.5pt}
				\textbf{Flusso eventi} & \begin{enumerate}
				\item L'utente scrive su terminale il comando di rimozione item
				\end{enumerate} \\
				\noalign{\hrule height 0.5pt}
				\textbf{Scenari alternativi} & Nessuno\\
				\noalign{\hrule height 0.5pt}
				\textbf{Lista requisiti\newline dedotti} & \begin{itemize}
				\item ...
				\end{itemize} 
			\end{tabularx}
			\caption{Caso d'uso UC 5.6.1 - Inserimento comando di rimozione item}
		 \end{table} 
		 
		 \subsection{UC 5.6.2: Inserimento chiave item da rimuovere}
	\textbf{Descrizione} 
	\\ \\
	L'utente intende rimuovere un item dalla mappa selezionata. Ha scritto su terminale il comando di rimozione item, ora deve inserire una chiave per ottenere l'item da eliminare. La chiave può essere inserita secondo le due modalità descritte in UC5.2.2 e UC5.2.3 ovvero come file o come stringa.
	\begin{table}[H]
			\begin{tabularx}{\textwidth}{r X}
				\textbf{Codice gerarchico} & UC5.6.2 \\
				\noalign{\hrule height 0.5pt}
				\textbf{Nome sintetico} & Inserimento chiave item da rimuovere \\
				\noalign{\hrule height 0.5pt}
				\textbf{Attore principale} & Utente autenticato\\
				\noalign{\hrule height 0.5pt}
				\textbf{Attori secondari} & Nessuno \\
				\noalign{\hrule height 0.5pt}
				\textbf{Pre-condizione} & L'utente intende rimuovere un item dalla mappa selezionata e ha scritto su terminale il comando di rimozione item \\
				\noalign{\hrule height 0.5pt}
				\textbf{Post-condizione} & L'utente ha inserito correttamente la chiave per l'item che vuole rimuovere\\
				\noalign{\hrule height 0.5pt}
				\textbf{Flusso eventi} & \begin{enumerate}
				\item L'utente inserisce la chiave come file o come stringa
				\end{enumerate} \\
				\noalign{\hrule height 0.5pt}
				\textbf{Scenari alternativi} & Nessuno\\
				\noalign{\hrule height 0.5pt}
				\textbf{Lista requisiti\newline dedotti} & \begin{itemize}
				\item ...
				\end{itemize} 
			\end{tabularx}
			\caption{Caso d'uso UC 5.6.2 - Inserimento chiave item da rimuovere}
		 \end{table} 
		 
		 
		 \subsection{UC 5.7: Errore inserimento item fallito}
	\textbf{Descrizione} 
	\\ \\
	L'utente ha tentato di inserire un nuovo item nella mappa selezionata. L'inserimento è fallito, l'utente riceve un messaggio informativo e la possibilità di richiedere una nuova operazione.
	\begin{table}[H]
			\begin{tabularx}{\textwidth}{r X}
				\textbf{Codice gerarchico} & UC5.7 \\
				\noalign{\hrule height 0.5pt}
				\textbf{Nome sintetico} & Errore inserimento item fallito\\
				\noalign{\hrule height 0.5pt}
				\textbf{Attore principale} & Utente autenticato\\
				\noalign{\hrule height 0.5pt}
				\textbf{Attori secondari} & Nessuno \\
				\noalign{\hrule height 0.5pt}
				\textbf{Pre-condizione} & L'utente ha cercato di inserire un nuovo item nella mappa selezionata e l'inserimento è fallito\\
				\noalign{\hrule height 0.5pt}
				\textbf{Post-condizione} & L'utente ha ricevuto un messaggio di errore informativo e la possibilità di inserire un nuovo comando\\
				\noalign{\hrule height 0.5pt}
				\textbf{Flusso eventi} & \begin{enumerate}
				\item L'utente riceve un messaggio di errore informativo sul terminale
				\item L'utente riceve la possibilità di inserire un nuovo comando
				\end{enumerate} \\
				\noalign{\hrule height 0.5pt}
				\textbf{Scenari alternativi} & Nessuno \\
				\noalign{\hrule height 0.5pt}
				\textbf{Lista requisiti\newline dedotti} & \begin{itemize}
				\item ...
				\end{itemize} 
			\end{tabularx}
			\caption{Caso d'uso UC 5.7 - Errore inserimento item fallito}
		 \end{table}	

\subsection{UC 5.8: Errore item non presente}
	\textbf{Descrizione} 
	\\ \\
	L'utente ha provato a ricercare, modificare o eliminare un item non presente nella mappa selezionata. L'utente riceve un messaggio informativo e la possibilità di richiedere una nuova operazione.
	\begin{table}[H]
			\begin{tabularx}{\textwidth}{r X}
				\textbf{Codice gerarchico} & UC5.8 \\
				\noalign{\hrule height 0.5pt}
				\textbf{Nome sintetico} & Errore item non presente\\
				\noalign{\hrule height 0.5pt}
				\textbf{Attore principale} & Utente autenticato\\
				\noalign{\hrule height 0.5pt}
				\textbf{Attori secondari} & Nessuno \\
				\noalign{\hrule height 0.5pt}
				\textbf{Pre-condizione} & L'utente ha cercato di ricercare, modificare o eliminare un item non presente nella mappa selezionata. \\
				\noalign{\hrule height 0.5pt}
				\textbf{Post-condizione} & Ricerca, modifica o rimozione dell'item fallita, l'utente ha ricevuto un messaggio di errore e la possibilità di inserire un nuovo comando\\
				\noalign{\hrule height 0.5pt}
				\textbf{Flusso eventi} & \begin{enumerate}
				\item L'utente riceve un messaggio di errore informativo sul terminale
				\item L'utente riceve la possibilità di inserire un nuovo comando
				\end{enumerate} \\
				\noalign{\hrule height 0.5pt}
				\textbf{Scenari alternativi} & Nessuno \\
				\noalign{\hrule height 0.5pt}
				\textbf{Lista requisiti\newline dedotti} & \begin{itemize}
				\item ...
				\end{itemize} 
			\end{tabularx}
			\caption{Caso d'uso UC 5.8 -Errore item non presente}
		 \end{table}	
		 
\subsection{UC 5.9: Errore modifica item fallita}
	\textbf{Descrizione} 
	\\ \\
	L'utente ha tentato di modificare il valore di un item della mappa selezionata. La modifica dell'item è fallita, l'utente riceve un messaggio informativo e la possibilità di richiedere una nuova operazione.
	\begin{table}[H]
			\begin{tabularx}{\textwidth}{r X}
				\textbf{Codice gerarchico} & UC5.9 \\
				\noalign{\hrule height 0.5pt}
				\textbf{Nome sintetico} & Errore modifica item fallita\\
				\noalign{\hrule height 0.5pt}
				\textbf{Attore principale} & Utente autenticato\\
				\noalign{\hrule height 0.5pt}
				\textbf{Attori secondari} & Nessuno \\
				\noalign{\hrule height 0.5pt}
				\textbf{Pre-condizione} & L'utente ha cercato di modificare un item nella mappa selezionata.\\
				\noalign{\hrule height 0.5pt}
				\textbf{Post-condizione} & La modifica dell'item è fallita, l'utente ha ricevuto un messaggio di errore e la possibilità di inserire un nuovo comando\\
				\noalign{\hrule height 0.5pt}
				\textbf{Flusso eventi} & \begin{enumerate}
				\item L'utente riceve un messaggio di errore informativo sul terminale
				\item L'utente riceve la possibilità di inserire un nuovo comando
				\end{enumerate} \\
				\noalign{\hrule height 0.5pt}
				\textbf{Scenari alternativi} & Nessuno \\
				\noalign{\hrule height 0.5pt}
				\textbf{Lista requisiti\newline dedotti} & \begin{itemize}
				\item ...
				\end{itemize} 
			\end{tabularx}
			\caption{Caso d'uso UC 5.9 - Errore modifica item fallita}
		 \end{table}
		 
\subsection{UC 5.10: Errore rimozione item fallita}
	\textbf{Descrizione} 
	\\ \\
	L'utente ha cercato di rimuovere un item dalla mappa selezionata. La rimozione di un item è fallita, l'utente riceve un messaggio informativo e la possibilità di richiedere una nuova operazione.
	\begin{table}[H]
			\begin{tabularx}{\textwidth}{r X}
				\textbf{Codice gerarchico} & UC5.10 \\
				\noalign{\hrule height 0.5pt}
				\textbf{Nome sintetico} & Errore rimozione item fallita\\
				\noalign{\hrule height 0.5pt}
				\textbf{Attore principale} & Utente autenticato\\
				\noalign{\hrule height 0.5pt}
				\textbf{Attori secondari} & Nessuno \\
				\noalign{\hrule height 0.5pt}
				\textbf{Pre-condizione} & L'utente ha cercato di rimuovere un item dalla mappa selezionata\\
				\noalign{\hrule height 0.5pt}
				\textbf{Post-condizione} & La rimozione dell'item è fallita, l'utente ha ricevuto un messaggio di errore e la possibilità di inserire un nuovo comando\\
				\noalign{\hrule height 0.5pt}
				\textbf{Flusso eventi} & \begin{enumerate}
				\item L'utente riceve un messaggio di errore informativo sul terminale
				\item L'utente riceve la possibilità di inserire un nuovo comando
				\end{enumerate} \\
				\noalign{\hrule height 0.5pt}
				\textbf{Scenari alternativi} & Nessuno \\
				\noalign{\hrule height 0.5pt}
				\textbf{Lista requisiti\newline dedotti} & \begin{itemize}
				\item ...
				\end{itemize} 
			\end{tabularx}
			\caption{Caso d'uso UC 5.10 - Errore rimozione item fallita}
		 \end{table}
		 
		 \subsection{UC 6: Disconnessione}
	\textbf{Descrizione} 
	\\ \\
	L'utente è autenticato, intende disconnettersi dal server utilizzando l'apposito comando da terminale.
	\begin{table}[H]
			\begin{tabularx}{\textwidth}{r X}
				\textbf{Codice gerarchico} & UC6 \\
				\noalign{\hrule height 0.5pt}
				\textbf{Nome sintetico} & Disconnessione\\
				\noalign{\hrule height 0.5pt}
				\textbf{Attore principale} & Utente autenticato\\
				\noalign{\hrule height 0.5pt}
				\textbf{Attori secondari} & Nessuno \\
				\noalign{\hrule height 0.5pt}
				\textbf{Pre-condizione} & L'utente intende disconnettersi dal server a cui è attualmente connesso\\
				\noalign{\hrule height 0.5pt}
				\textbf{Post-condizione} & L'utente risulta non connesso ad un server, può ora effettuare una connessione\\
				\noalign{\hrule height 0.5pt}
				\textbf{Flusso eventi} & \begin{enumerate}
				\item L'utente inserisce il comando di disconnessione e preme invio
				\end{enumerate} \\
				\noalign{\hrule height 0.5pt}
				\textbf{Scenari alternativi} & Nessuno \\
				\noalign{\hrule height 0.5pt}
				\textbf{Lista requisiti\newline dedotti} & \begin{itemize}
				\item ...
				\end{itemize} 
			\end{tabularx}
			\caption{Caso d'uso UC 6 - Disconnessione}
		 \end{table}
		 
		 \subsection{UC 7: Errore connessione fallita}
	\textbf{Descrizione} 
	\\ \\
	L'utente ha tentato di connettersi ad un server ma il tentativo di connessione è fallito. L'utente non risulta connesso ad un server, riceve un messaggio di errore informativo stampato su terminale e la possibilità di inserire un nuovo comando.
	\begin{table}[H]
			\begin{tabularx}{\textwidth}{r X}
				\textbf{Codice gerarchico} & UC7 \\
				\noalign{\hrule height 0.5pt}
				\textbf{Nome sintetico} & Errore connessione fallita \\
				\noalign{\hrule height 0.5pt}
				\textbf{Attore principale} & Utente non autenticato\\
				\noalign{\hrule height 0.5pt}
				\textbf{Attori secondari} & Nessuno \\
				\noalign{\hrule height 0.5pt}
				\textbf{Pre-condizione} & L'utente ha tentato di connettersi ad un server ma il tentativo di connessione è fallito\\
				\noalign{\hrule height 0.5pt}
				\textbf{Post-condizione} & L'utente non risulta connesso e ha ricevuto un messaggio di errore\\
				\noalign{\hrule height 0.5pt}
				\textbf{Flusso eventi} & \begin{enumerate}
				\item L'utente riceve un messaggio di errore informativo della mancata connessione
				\end{enumerate} \\
				\noalign{\hrule height 0.5pt}
				\textbf{Scenari alternativi} & Nessuno \\
				\noalign{\hrule height 0.5pt}
				\textbf{Lista requisiti\newline dedotti} & \begin{itemize}
				\item ...
				\end{itemize} 
			\end{tabularx}
			\caption{Caso d'uso UC 7 - Errore connessione fallita}
		 \end{table}
		 
		 
		 \subsection{Visione ad alto livello delle interazioni tramite interfaccia server}
		 	\begin{figure}[H]
				\centering
				\includegraphics[scale=0.2]{UC/"Actorbase server".png}
				\caption{Interazioni tramite interfaccia server}
			\end{figure}
			Il diagramma in figura illustra l'interfaccia CLI lato server. Tale interfaccia va avviata per far partire il server principale di \emph{Actorbase}. Se l'interfaccia server viene avviata su un server non configurato si avvia automaticamente la procedura di configurazione del server. 
			\\ 
			Se il server è già configurato, l'avvio dell'interfaccia server avvia automaticamente il server, e stampa il log delle varie operazioni che vengono effettuate su di esso.
			\\ 
			Per arrestare il server, l'amministratore del server deve inserire l'apposito comando da questa interfaccia.
		 
		 
		 
		 
		 
		 
	 
	\newpage \section{Requisiti}
		Di seguito vengono riportati tutti i requisiti individuati dal gruppo SWEeneyThreads. \\Tali requisiti
		derivano dai casi d'uso, dall'analisi del capitolato, dagli scambi di informazioni avvenuti con il
		\emph{proponente Cardin Riccardo}, oppure da necessità interne. \\
		La struttura di un requisito è definita nel documento \emph{Norme di progetto v1.1.1 sez 2.1.2}.
		I requisiti saranno elencati secondo un ordine. Ogni requisito seguirà la seguente codifica: \\
		\begin{center}
			R[Codice][Importanza][Tipo]
		\end{center}
		\textbf{Codice} \\ \\ Un codice univoco ed espresso in modo gerarchico;\\ \\
		\textbf{Importanza} \\ \\Può assumere i seguenti valori:
		\begin{itemize}
			\item \textbf{N:} necessario;
			\item \textbf{D:} desiderabile;
			\item \textbf{O:} opzionale.
		\end{itemize}
		\textbf{Tipo} \\ \\Può assumere i seguenti valori:
		\begin{itemize}
			\item \textbf{F:} funzionale;
			\item \textbf{Q:} di qualità;
			\item \textbf{P:} prestazionale;
			\item \textbf{V:} vincolo.
		\end{itemize}
	
	\subsection{Requisiti funzionali}
	I requisiti funzionali riguardano le funzioni vere e proprie del prodotto. Il soddisfacimento di uno di essi
	equivale all'implementazione di una funzionalità.
	\LTXtable{\textwidth}{tabelle_requisiti/funzionali.tex}
	\subsection{Requisiti di vincolo}
	I requisiti di vincolo rappresentano i vincoli che devono essere soddisfatti dal prodotto e che esulano dalle
	sue caratteristiche funzionali.
	\LTXtable{\textwidth}{tabelle_requisiti/vincolo.tex}
	\subsection{Requisiti di qualità}
	I requisiti di qualità specificano le operazioni da compiere per far raggiungere al prodotto il livello 
	qualitativo richiesto.
	\LTXtable{\textwidth}{tabelle_requisiti/qualita.tex}
	\subsection{Requisiti prestazionali}
	I requisiti prestazionali consentono, al loro soddisfacimento, di far raggiungere al prodotto un determinato
	livello di prestazioni, non si traducono in nuove funzionalità per l'utente ma intendono migliorare 
	la stabilità e la velocità del programma.
	\LTXtable{\textwidth}{tabelle_requisiti/prestazionali.tex}
	
	\subsection{Tracciamento fonti-requisiti}
	Di seguito si riporta in forma tabellare l'elenco di requisiti ricavati dalle diverse fonti.
	\LTXtable{\textwidth}{tabelle_requisiti/fontirequisiti.tex}
	
	\cleardoublepage
	\addcontentsline{toc}{section}{\listfigurename}
	\listoffigures
	
	\cleardoublepage
	\addcontentsline{toc}{section}{\listtablename}
	\listoftables
		
\end{document}
	
