%Document-Author: Davide Tommasin + Elia Maino
%Document-Date: 2016/05/12
%Document-Description: Manuale Utente del gruppo SWEeneyThreads 

\documentclass[a4paper]{article}
\usepackage[english, italian]{babel}
\usepackage[T1]{fontenc}
\usepackage[utf8]{inputenc}
\usepackage{url}
\usepackage{graphicx}
\usepackage[hidelinks]{hyperref}
\usepackage{booktabs}
\usepackage{eurosym}
\usepackage{tabularx}
\usepackage{pifont}
\usepackage[table]{xcolor}
\usepackage{float}
\usepackage[]{appendix}
\usepackage{ltxtable} 
\usepackage{geometry}
\geometry{margin=1in}
\usepackage{longtable}
\usepackage{multirow}

\graphicspath{{Immagini/}}

\newcolumntype{Y}{>{\centering\arraybackslash}X}
\newcolumntype{s}{>{\hsize=.21\hsize}X}
\newcolumntype{f}{>{\hsize=.37\hsize}X}
\newcolumntype{m}{>{\hsize=.42\hsize}X}
\newcolumntype{t}{>{\hsize=.1\hsize}X}
\newcolumntype{r}{>{\hsize=.3\hsize}X}
\newcolumntype{k}{>{\hsize=.4\hsize}X}

\renewcommand{\abstractname}{Tabella contenuti}

\begin{document}
	
	\begin{titlepage}
		% Defines a new command for the horizontal lines, change thickness here
		\newcommand{\HRule}{\rule{\linewidth}{0.5mm}} 
		\center  
		
		% HEADING SECTION
		\textsc{\LARGE SWEeneyThreads}\\[1.5cm] 
		\textsc{\Large Actorbase}\\[0.5cm] 
		\textsc{\large a NoSQL DB based on the Actor model}\\[0.5cm]
		
		
		% TITLE SECTION
		\HRule \\[0.4cm]
		{ \huge \bfseries Manuale Utente}\\[0.4cm] 
		\HRule \\[1.5cm]
		
		% AUTHOR SECTION
		\begin{minipage}{0.4\textwidth}
			\begin{flushleft} \large
				\emph{Redattori:}\\
				Maino Elia \\
				Tommasin Davide \\
			\end{flushleft}
		\end{minipage}
		~
		\begin{minipage}{0.4\textwidth}
			\begin{flushright} \large
				\emph{Approvazione:} \\
				\emph{Verifica:} 
			\end{flushright}
		\end{minipage}
		
		%immagine
		\begin{figure}[H]
			\centering
			\includegraphics[scale=0.8]{sweeney.png}
		\end{figure}
		\begin{center}
			Versione 1.0.0
		\end{center}
		% Date, change the \today to a set date if you want to be precise
		{\large \today}\\[3cm] 
		% Fill the rest of the page with whitespace
		\vfill  
	\end{titlepage}
	
	
	\tableofcontents
	
	\newpage
	\section*{Diario delle modifiche}
		\LTXtable{\textwidth}{Tabelle/tabelle_diario_modifiche/tabella_manualeutente.tex}	

	\newpage 
    \section{Introduzione}
	\subsection{Scopo del documento}
		Questo documento rappresenta il manuale utente per l'utilizzo del database NoSQL \emph{Actorbase}. In questo documento verranno descritte dettagliatamente tutte le caratteristiche dell’applicativo utilizzabili dall’utente.
	\subsection{Scopo del prodotto}
		Lo scopo del progetto è la realizzazione di un DataBase NoSQL key-value basato sul modello ad 
		Attori con l'obiettivo di fornire una tecnologia adatta allo sviluppo di moderne 
		applicazioni che richiedono brevissimi tempi di risposta e che elaborano enormi quantità 
		di dati. Lo sviluppo porterà al rilascio del software sotto licenza MIT.
	\subsection{Glossario}
		Al fine di evitare ambiguità di linguaggio e di massimizzare la comprensione dei documenti, il 
      gruppo ha steso un documento interno che è il \emph{Glossario v2.0.0}. In esso saranno definiti, in modo
      chiaro e conciso i termini che possono causare ambiguità o incomprensione del testo.
	\subsection{Riferimenti}
	\subsubsection{Informativi}	
		\begin{itemize}
			\item \textbf{Slide dell'insegnamento Ingegneria del software mod.A:} \\
			\url{http://www.math.unipd.it/~tullio/IS-1/2015/Dispense/E02.pdf}
		\end{itemize}
	\subsubsection{Normativi}
		\begin{itemize}
			\item \textbf{Norme di progetto:} \emph{Norme di progetto v2.0.0}
			\item \textbf{Capitolato d'appalto Actorbase (C1):} \\ 
			\url{http://www.math.unipd.it/~tullio/IS-1/2015/Progetto/C1p.pdf}
		\end{itemize}
	
	\newpage
	\section{Istruzioni per l'utilizzo}
		\subsection{Requisiti di sitema}
			\subsubsection{Requisiti Hardware}
			\subsubsection{Requisiti Software}
		\subsection{Installazione}
	
	\newpage
	\section{Guida alle funzionalità}
	
	
	\cleardoublepage
	\addcontentsline{toc}{chapter}{\listfigurename}
	\listoffigures
	
	\cleardoublepage
	\addcontentsline{toc}{chapter}{\listtablename}
	\listoftables
		
\end{document}