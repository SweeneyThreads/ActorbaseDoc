%Document-Author: Davide Tommasin
%Document-Date: 2016/05/12
%Document-Description: Manuale Utente del gruppo SWEeneyThreads 

\documentclass[a4paper]{article}
\usepackage[english, italian]{babel}
\usepackage[T1]{fontenc}
\usepackage[utf8]{inputenc}
\usepackage{url}
\usepackage{graphicx}
\usepackage[hidelinks]{hyperref}
\usepackage{booktabs}
\usepackage{eurosym}
\usepackage{tabularx}
\usepackage{pifont}
\usepackage[table]{xcolor}
\usepackage{float}
\usepackage[]{appendix}
\usepackage{ltxtable} 
\usepackage{geometry}
\geometry{margin=1in}
\usepackage{longtable}
\usepackage{multirow}

\graphicspath{{Immagini/}}

\newcolumntype{Y}{>{\centering\arraybackslash}X}
\newcolumntype{s}{>{\hsize=.21\hsize}X}
\newcolumntype{f}{>{\hsize=.37\hsize}X}
\newcolumntype{m}{>{\hsize=.42\hsize}X}
\newcolumntype{t}{>{\hsize=.1\hsize}X}
\newcolumntype{r}{>{\hsize=.3\hsize}X}
\newcolumntype{k}{>{\hsize=.4\hsize}X}

\renewcommand{\abstractname}{Tabella contenuti}

\begin{document}
	
	\begin{titlepage}
		% Defines a new command for the horizontal lines, change thickness here
		\newcommand{\HRule}{\rule{\linewidth}{0.5mm}} 
		\center  
		
		% HEADING SECTION
		\textsc{\LARGE SWEeneyThreads}\\[1.5cm] 
		\textsc{\Large Actorbase}\\[0.5cm] 
		\textsc{\large a NoSQL DB based on the Actor model}\\[0.5cm]
		
		
		% TITLE SECTION
		\HRule \\[0.4cm]
		{ \huge \bfseries Manuale Utente}\\[0.4cm] 
		\HRule \\[1.5cm]
		
		% AUTHOR SECTION
		\begin{minipage}{0.4\textwidth}
			\begin{flushleft} \large
				\emph{Redattori:}\\
				Tommasin Davide \\
			\end{flushleft}
		\end{minipage}
		~
		\begin{minipage}{0.4\textwidth}
			\begin{flushright} \large
				\emph{Approvazione:} \\
				\emph{Verifica:} 
			\end{flushright}
		\end{minipage}
		
		%immagine
		\begin{figure}[H]
			\centering
			\includegraphics[scale=0.8]{sweeney.png}
		\end{figure}
		\begin{center}
			Versione 1.0.1
		\end{center}
		% Date, change the \today to a set date if you want to be precise
		{\large \today}\\[3cm] 
		% Fill the rest of the page with whitespace
		\vfill  
	\end{titlepage}
	
	
	\tableofcontents
	
	\newpage
	\section*{Diario delle modifiche}
		\LTXtable{\textwidth}{Tabelle/tabelle_diario_modifiche/tabella_manualeutente.tex}	

	\newpage 
    \section{Introduzione}
	\subsection{Scopo del documento}
		Questo documento rappresenta il manuale utente per l'utilizzo del database NoSQL \emph{Actorbase}. In questo documento verranno descritte dettagliatamente tutte le caratteristiche dell’applicativo utilizzabili dall’utente.
	\subsection{Scopo del prodotto}
		Lo scopo del progetto è la realizzazione di un DataBase NoSQL key-value basato sul modello ad 
		Attori con l'obiettivo di fornire una tecnologia adatta allo sviluppo di moderne 
		applicazioni che richiedono brevissimi tempi di risposta e che elaborano enormi quantità 
		di dati. Lo sviluppo porterà al rilascio del software sotto licenza MIT.
	\subsection{Glossario}
		Al fine di evitare ambiguità di linguaggio e di massimizzare la comprensione dei documenti, il 
      gruppo ha steso un documento interno che è il \emph{Glossario v2.0.0}. In esso saranno definiti, in modo
      chiaro e conciso i termini che possono causare ambiguità o incomprensione del testo.
	\subsection{Riferimenti}
	\subsubsection{Normativi}
		\begin{itemize}
			\item \textbf{Norme di progetto:} \emph{Norme di progetto v2.0.4}
			\item \textbf{Capitolato d'appalto Actorbase (C1):} \\ 
			\url{http://www.math.unipd.it/~tullio/IS-1/2015/Progetto/C1p.pdf}
		\end{itemize}
	
	\newpage
	\section{Istruzioni per l'utilizzo}		
		\subsection{Requisiti server}
			Nei paragrafi sucessivi sono elencati i vari requisiti hardware e software necessari per poter utilizzare al meglio il prodotto, relativi alla parte server.
			\subsubsection{Requisiti hardware}
			Per il corretto funzionamento del database NoSQL \emph{Actorbase} è necessario un server avente come configurazione hardware minimale:
			\begin{itemize}
				\item Memoria RAM: 4GB o superiore;
			\end{itemize}
			Non viene richiesto l'utilizzo di uno specifico spazio su hard disk o l'uso di un particolare processore, anche se l'utilizzo di processori lenti e/o datati potrebbero rendere l'uso del programma meno fluido. 
			\subsubsection{Requisiti software}
			Per garantire il corretto funzionamento del database NoSQL \emph{Actorbase} lato server è necessario un computer con in esecuzione: 
			\begin{itemize}
				\item Sistema operativo: Windows 7 (x32bit, x64bit) / OS X 10.7 / Ubuntu 14.04 LTS
				\item Java Virtual Machine (JVM) 8 o superiore
			\end{itemize}
			\subsection{Requisiti client}
			Sucessivamente saranno elencati i requisiti hardware e software necessari per poter utilizzare al meglio il prodotto, relativi alla parte client.
			\subsubsection{Requisiti hardware}
			Per il corretto funzionamento del database NoSQL \emph{Actorbase} lato client si consiglia avere come configurazione hardware:
			\begin{itemize}
				\item Memoria RAM: 2GB o superiore;
			\end{itemize}
			Non viene richiesto l'utilizzo di uno specifico spazio su hard disk o l'uso di un particolare processore, anche se l'utilizzo di processori lenti e/o datati potrebbero rendere l'uso del programma meno fluido. 
			\subsubsection{Requisiti software}
			Per garantire il corretto funzionamento del database NoSQL \emph{Actorbase} lato client è necessario un computer con in esecuzione: 
			\begin{itemize}
				\item Sistema operativo: Windows 7 (x32bit, x64bit) / OS X 10.7 / Ubuntu 14.04 LTS;
				\item Java Virtual Machine (JVM) 8 o superiore.
			\end{itemize}
			\subsection{Installazione}
	
	\newpage
	\section{Guida alle funzionalità}
	In questo capitolo verranno presentati i comandi più utili per l'utilizzazione del database NoSQL \emph{Actorbase}.
	
		\subsection{Comandi di connessione, disconnessione e aiuto}
			\subsubsection{connect}
			Questo comando permette all'utente di connettersi ad un determinaro server, conoscendo l'indirizzo ip del server e avendo il proprio username e password.
			\begin{center}
				CONNECT indirizzoIP nomeUtente password 
			\end{center}
			\subsubsection{disconnect}
			Questo comando è l'ultimo comando da utilizzare dopo aver utilizzato il database, serve per disconnettersi dal server precedentemente selezionato col comando "connect".
			\begin{center}
				DISCONNECT
			\end{center}
			\subsubsection{help}
			Il comando "help" serve per avere informazioni dettagliate riguardo un comando da poter eseguire sul database. Questo comando puo essere di due tipologie:
			\begin{itemize}
				\item Help completo: stampa tutta la lista dei comandi con relativo esempio e spiegazione;
				\begin{center}
						HELP
				\end{center}
				\item Help specifico: stampa l'esempio e la spiegazione solo del comando interessato.
				\begin{center}
						HELP nomeComando
				\end{center}
			\end{itemize}
		\subsection{Comandi di operazioni su database}
			\subsubsection{listdb}
			Digitando il comando "listdb" verranno visualizzati tutti i database presenti all'interno dell'applicativo.
			\begin{center}
				LISTDB 
			\end{center}
			\subsubsection{selectdb}
			Utilizzando questo comando si andrà a selezionare il database nel quale si vorrà fare operazioni di visualizzazione, inserimento, modifica e cancellazione.
			\begin{center}
				SELECTDB nomeDatabase
			\end{center}
			\subsubsection{createdb}
			Il comando "createdb" ha la funzione di creare un nuovo database col nome specificato, se non gia esistente, altrimenti verrà visualizzato un messaggio d'errore.
			\begin{center}
				CREATEDB nomeDatabase
			\end{center}
			\subsubsection{deletedb}
			Questo comando, avendo i permessi corretti, permette di eliminare un database esistente, se si proverà ad eliminare un database non esistente verrà visulizzato un errore.
			\begin{center}
				DELETEDB nomeDatabase
			\end{center}
		\subsection{Comandi di operazioni sulle mappe}
			\subsubsection{listmap}
			Il comando "listmap" visualizza una lista di tutte le mappe all'interno del database precedentemente selezionato utilizzando il comando "selectdb".
			\begin{center}
				LISTMAP
			\end{center}
			\subsubsection{selectmap}
			Digitando questo comando si potrà selezionare una determinata mappa per poter andare a fare operazioni sulle chiavi e i valori al suo interno. 
			\begin{center}
				SELECTMAP map
			\end{center}
			\subsubsection{createmap}
			Utilizzando il comando "createmap", se si posseggono i corretti permessi e se non esiste una mappa con lo stesso nome, creerà una mappa col nome specificato all'interno del database precedentemente selezionato.
			\begin{center}
				CREATEMAP map
			\end{center}
			\subsubsection{deletemap}
			Il comando "deletemap" permette di eliminare una mappa all'interno di u database, se si proverà ad eliminare una mappa insesistente verrà visualizzato un adeguato messaggio d'errore.
			\begin{center}
				DELETEMAP map
			\end{center}
		\subsection{Comandi di operazioni sulle mappe}
			\subsubsection{keys}
			Questo comando permette di visualizzare tutte le chiavi, e non i rispettivi valori, all'interno di una mappa precedentemente selezionata col comando "selectmap".
			\begin{center}
				KEYS
			\end{center}
			\subsubsection{find}
			Il comando "find" ritorna il valore associato alla chiave specificata.
			\begin{center}
				FIND key
			\end{center}
			\subsubsection{remove}
			Se si posseggono i giusti permessi, questo comando permettere di rimuovere la chiave e il relativo valore dalla mappa selezionata in precedenza.
			\begin{center}
				REMOVE key
			\end{center}
			\subsubsection{insert}
			Il comando "insert" permette di inserire una nuova chiave e il suo valore, se la chiave non è gia esistente, altrimenti verrà visualizzato un messaggio d'errore.
			\begin{center}
				INSERT key valore
			\end{center}
			\subsubsection{update}
			Digitando il comando "update" si potrà aggiornare la chiave e il rispettivo valore all'interno di una mappa precedentemente selezionata.
			\begin{center}
				UPDATE key valore
			\end{center}
			
	\cleardoublepage
	\addcontentsline{toc}{chapter}{\listfigurename}
	\listoffigures
	
	\cleardoublepage
	\addcontentsline{toc}{chapter}{\listtablename}
	\listoftables
		
\end{document}