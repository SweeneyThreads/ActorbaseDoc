%Document-Author: Davide Tommasin
%Document-Date: 2016/05/12
%Document-Description: Manuale Utente del gruppo SWEeneyThreads 

\documentclass[a4paper]{article}
\usepackage[english, italian]{babel}
\usepackage[T1]{fontenc}
\usepackage[utf8]{inputenc}
\usepackage{url}
\usepackage{graphicx}
\usepackage[hidelinks]{hyperref}
\usepackage{booktabs}
\usepackage{eurosym}
\usepackage{tabularx}
\usepackage{pifont}
\usepackage[table]{xcolor}
\usepackage{float}
\usepackage[]{appendix}
\usepackage{ltxtable} 
\usepackage{geometry}
\geometry{margin=1in}
\usepackage{longtable}
\usepackage{multirow}

\graphicspath{{Immagini/}}

\newcolumntype{Y}{>{\centering\arraybackslash}X}
\newcolumntype{s}{>{\hsize=.21\hsize}X}
\newcolumntype{f}{>{\hsize=.37\hsize}X}
\newcolumntype{m}{>{\hsize=.42\hsize}X}
\newcolumntype{t}{>{\hsize=.1\hsize}X}
\newcolumntype{r}{>{\hsize=.3\hsize}X}
\newcolumntype{k}{>{\hsize=.4\hsize}X}

\renewcommand{\abstractname}{Tabella contenuti}

\begin{document}
	
	\begin{titlepage}
		% Defines a new command for the horizontal lines, change thickness here
		\newcommand{\HRule}{\rule{\linewidth}{0.5mm}} 
		\center  
		
		% HEADING SECTION
		\textsc{\LARGE SWEeneyThreads}\\[1.5cm] 
		\textsc{\Large Actorbase}\\[0.5cm] 
		\textsc{\large a NoSQL DB based on the Actor model}\\[0.5cm]
		
		
		% TITLE SECTION
		\HRule \\[0.4cm]
		{ \huge \bfseries Manuale Utente}\\[0.4cm] 
		\HRule \\[1.5cm]
		
		% AUTHOR SECTION
		\begin{minipage}{0.4\textwidth}
			\begin{flushleft} \large
				\emph{Redattori:}\\
				Maino Elia \newline
				Tommasin Davide \\
			\end{flushleft}
		\end{minipage}
		~
		\begin{minipage}{0.4\textwidth}
			\begin{flushright} \large
				\emph{Approvazione:} \\
				\emph{Verifica:} \\
				Nicoletti Luca
			\end{flushright}
		\end{minipage}
		
		%immagine
		\begin{figure}[H]
			\centering
			\includegraphics[scale=0.8]{sweeney.png}
		\end{figure}
		\begin{center}
			Versione 1.0.2
		\end{center}
		% Date, change the \today to a set date if you want to be precise
		{\large \today}\\[3cm] 
		% Fill the rest of the page with whitespace
		\vfill  
	\end{titlepage}
	
	
	\tableofcontents
	
	\newpage
	\section*{Diario delle modifiche}
		\LTXtable{\textwidth}{Tabelle/tabelle_diario_modifiche/tabella_manualeutente.tex}	

	\newpage 
    \section{Introduzione}
	\subsection{Scopo del documento}
		Questo documento rappresenta il manuale utente per l'utilizzo del database NoSQL \emph{Actorbase}. In questo documento verranno descritte dettagliatamente tutte le caratteristiche dell’applicativo utilizzabili dall’utente. Il manuale è diviso in tre sezioni realtive alle diverse componenti di \emph{Actorbase}: Client, Server e Driver.
	\subsection{Scopo del prodotto}
		Lo scopo del progetto è la realizzazione di un DataBase NoSQL key-value basato sul modello ad 
		Attori con l'obiettivo di fornire una tecnologia adatta allo sviluppo di moderne 
		applicazioni che richiedono brevissimi tempi di risposta e che elaborano enormi quantità 
		di dati. Lo sviluppo porterà al rilascio del software sotto licenza MIT.
	\subsection{Glossario}
		Al fine di evitare ambiguità di linguaggio e di massimizzare la comprensione dei documenti, il 
      gruppo ha steso un documento interno che è il \emph{Glossario v2.0.0}. In esso saranno definiti, in modo
      chiaro e conciso i termini che possono causare ambiguità o incomprensione del testo.
	\subsection{Riferimenti}
	\subsubsection{Normativi}
		\begin{itemize}
			\item \textbf{Norme di progetto:} \emph{Norme di progetto v3.0.0}
			\item \textbf{Capitolato d'appalto Actorbase (C1):} \\ 
			\url{http://www.math.unipd.it/~tullio/IS-1/2015/Progetto/C1p.pdf}
		\end{itemize}
	\newpage

	\section{Actorbase}
	\emph{Actorbase} è un database NoSQL key-value basato sul modello matematico ad attori che garantisce un alto livello di scalabilità, resilienza e prestazioni. Permette di gestire facilmente e flessibilmente i propri dati usufruendo dei principali vantaggi offerti dal modello ad attori, in modo
	da supportare lo sviluppo di applicazioni moderne e performanti. 
	\\ \\
	\emph{Actorbase} fornisce un'interfaccia client a linea di comando che offre una facile gestione dei dati come stringhe, grazie ad essa è possibile comunicare rapidamente e intuitivamente con un server. 
	\\ \\
	Per una gestione più flessibile delle query è disponibile il driver per \emph{Scala}, integrabile con qualsiasi applicazione.
	\begin{figure}[H]
		\centering
		\includegraphics[scale=0.4]{actorbaseLogo.png}
		\caption{Logo di Actorbase}
	\end{figure}

	\newpage



	\section{Requisiti di sistema}	
	L'esecuzione corretta di \emph{Actorbase} è garantita su macchine che soddisfano le seguenti caratteristiche hardware e software.
	\\ \\
	Sistema operativo:
	\begin{itemize}
		\item Windows 7 o superiori
		\item OS X 10.7 o superiori
		\item Ubuntu 14.04 o superiori
	\end{itemize}
	Java Virtual Machine (JVM) versione 8 o superiore.
	\\ \\
	RAM:
	\begin{itemize}
		\item Applicativo client: 2GB minimo
		\item Applicativo server: 4GB minimo, 8GB consigliato
	\end{itemize}
	Non sono presenti espliciti vincoli per quanto riguarda il processore, chiaramente l'utilizzo di hardware eccessivamente datato può influenzare le prestazioni del sistema.
	
	\section{Installazione}
	\emph{Actorbase} viene eseguito su JVM, per tale motivo non è necessaria una procedura di installazione per utilizzare il software. 
	\newpage



	\section{Applicativo server}
	L'applicativo server consente di avviare un'istanza \emph{Actorbase} server sulla macchina. Una volta avviato permette ai client di effettuare connessioni alla macchina e di interrogare un database. 
	
	\subsection{Configurazione del server}
	La configurazione della macchina server avviene tramite la modifica del file \texttt{server.conf}. In esso vanno settati indirizzo IP e porta di connessione al server.
	\\
	Nel caso si tentasse di avviare l'applicativo server senza aver definito opportunamente i parametri di configurazione, il server mostrerà un messaggio di errore e non si avvierà.
	
	\subsection{Interfaccia server}
	Una volta modificato il file di configurazione è possibile avviare il server cliccando sull'apposita icona. L'applicativo server fornisce all'utente un'interfaccia a linea dicomando che mostra in tempo reale il log delle operazioni effettuate sul server \emph{Actorbase}.  
	\begin{figure}[H]
		\centering
		\includegraphics[width=\textwidth]{logServer.png}
		\caption{Interfaccia di log del server}
	\end{figure}
	\newpage
	

	\section{Applicativo client}
	L'applicativo client fornisce all'utente un interfaccia a linea dicomando per connettersi ad un server \emph{Actorbase} ed effettuare interrogazioni sul database. Come per il server, l'avvio dell'interfaccia client avviene cliccando sull'apposita icona. Appena avviato il client presenta all'utente un banner di benvenuto seguito da una breve descrizione della configurazione software utilizzata (sistema operativo e JVM).
	\begin{figure}[H]
		\centering
		\includegraphics[width=\textwidth]{welcomeClient.png}
		\caption{Informazioni di benvenuto applicativo client}
	\end{figure}
	L'interazione con l'interfaccia avviene tramite l'immissione di comandi testuali, essi possono essere composti da più campi separati da un carattere di spazio.
	
	\subsection{Gestione della connessione}
	Una volta avviato il client per prima cosa è necessario effettuare una connessione ad un server, la gestione della connessione si basa su due comandi: \texttt{connect} e \texttt{disconnect}. L'applicativo client fornito non permette di gestire più connessioni contemporaneamente.
	
	\subsubsection{Comando \texttt{connect}}
	Questo comando permette all'utente di connettersi ad un server, la struttura del comando è la seguente:
	\\ \\
	\texttt{actorbase>	connect <indirizzo> <username> <password>}
	\\ \\
	In particolare l'indirizzo del server deve essere fornito nel formato \texttt{indirizzoServer:porta}. \\ \\
	Nel caso la richiesta di connessione abbia successo l'utente riceve un messaggio di conferma (\texttt{"You are connected!"}), in caso contrario un messaggio di errore (\texttt{"Connection failed!"}).
	
	\subsubsection{Comando \texttt{disconnect}}
	Per effettuare la disconnessione dal server a cui si è connessi è necessario inserire il comando \texttt{disconnect} e premere invio.
	

	\subsection{Aiuto inline}
	\'E possibile ottenere un aiuto per quanto riguarda le operazioni effettuabili, direttamente a linea dicomando. Il comando \texttt{help} permette di ottenere sia un aiuto generico che un aiuto specifico.
	
	\subsubsection{Comando \texttt{help} generico}
	Il comando di aiuto generico \texttt{help}, non richiede parametri ulteriori e stampa a video la lista dei comandi di \emph{Actorbase}. Ogni comando è seguito da una breve descrizione che ne illustra il funzionamento.
	
	\subsubsection{Comando \texttt{help} specifico}
	Il comando di aiuto specifico presenta la seguente struttura:
	\\ \\
	\texttt{actorbase>	help <nomeComando>}
	\\ \\
	Permette di richiedere informazioni per un comando specifico, ottenendone una breve descrizione.
	
	\subsection{Comandi per operazioni a livello server}
	Una volta connesso l'utente si trova a livello server. A tale livello si possono effettuare operazioni sui database quali:
	\begin{itemize}
		\item Visualizzazione della lista di database
		\item Selezione di un database
		\item Creazione di un database
		\item Rimozione di un database
	\end{itemize}
	
	\subsubsection{Comando \texttt{listdb}}
	Questo comando permette di ottenere, stampata a video, la lista dei database di cui dispone di permessi di accesso (siano essi di visualizzazione o modifica). Tale comando non richiede parametri aggiuntivi.
	\\ \\
	\texttt{actorbase>	listdb}

	\subsubsection{Comando \texttt{selectdb}}
	Questo comando permette di selezionare un database. Una volta selezionato un database l'utente può effettuare operazioni sulle mappe di esso. La struttura del comando di selezione database è la seguente:
	\\ \\
	\texttt{actorbase>	selectdb <nomeDatabase>}
	\\ \\
	Nel caso l'utente disponga dei permessi necessari alla selezione del database richiesto (\texttt{x}), riceve a video un messaggio di conferma: \texttt{"Database x selected"}. In caso contrario viene riportata un operazione non valida (\texttt{"Invalid operation"}).

	\subsubsection{Comando \texttt{createdb}}
	Questo comando permette di creare un nuovo database col nome specificato, nel caso  un database con tale nome non fosse già presente.
	\\ \\
	\texttt{actorbase>	createdb <nomeDatabase>}
	\\ \\
	Nel caso esista già un database con il nome inserito la creazione fallisce, l'utente riceve un messaggio di errore: \texttt{"A database with the requested name already exists"}.

	\subsubsection{Comando \texttt{deletedb}}
	Questo comando permette di eliminare un database dal server:
	\\ \\
	\texttt{actorbase>	deletedb <nomeDatabase>}
	\\ \\
	Nel caso si tenti di rimuovere un database inesistente o di cui non si dispone dei permessi di modifica, l'utente riceve un messaggio di operazione non valida (\texttt{"Invalid operation"}).
	

	\subsection{Comandi per operazioni a livello database}
	Una volta selezionato un database tramite il comando \texttt{selectdb}, l'utente si trova a livello database. A tale livello si possono effettuare queste operazioni:
	\begin{itemize}
		\item Visualizzazione della lista delle mappe che compongono il database
		\item Selezione di una mappa
		\item Creazione di una mappa
		\item Eliminazione di una mappa
	\end{itemize}

	\subsubsection{Comando \texttt{listmap}}
	Questo comando stampa una lista di tutte le mappe che compongono il database precedentemente selezionato.
	\\ \\
	\texttt{actorbase>	listmap}

	\subsubsection{Comando \texttt{selectmap}}
	Questo comando permette di selezionare una mappa, utilizzando la seguente sintassi:
	\\ \\
	\texttt{actorbase>	selectmap <nomeMappa>}
	\\ \\
	La selezione viene confermata in caso di successo con il messaggio \texttt{"Map x selected"}, altrimenti si 
	riceve un messaggio di operazione non valida (\texttt{"Invalid operation"}).

	\subsubsection{Comando \texttt{createmap}}
	Questo comando permette di creare una mappa con il nome inserito, all'interno del database selezionato.
	\\ \\
	\texttt{actorbase>	createmap <nomeMappa>}
	\\ \\
	La creazione viene confermata in caso di successo con il messaggio \texttt{"Map x created"}. Nel caso non si 
	disponga dei permessi di modifica al database, o esista già una mappa con il nome inserito, si riceve un messaggio 
	di operazione non valida (\texttt{"Invalid operation"}).

	\subsubsection{Comando \texttt{deletemap}}
	Questo comando permette di eliminare una mappa dal database selezionato. 
	\\ \\
	\texttt{actorbase>	deletemap <nomeMappa>}
	\\ \\
	L'eliminazione viene confermata in caso di successo con il messaggio \texttt{"Map x removed"}. Nel caso non si disponga 
	dei permessi di modifica al database, o la mappa richiesta non esista, si riceve un messaggio di operazione 
	non valida (\texttt{"Invalid operation"}).
	

	\subsection{Comandi per operazioni a livello mappa}
	Una volta selezionata una mappa tramite il comando \texttt{selectmap}, l'utente si trova a livello mappa. A tale livello si possono effettuare queste operazioni:
	\begin{itemize}
		\item Visualizzazione della lista delle chiavi
		\item Ricerca di un item per chiave
		\item Inserimento di un item 
		\item Aggiornamento di un item
		\item Rimozione di un item
	\end{itemize}

	\subsubsection{Comando \texttt{keys}}
	Questo comando permette di visualizzare la lista di tutte le chiavi degli item che compongono la mappa selezionata.
	\\ \\
	\texttt{actorbase>	keys}

	\subsubsection{Comando \texttt{find}}
	Questo comando permette di ottenere il valore di un item della mappa, effettuando la ricerca tramite la sua chiave.
	\\ \\
	\texttt{actorbase>	find '<key>'}
	\\ \\
	Nel caso sia presente un item corrispondente alla chiave inserita, si riceve il valore dell'item in formato stringa, stampato a video. Se l'item non viene trovato si riceve un messaggio di errore.
	\subsubsection{Comando \texttt{remove}}
	Questo comando permette di rimuovere un item dalla mappa, ricercandolo tramite la chiave.
	\\ \\
	\texttt{actorbase>	remove '<key>'}
	\\ \\
	L'eliminazione viene confermata in caso di successo con il messaggio \texttt{"Item removed"}. Nel caso non si disponga dei permessi di modifica al database, o l'item 
	richiesto non esista, si riceve un messaggio di operazione non valida (\texttt{"Invalid operation"}).

	\subsubsection{Comando \texttt{insert}}
	Questo comando permette di inserire un item (coppia chiave-valore) nella mappa:
	\\ \\
	\texttt{actorbase>	insert '<key>' <value>}
	\\ \\
	L'inserimento viene confermato in caso di successo con il messaggio \texttt{"Item inserted"}. Nel caso non si disponga dei permessi di modifica al database, o sia già presente un item con la chiave inserita, si riceve un messaggio di operazione non valida (\texttt{"Invalid operation"}).
	
	\subsubsection{Comando \texttt{update}}
	Questo comando permette di aggiornare un item dalla mappa, ricercandolo tramite la chiave e inserendo il valore aggiornato.
	\\ \\
	\texttt{actorbase>	update '<key>' <value>}
	\\ \\
	L'aggiornamento viene confermato in caso di successo con il messaggio \texttt{"Item updated"}. Nel caso non si disponga dei permessi di modifica al database, o l'item 
	richiesto non esista, si riceve un messaggio di operazione non valida (\texttt{"Invalid operation"}).
			
	\cleardoublepage
	\addcontentsline{toc}{section}{\listfigurename}
	\listoffigures
	
	\cleardoublepage
	\addcontentsline{toc}{section}{\listtablename}
	\listoftables
		
\end{document}