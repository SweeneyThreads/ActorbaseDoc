%Document-Author: Nicoletti Luca + Padovan Tommaso
%Document-Date: 2016/01/13
%Document-Description: Documento di Piano di Progetto del gruppo SWEeneyThreads 

\documentclass[a4paper]{report}
\usepackage[english, italian]{babel}
\usepackage[T1]{fontenc}
\usepackage[utf8]{inputenc}
\usepackage{url}
\usepackage{graphicx}
\usepackage[hidelinks]{hyperref}
\usepackage{booktabs}
\usepackage{tabularx}
\usepackage{pifont}
\usepackage[table]{xcolor}
\usepackage{float}
\usepackage[]{appendix}

\graphicspath{{Immagini/}}

\newcolumntype{s}{>{\hsize=.21\hsize}X}
\newcolumntype{f}{>{\hsize=.37\hsize}X}
\newcolumntype{m}{>{\hsize=.42\hsize}X}

\newcommand{\mychapter}[2]{
	\setcounter{chapter}{#1}
	\setcounter{section}{0}
	\setcounter{subsection}{1}
	\chapter*{#2}
	\addcontentsline{toc}{chapter}{#2}
}

\renewcommand{\abstractname}{Tabella contenuti}

\begin{document}

	\begin{titlepage}
		% Defines a new command for the horizontal lines, change thickness here
		\newcommand{\HRule}{\rule{\linewidth}{0.5mm}} 
		\center  
		
		% HEADING SECTION
		\textsc{\LARGE SweeneyThreads}\\[1.5cm] 
		\textsc{\Large Actorbase}\\[0.5cm] 
		\textsc{\large a NoSQL DB based on the Actor model}\\[0.5cm]
		
		
		% TITLE SECTION
		\HRule \\[0.4cm]
		{ \huge \bfseries Piano di progetto}\\[0.4cm] 
		\HRule \\[1.5cm]
		
		% AUTHOR SECTION
		\begin{minipage}{0.4\textwidth}
			\begin{flushleft} \large
				\emph{Redattori:}\\
				Nicoletti \textsc{Luca} \\
				Padovan \textsc{Tommaso} \\
			\end{flushleft}
		\end{minipage}
		~
		\begin{minipage}{0.4\textwidth}
			\begin{flushright} \large
				\emph{Approvazione:} \\
				\dots \\
				\emph{Verifica:} \\
				Biggeri \textsc{Mattia} \\
				Tommasin \textsc{Davide} \\
			\end{flushright}
		\end{minipage}
		
		%immagine
		\begin{figure}[H]
			\centering
			\includegraphics[scale=0.8]{sweeney.png}
		\end{figure}
		\begin{center}
			Versione 1.0.0
		\end{center}
		% Date, change the \today to a set date if you want to be precise
		{\large \today}\\[3cm] 
		% Fill the rest of the page with whitespace
		\vfill  
	\end{titlepage}
	
	\tableofcontents
	
	\mychapter{0}{Diario delle modifiche}
		\begin{table}[H]
			\begin{tabularx}{\textwidth}{sfmX}
				\noalign{\hrule height 1.5pt}
				\rowcolor{orange!85} Versione & Data & Autore & Descrizione \\
				\noalign{\hrule height 1.5pt}
				1.0.1 & 2016-01-14 & \emph{Amministratore} Nicoletti Luca & Stesura analisi dei rischi e introduzione del documento \\
				\noalign{\hrule height 0.5pt}
				1.0.0 & 2016-01-13 & \emph{Amministratore} Nicoletti Luca & Scrittura scheletro logico del documento \\
				\noalign{\hrule height 1.5pt}
			\end{tabularx}
			\caption{Diario delle modifiche \label{tab:table_label}}
		\end{table}
	
	\begin{appendices}
	\mychapter{1}{Organigramma}
		\section{Redazione}
		\begin{table}[H]
			\begin{tabularx}{\textwidth}{X X X}
				\noalign{\hrule height 1.5pt}
				\rowcolor{orange!85}Nominativo & Data di redazione & Firma\\
				\noalign{\hrule height 1.5pt}
				Nicoletti Luca & 2016-01-13 & \\
				\noalign{\hrule height 0.5pt}
				Padovan Tommaso & 2016-01-13 &\\
				\noalign{\hrule height 1.5pt}
			\end{tabularx}
			\caption{Redazione documento \label{tab:table_label}}
		\end{table}
		\section{Approvazione}
		\begin{table}[H]
			\begin{tabularx}{\textwidth}{X X X}
				\noalign{\hrule height 1.5pt}
				\rowcolor{orange!85}Nominativo & Data di redazione & Firma\\
				\noalign{\hrule height 1.5pt}
				Nicoletti Luca & 2016-01-13 & \\
				\noalign{\hrule height 0.5pt}
				Padovan Tommaso & 2016-01-13 & \\
				\noalign{\hrule height 0.5pt}
				Prof. Vardanega Tullio & & \\
				\noalign{\hrule height 1.5pt}
			\end{tabularx}
			\caption{Approvazione documento \label{tab:table_label}}
		\end{table}
		\section{Componenti}
		\begin{table}[H]
			\begin{tabular}{p{3cm} p{1.6cm} p{5.7cm} p{1cm}}
				\noalign{\hrule height 1.5pt}
				\rowcolor{orange!85}Nominativo & Matricola & E-mail \\
				\noalign{\hrule height 1.5pt}
				Paolo Bonato & 1023655 & paolo.bonato.3@studenti.unipd.it\\
				\noalign{\hrule height 0.5pt}
				Matteo Bortolazzo & 1073194 & matteo.bortolazzo.1@studenti.unipd.it \\
				\noalign{\hrule height 0.5pt}
				Mattia Biggeri & 1074269 & mattia.biggeri@studenti.unipd.it \\
				\noalign{\hrule height 0.5pt}
				Elia Maino & 1069880 & elia.maino@studenti.unipd.it \\
				\noalign{\hrule height 0.5pt}
				Luca Nicoletti & 1070634 & luca.nicoletti.2@studenti.unipd.it \\
				\noalign{\hrule height 0.5pt}
				Tommaso Padovan & 1054128 & tommaso.padovan@studenti.unipd.it \\
				\noalign{\hrule height 0.5pt}
				Davide Tommasin & 1073541 & davide.tommasin.1@studenti.unipd.it \\
				\noalign{\hrule height 1.5pt}
			\end{tabular}
			\caption{Componenti SWEeneyThreads \label{tab:table_label}}
		\end{table}
		\section{Accetazione componenti}
		\begin{table}[H]
			\begin{tabularx}{\textwidth}{X X X}
				\noalign{\hrule height 1.5pt}
				\rowcolor{orange!85}Nominativo & Data di accettazione & Firma\\
				\noalign{\hrule height 1.5pt}
				Paolo Bonato & 2015-18-12 & \\
				\noalign{\hrule height 0.5pt}
				Matteo Bortolazzo & 2015-18-12 & \\
				\noalign{\hrule height 0.5pt}
				Mattia Biggeri & 2015-18-12 & \\
				\noalign{\hrule height 0.5pt}
				Elia Maino & 2015-18-12 & \\
				\noalign{\hrule height 0.5pt}
				Luca Nicoletti & 2015-18-12 & \\
				\noalign{\hrule height 0.5pt}
				Tommaso Padovan & 2015-18-12 & \\
				\noalign{\hrule height 0.5pt}
				Davide Tommasin & 2015-18-12 & \\
				\noalign{\hrule height 1.5pt}
			\end{tabularx}
			\caption{Accettazione componenti \label{tab:table_label}}
		\end{table}
		\section{Definizione dei ruoli}
			Nel corso dello sviluppo del progetto i membri del gruppo dovranno ricoprire diversi ruoli, rappresentanti 
			figure aziendali specializzate, indispensabili per il buon esito del progetto. Ogni singolo componente del 
			gruppo può ricoprire più ruoli, sia contemporaneamente che in distinte fasi del progetto in ogni caso 
			sempre garantendo assenza di conflitto di interesse tra i ruoli assunti.
			Tali ruoli sono:
			\begin{itemize}
				\item \textbf{Responsabile:} È il responsabile ultimo, per conto del suo gruppo, dei risultati del progetto.
				\begin{itemize}
					\item Elabora ed emana piani e scadenze.
					\item Approva l'emissione di documenti.
					\item Coordina le attività del gruppo.
					\item Si relaziona con il controllo di qualità interno al progetto.
					\item Redige Organigramma e Piano di Progetto.
					\item Approva l'Offerta e i relativi allegati.
				\end{itemize}
				\item \textbf{Amministratore:} È responsabile dell'efficienza e dell'operatività dell'ambiente di sviluppo 
				oltre che della redazione e attuazione di piani e procedure di Gestione per la Qualità.
				\begin{itemize}
					\item Controlla versioni e configurazioni del prodotto. 
					\item Gestisce l'archivio della documentazione di progetto. 
					\item Collabora alla redazione del Piano di Progetto. 
					\item Redige le Norme di Progetto per conto del Responsabile.
				\end{itemize}
				\item \textbf{Analista:} È responsabile delle attività di analisi. 
				\begin{itemize}
					\item Redige lo Studio di Fattibilità (documento interno al gruppo) e l'Analisi dei Requisiti.
				\end{itemize}
				\item \textbf{Progettista:} È responsabile delle attività di progettazione. 
				\begin{itemize}
					\item Redige Specifica Tecnica, Definizione di Prodotto e la parte programmatica del Piano di Qualifica.
				\end{itemize}
				\item \textbf{Programmatore:} È responsabile delle attività di codifica miranti alla realizzazione del 
				prodotto e delle componenti di ausilio necessarie per l'esecuzione delle prove di verifica e validazione.
				\item \textbf{Verificatore:} È responsabile delle attività di verifica.
				\begin{itemize}
					\item Redige la parte retrospettiva del Piano di Qualifica che illustra l'esito e la completezza delle 
					verifiche e delle prove effettuate secondo il piano.
				\end{itemize}
			\end{itemize}
	\end{appendices}
	
	\mychapter{1}{Introduzione}
		\section{Scopo del documento}
			Il presente documento ha l'intento di specificare la pianificazione secondo la quale saranno portati avanti i 
			lavori dal gruppo SWEeneyThreads in merito al progetto \emph{Actorbase}.
			Gli scopi del presente documento sono:
			\begin{itemize}
				\item Presentare la pianificazione dei tempi e delle attività, definendo le scadenze e la suddivisione dei lavori
				\item Preventivare l’utilizzo delle risorse, descrivendo i costi in relazione alla suddivisione del lavoro
				\item Consuntivare l’utilizzo delle risorse durante l’evolversi dei lavori
				\item Analizzare i possibili fattori di rischio e descrivere i relativi strumenti di controllo
			\end{itemize}
		\section{Riferimenti}
			\subsection{Informativi}
				\begin{itemize}
					\item 
				\end{itemize}
			\subsection{Normativi}
				\begin{itemize}
					\item \textbf{Capitolato d'appalto: } \\ \url{http://www.math.unipd.it/~tullio/IS-1/2015/Progetto/C1p.pdf}
					\item \textbf{Vincoli di organigramma e dettagli tecnici-economici} \\ 
					\url{http://www.math.unipd.it/~tullio/IS-1/2015/Progetto/PD01b.html}
				\end{itemize}
		\section{Ciclo di vita}
			Il gruppo ha deciso, uninamente di adottare il \textbf{modello incrementale} come modello di ciclo di vita.
			Tale scelta è stata presa per i vari vantaggi che comporta questo modello di Ciclo di vita\ped{\emph{G}}:
			\begin{itemize}
				\item rende più facile la gestione e il controllo del progetto e quindi la stima del preventivo;
				\item aiuta a definire in moto specifico un'unità permettendo di effettuare test di maggiore dettaglio, ma 
				in numero contenuto, in modo da riuscire a mantenere positivo il rapporto costi/benefici dei test;
				\item prevede rilasci multipli e successivi;
				\item riduce il rischio di fallimento ad ogni iterazione, il che lo rende particolarmente adatto a gruppi 
				inesperti.
			\end{itemize}
			
			Nello specifico, queste parti si adattano bene al capitolato scelto: ad esempio la modularità. I rilasci 
			multipli e successivi ci permetteranno di rilasciare prototipi dei singoli \emph{Actors} permettendoci di
			isolare i requisiti per i successivi incrementi. Questo ciclo di vita permetterà al gruppo di raffinare e 
			di rivedere, di rilascio in rilascio, i requisiti relativi ai vari \emph{Actors}. Inoltre, prevedendo 
			\emph{Actors} differenti e indipendenti tra loro, non tutti obbligatori, il \textbf{modello incrementale} 
			si adatta alla perfezione.
			
			Il gruppo ha anche valutato l'utilizzo del \textbf{modello evolutivo}. Esso risulta però inadeguato per alcune sue 
			caratteristiche non perfettamente aderenti al nostro capitolato:
			\begin{itemize}
				\item lo scopo fondamentale di un modello evolutivo è rilasciare molte versioni di un sistema che va 
				poi sempre più raffinato: nel caso preso in considerazione non è l'intero sistema a dover essere raffinato, 
				semplicemente si aggiungeranno nuove parti che verrano integrate in modo incrementale;
				\item il modello incrementale inoltre aiuta a rispondere a bisogni non inizialmente preventivabili, mentre 
				quelli di \emph{Actorbase} sono perlopiù definiti a priori.
			\end{itemize}
			Queste due qualità, come indicato, non sono largamente necessarie; l'adozione di questo modello porterebbe dei 
			vantaggi modesti a fronte della necessità, molto onerosa, di riattraversare diverse fasi del ciclo di vita.
		\section{Scadenze}
			Di seguito vengono riportate le scadenze che il gruppo SWEeneyThreads ha deciso di rispettare e sulle quali 
			si baserà la pianificazione del progetto:
			\begin{enumerate}
				\item Revisione dei requisiti [RR]: 2016-02-16
				\item Revisione di progettazione [RP]: 2016-04-18
				\item Revisione di qualifica [RQ]: 2016-05-23
				\item Revisione di accettazione [RA]: 2016-06-17
			\end{enumerate}
	\mychapter{2}{Pianificazione}
		\section{Analisi}
		\section{Analisi di dettaglio}
		\section{Progettazione architetturale}
		\section{Progettazione di dettaglio}
		\section{Verifica e validazione}
	\mychapter{3}{Suddivisione del lavoro}
		Ogni membro del gruppo, dovrà, durante tutta la durata del progetto, ricoprire almeno una volta ciascuno dei
		ruoli descritti nell'appendice, sezione A.5. Durante ogni fase\ped{\emph{G}} ogni membro può ricoprire
		più ruoli contemporaneamente, a patto che non si verifichino dei conflitti di interesse tra i ruoli ricoperti. 
		Un membro del gruppo non può verificare il suo stesso lavoro.
		\section{Dettaglio delle fasi}
			\subsection{Analisi}
			\subsection{Analisi di dettaglio}
			\subsection{Progettazione architetturale}
			\subsection{Progettazione di dettaglio}
			\subsection{Verifica e validazione}
		\section{Totali}
			\subsection{Totale ore di investimento}
			\subsection{Totale ore rendicontate}
	\mychapter{4}{Prospetto economico}
		\section{Analisi}
		\section{Analisi di dettaglio}
		\section{Progettazione architetturale}
		\section{Progettazione di dettaglio}
		\section{Verifica e validazione}
		\section{Totali}
			\subsection{Investimento}
			\subsection{Preventivo}
			\subsection{Conclusione}
	\mychapter{5}{Analisi dei rischi}
		Al fine di migliorare l'avanzamento del progetto, è stata effettuata un'accurata analisi dei rischi suddivisa in:
		\begin{itemize}
			\item \textbf{Identificazione:} in cui si identificano i principali fattori di rischi come:
			\begin{itemize}
				\item Variabilità della disponibilità del personale;
				\item Variabilità delle tecnologie;
				\item Ritardo o mutazione di requisiti fondamentali;
				\item Specifiche in ritardo.
			\end{itemize}
			\item \textbf{Analisi:} durante la quale si individua la possibilità di occorrenza di ogni rischio, e le conseguenze a cui porterebbe.
			\item \textbf{Pianificazione:} scelta di tecniche per evitare il verificarsi dei rischi verificati, o per mitigarne gli effetti.
			\item \textbf{Controllo:} attività svolta durante tutto il ciclo di vita del progetto per prevedere il verificarsi dei rischi
			ed evitare che si verifichino.
		\end{itemize}
		Per ogni rischio individuato viene quindi stilata una lista di attributi quali: la sua probabilità di occorrenza, la gravità delle 
		conseguenze a cui porterebbe il suo verificarsi, una descrizione, le strategie da utilizzare per la sua rilevazione preventiva e le 
		contromisure da adottare (nel caso in cui il rischio si verifichi, o nel caso in cui si noti che il rischio sta per verificarsi). \\
		L'identificazione dei rischi viene gestita a livelli.
		
		\section{Livello tecnologico}
			\subsection{Tecnologie adottate}
				\subsubsection{Probabilità:}
					Bassa.
				\subsubsection{Gravità:}
					Alta.
				\subsubsection{Descrizione:}
					Nessun membro del gruppo ha una conoscenza in tutte le tecnologie utilizzate nel 
					progetto. È quindi possibile che il gruppo incontri inconvenienti nell'utilizzo 
					di determinati strumenti o tecnologie.
				\subsubsection{Contromisure:}
					L'\emph{Amministratore} è tenuto a fornire documentazione sufficiente riguardante 
					le tecnologie adottate, in tempo utile per permettere all'intero gruppo di documentarsi 
					in maniera autonoma.
			\subsection{Rotture hardware}
				\subsubsection{Probabilità:}
					Bassa.
				\subsubsection{Gravità:}
					Bassa.
				\subsubsection{Descrizione:}
					Il gruppo ha deciso di sfruttare servizi online gratuiti o software open-source per lo 
					sviluppo del progetto. È quindi da tenere in considerazione il possibile malfunzionamento 
					di host o di qualche servizio/piattaforma.
				\subsubsection{Contromisure:}
					Il gruppo si impegna ad effettuare un backup periodicamente in modo da preventivare un'eventuale 
					perdita di dati. Questo è compito del \emph{Responsabile di progetto}. La copia di backup sarà mantenuta 
					sia sul \emph{Drive} del gruppo, sia su un disco rimovibile. Tutti i membri del gruppo hanno a disposizione 
					un computer di supporto per poter rimanere operativi anche in caso di guasti hardware alle proprie macchine.
		\section{Livello del personale}
			\subsection{Inesperienza del gruppo}
				\subsubsection{Probabilità:} 
					Media.
				\subsubsection{Gravità:} 
					Media.
				\subsubsection{Descrizione:} 
					Il gruppo per lo sviluppo del progetto didattico andrà ad utilizzare una tecnologia 
					con la quale nessuno ha particolare familiarità, questo può portare a ritardi 
					nella fase di sviluppo dovuti a risoluzione di problemi di primo approccio ad una 
					nuova tecnologia non conosciuta.
				\subsubsection{Contromisure:}
					Il gruppo, per prevenire il verificarsi di questo rischio, ha stabilito di leggersi
					più di un manuale riguardante \emph{Scala}, il linguaggio di programmazione richiesto
					dal capitola d'appalto. Inoltre, il gruppo si sta già formando all'utilizzo delle
					altre tecnologie previste per il corretto svolgimento di ogni fase.
			\subsection{Variazione disponibilità}
				\subsubsection{Probabilità:}
					Media.
				\subsubsection{Gravità:} 
					Medio-alta.
				\subsubsection{Descrizione:}
					Ogni membro del gruppo ha deciso di dedicare un certo monte ore allo sviluppo del 
					progetto didattico. Questo monte ore, purtroppo, potrebbe non essere mantenuto da 
					ciascuno dei membri del gruppo, in quanto possono capitare imprevisti, o sviste.
				\subsubsection{Contromisure:}
					Il \emph{Responsabile di progetto} è tenuto ad avvisare qualsiasi membro del gruppo 
					nel caso in cui, all'avvicinarsi della terminazione di un compito a lui assegnato, 
					essi mancasse ancora di molte ore, superiori al carico giornaliero previsto. 
					
					Come già specificato, ogni membro del gruppo ha preso l'impegno di dedicare tempo 
					al progetto, e nel caso in cui qualcuno non rispetterà quanto detto, si presume 
					non sia una cosa voluta o pianificata.
			\subsection{Problemi tra componenti}
				\subsubsection{Probabilità:}
					Media.
				\subsubsection{Gravità:}
					Alta.
				\subsubsection{Descrizione:}
					SWEeneyThreads è un gruppo nato per questo progetto. Nessuno dei membri al suo interno 
					ha mai lavorato con tutti gli altri a qualche altro progetto. Inoltre, nessuno dei membri 
					del gruppo ha mai lavorato in un team così numeroso ad un progetto di questo livello. 
					Questo potrebbe portare a problemi di collaborazione, ad un carico eccessivo da parte di 
					alcuni, per sistemare una carenza da parte di altri; questo porterebbe ad avere un clima 
					poco profiquo durante lo svolgimento del progetto.
				\subsubsection{Contromisure:}
					È compito del \emph{Responsabile di progetto} monitorare la nascita di problematiche tra più 
					individui. Se questo si verificasse, è sempre compito del \emph{Responsabile di progetto} 
					cercare di organizzare il lavoro cercando di diminuire il più possibile la cooperazione 
					dei suddetti individui. 
					
					La differenza di opinioni in forte contrasto tra due individui verrà esposta al resto del 
					gruppo che deciderà, per maggioranza, la strada da intraprendere.
		\section{Livello organizzativo}
			\subsubsection{Probabilità:} 
				Media.
			\subsubsection{Gravità:}
				Alta.
			\subsubsection{Descrizione:}
				Durante la pianificazione di progetto, è possibile che la stima dei tempi, e quindi il
				preventivo, risulti errata. In particolare, una sottostima dei costi di produzione può 
				portare ad un ritardo nella consegna dei materiali previsti. 
			\subsubsection{Contromisure:}
				La caratteristiche del rischio rilevato implica il dovere, da parte di ogni membro del 
				gruppo, di controllare periodicamente lo stato dei tickets, in modo da rendersi conto 
				immediatamente di eventuali ritardi nello svolgimento di \emph{Task}. Particolare attenzione
				va posta alle attività contrassegnate come critiche. 
				
				Per le attività critiche si è deciso di inserire, già durante la loro pianificazione, delle 
				ore di slack, in modo che un eventuale ritardo non influenzi la durata totale del progetto. 
				Inoltre il preventivo fornito è maggiorato (se pur non di molto)rispetto a quello calcolato, 
				il che permette di avere delle ore bonus a disposizione in caso di ritardo.
		\section{Livello dei requisiti}
			\subsubsection{Probabilità:} 
				Media.
			\subsubsection{Gravità:}
				Media.
			\subsubsection{Descrizione:}
				Durante lo studio del capitolato e la stesura dei requisiti, è possibile che essi non vengano 
				capiti totalmente dagli analisti. È anche possibile che alcuni aspetti vengano studiati in modo 
				incompleto o peggio ancora in modo errato. Questo porterebbe a differenze tra le aspettative del 
				committente e la visione del prodotto del gruppo di lavoro.
			\subsubsection{Contromisure:}
				Per evitare che questo rischio si verifichi, durante le fasi di analisi si terranno più incontri 
				con il committente, in modo da chiarire incertezze su requisiti, o correggere errate interpretazioni 
				dei requisiti espressi. Inoltre, ogni documento verrà consegnato e valutato dal committente, ad ogni 
				revisione.
				
				Se si verificassero incongruenze tra le due visioni sul proddo, è importante che esse vengano comunicate 
				al gruppo dal committente al termine di ogni revisione, in modo che le analisi subiscano un miglioramento 
				incrementale permettendo di ottenerne di affidabili.
		\section{Livello di valutazione dei costi}
			\subsubsection{Probabilità:}
				Bassa.
			\subsubsection{Gravità:}
				Alta.
			\subsubsection{Descrizione:}
				Il costo per ora di ogni ruolo è stato definito a priori, non era compito del grupo. Spetta invece al 
				gruppo la stima delle ore di lavoro necessarie per svolgere il progetto.
			\subsubsection{Contromisure:}
				Il preventivo è maggiorato, e anche le ore. I prezzi orari per ogni ruolo non è di pertinenza dei membri 
				del gruppo.
	\mychapter{6}{Meccanismi di controllo e rendicontazione}
		\section{Controllo}
			\subsection{Meccanismi di controllo}
				All'interno dell'ambiente di lavoro sono stati predisposti meccanismi per:
				\begin{itemize}
					\item Controllare l'andamento delle attività ed eventuali ritardi;
					\item \dots
				\end{itemize}
			\subsection{Andamento delle ttività}
				Per monitorare i ritardi sulle attività e acquisire maggiore esperienza per stime future si adotta la 
				funzione timer di Teamwork. Ogni componente del gruppo è invitato a tenere attivo il timer durante 
				tutto lo svolgimento delle attività a lui assegnate. In questo modo si può avere una misurazione del 
				tempo effettivo impiegato da ogni membro per svolgere le attività, che può poi essere confrontata con 
				la stima fatta a priori.
				%image
				\begin{figure}[H]
					\centering
					\includegraphics[scale=0.4]{teamworkTimer}
					\caption{Timer da attivare durante il lavoro svolto.}
				\end{figure}
				Inoltre per ogni attività è predisposta anche una due to date, ovvero la data entro la quale la task deve 
				essere soddisfatta. Teamwork segnala ogni attività nel riepilogo non completata entro la data di fine con 
				una scritta rossa che riporta il ritardo. È facile per il responsabile individuare a colpo d'occhio le 
				task in ritardo e provvedere a comunicare con il/i compenti del gruppo a cui essa è assegnata per capire 
				le motivazioni del ritardo ed eventualmente rivedere le stime future.
				%image
				\begin{figure}[H]
					\centering
					\includegraphics[scale=0.4]{teamworkTaskinRitardo}
					\caption{Visualizzazione data di scadenza di un task.}
				\end{figure}
				Se necessario è possibile impostare notifiche automatiche in prossimità o al superare di una scadenza.
		\section{Rendicontazione}
			Il sistema di ticketing adottato, mette a disposizione la rendicontazione delle ore di lavoro svolto da 
			ciascun componente del gruppo. Inoltre offre la possibilità di esportare tale rendicontazione in formato 
			compatibile con projectLibre. Per automatizzare questo processo è stato reso disponibile, per uso interno, 
			a tutti i componenti del gruppo un servizio online (accessibile al link:
			\url{sweeneytreadaas.altervista.org/roleparser/parser.php} ) che permette di effettuare delle modifiche al 
			file esportato da Teamwork al fine di:
			\begin{itemize}
				\item  Riassegnare ogni attività non più alla singola persona (che ha ruolo, e quindi stipendio variabile) 
				ma al ruolo che quella stessa persona ricopriva in quel periodo;
				\item Assegnare, automaticamente, ad ogni ruolo lo stipendio indicato dalle regole di progetto.
			\end{itemize}
			Il nuovo file è poi importato all'interno di projectLibre dove è possibile accedere a molte funzionalità utili 
			alla rendicontazione, come:
			\begin{itemize}
				\item Tabelle d'uso delle risorse umane e non;
				\item RBS delle risorse;
				\item Report dettagliati di preventivo/consuntivo ad ogni milestone;
				\item Altri grafici personalizzabili a seconda delle esigenze del gruppo o delle richieste del committente 
				(ad esempio per singolo componente o singolo ruolo).
			\end{itemize}


	\mychapter{7}{Consuntivo a finire}
		Questa sezione, lasciata per ultima perchè incrementale, riporta il prospetto economico con i 
		costi effettivamente sostenuti. Per ogni fase verrà calcolato un conguaglio, offero la differenza
		tra ore preventivate e spese, esso portà essere:
		\begin{itemize}
			\item \textbf{Positivo:} se il preventivo ha superato il consuntivo;
			\item \textbf{In pari:} se il preventivo e il consuntivo coincidono;
			\item \textbf{Negativo:} se il consuntivo ha superato il preventivo.
		\end{itemize}
		\section{Analisi}
		%Da aggiungere man-mano
		%\section{Analisi di dettaglio}
		%\section{Progettazione architetturale}
		%\section{Progettazione di dettaglio}
		%\section{Verifica e validazione}


\end{document}
