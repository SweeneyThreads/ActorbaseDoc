%Document-Author: Nicoletti Luca + Maino Elia
%Document-Date: 2016/01/08
%Document-Description: Documento di norme di progetto, include tutte le norme del gruppo SWEeneyThreads


\documentclass[a4paper]{report}
\usepackage[english, italian]{babel}
\usepackage[T1]{fontenc}
\usepackage[utf8]{inputenc}
\usepackage{url}
\usepackage{graphicx}
\graphicspath{{Immagini/}}
\usepackage[hidelinks]{hyperref}
\usepackage{lipsum}
\usepackage{booktabs}
\usepackage{tabularx}
\usepackage{pifont}
\usepackage[table]{xcolor}

\newcommand{\mychapter}[2]{
	\setcounter{chapter}{#1}
	\setcounter{section}{0}
	\setcounter{subsection}{1}
	\chapter*{#2}
	\addcontentsline{toc}{chapter}{#2}
}

\begin{document}
	
	\begin{titlepage}
		% Defines a new command for the horizontal lines, change thickness here
		\newcommand{\HRule}{\rule{\linewidth}{0.5mm}} 
		\center  
		
		% HEADING SECTION
		\textsc{\LARGE SweeneyThreads}\\[1.5cm] 
		\textsc{\Large Actorbase}\\[0.5cm] 
		\textsc{\large a NoSQL DB based on the Actor model}\\[0.5cm]
		
		
		% TITLE SECTION
		\HRule \\[0.4cm]
		{ \huge \bfseries Norme di progetto}\\[0.4cm] 
		\HRule \\[1.5cm]
		
		% AUTHOR SECTION
		\begin{minipage}{0.4\textwidth}
			\begin{flushleft} \large
				\emph{Redattori:}\\
				Maino \textsc{Elia} \\
				Nicoletti \textsc{Luca} \\
			\end{flushleft}
		\end{minipage}
		~
		\begin{minipage}{0.4\textwidth}
			\begin{flushright} \large
				\emph{Approvazione:} \\
				\dots
			\end{flushright}
		\end{minipage}
		
		\begin{figure}[h!]
			\centering
			\includegraphics[scale=0.8]{sweeney.png}
		\end{figure}
		\begin{center}
			Versione 1.0.3
		\end{center}
		% Date, change the \today to a set date if you want to be precise
		{\large \today}\\[3cm] 
		% Fill the rest of the page with whitespace
		\vfill  
	\end{titlepage}
	
	
	\tableofcontents
	
	\mychapter{1}{Introduzione}
	\section{Scopo del documento}
	Lo scopo del seguente documento è quello di definire le norme che l'intero gruppo SWEeneyThreads si impegna a rispettare durante lo svolgimento 
	del progetto ActorBase. 
	
	Ogni membro è tenuto a leggere il documento e a rispettare le norme al fine di dare maggiore uniformità allo svolgimento dei processi, 
	migliorandone l'efficacia, riducendo il numero di errori e i tempi di sviluppo.
	
	Poichè il gruppo ha deciso di basarsi sulla struttura a processi \emph{ISO/IEC 12207}\ped{\textit{G}} la struttura di questo documento ne rispecchia 
	l'organizzazione. In particolare la suddivisione in processi primari, di supporto e organizzativi.
	\begin{figure}[h!]
		\centering
		\includegraphics[scale=0.4]{Immagini/"processi 12207".png}
		\caption{Processi ISO/IEC 12207}
	\end{figure}
	%immagine
	\section{Scopo del prodotto}
	Lo scopo del progetto è la realizzazione di un DataBase NoSQL key-value basato sul modello ad Attori\ped{\textit{G}} con l'obiettivo di fornire una 
	tecnologia adatta allo sviluppo di moderne applicazioni che richiedono brevissimi tempi di risposta e che elaborano enormi quantità 
	di dati. Lo sviluppo porterà al rilascio del software sotto licenza MIT.
	\section{Glossario}
	Con lo scopo di evitare ambiguità di linguaggio e di massimizzare la comprensione dei documenti, il gruppo ha steso un documento 
	interno che è il \emph{Glossario v1.0.0}. Ogni termine contenuto nel \emph{Glossario} e presente in questo documento verrà marcato 
	con una "\textit{G}" maiuscola in pedice.
	\section{Riferimenti}
	\subsection{Informativi}
	\begin{itemize}
		\item Specifiche UTF-8\ped{\textit{G}}: \\ \url{http://unicode.org/faq/utf_bom.html}
		\item Licenza MIT: \\ \url{https://opensource.org/licenses/MIT}
		\item Scala Programming Language: \\ \url{http://www.scala-lang.org/}
		\item ISO/IEC 12207\ped{\textit{G}}: \\ \url{http://www.iso.org/iso/catalogue_detail?csnumber=43447}
		\item ISO 8601:2004\ped{\textit{G}}: \\ \url{http://www.iso.org/iso/home/standards/iso8601.htm}
		\item \LaTeX: \\ \url{https://www.latex-project.org}
		\item UML: \\ \url{http://www.uml.org}
		\item StarUML: \\ \url{http://staruml.io}
		\item Telegram: \\ \url{https://telegram.org}
		\item Google Drive: \\ \url{https://www.google.com/intl/it_it/drive/}
		\item Google Hangouts: \\ \url{https://hangouts.google.com}
		\item Piano di progetto: \\ \emph{Piano di Progetto v1.0.0}
		\item Piano di qualifica: \\ \emph{Piano di Qualifica v1.0.0}
	\end{itemize}
	\subsection{Normativi}
		\begin{itemize}
			\item Capitolato d'appalto Actorbase (C1): \\ 
			\url{http://www.math.unipd.it/~tullio/IS-1/2015/Progetto/C1p.pdf}
		\end{itemize}
	\section{Diario modifiche}
	Di seguito viene riportato il diario delle modifiche, in cui vengono elencate le modifiche apportate al documento stesso e da utilizzare 
	come template in tutti gli altri documenti.
	\begin{table}[!h]
		\begin{tabularx}{\textwidth}{*4{>{\centering\arraybackslash}X}}
			\noalign{\hrule height 1.5pt}
			\rowcolor{orange!85} Versione & Data & Autore & Descrizione \\
			\noalign{\hrule height 0.5pt}
			1.0.0 & 2016-01-08 & Maino Elia Nicoletti Luca & Creazione scheletro documento e struttura organizzativa,
			 stesura capitolo 1 \\
			\noalign{\hrule height 0.5pt}
			1.0.1 & 2016-01-08 & Nicoletti Luca & Stesura capitolo 2 \\
			\noalign{\hrule height 0.5pt}
			1.0.2 & 2016-01-08 & Maino Elia & Stesura sezione 3.1 \\
			\noalign{\hrule height 0.5pt}
			1.0.3 & 2016-01-09 & Nicoletti Luca & Stesura capitolo 4 \\
			\noalign{\hrule height 0.5pt}
			1.0.3 & 2016-01-09 & Maino Elia & Completamento stesura capitolo 3 ed integrazione capitoli 2 e 4 \\
			\noalign{\hrule height 0.5pt}
			1.0.4 & 2016-01-09 & Nicoletti Luca & Completata stesura documento, pronto per essere verificato. \\
			\noalign{\hrule height 1.5pt}
		\end{tabularx}
		\caption{Changelog del file \label{tab:table_label}}
	\end{table}
	\mychapter{2}{Processi primari}
	Sono state definite delle norme relative ai processi primari che maggiormente riguardano le attività svolte 
	dal gruppo: fornitura e sviluppo.
	\section{Fornitura}
	\subsection{Studio di fattibilità}
	I responsabili dello studio di fattibilità del progetto sono i membri che ricoprono il ruolo di \emph{Analisti}. In base alle prime
	riunioni effettuate, decise dal \emph{Responsabile di progetto}, le preferenze e le idee emerse per ogni singolo 
	membro del gruppo, essi dovranno stendere il documento (che verrà poi analizzato e valutato da altri membri del gruppo). Lo studio 
	di fattibilità deve contenere:
	\begin{itemize}
		\item \textbf{Dominio:} conoscenza delle tecnologie richieste e del dominio applicativo;
		\item \textbf{Rapporto costo/benefici:} eventuali prodotti simili già presenti sul mercato, competitori, costo della 
		realizzazione del prodotto e quantità di requisiti obbligatori;
		\item \textbf{Individuazione dei rischi:} evidenziare lacune tecniche e di conoscenza del dominio dei membri del gruppo, comprensione
		dei punti critici, difficoltà nel determinare i requisiti obbligatori e opzionali e nella loro classificazione; 
	\end{itemize}				 
	\subsection{Tecniche di analisi e classificazione requisiti}
	Sempre compito degli analisti, sarà quello di stilare l'analisi dei requisiti. Essi potranno ricavarli da eventuali \emph{Casi 
		d'uso} emersi da Brainstorming\ped{\textit{G}} o riunioni con il committente.
	
	I requisiti saranno elencati secondo un ordine. Ogni requisito seguirà la seguente codifica: \\
	\begin{center}
		R[Codice][Importanza][Tipo]
	\end{center}
	\textbf{Codice} \\ \\ Un codice univoco ed espresso in modo gerarchico;\\ \\
	\textbf{Importanza} \\ \\Può assumere i seguenti valori:
	\begin{itemize}
		\item \textbf{N:} Necessary (obbligatorio);
		\item \textbf{D:} Desiderable (desiderabile, a valore aggiunto);
		\item \textbf{O:} Optional (opzionale).
	\end{itemize}
	\textbf{Tipo} \\ \\Può assumere i seguenti valori:
	\begin{itemize}
		\item \textbf{F:} funzionale;
		\item \textbf{Q:} di qualità;
		\item \textbf{P:} prestazionale;
		\item \textbf{V:} vincolo.
	\end{itemize}
	\subsection{Casi d'uso}
		Come detto in precedenza alcuni requisiti possono essere ricavati dai \emph{Casi d'uso}, ad essi si può fare
		 riferimento anche con la dicitura \emph{use case} o con l'acronimo \emph{UC} nel caso fosse necessario  
		 utilizzarli in tabelle o diagrammi. 
		 
		 I \emph{Casi d'uso} vanno identificati dagli \emph{Analisti}, attraverso una procedura che va dal generale al
		  particolare. 
		  \\ \\
		 Un \emph{caso d'uso} richiede la definizione dei seguenti campi:
		 \begin{itemize}
		 	\item Codice gerarchico 
		 	\item Nome sintetico
		 	\item Attori
		 	\begin{itemize}
		 		\item Principali
		 		\item Secondari
		 	\end{itemize}
		 	\item Precondizione
		 	\item Postcondizione
		 	\item Flusso degli eventi relativi allo scenario principale
		 	\item Eventuali scenari alternativi
		 	\item Lista di requisiti dedotti 
		 \end{itemize}
		 A tale insieme di informazioni va associato un diagramma di caso d'uso in UML. Per la scrittura dei
		  \emph{casi d'uso} in UML il gruppo ha deciso di utilizzare il software StarUML alla versione più recente. La scelta è 
		 caduta su tale programma poichè è disponibile per tutti i principali sistemi operativi desktop e permette di
		  utilizzare una versione di prova completa per tutto il tempo del progetto.
		  \\ \\
		  \'E degli \emph{Analisti} il compito di inserire tutte le informazioni relative ai casi d'uso e i loro diagrammi nel 
		  documento di \emph{Analisi dei requisiti}.
		  \\ \\
		 Il codice gerarchico del \emph{caso d'uso} ha la forma seguente:
		 \begin{center}
		 	UC[codice univoco del padre].[codice progressivo]
		 \end{center}
		 il codice progressivo può definire diversi livelli di gerarchia separati da un punto.
	\subsection{Tecniche di tracciamento dei requisiti}
	Tutti i requisiti e casi d'uso saranno inseriti in un Database, strutturato in modo funzionale dal gruppo. Nel presente Database verranno
	inserite anche le \emph{Milestones} definite dal responsabile di turno, in modo da poter collegare requisiti e milestones. 
	Verranno creati degli appositi triggers per automatizzare la validazione dei requisiti tramite completamento manuale delle
	milestones presenti. 
	\\ \\ 
	Il Database verrà scritto utilizzando il linguaggio SQL e l'ambiente mySQL disponibile gratuitamente.
	\\ \\
	Questo compito è competenza dell'\emph{Amministratore}.
	\subsection{Gestione cambiamento requisiti}
	Per quanto riguarda il cambiamento dei requisiti, nel Database verrà tenuta una tabella di backup dei "vecchi" requisiti, 
	nella loro forma di dichiarazione, con un puntatore al "nuovo" requisito, che invece avrà le specifiche aggiornate. È compito del
	\emph{Responsabile di progetto} mantenere aggiornata la tabella dei requisiti, copiando prima il requisito che necessita di cambiamento
	nella tabella dei vecchi requisiti, e poi aggiornando il requisito stesso.
	
	Tutto questo al fine unico di avere una tracciabilità del cambiamento dei requisiti visionabile a posteriori.
	\section{Sviluppo}
	\subsection{Codifica e convenzioni}
	Di seguito sono riportate le norme che il gruppo andrà a seguire durante la stesura di codice, qualsiasi esso sia.
	\\ \\
	\textbf{\LaTeX} \\ \\ 
	Regole riguardanti \LaTeX :
	\begin{itemize}
		\item Ogni file deve iniziare con 3 righe di commento come quelle riportate in seguito:
		%immagine
		\begin{figure}[h!]
			\centering
			\includegraphics[scale=0.4]{Immagini/"3commentsline".png}
			\caption{Commenti ad inizio file}
		\end{figure}
		\item Ogni file deve contenere nella prima parte tutti gli \verb|\usepackage{}| necessari
		\item I commenti andranno inseriti in una riga vuota, eventualmente prima della riga di codice a cui fanno riferimento
		\item I commenti su più righe useranno il comando \verb|\begin{comment} - \end{comment}|
		\item Tra ogni \verb|\begin{PART}| e \verb|\end{PART}| tutto il testo e il codice andrà indentato:
		%immagine
		\begin{figure}[h!]
			\centering
			\includegraphics[scale=0.4]{Immagini/"indent".png}
			\caption{Indentazione 1}
		\end{figure}
		\item Per quanto riguarda il comando personalizzato \verb|\mychapter{}{}| o altre sezione \verb|\section{}|, \verb|\subsection{}| 
		verranno comunque indentate le parti innestate al loro interno come segue:
		%immagine
		\begin{figure}[h!]
			\centering
			\includegraphics[scale=0.4]{Immagini/"indent2".png}
			\caption{Indentazione 2}
		\end{figure}
		\item Verrà utilizzato T1 come encoding del font: \verb|\usepackage[T1]{fontenc}|
		\item Verrà utilizzato utf8 come encoding dell'input: \verb|\usepackage[utf8]{inputenc}|
		\item Verrà utilizzato english, italian come parametro per babel: \verb|\usepackage[english, italian]{babel}| in modo da 
		usare inglese e italiano nello stesso documento tenendo italiano come lingua principale
		\item Prima di ogni immagine, verrà inserito un commento su una riga, come definito sopra, per facilitarne l'individuazione:
		%immagine
		\begin{figure}[h!]
			\centering
			\includegraphics[scale=0.4]{Immagini/"immagini".png}
			\caption{Commento prima di ogni immagine}
		\end{figure}
		\item A fine documento, come commento su più righe, andrà inserita la documentazione e la descrizione (anche breve) del file
	\end{itemize}
	\textbf{\emph{Scala}} \\ \\ 
	Tutte le regole di indentazione, assegnazione dei nomi, scrittura delle parentesi, nominazione file, e documentazione sono quelle
	definite dalla documentazione ufficiale di \emph{Scala}: \url{http://docs.scala-lang.org/style/}
	\mychapter{3}{Processi di supporto}
	\section{Documentazione} %
	In questo capitolo si descrivono le convenzioni definite e adottate dal gruppo riguardanti le 
	modalità di redazione, verifica e approvazione dei documenti.
	
	Tutti i documenti ufficiali prodotti da SWEeneyThreads sono scritti utilizzando il linguaggio \LaTeX, compilati e
	forniti in formato PDF (per quanto riguarda le versioni digitali).
	\subsection{Template}
	Al fine di rendere più rapida e meno incline a differenziazioni la stesura dei diversi documenti è stato prodotto un
	template \LaTeX, reperibile nel repository in \verb|Actorbase/LaTeX/Templates| %non sono sicuro del path
	\subsection{Struttura documenti}
	La struttura dei documenti presenta una suddivisione in capitoli, sezioni e sottosezioni. 
	
	Per quanto riguarda i capitoli, è stato definito un comando personalizzato \LaTeX \space denominato
	\verb|\mychapter{}{}|:
	\begin{verbatim}
		\newcommand{\mychapter}[2]{
			\setcounter{chapter}{#1}
			\setcounter{section}{0}
			\setcounter{subsection}{1}
			\chapter*{#2}
			\addcontentsline{toc}{chapter}{#2}
		}
	\end{verbatim}
	Per quanto riguarda sezioni e sottosezioni sono stati utilizzati i comandi standard \LaTeX \space \verb|\section{}| e
	\verb|\subsection{}| 

	La numerazione delle sezioni è utilizzata fino al terzo livello di profondità (x.y.z), dal quarto livello in poi le sottosezioni non 
	presentano numerazione. Tale scelta è stata presa al fine di rendere più leggibile l'indice.
	\\ \\
	Di seguito viene fornita una descrizione più dettagliata di alcuni elementi di un documento: \\ \\
	\textbf{Prima pagina} \\ \\ 
	La prima pagina di un documento presenta gli elementi seguenti:
	\begin{itemize}
		\item Nome del gruppo
		\item Nome del progetto
		\item Sottotitolo del progetto
		\item Titolo del documento
		\item Cognome e nome dei redattori del documento
		\item Cognome e nome dei verificatori del documento
		\item Cognome e nome di chi approva il progetto in qualità di responsabile
		\item Logo del gruppo
		\item Numero di versione del documento
		\item Data di rilascio del documento
	\end{itemize}
	La prima pagina è parte del template disponibile nel repository. \\ \\
	\textbf{Indice} \\ \\
	In ogni documento sono presenti in ordine
	\begin{itemize}
		\item Un indice delle sezioni
		\item Un indice delle tabelle
		\item Un indice delle figure
	\end{itemize}
	Tali indici sono generati automaticamente tramite appositi comandi \LaTeX, l'assenza di figure 
	e/o tabelle nel documento comporta l'omissione del corrispondente indice. 
	
	Data la natura secondaria degli indici relativi alle tabelle e alle figure, si è deciso di posizionarli alla fine del documento.
	L'indice dei contenuti si trova invece subito dopo la pagina iniziale.	
	\\ \\
	\textbf{Formattazione generale delle pagine} \\ \\
	La formattazione generale di una pagina non prevede particolari personalizzazioni, si basa 
	sulla formattazione standard di \LaTeX \space usata per documenti di classe "Report". 
	\subsection{Norme tipografiche}
	Questa sezione contiene norme tipografiche e ortografiche adottate dal gruppo al fine di garantire uno stile
	uniforme e una semantica coerente per tutti i documenti.  \\ \\
	\textbf{Stile del testo} 
	\begin{itemize}
		\item \textbf{Corsivo:} il corsivo va utilizzato nei casi seguenti:
		\begin{itemize}
			\item Citazioni
			\item Nomi particolari
			\item Documenti
			\item Riferimenti
		\end{itemize}
		A seconda della semantica del testo si utilizzano i comandi \LaTeX \space \verb|\emph{}| e \verb|\textit{}|
		\item \textbf{Grassetto:} il grassetto va utilizzato nei casi seguenti:
		\begin{itemize}
			\item Elenchi puntati: evidenzia il concetto sviluppato nella continuazione del punto
			\item Titoli di sottosezioni non numerate
		\end{itemize}
		\item \textbf{Maiuscolo:} una parola completamente in maiuscolo deve indicare un acronimo o una sigla.
		\item \textbf{\LaTeX:} ogni riferimento al linguaggio \LaTeX \space va scritto utilizzando il comando
		\verb|\LaTeX|
	\end{itemize}
	\textbf{Formati} 
	\begin{itemize}
		\item \textbf{Percorsi:} 
		\begin{itemize}
			\item Indirizzi email : comando \LaTeX \space \verb|\href{mailto:nome@dominio}{nome@dominio}|
			\item Indirizzi web completi: comando \LaTeX \space \verb|\url|
			\item Indirizzi relativi: comando \LaTeX  \space \verb|\verb|
		\end{itemize}
		\item \textbf{Date:} le date presenti nei documenti seguono lo standard ISO 8601:2004\ped{\textit{G}}:
		\begin{center}
			AAAA - MM - GG
		\end{center}
		Dove:
		\begin{itemize}
			\item AAAA rappresenta l'anno 
			\item MM rappresenta il mese
			\item GG rappresenta il giorno
		\end{itemize}
		\item \textbf{Ruoli di progetto:} quando si fa riferimento ad un ruolo di progetto questo va scritto in corsivo
		e con la prima lettera maiuscola (es. \textit{Responsabile})
		\item \textbf{Documenti:} i riferimenti vanno scritti in corsivo (es. \textit{Analisi dei requisiti})
		\item \textbf{Nomi dei file:} i nomi dei file vanno scritti utilizzando il comando \LaTeX \space 
		\verb|\verb| (es. \verb|immagine.png|)
		\item \textbf{Nomi propri:} I nomi propri seguono la forma "Cognome Nome"
		\item \textbf{Nome del gruppo:} il nome del gruppo è SWEeneyThreads, la distinzione tra lettere maiuscole e
		minuscole va rispettata ogni volta che vi si fa riferimento
	\end{itemize}
	\textbf{Sigle} \\ \\
	L'utilizzo di sigle e abbreviazioni per riferirsi a documenti va limitato il più possibile, tuttavia nel caso il loro uso
	fosse funzionale alla lettura (come nel caso di tabelle o diagrammi) il loro uso è consentito:
	\begin{itemize}
		\item \textbf{SdF:} Studio di Fattibilità
		\item \textbf{AdR:} Analisi dei Requisiti
		\item \textbf{GL:} Glossario
		\item \textbf{NdP:} Norme di Progetto
		\item \textbf{PdQ:} Piano di Qualifica
		\item \textbf{PdP:} Piano di Progetto
		\item \textbf{ST:} Specifica Tecnica
		\item \textbf{RR:} Revisione dei Requisiti
		\item \textbf{RP:} Revisione di Progettazione
		\item \textbf{RQ:} Revisione di Qualifica
		\item \textbf{RA:} Revisione di Accettazione
	\end{itemize}
	\subsection{Componenti grafiche}
	Le componenti grafiche previste all'interno dei documenti sono immagini e tabelle. Ogni occorrenza di un
	elemento grafico è accompagnata da una didascalia indicizzata, in modo da poterla associare alla sezione 
	relativa del documento. \\ \\
	\textbf{Tabelle} \\ \\ 
	Le tabelle sono definite utilizzando un template in \LaTeX \space realizzato dal gruppo e disponibile nel 
	repository all'indirizzo \verb|Actorbase/LaTeX/Templates| \\ \\ 
	\textbf{Immagini}  \\ \\
	Il formato scelto per le immagini è Portable Network Graphics (PNG). \\
	Le immagini vanno sempre inserite utilizzando la seguente sequenza di comandi \LaTeX:
	\begin{verbatim}
		\begin{figure}[h!]
			\includegraphics[scale=0-1]{Immagini/nome.png}
			\caption{Titolo - didascalia}
		\end{figure}
	\end{verbatim}
	\subsection{Classificazione documenti}
	I documenti prodotti dal gruppo si dividono in formali e informali. \\ \\
	\textbf{Documenti Formali} \\ \\
	Quando un documento riceve l'approvazione del \emph{Responsabile} viene definito formale e risulta idoneo
	al rilascio all'esterno del gruppo. \\
	Per risultare approvato un documento deve aver completato con successo il percorso di verifica e validazione 
	descritto nel \emph{Piano di Qualifica}. \\ \\
	\textbf{Documenti informali} \\ \\
	Un documento rimane informale finché non viene approvato dal \emph{Responsabile}, durante tale fase 
	il suo uso è da considerarsi esclusivamente interno al gruppo. \\
	Alcuni documenti prodotti dal gruppo possono rimanere informali per l'intera durata del loro ciclo di vita.
	\subsection{Versionamento documenti}
	I documenti prodotti dal gruppo devono essere sempre identificati da un numero di versione del tipo:
	\begin{center}
		X.Y.Z
	\end{center}
	Dove:
	\begin{itemize}
		\item X: è il numero principale di versione, viene incrementato ad ogni uscita formale del documento
		\item Y: viene incrementato quando si apportano modifiche sostanziali al documento
		\item Z: viene incrementato quando si apportano modifiche minori al documento
	\end{itemize}
	All'interno di un documento quando si intende fare riferimento ad una specifica versione di un altro documento la
	notazione da utilizzare è: 
	\begin{center}
		\emph{Nome Documento vX.Y.Z}.
	\end{center}
	Mentre per fare riferimento ad un file vero e proprio:
	\begin{center}
		\verb|NomeDocumento_vX.Y.Z.estensione|
	\end{center}
	\subsection{Ciclo di vita dei documenti}
	Ogni documento prodotto dal gruppo rispetta il seguente ciclo di vita:
	\begin{itemize}
		\item \textbf{Lavorazione/Modifica:} il documento entra in questa fase al momento della sua creazione e vi
		rimane per tutto il tempo in cui il suo contenuto viene modificato.
		\item \textbf{Verifica:} quando termina la fase di modifica, il documento passa nelle mani dei
		\emph{Verificatori}
		che lo analizzano al fine di individuare eventuali errori o incongruenze sintattiche e semantiche.
		\item \textbf{Approvazione:} dopo essere stato verificato il documento deve essere approvato dal
		\emph{Responsabile}. Se il documento ottiene l'approvazione diventa ufficiale e raggiunge lo stato finale del
		suo ciclo di vita per quanto riguarda la corrente versione.
	\end{itemize}
	Ogni documento prodotto può attraversare più volte ogni fase del suo ciclo di vita, allo stesso modo può non
	attraversarle tutte. Quando si inizia una revisione formale su un documento già approvato questo ricomincia il
	ciclo da capo con un numero di versione incrementato.
	%\section{Accertamento qualità} %da valutare
	%almeno 2 revisionatori per documento e/o file
	%\section{Qualifica} 
	%dopo
	\section{Risoluzione di problemi} 
	Per quanto riguarda l'individuazione, il tracciamento e la risoluzione di bug e problemi il gruppo ha deciso di 
	affidarsi ad il servizio di issue tracking \emph{YouTrack} offerto da \emph{JetBrains}. \\ 
	Per i progetti fino a dieci utenti, il servizio offre 10GB di spazio sul cloud dell'azienda in maniera gratuita. 
	
	La dashboard \emph{YouTrack} per quanto riguarda il progetto Actorbase è reperibile all'indirizzo
	 \url{https://actorbase.myjetbrains.com/youtrack/dashboard}
	\begin{figure}[h!]
		\centering
		\includegraphics[scale=0.4]{Immagini/"youtrack".png}
		\caption{YouTrack logo}
	\end{figure}
	\mychapter{4}{Processi organizzativi}
	\section{Processi di gestione dell'infrastruttura}
	\subsection{Ticketing}	
	Per quanto riguarda l'emissione e la gestione dei ticket si è scelto di affidarsi alla piattaforma \emph{Teamwork}
	in quanto:
	\begin{itemize}
		\item Ha ottenuto buoni punteggi da reviews di utenti e di critica
		\item Fornisce 100Mb di storage e la possibilità di avere due progetti attivi, contemporaneamente
		\item Fornisce un analizzatore di rischi e benefici
		\item Genera automaticamente diagrammi di Gantt interattivi
		\item Include un ottimo Task management (priorità, task history, possibilità di aggiungere in automatico task ricorrenti)
		\item Notifiche sms e \emph{Notification group}
	\end{itemize}
	
	
	La principale alternativa presa in considerazione è stata \emph{Zoho}, ma non è stata ritenuta all'altezza in quanto offre 
	meno features. Segue una breve lista per mettere a confronto le principali funzionalità messe a disposizione dalle due piattaforme: \\
	
	\begin{table}[!h]
		\begin{tabularx}{\textwidth}{*2{>{\centering\arraybackslash}X}}
			\noalign{\hrule height 1.5pt}
			\rowcolor{orange!85} ZOHO & TEAMWORK \\
			\noalign{\hrule height 0.5pt}
			Calendar & Calendar \\
			Gant & Gantt interattivi \\
			Task management & To-do list\\
			Time tracking & Track Project Hours\\
			Bug tracking & Analizzatore rischi/benefici  \\
			Document management & Template di progetto \\
			& Priorities \\
			& Track Burn Rate \\
			& Track Staff Hours \\
			& SMS di notifica \\
			\noalign{\hrule height 1.5pt}
		\end{tabularx}
		\caption{Lista features ZOHO - TEAMWORK\label{tab:table_label}}
	\end{table}
	
	\begin{figure}[!h]
		\centering
		\includegraphics[scale=0.4]{Immagini/"zohovstw".png}
		\caption{Rating delle features a confronto}
	\end{figure}
	
	Secondo \emph{SoftwareInsider} (\url{softwareinsider.com}) sono molto simili nelle funzionalità principali;
	ma \emph{Teamwork} offre alcuni strumenti in più per la gestione di processi software tradizionali.
	
	I principali:\\
	\begin{table}[!h]
		\begin{tabularx}{\textwidth}{*3{>{\centering\arraybackslash}X}}
			\noalign{\hrule height 1.5pt}
			\rowcolor{orange!85} & ZOHO & TEAMWORK \\
			\noalign{\hrule height 0.5pt}
			Calendar & \ding{51} & \ding{51} \\
			Gantt interattivi & \ding{51} & \ding{51} \\
			Template di progetto & \ding{51} & \ding{51} \\
			Risk/benefits analyzer & \ding{53} & \ding{51} \\
			Scheduling & \ding{53} & \ding{51} \\
			\noalign{\hrule height 1.5pt}
		\end{tabularx}
		\caption{Differenza strumenti ZOHO - TEAMWORK \label{tab:table_label}}
	\end{table}
	
	Come task management \emph{Zoho} offre solamente delle To-do List, mentre \emph{Teamwork} ha anche le seguenti
	feature:
	\begin{itemize}
		\item Add Recurring Tasks
		\item Group Tasks by Projects
		\item Set Priorities
		\item Task History
	\end{itemize}
	\emph{Zoho} offre alcune funzionalità in più in quanto a comunicazione real-time tra membri del gruppo,
	ma questo risulta irrilevante per il nostro gruppo, in quanto per la comunicazione real-time viene
	adottato un sistema diverso.
	\subsection{Versioning}
	Per gestire il versionemento il gruppo utilizza \emph{GitHub}. Tale scelta è dovuta sia ad un apprezzamento
	comune da parte dei membri del gruppo per la piattaforma, che ad una richiesta esplicita di pubblicazione del 
	progetto sulla stessa da parte del committente. \\
	\'E stato creato un account ufficiale del gruppo, raggiungibile all'indirizzo 
	\url{https://github.com/SweeneyThreads}
	\begin{figure}[h!]
		\centering
		\includegraphics[scale=0.25]{Immagini/"sweeneygithub".png}
		\caption{Account GitHub SWEeneyThreads}
	\end{figure}
	\subsection{Repository}
	%come gestiamo GitHub (immgine albero cartelle)
	Sono state previste diverse \emph{Repository} necessarie allo sviluppo del progetto: Actorbase, RR, RP, RQ, RA. 
	Actorbase conterrà tutti i file del prodotto da sviluppare, mentre le \emph{Repository} RR, RP, RQ e RA si riferiscono 
	alle 4 consegne del progetto previste: revisione dei requisiti, revisione di progettazione, revisione di qualifica, 
	revisione di accettazione. Effettuata la consegna del materiale, la \emph{Repository} \verb|Actorbase/| verrà copiata 
	in quella corrispondente, che servirà quindi come backup della \emph{Baseline} a cui fa riferimento.
	
	In \verb|Actorbase/| saranno presenti le seguenti sottocartelle:
	\begin{itemize}
		\item Documenti
		\item LaTeX
		\item Progetto
	\end{itemize}
	\textbf{Documenti} \\ \\
	Nella cartella \verb|Actorbase/Documenti/| verranno inseriti tutti i pdf generati dal comando \verb|pdflatex nome-documento.tex|. Non 
	saranno presenti altri file in questa cartella. I documenti saranno divisi in \emph{Interni} ed \emph{Esterni} e per questo saranno create
	delle sottodirectory: \verb|Actorbase/Documenti/Interni| e\\  \verb|Actorbase/Documenti/Esterni|. \\ \\
	\textbf{LaTeX} \\ \\
	Nella cartella \verb|Actorbase/LaTeX/| saranno presenti tutti i file \verb|*.tex| pronti per la compilazione. In questa cartella verrà inserita
	anche una cartella\\ \verb|Actorbase/LaTeX/Immagini/| contenente tutte le immagini necessarie alla compilazione dei file.
	Inoltre verrà aggiunta una cartella \verb|Actorbase/LaTeX/Templates/| contenente i templates per la stesura di documenti e per il
	disegno appropriato di tabelle. \\ \\
	\textbf{Progetto} \\ \\
	La cartella \verb|Actorbase/Progetto/| contiene tutti i file che compongono il prodotto richiesto nel capitolato. Sarà quindi un progetto in
	\emph{Scala} che seguirà gli standard sopra citati.
	\section{Processi di management}
	\subsection{Ruoli}
	Per quanto riguarda i ruoli, il gruppo utilizzerà quelli definiti nelle slide 7-11 disponibili all'indirizzo: 
	\url{http://www.math.unipd.it/~tullio/IS-1/2015/Dispense/L04.pdf}. È stato deciso, unanimamente, che le rotazioni dei ruoli principali
	come \emph{Amministratore} e \emph{Responsabile di progetto} avverrano ogni 2 settimane. Come stabilito, una persona può ricoprire
	contemporaneamente più ruoli, la rotazione di altri ruoli come \emph{Analista}, \emph{Progettista} e \emph{Programmatore} potrà avvenire 
	meno frequentemente, in quanto potrebbe risultare dannoso dover abbandonare un'attività di analisi o di programmazione prima della sua 
	conclusione. 
	
	Il gruppo stabilisce la rotazione dei ruoli in base alle attività, ai task, e alla disponibilità fornita da ogni membro.
	%Definizione e rotazione
	
	È possibile ottenere in qualsiasi momento una panoramica dei ruoli assegnati tramite l'apposita sezione del 
	progetto creato su \emph{Teamwork}, all'indirizzo\\ \url{https://actorbase.teamwork.com/projects/188894/projectroles} 
	\subsection{Comunicazioni}
	\label{sec:Comunicazioni}
	Le comunicazioni possono avvenire tra membri del gruppo (interne), o tra il gruppo e terzi (esterne). Gli
	strumenti utilizzati differiscono a seconda della tipologia della comunicazione. \\ \\
	\textbf{Interne} 
	\begin{itemize}
		\item \textbf{Chat:} Per le comunicazioni interne il gruppo ha deciso di adottare una chat di messaggistica 
		istantanea: \emph{Telegram}. All'interno di questo mezzo di comunicazione verranno concordate date e orari 
		delle riunioni; comunicati eventuali ritardi ai meeting; proposte idee informali, che verranno poi riproposte 
		in modo ufficiale alle riunioni (questo per evitare di dimenticarsene o per lasciare tempo agli altri membri 
		del gruppo di ragionare più tempo su una proposta); inoltre \emph{Telegram} verrà utilizzato per l'invio di files
		temporanei, di documentazione o informativi. Sarà compito del \emph{Responsabile di progetto} prelevare file di documentazione 
		e riportarli nella repository adatta, e nel Drive del gruppo.
		
		La scelta di \emph{Telegram} è dovuta alla possibilità di utilizzare il servizio sia da desktop che da mobile, e alla 
		possibilità di inviare qualsiasi tipo di file.
		\item \textbf{Videoconferenze:} Per le videoconferenze di gruppo si utilizzerà \emph{Google Hangouts}. 
		
		\'E utilizzabile da tutti i dispositivi e richiede semplicemente un account \emph{Google} di cui disponevano 
		già tutti i membri del gruppo.
	\end{itemize}
	\textbf{Esterne}  \\ \\ 
	%Redirect mail al responsabile di turno 
	Per tutte le comunicazioni esterne va utilizzata la mail ufficiale del gruppo: \href{mailto:sweeneythreads@gmail.com}%
	{sweeneythreads@gmail.com}. \\ La gestione di 
	tale indirizzo email spetta al \emph{Responsabile} che dunque risulta essere l'unico componente del gruppo a poter comunicare
	con il committente i maniera ufficiale. Il \emph{Responsabile} ha il compito di informare gli altri membri del gruppo sulle 
	discussioni avute con il committente, tale aggiornamento può avvenire a voce durante le riunioni e gli incontri oppure tramite
	l'inoltro delle email ricevute agli indirizzi personali dei componenti interessati.
	\\ \\
	Le email ufficiali devono rispettare le seguenti linee guida:
	\begin{itemize}
		\item \textbf{Destinatario:} poichè questo indirizzo email va usato esclusivamente per comunicazioni ufficiali il destinatario
		del messaggio va salvato tra i contatti (funzione di Gmail), nel caso non dovesse già farne parte.
		\item \textbf{Oggetto:} l'oggetto deve esprimere in maniera chiara ed esaustiva il contenuto dell'email, deve essere breve
		 e non deve rendere l'email confondibile con le altre preesistenti. 
		 Nel caso il messaggio fosse una risposta l'oggetto deve essere preceduto dalla particella "Re:", nel caso di un inoltro dalla
		 particella "I:"
		\item \textbf{Corpo:} nel caso il messaggio fosse una risposta o un inoltro, il contenuto aggiunto va sempre scritto in testa al
		fine di non costringere i lettori a scorrere tutta l'email. La cancellazione della restante parte del messaggio è sconsigliata, per
		facilitare una visione completa della conversazione.
		\item \textbf{Allegati:} L'aggiunta di allegati al messaggio è consentita con l'unico vincolo di inviare file che possiedono un nome
		esplicativo o di specificare il contenuto dell'allegato nel corpo se il nome del file potrebbe essere poco comprensibile.
	\end{itemize}										
	\subsection{Riunioni}
	%incontri con il committente, e formali ogni 2 settimane
	\textbf{Ufficiali} \\ \\
		Le riunioni sono divise in due: con o senza presenza del committente. Il gruppo si impegna a tenere almeno una riunione
		ufficiale senza presenza del committente ogni due settimane. Le riunioni hanno una durata minima di due ore, che potrà 
		essere prolungata a piacere, in questo caso, nel verbale di riunione dovrà comparire di quanto si è superato il tempo
		previsto durante la riunione, e il motivo del prolungamento. Queste modifiche sono a carico del \emph{Responsabile di progetto}.
		
		Le riunioni con presenza del committente, andranno concordate secondo le norme di comunicazioni esterne con quest'ultimo
		e comunicate tramite i mezzi di comunicazione interni a tutti i membri del gruppo, ognuno dei quali è fortemente tenuto ad
		essere presente. Potranno verificarsi casi in cui non tutti i membri del gruppo potranno presentarsi alle riunioni con 
		presenza del committente, ma NON potrà verificarsi l'assenza del \emph{Responsabile di progetto} e dell'\emph{Amministratore}.
		I quali sono tenuti a riferire quanto emerso dalle riunioni a tutti i restanti membri assenti. \\ \\
	\textbf{Non ufficiali} \\ \\
		Le riunioni non ufficiali sono da considerarsi riunioni tra pochi membri del gruppo, ad esempio tra i due realizzatori di questo
		stesso documento, o incontri occasionali avvenuti senza comunicazioni nei canali ufficiali. Queste riunioni non necessitano di 
		una stesura di un verbale; se da queste riunioni emergesse un grave errore, o una comunicazione importante, i membri presenti
		sono tenuti a richiedere una riunioni ufficiale straordinaria, che dovrà essere approvata dall'\emph{Amministratore}. In caso 
		contrario, tutte le scelte non rilevanti non necessitano di approvazione. \\ \\
	\textbf{Brainstorming} \\ \\
		I \emph{Brainstorming}\ped{\emph{G}} vengono tenuti sotto richiesta di qualsiasi membro del gruppo, e approvati, se per motivazioni
		valide, dal \emph{Responsabile di progetto}. Un \emph{Brainstorming}\ped{\emph{G}} ha durata minima di un'ora e massima di due; durante il quale
		ogni membro ricopre un ruolo di egual importanza rispetto agli altri, le decisioni vengono prese all'unisono o con la maggioranza 
		dei membri a favore, non è compito del \emph{Responsabile di progetto} approvare le soluzioni emerse da un \emph{Brainstorming}\ped{\emph{G}}.
		
		Durante un \emph{Brainstorming}\ped{\emph{G}} ci sarà un membro con il compito di scrivere le \emph{Minute}, ovvero un \emph{Rapportator}. 
		Ad ogni \emph{Brainstorming}\ped{\emph{G}} sarà anche scelto un \emph{Facilitator} che ricoprirà un ruolo di servizio. Ovvero dovrà far 
		rispettare le regole di base. Una volta finito il \emph{Brainstorming}\ped{\emph{G}}, il \emph{Rapportator} consegnerà al \emph{Facilitator}
		gli appunti, che dovrà rielaborarli e stendere un verbale.
	
	\cleardoublepage
	\addcontentsline{toc}{chapter}{\listfigurename}
	\listoffigures
	
	\cleardoublepage
	\addcontentsline{toc}{chapter}{\listtablename}
	\listoftables
	
\end{document}
