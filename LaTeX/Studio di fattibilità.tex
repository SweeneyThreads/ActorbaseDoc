%Document-Author: Bonato Paolo + Bortolazzo Matteo + Maino Elia
%Document-Date: 2016-01-12
%Document-Description: Studio di fattibilità relativo al capitolato C1


\documentclass[a4paper]{report}
\usepackage[english, italian]{babel}
\usepackage[T1]{fontenc}
\usepackage[utf8]{inputenc}
\usepackage{url}
\usepackage{graphicx}
\graphicspath{{Immagini/}}
\usepackage[hidelinks]{hyperref}
\usepackage{lipsum}
\usepackage{booktabs}
\usepackage{tabularx}
\usepackage{pifont}
\usepackage[table]{xcolor}
\usepackage{float}
\usepackage{geometry}
\geometry{margin=1in}

\newcolumntype{s}{>{\hsize=.21\hsize}X}
\newcolumntype{f}{>{\hsize=.37\hsize}X}
\newcolumntype{m}{>{\hsize=.42\hsize}X}

\newcommand{\mychapter}[2]{
	\setcounter{chapter}{#1}
	\setcounter{section}{0}
	\setcounter{subsection}{1}
	\chapter*{#2}
	\addcontentsline{toc}{chapter}{#2}
}

\renewcommand{\abstractname}{Tabella contenuti}

\begin{document}
	
	\begin{titlepage}
		% Defines a new command for the horizontal lines, change thickness here
		\newcommand{\HRule}{\rule{\linewidth}{0.5mm}} 
		\center  
		
		% HEADING SECTION
		\textsc{\LARGE SweeneyThreads}\\[1.5cm] 
		\textsc{\Large Actorbase}\\[0.5cm] 
		\textsc{\large a NoSQL DB based on the Actor model}\\[0.5cm]
		
		
		% TITLE SECTION
		\HRule \\[0.4cm]
		{ \huge \bfseries Studio di fattibilità}\\[0.4cm] 
		\HRule \\[1.5cm]
		
		% AUTHOR SECTION
		\begin{minipage}{0.4\textwidth}
			\begin{flushleft} \large
				\emph{Redattori:}\\
				Bonato Paolo \\
				Bortolazzo Matteo \\
				Maino Elia \\
			\end{flushleft}
		\end{minipage}
		~
		\begin{minipage}{0.4\textwidth}
			\begin{flushright} \large
				\emph{Approvazione:} \\
				\dots \\
				\emph{Verifica:} \\
				\dots
			\end{flushright}
		\end{minipage}
		
		%immagine
		\begin{figure}[H]
			\centering
			\includegraphics[scale=0.8]{sweeney.png}
		\end{figure}
		\begin{center}
			Versione 1.0.2
		\end{center}
		% Date, change the \today to a set date if you want to be precise
		{\large \today}\\[3cm] 
		% Fill the rest of the page with whitespace
		\vfill  
	\end{titlepage}
	
	
	\tableofcontents
	
	\mychapter{0}{Diario delle modifiche}
		\begin{table}[H]
			\begin{tabularx}{\textwidth}{s f m X}
				\noalign{\hrule height 1.5pt}
				\rowcolor{orange!85} Versione & Data & Autore & Descrizione \\
				\noalign{\hrule height 1.5pt}
				1.0.2 & 2016-01-14 & \emph{Analista} Maino Elia & 
				Stesura considerazioni finali \\
				\noalign{\hrule height 0.5pt}
				1.0.1 & 2016-01-13 & \emph{Analisti} Bonato Paolo, Bortolazzo Matteo, Maino Elia & 
				Stesura sezioni riguardanti il dominio, costi e benefici \\
				\noalign{\hrule height 0.5pt}
				1.0.0 & 2016-01-12 & \emph{Analista} Maino Elia & Creazione scheletro del documento e stesura introduzione \\
				\noalign{\hrule height 1.5pt}
			\end{tabularx}
			\caption{Diario delle modifiche \label{tab:table_label}}
		\end{table}

	\mychapter{1}{Introduzione}
	\section{Scopo del documento}
		Lo scopo del seguente documento è descrivere le motivazioni che hanno portato alla scelta del 
		capitolato C1 da parte del gruppo SWEeneyThreads. 
	\section{Scopo del prodotto}
		Lo scopo del progetto è la realizzazione di un DataBase NoSQL key-value basato sul modello ad 
		Attori con l'obiettivo di fornire una tecnologia adatta allo sviluppo di moderne 
		applicazioni che richiedono brevissimi tempi di risposta e che elaborano enormi quantità 
		di dati. Lo sviluppo porterà al rilascio del software sotto licenza MIT.
	\section{Glossario}
		Con lo scopo di evitare ambiguità di linguaggio e di massimizzare la comprensione dei documenti, il 
      gruppo ha steso un documento interno che è il \emph{Glossario v1.0.3}. In esso saranno definiti, in modo
      chiaro e conciso i termini che possono causare ambiguità o incomprensione del testo.
	\section{Riferimenti}
	\subsection{Informativi}	
		\begin{itemize}
			\item \textbf{Capitolato d'appalto C1:} Actorbase: a NoSQL DB based on the Actor model \\
			\url{http://www.math.unipd.it/~tullio/IS-1/2015/Progetto/C1p.pdf}
			\item \textbf{Capitolato d'appalto C2:} CLIPS: Communication \& Localisation with Indoor
			 Positioning Systems \\
			\url{http://www.math.unipd.it/~tullio/IS-1/2015/Progetto/C2p.pdf}
			\item \textbf{Capitolato d'appalto C3:} UMAP: un motore per l'analisi predittiva in ambiente
			 Internet of Things \\
			\url{http://www.math.unipd.it/~tullio/IS-1/2015/Progetto/C3p.pdf}
			\item \textbf{Capitolato d'appalto C4:} MaaS: MongoDB as an admin Service \\
			\url{http://www.math.unipd.it/~tullio/IS-1/2015/Progetto/C4p.pdf}
			\item \textbf{Capitolato d'appalto C5:} Quizzipedia: software per la gestione di questionari \\
			\url{http://www.math.unipd.it/~tullio/IS-1/2015/Progetto/C5p.pdf}
			\item \textbf{Capitolato d'appalto C6:} SiVoDiM: Sintesi Vocale per Dispositivi Mobili \\
			\url{http://www.math.unipd.it/~tullio/IS-1/2015/Progetto/C6p.pdf}
		\end{itemize}
	\subsection{Normativi}
		\begin{itemize}
			\item \textbf{Norme di progetto:} \emph{Norme di progetto v1.1.1}
		\end{itemize}
	\mychapter{2}{Scelta del capitolato C1}
		\section{Descrizione del capitolato}
			Il capitalato C1, il cui committente e proponente è \emph{Cardin Riccardo}, riguarda lo sviluppo 
			di un database NoSQL di tipo key-value basato sul modello ad attori.
			\\ \\
			Le richieste principali esposte nel capitolato sono le seguenti:
			\begin{itemize}
				\item L'utilizzo della libreria Akka;
				\item L'utilizzo del linguaggio Scala o Java;
				\item Lo sviluppo dei tre attori principali: \textsc{Storekeeper},
				 \textsc{Storefinder} e \textsc{Warehouseman}.
			\end{itemize}
		\section{Dominio applicativo}
			Il capitolato si prefigge l’obiettivo di creare un database di tipo Key-Value NoSQL basato sul
			modello ad attori. \\
			Negli ultimi anni i database NoSQL stanno diventando sempre più popolari  rispetto ai tradizionali
			database relazionali (un esempio può essere MongoDB) per un insieme di caratteristiche tra cui: 
			\begin{itemize}
				\item Maggiore libertà progettuale;
				\item Scalabilità orizzontale;
				\item Supporto per strutture dati diverse.
			\end{itemize}
			Con l’avanzare del cloud c’è bisogno di sistemi flessibili e distribuiti che scalino facilmente in modo
			orizzontale. \\
			Il database si propone di essere reattivo, quindi orientato agli eventi, responsivo, resiliente e
			scalabile per far fronte ad enormi moli di dati con tempi di risposta molto brevi. \\
			L’intera applicazione verrà eseguita sulla Java Virtual Machine (JVM) in ambiente
			 desktop.
		\section{Dominio tecnologico}
			Visto il dominio applicativo dei database NoSQL, per la realizzazione del progetto al gruppo è
			necessaria una conoscenza approfondita dei seguenti campi:
			\begin{itemize}
				\item \textbf{Database NoSQL di tipo Key-Value:} conoscenza della struttura di questo 
				particolare tipo di database;
				\item \textbf{Modello ad attori:} conoscenza del modello matematico di 
				esecuzione concorrente su cui si basa l'implementazione del database;
				\item \textbf{Akka:} conoscenza approfondita di questa libreria che fornisce 
				 un'implementazione del modello ad attori su JVM;
				\item \textbf{Scala:} conoscenza di questo specifico linguaggio di programmazione;
				\item \textbf{JVM:} conoscenze di base del funzionamento della macchina virtuale di Java.
			\end{itemize}
		\section{Criticità potenziali e costi}
			Tutte le tecnologie richieste per la realizzazione del progetto sono gratuite quindi non è richiesto
			un'impegno monetario per utilizzarle, tuttavia essendo in gran parte nuove per i membri del gruppo 
			l'acquisizione delle competenze necessarie richiederà un investimento non banale in termini di
			 tempo.
			\\ \\
			Nello specifico l'uso delle seguenti tecnologie può essere fonte di criticità:
			\begin{itemize}
				\item \textbf{Database NoSQL Key-Value:} il gruppo ha buona padronanza con l'ambiente dei 
				database di tipo relazionale (SQL). Lo studio dei database NoSQL parte dunque 
				da una buona conoscenza di base per quanto riguarda l'ambito dei database in generale;
				\item \textbf{Scala:} il linguaggio consigliato per lo sviluppo del progetto risulta essere Scala, 
				nessun componente del gruppo possiede padronanza in tale linguaggio, essa andrà dunque 
				acquisita prima della fase di progettazione;
				\item \textbf{Modello ad attori e libreria Akka:} i componenti del gruppo conoscono
				 discretamente il modello ad attori il cui studio va quindi semplicemente approfondito. La 
				 libreria, invece, non è mai stata utilizzata da nessun componente, il suo utilizzo richiede dunque
				 uno studio completo che dovrà avvenire prima della fase di progettazione;
				 \item \textbf{JVM:} il gruppo possiede una buona conoscenza del linguaggio Java e una 
				 discreta conoscenza del suo ambiente di esecuzione JVM, tale conoscenza va quindi
				  approfondita.
			\end{itemize}
		\section{Analisi del mercato e benefici}
			Attualmente sul mercato esistono pochi database distribuiti open-source, ancor meno con una
			struttura Key-Value e nessuno che applichi il modello ad attori, il che rende Actorbase un prodotto
			potenzialmente unico. 
			\\ \\
			Inoltre citando il capitolato stesso:
			\begin{center}
				\emph{Le capacità di calcolo richieste dalle applicazioni attuali rendono obsoleto e inadeguato
				il modello di gestione dei dati relazionale. \dots \space Dal 2009, sempre più insistentemente si
				stanno facendo strada una serie di modelli non relazionali, definiti NoSQL.}
			\end{center}
			Il prodotto quindi non rappresenta solamente una novità nel mondo dei database, ma risponde 
			anche alle esigenze concrete del mercato attuale, basandosi sul modello NoSQL.\\
			Il rilascio su licenza MIT permetterà infine una potenziale rapida crescita del progetto grazie al
			possibile apporto della comunità.
		\section{Considerazioni e valutazioni finali}
			Conseguentemente alle considerazioni esposte nelle sezioni precedenti il gruppo ha definito un 
			insieme di aspetti positivi e negativi del capitolato:
			\subsection{Aspetti positivi}
				\begin{itemize}
					\item \textbf{Interesse:} i componenti del gruppo hanno manifestato un interesse elevato 
					nei confronti del dominio applicativo e delle tecnologie necessarie allo sviluppo;
					\item \textbf{Novità:} il prodotto rappresenta un'implementazione originale del modello 
					NoSQL Key-Value, non ancora disponibile nel mercato open source;
					\item \textbf{Esperienza:} lo sviluppo del prodotto permetterà ai membri del gruppo di 
					acquisire competenze utili nel proseguimento della carriera, come Scala e i database NoSQL;
					\item \textbf{Curiosità nei confronti del linguaggio di programmazione:} il progetto permette 
					di prendere confidenza con un linguaggio di programmazione moderno come Scala.
					\item \textbf{Licenza:} il rilascio del prodotto con licenza MIT fornisce interessanti
					 prospettive future di utilizzo e sviluppo;
					\item \textbf{Sviluppo futuro:} Il proponente si è dichiarato disponibile a proporre un piano
					successivo di sviluppo del prodotto nel caso in cui i risultati ottenuti e le prospettive lo
					 consentano.
				\end{itemize}
			\subsection{Aspetti negativi}
				Gli aspetti negativi individuati nel capitolato sono fondamentalmete collegati alle competenze 
				richieste, che al momento il gruppo non possiede e i cui tempi di acquisizione possono essere
				potenziale causa di ritardi sullo sviluppo.
	%\cleardoublepage
	%\addcontentsline{toc}{chapter}{\listfigurename}
	%\listoffigures
	
	\cleardoublepage
	\addcontentsline{toc}{chapter}{\listtablename}
	\listoftables
		
\end{document}
	