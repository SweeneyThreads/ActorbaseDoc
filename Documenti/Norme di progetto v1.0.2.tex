\documentclass[a4paper]{report}
\usepackage[english, italian]{babel}
\usepackage[T1]{fontenc}
\usepackage[utf8]{inputenc}
\usepackage{url}
\usepackage{graphicx}
\usepackage[hidelinks]{hyperref}
\usepackage{booktabs}
\usepackage{tabularx}
\usepackage[table]{xcolor}

\newcommand{\mychapter}[2]{
    \setcounter{chapter}{#1}
    \setcounter{section}{0}
    \setcounter{subsection}{1}
    \chapter*{#2}
    \addcontentsline{toc}{chapter}{#2}
}

\begin{document}


	\tableofcontents
	
	\mychapter{1}{Introduzione}
		\section{Scopo del documento}
			Lo scopo del seguente documento è quello di definire le norme che l'intero gruppo SWEeneyThreads si impegna a rispettare durante lo svolgimento 
			del progetto ActorBase. 
			
			Ogni membro è tenuto a leggere il documento e a rispettare le norme al fine di dare maggiore uniformità allo svolgimento dei processi, 
			migliorandone l'efficacia e riducendo il numero di errori.
			
			Poichè il gruppo ha deciso di basarsi sulla struttura a processi \emph{ISO/IEC 12207} la struttura di questo documento ne rispecchia 
			l'organizzazione. In particolare la suddivisione in processi primari, di supporto e organizzativi come in \emph{Figura 1.1}
			%immagine
			\begin{figure}[h!]
				\centering
				\includegraphics[scale=0.4]{Immagini/"processi 12207".png}
				\caption{Processi 12207}
			\end{figure}
		\section{Scopo del prodotto}
			Lo scopo del progetto è la realizzazione di un DataBase NoSQL key-value basato sul modello ad Attori con l'obiettivo di fornire una 
			tecnologia adatta allo sviluppo di moderne applicazioni che richiedono brevissimi tempi di risposta e che elaborano enormi quantità 
			di dati. Lo sviluppo porterà al rilascio del software sotto licenza MIT.
		\section{Glossario}
			Con lo scopo di evitare ambiguità di linguaggio e di massimizzare la comprensione dei documenti, il gruppo ha steso un documento 
			interno che è il \emph{Glossario v1.0.0}. Ogni termine contenuto nel \emph{Glossario} e presente in questo documento verrà marcato 
			con una "\textit{G}" maiuscola in pedice.
		\section{Riferimenti}
			\subsection{Informativi}
				\begin{itemize}
					\item Specifiche UTF-8\ped{\textit{G}}: \\ \url{http://unicode.org/faq/utf_bom.html}
					\item Licenza MIT: \\ \url{https://opensource.org/licenses/MIT}
					\item Scala Programming Language: \\ \url{http://www.scala-lang.org/}
					\item Libreria Akka: \\ \url{http://akka.io/}
					\item ISO/IEC 12207: \\ \url{http://www.iso.org/iso/catalogue_detail?csnumber=43447}
					\item Piano di progetto: \\ \emph{Piano di Progetto v1.0.0}
					\item Piano di qualifica: \\ \emph{Piano di Qualifica v1.0.0}
				\end{itemize}
			\subsection{Normativi}
				da fare dopo
		\section{Diario modifiche}
			Di seguito viene riportato il diario delle modifiche, in cui vengono elencate le modifiche apportate al documento stesso e da utilizzare 
			come template in tutti gli altri documenti.
			
			\begin{tabularx}{\textwidth}{*4{>{\centering\arraybackslash}X}}
				\noalign{\hrule height 1.5pt}
				\rowcolor{orange!85} Versione & Data & Autore & Descrizione \\
				\noalign{\hrule height 0.5pt}
				1.0.1 & 2016-01-08 & Nicoletti Luca & Aggiunto i capitoli 2 e 3 del documento di Norme di Progetto \\
				\noalign{\hrule height 1.5pt}
				1.0.2 & 2016-01-08 & Nicoletti Luca & Merge dei cambiamenti effettuati da me e da Maino Elia \\
				\noalign{\hrule height 1.5pt}
			\end{tabularx}
	\mychapter{2}{Processi primari}
		Sono state definite delle norme relative ai processi che maggiormente riguardano le attività svolte dal gruppo: fornitura e sviluppo 
		all'interno di quelli primari.
		\section{Fornitura}
			\subsection{Studio di fattibilità}
				I responsabili dello studio di fattibilità del progetto sono i membri che ricoprono il ruolo di \emph{Analisti}. In base alle prime
				riunioni effettuate, decise dal \emph{Responsabile di progetto}, le preferenze e le idee emerse per ogni singolo 
				membro del gruppo, essi dovranno stendere il documento (che verrà poi analizzato e valutato da altri membri del gruppo). Lo studio 
				di fattibilità deve contenere:
				\begin{itemize}
					\item \textbf{Dominio:} conoscenza delle tecnologie richieste e del dominio applicativo;
					\item \textbf{Rapporto costo/benefici:} eventuali prodotti simili già presenti sul mercato, competitori, costo della 
					realizzazione del prodotto e quantità di requisiti obbligatori;
					\item \textbf{Individuazione dei rischi:} evidenziare lacune tecniche e di conoscenza del dominio dei membri del gruppo, comprensione
					dei punti critici, difficoltà nel determinare i requisiti obbligatori e opzionali e nella loro classificazione; 
				\end{itemize}				 
			\subsection{Tecniche di analisi e classificazione requisiti}
				Sempre compito degli analisti, sarà quello di stilare l'analisi dei requisiti. Essi potranno ricavarli da eventuali \emph{Casi 
				d'uso} emersi da Brainstorming\ped{\textit{G}} o riunioni con il committente.
				
				I requisiti saranno elencati secondo un ordine. Ogni requisito seguirà la seguente codifica: \\
				\begin{center}
					R[Codice][Importanza][Tipo]
				\end{center}
					\textbf{Codice} \\ \\ Un codice univoco ed espresso in modo gerarchico;\\ \\
					\textbf{Importanza} \\ \\Può assumere i seguenti valori:
					\begin{itemize}
						\item \textbf{N:} Necessary (obbligatorio);
						\item \textbf{D:} Desiderable (desiderabile, a valore aggiunto);
						\item \textbf{O:} Optional (opzionale).
					\end{itemize}
					\textbf{Tipo} \\ \\Può assumere i seguenti valori:
					\begin{itemize}
						\item \textbf{F:} funzionale;
						\item \textbf{Q:} di qualità;
						\item \textbf{P:} prestazionale;
						\item \textbf{V:} vincolo.
					\end{itemize}
			\subsection{Tecniche di tracciamento dei requisiti}
				Tutti i requisiti saranno inseriti in un Database, strutturato in modo funzionale dal gruppo. Nel presente Database verranno
				inserite anche le \emph{Milestones} definite dal responsabile di turno, in modo da poter collegare requisiti e milestones. 
				Verranno creati degli appositi triggers per automatizzare la validazione dei requisiti tramite completamento manuale delle
				milestones presenti. Questo compito è competenza dell'\emph{Amministratore}.
			\subsection{Gestione cambiamento requisiti}
		\section{Sviluppo}
			\subsection{Codifica e convenzioni}
				Di seguito sono riportate le norme che il gruppo andrà a seguire durante la stesura di codice, qualsiasi esso sia.
				\\ \\
				\textbf{\LaTeX} \\ \\ 
				Regole riguardanti \LaTeX :
				\begin{itemize}
					\item Ogni file deve iniziare con 3 righe di commento come le seguenti:
					%immagine
					\begin{figure}[h!]
						\centering
						\includegraphics[scale=0.4]{Immagini/"3commentsline".png}
						\caption{Commenti ad inizio file}
					\end{figure}
					\item Ogni file deve contenere nella prima parte tutti gli **usepackage** necessari
					\item Tra ogni **begin() e end()** tutto il testo e il codice andrà indentato
					\item I commenti andranno inseriti in una riga vuota, eventualmente prima della riga di codice a cui fanno riferimento
					\item I commenti su più righe useranno il comando **begin{comment} - end{comment}**:
					%immagine
					\begin{figure}[h!]
						\centering
						\includegraphics[scale=0.4]{Immagini/"indent".png}
						\caption{Indentazione}
					\end{figure}
					\item Verrà utilizzato T1 come encoding del font: 
					\item Verrà utilizzato utf8 come encoding dell'input:
					\item Verrà utilizzato english, italian come parametro per label: ** in modo da usare inglese e italiano nello stesso documento
					tenendo italiano come lingua principale
					\item A fine documento, come commento su più righe, andrà inserita la documentazione e la descrizione (anche breve) del file
				\end{itemize}
				\textbf{\emph{Scala}} \\ \\ 
				Regole riguardanti \emph{Scala} :
				\begin{itemize}
					\item
					\item
				\end{itemize}
	\mychapter{3}{Processi di supporto}
		\section{Documentazione} %
			In questo capitolo si descrivono le convenzioni definite ed adottate dal gruppo riguardanti le 
			modalità di redazione, verifica e approvazione dei documenti. \\
			Tutti i documenti ufficiali prodotti da SWEeneyThreads sono scritti utilizzando il linguaggio \LaTeX\ped{\textit{G}}, compilati e
			 forniti in formato PDF\ped{\textit{G}} (per quanto riguarda le versioni digitali).
			\subsection{Template}
				Al fine di rendere più rapida e meno incline a differenziazioni la stesura dei diversi documenti è stato prodotto un
				 template \LaTeX, reperibile nel repository in \verb|Documenti/LaTeX/Codice| %non sono sicuro del path
			\subsection{Struttura documenti}
				La struttura dei documenti presenta una suddivisione in capitoli, sezioni e sottosezioni. \\
				Per quanto riguarda i capitoli, è stato definito un comando personalizzato \LaTeX \space denominato
				 \verb|\mychapter{}{}|:
				\begin{verbatim}
					\newcommand{\mychapter}[2]{
    				\setcounter{chapter}{#1}
    				\setcounter{section}{0}
    				\setcounter{subsection}{1}
    				\chapter*{#2}
    				\addcontentsline{toc}{chapter}{#2}
					}
				\end{verbatim}
				Per quanto riguarda sezioni e sottosezioni sono stati utilizzati i comandi standard \LaTeX \verb|\section{}| e
				 \verb|\subsection{}| \\ \\
				Di seguito viene fornita una descrizione più dettagliata di alcuni elementi di un documento: \\ \\
				\textbf{Prima pagina} \\ \\ 
					La prima pagina di un documento presenta gli elementi seguenti:
					\begin{itemize}
						\item Nome del gruppo
						\item Nome del progetto
						\item Sottotitolo del progetto
						\item Titolo del documento
						\item Cognome e nome dei redattori del documento
						\item Cognome e nome dei verificatori del documento
						\item Cognome e nome di chi approva il progetto in qualità di responsabile
						\item Logo del gruppo
						\item Numero di versione del documento
						\item Data di rilascio del documento
					\end{itemize}
					La prima pagina è parte del template disponibile nel repository. \\ \\
				\textbf{Indice} \\ \\
					In ogni documento sono presenti in ordine
					\begin{itemize}
						\item Un indice delle sezioni
						\item Un indice delle tabelle
						\item Un indice delle figure
					\end{itemize}
					Tali indici sono generati automaticamente tramite appositi comandi \LaTeX, l'assenza di figure 
					e/o tabelle nel documento comporta l'omissione del corrispondente indice. \\ \\
				\textbf{Formattazione generale delle pagine} \\ \\
					La formattazione generale di una pagina non prevede particolari personalizzazioni, si basa 
					sulla formattazione standard di \LaTeX \space usata per documenti di classe "Report". 
			\subsection{Norme tipografiche}
				Questa sezione contiene norme tipografiche e ortografiche adottate dal gruppo al fine di garantire uno stile
				 uniforme e una semantica coerente per tutti i documenti.  \\ \\
				\textbf{Stile del testo} 
					\begin{itemize}
					 \item \textbf{Corsivo:} il corsivo va utilizzato nei casi seguenti:
					 \begin{itemize}
					 	\item Citazioni
					 	\item Nomi particolari
					 	\item Documenti
					 	\item Riferimenti
					 \end{itemize}
					 A seconda della semantica del testo si utilizzano i comandi \LaTeX \space \verb|\emph{}| e \verb|\textit{}|
					 \item \textbf{Grassetto:} il grassetto va utilizzato nei casi seguenti:
					 \begin{itemize}
					 	\item Elenchi puntati: evidenzia il concetto sviluppato nella continuazione del punto
					 	\item Titoli di sottosezioni non numerate
					 \end{itemize}
					 \item \textbf{Maiuscolo:} una parola completamente in maiuscolo deve indicare un acronimo o una sigla.
					 \item \textbf{\LaTeX:} ogni riferimento al linguaggio \LaTeX \space va scritto utilizzando il comando
					  \verb|\LaTeX|
					\end{itemize}
				\textbf{Formati} 
				\begin{itemize}
					\item \textbf{Percorsi:} 
					\begin{itemize}
						\item Indirizzi email e indirizzi web completi: comando \LaTeX \space \verb|\url|
						\item Indirizzi relativi: comando \LaTeX  \space \verb|\verb|
					\end{itemize}
					\item \textbf{Date:} le date presenti nei documenti seguono lo standard ISO 8601:2004\ped{\textit{G}}:
					\begin{center}
						AAAA - MM - GG
					\end{center}
					Dove:
					\begin{itemize}
						\item AAAA rappresenta l'anno 
						\item MM rappresenta il mese
						\item GG rappresenta il giorno
					\end{itemize}
					\item \textbf{Ruoli di progetto:} quando si fa riferimento ad un ruolo di progetto questo va scritto in corsivo
					 (es. \textit{Responsabile})
					\item \textbf{Documenti:} i riferimenti vanno scritti in corsivo (es. \textit{Analisi dei requisiti})
					\item \textbf{Nomi dei file:} i nomi dei file vanno scritti utilizzando il comando \LaTeX \space 
					\verb|\verb| (es. \verb|immagine.png|)
					\item \textbf{Nomi propri:} I nomi propri seguono la forma "Cognome Nome"
					\item \textbf{Nome del gruppo:} il nome del gruppo è SWEeneyThreads, la distinzione tra lettere maiuscole e
					 minuscole va rispettata ogni volta che vi si fa riferimento
				\end{itemize}
				\textbf{Sigle} \\ \\
				L'utilizzo di sigle e abbreviazioni per riferirsi a documenti va limitato il più possibile, tuttavia nel caso il loro uso
				 fosse funzionale alla lettura (come nel caso di tabelle o diagrammi) il loro uso è consentito:
				\begin{itemize}
					\item \textbf{SdF:} Studio di Fattibilità
					\item \textbf{AdR:} Analisi dei Requisiti
					\item \textbf{GL:} Glossario
					\item \textbf{NdP:} Norme di Progetto
					\item \textbf{PdQ:} Piano di Qualifica
					\item \textbf{PdP:} Piano di Progetto
					\item \textbf{ST:} Specifica Tecnica
					\item \textbf{RR:} Revisione dei Requisiti
					\item \textbf{RP:} Revisione di Progettazione
					\item \textbf{RQ:} Revisione di Qualifica
					\item \textbf{RA:} Revisione di Accettazione
				\end{itemize}
			\subsection{Componenti grafiche}
				Le componenti grafiche previste all'interno dei documenti sono immagini e tabelle. Ogni occorrenza di un
				 elemento grafico è accompagnata da una didascalia indicizzata, in modo da poterla associare alla sezione 
				 relativa del documento. \\ \\
				\textbf{Tabelle} \\ \\ 
					Le tabelle sono definite utilizzando un template in \LaTeX \space realizzato dal gruppo e disponibile nel 
					repository all'indirizzo \verb|Documenti/Latex/Codice| \\ \\ 
				\textbf{Immagini}  \\ \\
					Il formato scelto per le immagini è Portable Network Graphics (PNG\ped{\textit{G}}). \\
					Le immagini vanno sempre inserite utilizzando la seguente sequenza di comandi \LaTeX:
					\begin{verbatim}
						\begin{figure}[h!]
							\includegraphics[scale=0-1]{Immagini/nome.png}
							\caption{Titolo - didascalia}
						\end{figure}
					\end{verbatim}
			\subsection{Classificazione documenti}
				I documenti prodotti dal gruppo si dividono in formali ed informali. \\ \\
				\textbf{Documenti Formali} \\ \\
					Quando un documento riceve l'approvazione del \emph{Responsabile} viene definito formale e risulta idoneo
					al rilascio all'esterno del gruppo. \\
					Per risultare approvato un documento deve aver completato con successo il percorso di verifica e validazione 
					descritto nel \emph{Piano di Qualifica}. \\ \\
				\textbf{Documenti informali} \\ \\
					Un documento rimane informale finché non viene approvato dal \emph{Responsabile}, durante tale fase 
					il suo uso è da considerarsi esclusivamente interno al gruppo. \\
					Alcuni documenti prodotti dal gruppo possono rimanere informali per l'intera durata del loro ciclo di vita.
			\subsection{Versionamento documenti}
				I documenti prodotti dal gruppo devono essere sempre identificati da un numero di versione del tipo:
				\begin{center}
					 X.Y.Z
				\end{center}
				Dove:
				\begin{itemize}
					\item X: è il numero principale di versione, viene incrementato ad ogni uscita formale del documento
					\item Y: viene incrementato quando si apportano modifiche sostanziali al documento
					\item Z: viene incrementato quando si apportano modifiche minori al documento
				\end{itemize}
				All'interno di un documento quando si intende fare riferimento ad una specifica versione di un altro documento la
				 notazione da utilizzare è: 
				\begin{center}
					\emph{Nome Documento vX.Y.Z}.
				\end{center}
				Mentre per fare riferimento ad un file vero e proprio:
				\begin{center}
					\verb|NomeDocumento_vX.Y.Z.estensione|
				\end{center}
		%\section{Accertamento qualità} %da valutare
			%almeno 2 revisionatori per documento e/o file
		\section{Qualifica} 
			%dopo
		\section{Risoluzione di problemi} 
			%issue tracking
	\mychapter{4}{Processi organizzativi}
		\section{Processi di gestione dell'infrastruttura}
			\subsection{Ticketing}	
				%Teamwork
			\subsection{Versioning}
				%Usiamo GitHub
			\subsection{Repository}
				%come gestiamo GitHub (immgine albero cartelle)
		\section{Processi di management}
			\subsection{Ruoli}
				%Definizione e rotazione
			\subsection{Comunicazioni}
				\textbf{Interne}
				\textbf{Esterne} %Redirect mail al responsabile di turno
			\subsection{Riunioni}
				\textbf{Ufficiali} %incontri con il committente, e formali ogni 2 settimane
				\textbf{Non ufficiali}
				\textbf{Brainstorming}

\end{document}
